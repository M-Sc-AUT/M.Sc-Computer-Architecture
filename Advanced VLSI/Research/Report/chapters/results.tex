
\فصل{مقایسه مقاله های \مرجع{article1, article2} و نتیجه گیری}

هر دو مقاله بررسی شده، به ارائه ساختاری متن‌باز جهت انجام دو فرایند مهم در طراحی به نام های چینش و مسیریابی با استفاده از تکنیک های یادگیری عمیق پرداخته است. با این تفاوت که در مقاله \مرجع{article1} محدودیتی بر روی مدار یا ساختار مورد بخث گذاشته نشده است و می‌توان الگوریتم را برای هر مدار فشرده و خیلی فشرده «VLSI» ای آموزش داد و از آن در فرایند طراحی استفاده نمود. اما در مقاله \مرجع{article2} الگوریتم ارائه شده فقط مخصوص چینش و مسیریابی FPGA هایی با منابع داخلی فراوان است. از این رو، کاربرد و عمومیت مقاله \مرجع{article1} نسبت به مقاله \مرجع{article2} بیشتر است.

هر دو مقاله برای انجام فرایند آموزش از سخت‌افزار های پیشرفته و گران قیمتی استفاده کرده اند.

در \مرجع{article1} از سروری ۴۰ هسته لینوکسی با سی‌پی‌یو 4V 2698-5E Intel با فرکانس کاری ۲٫۲ گیگاهرتز و کارت‌گرافیک 	100V Tesla NVIDIA برای فرایند آموزش استفاده شده است.

و در \مرجع{article2} از سروری لینوکس بیس که CPU آن 6230 Xeon Intel با فرکانس کاری ۲٫۱۰ گیگاهرتز و ۴۰ هسته 
هسته، با کارت گرافیک Ti2080 RTX NVIDIA و حافظه رم ۵۱۲ گیگابایتی استفاده شده است.

در هر دو مقاله سیستم های استفاده شده، سیستم های قوی و خاص منظوره ای هستند که کمتر در دسترس عموم مردم است. به همین دلیل ممکن است نتوان این الگوریتم ها را با سیستم‌های معمولی توسعه داد و این شاید به نوعی یکی از عیب‌های این دو الگوریتم به حساب آید.

در \مرجع{article1} الگوریتم با دو بنچ‌کارک اکادمیک و صنعتی تست شده است که خروجی‌های آن به صورت زیر است: «شکل \رجوع{شکل:خروجی های زمانی الگوریتم DREAMPlace برای دو بنچ‌مارک مختلف}»
	
\شروع{شکل}[ht]
\centerimg{img9.png}{15cm}
\شرح{خروجی های زمانی الگوریتم DREAMPlace برای دو بنچ‌مارک مختلف}
\برچسب{شکل:خروجی های زمانی الگوریتم DREAMPlace برای دو بنچ‌مارک مختلف}
\پایان{شکل}

در \مرجع{article2} نیز خروجی به ازای بنچ‌مارک 2017 ISPD ارائه شده است. «شکل \رجوع{شکل:خروجی های زمانی الگوریتم OpenPARF برای بنچ‌مارک 2017 ISPD}»

\شروع{شکل}[ht]
\centerimg{img10.png}{14cm}
\شرح{خروجی های زمانی الگوریتم OpenPARF برای بنچ‌مارک 2017 ISPD}
\برچسب{شکل:خروجی های زمانی الگوریتم OpenPARF برای بنچ‌مارک 2017 ISPD}
\پایان{شکل}

همانطور که در « شکل\رجوع{شکل:خروجی های زمانی الگوریتم DREAMPlace برای دو بنچ‌مارک مختلف}» مشاهده می‌شود، در الگوریتم DREAMPlace با افزایش تعداد سلول های طراحی، زمان مراحل GP و LG و ... افزایش پیدا می‌کند و سلول ها تا ۲ میلیون و ۱۷۷ هزار عدد افزایش داده شده است که برای این مقدار سلول، مرحله GP ، ۴۳ ثانیه طول کشیده است.


برای الگوریتم OpenPARF هم هرچقدر دیزاین ها پیچیده تر می‌شود، هم تعداد منابع مصرفی داخلی FPGA بیشتر مصرف می‌شود و هم زمان‌بندی فاز چینش و مسیریابی نیز زیاد تر می‌شود.


نتیجه گیری ای که این‌جانب از این مطالعات دارم بدین صورت است:

با پیشرفت ابزار‌های هوش مصنوعی، کم کم ابزار‌های قدیمی طراحی دارند جای خود را به ابزار‌های هوشمند می‌دهند. هرچند که همچنان راه زیادی وجود دارد تا جایگذینی کامل این ابزار‌ها با هم اما بلاخره روزی فرا‌خواهد رسید که قرایند طراحی از مرحله طراحی شماتیک تا Layout به صورت اتوماتیک و بدون دخالت انسان انجام می‌شود.

این دو مقاله به پیاده‌سازی ۲ مرحله مهم از طراحی، یعنی چینش قطعات و مسیر‌یابی آنها گام کوچ و موثری در تحقق این هدف برداشته اند.

امید است که بتوانیم با تحقیقات بیشتر و ارائه ساختار‌های جدید به پیشرفت تکنولوژی در این مسیر کمک کنیم.