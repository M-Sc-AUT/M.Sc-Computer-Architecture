
\فصل{مقدمه}

مسئله چینش\پاورقی{Placement} و سیم‌کشی\پاورقی{Routing} از گذشته تا به امروز جزئی از مهمترین مسائل طراحی آیسی است. با پیشرفت شبکه‌های عصبی مصنوعی و استفاده گسترده آنها در کاربرد‌ها و اپلیکیشن‌های متفاوت محققان حوزه طراحی IC در صدد برآمدند که بخش چینش و سیم‌کشی طراحی را که یکی از وقت‌گیر ترین مراحل طراحی است را به کمک شبکه‌های عصبی انجام دهند.

%\شروع{شکل}[ht]
%\centerimg{img0}{15cm}
%\شرح{ساختار کلی حافظه‌ها}
%\برچسب{شکل:ساختار حافظه‌ها}
%\پایان{شکل}



\قسمت{تعریف مسئله}




مراحل طراحی IC را می‌توان به ۴ مرحله زیر تقسیم کرد:
\شروع{شمارش}
\فقره طراحی شماتیک بخش‌های مختلف مدار
\فقره آنالیز و بررسی طراحی انجام شده و اطمینان از صحت عملکرد مدار به وسیله نرم‌افزار های شبیه‌سازی مانند SPICE و Cadence
\فقره مرحله Layout که شامل Placement و Routing است
\فقره ساخت آیسی یا Fabrication
\پایان{شمارش}

مرحله ۱ و ۲ باید به صورت دستی و توسط انسان انجام شود. چرا که شخص طراح می‌بایست بر همه بخش‌های طراحی خود مسلط باشد و بتواند اگر نیاز بود بخش‌های دیگیری به طراحی اضافه و یا از آن کم شود، آن را اعمال کند. اما هوش مصنوعی به این مرحله نیز وارد شده است و ابزار‌هایی ماندد Magic EDA\زیرنویس{برای اطلاعات بیشتر می‌توان به اینجا مراجعه کرد: \href{www.snapmagic.com}{snapmagic.com}} مخصوص این کار آموزش داده شده است که با دادن اطلاعات مورد نیاز خود برای طراحی، مدار مورد نیاز ما را به صورت کامل طراحی می‌کند. که در این گزارش به آن نمی‌پردازیم.


در مرحله ساخت آیسی\پاورقی{Fabrication} نیز هوش‌مصنوعی به صورت محدود وارد شده است و همچنان مرحله ساخت به صورت قدیمی و سنتی انجام می‌شود.


در گذشته، در مرحله ۳، طراحی‌ ها با استفاده از ابزار‌های کامپیوتری CAD\پاورقی{Computer Aided Design} انجام می‌شود. از مزایا ابزار‌های CAD می‌توان به موارد زیر اشاره کرد:

\شروع{فقرات}
\فقره تحلیل‌ دقیق
\فقره تولید خروجی باکیفیت
\فقره پشتیبانی گسترده نرم‌افزاری
\پایان{فقرات}

اما در کنار مزایای نامبرده می‌توان به معایب آن هم اشاره کرد:
\شروع{فقرات}
\فقره لزوم وجود کاربر انسانی\پاورقی{Designer} برای انجام طراحی
\فقره زمان زیاد برای انجام
\فقره هزینه بسیار بالای ابزار‌های CAD
\پایان{فقرات}


با پیشرفت ابزار‌های هوش‌مصنوعی مانند شبیه‌های عصبی\پاورقی{Neural Network} ابزار‌های مختلفی که برپایه شبکه‌های عصبی کار می‌کنند معرفی شده است.  این ابزار‌ها با حذف اپراتور انسانی در فرایند Place\&Route و کاهش زمان انجام این فاز از طراحی، کمک بزرگی به این زمینه کرده است.



\قسمت{اهمیت موضوع}
در طراحی‌های تجاری، طراح‌ها مجبور‌اند چندین بار طراحی خود را برای دست‌یابی به بهترین و بهینه\پاورقی{Optimum} ترین حالت عوض کنند. استفاده از روش‌های طراحی‌ سنتی قدیمی، برای مدار‌های بزرگ\پاورقی{Complex} امروزی، بسیار فرایندی طولانی و کند است که فرایند تکرار طراحی برای دستیابی به بهینه‌ترین حالت جایگذاری و سیم‌کشی را به شدت کند می‌کند.





\قسمت{اهداف پژوهش}
در اصل، در این مقالات، یک مسئله بهینه‌سازی غیر خطی حل شده است و به مسئله جایگذاری و سیم‌کشی به عنوان یک فرایند غیرخطی نگاه شده است که قرار است آن را بهینه کنیمو به طوری اهداف ما یعنی پیدا کردن بهترین محل قرار گیری سلول‌های طراحی با حداقل سیم‌کشی ممکن که کمترین همپوشانی را داشته باشد ارضا شود.



%
%
%در این پروژه هدف طراحی و شبیه‌سازی file Register ای با اندازه ۱۲۸ کلمه ۳۲ بیت است. هدف از انجام این پروژه آشنایی و انواع حافظه‌ها و نحوه شبیه‌سازی و پیاده‌سازی آنهاست. حافظه‌ها معمولا از دو بخش تشکیل می‌شوند:
%\صفحه‌جدید
%\شروع{فقرات}
%\فقره بخش حافظه
%\فقره بخش سخت‌افزار
%\پایان{فقرات}
%
%بخش حافظه مبتنی‌ست بر تکرار یک طراحی مشخص از یک سلول\پاورقی{Cell} حافظه با چینشی مشخص. بخش سخت افزار آن متشکل است از دیکدر\پاورقی{Decoder} آدرس و مالتی‌پلکسر\پاورقی{Multiplaxer} داده ها.
%
%
%
%\قسمت{مراحل انجام پروژه}
%
%برای اطمینان از انجام پروژه و مقایسه بین مدل  RTL و مدل سطح ترانزیستوری، این پروژه به دو بخش تقسیم شده است.
%
%\زیرقسمت{فاز اول}
%در فاز اول با استفاده از زبان توصیف سخت‌افزار\پاورقی{Hardware describtion language} VHDL کد RTL حافظه SRAM نوشته و شبیه‌سازی شده است.
%
%برای مقایسه بین دو مدل RTL و رفتاری\پاورقی{Behavioral} حافظه، یک کد مجزا هم برای شبیه‌سازی مدل رفتاری نوشته شده است.
%
%
%\زیرقسمت{فاز دوم}
%در فاز دوم پروژه مقیاس طراحی را به سطح ترانزیستور می‌آوریم و به کمک نرم‌افزار HSpice مدل شبیه‌سازی شده در فاز اول را در سطح ترانزیستور شبیه‌سازی می‌کنیم.
%
%\مهم{*** به دلیل عدم تکمیل فاز دوم پروژه در این ددلاین، این بخش از گزارش پس از تکمیل فاز دو تکمیل خواهد شد.***}
