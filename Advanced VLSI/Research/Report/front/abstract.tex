% -------------------------------------------------------
%  Abstract
% -------------------------------------------------------


\شروع{وسط‌چین}
\مهم{چکیده}
\پایان{وسط‌چین}
\بدون‌تورفتگی

جایگذاری قطعات و سیم‌کشی مدار‌های مجتمع در مقیاس های خیلی بزرگ، همواره یکی از چالشی ترین مراحله ها در فرایند طراحی مدار است. در گذشته این دو مرحله به صورت دستی توسط اپراتور انسانی انجام می‌شد. به طوری که ممکن بود بار و بارها طراحی انجام شده به دلیل برخی از ملاحظات فنی عوض می‌شد و این تغییر دادن طراح و سیم‌کشی مجدد آن تایم زیادی را نیاز دارد که با بزرگ شدن طراحی‌ها در مدار‌های مجتمع امروزی فرایندی سخت و حتی شاید غیر ممکن به نظر برسد. پیشرفت تکنولوژی در حوزه هوش مصنوعی و یادگیری عمیق، دنیای طراحی مدار‌های مجتمع را هم دستخوش تغییراتی کرده است. به‌طوری که امروزه دنیای طراحی آیسی به سمتی می‌رود که فرایند جایگذاری و سیم‌کشی آیسی بدون دخالت انسال و به‌طور کاملا خودکار با دقت بالا و خطای بسیار کم انجام شود.

در این گزارش به بررسی، بیان مزایا و معایب دو روش مشابه برای این‌کار که در مقاله های \مرجع{article1} و \مرجع{article2} پیشنهاد شده است پرداخته ایم.

\پرش‌بلند
\بدون‌تورفتگی \مهم{کلیدواژه‌ها}: 
جایگذاری، سیم‌کشی، یادگیری عمیق
\صفحه‌جدید
