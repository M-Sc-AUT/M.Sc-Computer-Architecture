
% -------------------------------------------------------
%  English Abstract
% -------------------------------------------------------


\pagestyle{empty}

\begin{latin}

\begin{center}
\textbf{Abstract}
\end{center}
\baselineskip=.8\baselineskip

The placement of components and routing of integrated circuits on very large scales has always been one of the most challenging stages in the circuit design process. In the past, these two stages were manually performed by human operators. This manual process often led to repeated designs due to certain technical considerations, requiring the redesign and rerouting of the circuit, which consumed a considerable amount of time. As circuit designs have grown in complexity in modern integrated circuits, this process has become increasingly difficult and perhaps even seemingly impossible. Technological advancements in the field of artificial intelligence and deep learning have brought about changes in the world of integrated circuit design. Today, the world of IC design is moving towards a direction where the placement and routing of ICs can be done without human intervention, entirely automatically, with high precision and very low error rates.

In this report, we will examine and discuss the advantages and disadvantages of two similar methods proposed in the articles \cite{article1} and \cite{article2} for achieving this.


\bigskip\noindent\textbf{Keywords}:
Placement, Routing, Deep Learning

\end{latin}
