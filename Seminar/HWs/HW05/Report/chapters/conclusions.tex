
\فصل{نتیجه‌گیری} \label{conclusion}



پیش‌بینی عمر باقی‌مانده مفید یکی از مسائل کلیدی در نگهداری پیش‌گیرانه و مدیریت سلامت سیستم‌های صنعتی است. با افزایش پیچیدگی سیستم‌ها و اهمیت بهینه‌سازی عملکرد، استفاده از روش‌های پیشرفته برای پیش‌بینی دقیق‌تر و سریع‌تر \lr{RUL} ضروری شده است. شبکه‌های عصبی مصنوعی، به‌ویژه شبکه‌های ترنسفرمر، به دلیل توانایی بسیار بالای آنها در درک وابستگی‌ها در داده‌های سری زمانی به عنوان یکی از ابزارهای مؤثر در این زمینه مطرح شده‌اند. شبکه‌های ترنسفرمر با بهره‌گیری از مکانیسم توجه، قادر به یادگیری وابستگی‌های بلندمدت در داده‌ها هستند و این ویژگی آن‌ها را برای کاربردهایی که نیاز به تحلیل دقیق و سریع داده‌های حسگر دارند، بسیار مناسب می‌سازد.


استفاده از \lr{FPGA} برای پیاده‌سازی شبکه‌های عصبی می‌تواند بسیاری از محدودیت‌های موجود در زمینه پردازش داده‌ها و مصرف انرژی را برطرف کند. \lr{FPGA}‌ها به دلیل قابلیت پیکربندی مجدد و پردازش موازی، می‌توانند سرعت پردازش را به طور قابل توجهی افزایش و مصرف انرژی را کاهش دهند. این ویژگی‌ها به‌ویژه در کاربردهای حیاتی مانند خودروهای خودران که نیاز به پیش‌بینی‌های بلادرنگ و با دقت بالا دارند، بسیار حائز اهمیت است. به عنوان مثال، پیاده‌سازی فاز استنتاج شبکه‌های \lr{BERT} و \lr{GPT} بر روی \lr{FPGA} نشان داده است که سرعت و کارایی این شبکه‌ها به‌طور چشمگیری بهبود یافته و مصرف انرژی کاهش یافته است.


یکی از چالش‌های اصلی در پیاده‌سازی شبکه‌های ترنسفرمر بر روی \lr{FPGA}، مدیریت حافظه محدود این سخت‌افزارهاست. برای حل این مشکل، می‌توان از تکنیک‌هایی مانند \lr{FIFO} استفاده کرد که به ما امکان می‌دهد تا داده‌ها را به‌صورت مرحله‌ای پردازش و ذخیره کنیم. با تنظیم حجم \lr{FIFO} براساس زمان پردازش سیستم، می‌توان اطمینان حاصل کرد که تمام داده‌های ورودی بدون از دست رفتن اطلاعات مهم پردازش شوند. این روش نه تنها کارایی پردازش را بهبود می‌بخشد بلکه نیاز به حافظه بزرگ را کاهش داده و از محدودیت‌های موجود در \lr{FPGA} بهره‌برداری می‌کند. به این ترتیب، ترکیب شبکه‌های ترانسفورمر با \lr{FPGA} می‌تواند راه‌حلی کارآمد و بهینه برای پیش‌بینی \lr{RUL} در سیستم‌های صنعتی و خودروهای خودران ارائه دهد.





%گزارشی که مطالعه نمودید، گزارش کتبی و نهایی اینجانب برای درس سمینار بود. در این گزارش سعی شد موضوع، پیش‌نیازها و صورت‌مسئله دقیق پایان‌نامه اینجانب شرح داده شود. در یک فصل به طور کامل و جامع عمده کارهای پژوهشی اخیر در این حوزه بررسی شد و در نهایت چالش‌ها و نوآوری‌های این‌جانب در این پژوهش آورده شده است.
%
%تمامی اشکال به‌کاررفته در این گزارش (به‌غیراز اشکالی که ارجاع داده شده است) توسط اینجانب و با استفاده از ابزار \texttt{\textcolor{magenta}{Ipe}}\زیرنویس{می‌توانید این نرم‌افزار را از اینجا دریافت کنید: \href{https://ipe.otfried.org/}{\texttt{ipe.otfried.org}}} کشیده شده است.
%
%
%از ابتدای نوشتن این گزارش هدف آن بود که ان‌شاءالله بتوانم بخش زیادی از مطالب نوشته شده در این گزارش را مستقیماً در پایان‌نامه نهایی‌ام مورداستفاده قرار دهم. امید است این امر محقق شود.