
\فصل{مفاهیم اولیه}\label{basic}


\قسمت{عمر باقی‌مانده مفید}
مانند انسان‌ها، همه دستگاه‌ها و قطعات نیز عمری دارند و برای پایش سلامت دستگاه نیاز است که بتوانیم از عمر باقی‌مانده قطعه مطلع باشیم.

عمر یک قطعه را می‌توان به‌وسیله یک تابع خطی که آن را تابع \lr{RUL} می‌نامیم «شکل \رجوع{شکل:تابع تقریب زننده عمر باقی مانده یک دستگاه}» تقریب بزنیم.



\شروع{شکل}[ht]
\centerimg{img1.pdf}{10cm}
\شرح{تابع تقریب زننده عمر باقی مانده یک دستگاه}
\برچسب{شکل:تابع تقریب زننده عمر باقی مانده یک دستگاه}
\پایان{شکل}


محور عمودی در شکل «\رجوع{شکل:تابع تقریب زننده عمر باقی مانده یک دستگاه}» نشان‌دهنده میزان سلامت دستگاه و محور افقی نشان‌دهنده زمان است که معمولاً برحسب دقیقه بیان می‌شود.

همه دستگاه‌ها زمانی که در آستانه بروز خطا و خرابی قرار می‌گیرند رفتار غیرعادی از خودشان نشان می‌دهند. از جمله این رفتارها می‌توان به نوسانات غیرطبیعی، افزایش دمای دستگاه، افزایش سروصدا\پانویس{Noise} در دستگاه اشاره نمود.


برای مثال در \مرجع{wei2024conditional} سیگنال ارتعاشات یک بلبرینگ به‌عنوان یکی از اصلی‌ترین قطعات صنعتی از ابتدای شروع به کار تا زمان بروز اولین تخریب\پانویس{First Prediction Time} و تخریب کامل\پانویس{End of Life} در مدت‌زمان ۶۹ روز جمع‌آوری شده است. «شکل \رجوع{شکل:سیگنال ارتعاش بلبرینگ از ابتدای فعالیت تا تخریب کامل}»




\شروع{شکل}[ht]
\centerimg{img2.png}{11cm}
\شرح{سیگنال ارتعاش بلبرینگ از ابتدای فعالیت تا تخریب کامل \مرجع{wei2024conditional}}
\برچسب{شکل:سیگنال ارتعاش بلبرینگ از ابتدای فعالیت تا تخریب کامل}
\پایان{شکل}





\زیرقسمت{اولین زمان خرابی}
اولین زمانی را که دستگاه دچار نوسانات شدید می‌شود را به‌عنوان اولین زمان شروع فرایند تخریب در نظر می‌گیریم و آن را «\lr{FPT}» می‌نامیم.


\زیرقسمت{عمر پایانی دستگاه}
با افزایش دامنه نوسانات ثبت شده از دستگاه، تخریب دستگاه بیشتر شده و دستگاه گرم‌تر می‌شود. از این فرایند به‌عنوان یک بازخورد\پانویس{Feedback} مثبت یاد می‌شود که افزایش گرما، نوسانات را بیشتر کرده و نوسانات بیشتر نیز گرمای دستگاه را افزایش می‌دهد. با تشدید هرچه بیشتر نوسانات، دستگاه به پایان عمر خود نزدیک‌تر شده و درنهایت از کار می‌افتد. زمان از کار فتادن نهایی دستگاه را به‌عنوان زمان پایان زندگی «\lr{EOF}» تعریف می‌کنیم.


و درنهایت سیگنال \lr{RUL} به‌صورت تفاضل این دوزمان تعریف می‌شود:

\شروع{equation}
T_{RUL} = T_{EOF} - T_{FPT}
\پایان{equation}

این سیگنال از جنس زمان است و مقدار \lr{RUL} در این باز زمانی از رابطه زیر پیروی می‌کند:

\شروع{equation}
RUL(t) = -t
\پایان{equation}


برای مثال سیگنال \lr{RUL} برای شکل «\رجوع{شکل:سیگنال ارتعاش بلبرینگ از ابتدای فعالیت تا تخریب کامل}» به صورت زیر محاسبه می‌شود:

\شروع{شکل}[ht]
\centerimg{img3.png}{12cm}
\شرح{سیگنال \lr{RUL} شکل \رجوع{شکل:سیگنال ارتعاش بلبرینگ از ابتدای فعالیت تا تخریب کامل}، \مرجع{wei2024conditional}}
\برچسب{شکل:تابع تقریب زننده عمر باقی مانده یک دستگاه}
\پایان{شکل}






\قسمت{داده‌ها}
برای پیش‌بینی عمر باقی‌مانده مفید، چندین مجموعه‌داده\پانویس{Dataset} وجود دارد که در ادامه آنها را معرفی و بررسی می‌کنیم.

\زیرقسمت{مجموعه داده \lr{XJTU-SY}}
این مجموعه‌داده\زیرنویس{می‌توانید این مجموعه‌داده را از اینجا دانلود کنید: \href{https://biaowang.tech/xjtu-sy-bearing-datasets/}{\textcolor{magenta}{\texttt{biaowang.tech/xjtu-sy-bearing-datasets/}}}} شامل داده‌های ثبت شده از ۱۵ بلبرینگ است که با انجام آزمایش‌های تخریب سریع، دچار تخریب شده‌اند. \مرجع{wang2018hybrid} 



این مجموعه‌داده توسط سیستمی که در شکل «\رجوع{شکل:بستر تهیه مجموعه‌داده XJTU-SY}» نشان‌داده‌شده است، متشکل از یک موتور القایی جریان متناوب\پانویس{Alternating Current}، یک کنترل‌کننده سرعت موتور، یک محور\پانویس{Shaft} پشتیبان، دو بلبرینگ پشتیبان (بلبرینگ‌های سنگین) و یک سیستم بارگذاری هیدرولیک تشکیل شده است.

این بستر آزمون برای انجام آزمایش‌های تخریب تسریع‌شده بلبرینگ‌، تحت شرایط مختلف عملیاتی (نیروی شعاعی و سرعت چرخشی مختلف) طراحی شده است. نیروی شعاعی توسط سیستم بارگذاری هیدرولیک تولید شده و به محفظه بلبرینگ‌های آزمایش شده اعمال می‌شود و سرعت چرخش نیز توسط کنترلر سرعت موتور القایی \lr{AC} تنظیم و نگه داشته می‌شود.








\شروع{شکل}[ht]
\centerimg{img4.png}{11cm}
\شرح{بستر تهیه مجموعه‌داده \lr{XJTU-SY}، \مرجع{wang2018hybrid}}
\برچسب{شکل:بستر تهیه مجموعه‌داده XJTU-SY}
\پایان{شکل}




بلبرینگ‌های مورداستفاده در این آزمایش از نوع \lr{LDK UER204} هستند که پارامترهای دقیق آنها در جدول «\رجوع{جدول:پارامتر‌های بلبرینگ‌های آزمایش شده}» آورده شده است.


\شروع{لوح}[ht]
\تنظیم‌ازوسط
\شرح{پارامتر‌های بلبرینگ‌های آزمایش شده}

\شروع{جدول}{c c c c}
\خط‌پر 
\سیاه پارامتر & \سیاه مقدار & \سیاه پارامتر & \سیاه مقدار \\ 
\خط‌پر \خط‌پر 
قطر مسیر بیرونی & \lr{mm} ۸۰٫۳۹ & قطر مسیر داخلی &  \lr{mm} ۳۰٫۲۹ \\ 
قطر متوسط بلبرینگ &  \lr{mm} ۵۵٫۳۴ & قطر توپ &  \lr{mm} ۹۲٫۷ \\ 
تعداد توپ‌ها & ۸ & زاویه تماس &  \textdegree ۰ \\ 
بار استاتیک & \lr{kN} ۶۵٫۶ & بار دینامیک & \lr{kN} ۸۲٫۱۲ \\ 
\خط‌پر
\پایان{جدول}

\برچسب{جدول:پارامتر‌های بلبرینگ‌های آزمایش شده}
\پایان{لوح}




این آزمایش، تحت ۳ شرط عملیاتی مختلف انجام شده است و هر ۵ بلبرینگ موجود در این آزمایش تحت این سه شرط عملیاتی قرار گرفته‌اند. این شرایط عملیاتی شامل موارد زیر هستند:

\شروع{فقرات}
\فقره ۲۱۰۰ دور در دقیقه\پانویس{RPM} (۳۵ هرتز) و بار دینامیکی ۱۲ کیلو نیوتون
\فقره ۲۲۵۰ دور در دقیقه (۵٫۳۷ هرتز) و بار دینامیکی ۱۱ کیلو نیوتون
\فقره ۲۴۰۰ دور در دقیقه (۴۰ هرتز) و بار دینامیکی ۱۰ کیلو نیوتون

\پایان{فقرات}


برای جمع‌آوری سیگنال‌های ارتعاشی بلبرینگ‌های آزمایش شده، همان‌طور که در شکل «\رجوع{شکل:بستر تهیه مجموعه‌داده XJTU-SY}» نشان‌داده‌شده است، دو شتاب‌سنج از نوع 
۳۳\lr{C}۳۵۲\lr{PCB}
 در زاویه ۹۰ درجه بر روی محفظه بلبرینگ‌های آزمایش شده قرار داده شده است، یعنی یکی بر روی محور افقی و دیگری بر روی محور عمودی نصب شده است.


همچنین فرکانس نمونه‌برداری بر روی ۶٫۲۵ کیلوهرتز تنظیم شده است. همانطور که در شکل «\رجوع{شکل:تنظیمات نمونه‌برداری برای سیگنال‌های ارتعاشی}» نشان داده شده است، در مجموع ۳۲۷۶۸ نقطه داده (به مدت ۲۸٫۱ ثانیه) برای هر نمونه‌برداری ثبت می‌شوند و دوره نمونه‌برداری برابر با 1 دقیقه است.


\شروع{شکل}[ht]
\centerimg{img5.pdf}{12cm}
\شرح{تنظیمات نمونه‌برداری برای سیگنال‌های ارتعاشی، \مرجع{wang2018hybrid}}
\برچسب{شکل:تنظیمات نمونه‌برداری برای سیگنال‌های ارتعاشی}
\پایان{شکل}




برای هر نمونه‌برداری، داده‌های به‌دست‌آمده در یک فایل \texttt{csv} ذخیره شده است که در آن ستون اول، سیگنال‌های ارتعاشی افقی و ستون دوم سیگنال‌های ارتعاشی عمودی را شامل می‌شود. جدول «\رجوع{اطلاعات مجموعه‌داده XJTU-SY}» اطلاعات دقیق هر بلبرینگ آزمایش شده، شامل تعداد فایل‌های \texttt{csv}، عمر بلبرینگ و عنصر خرابی را فهرست می‌کند.



\begin{table}[h!]
	\centering
	\caption{اطلاعات مجموعه‌داده \lr{XJTU-SY}}
	\label{اطلاعات مجموعه‌داده XJTU-SY}
	\resizebox{\textwidth}{!}{
		\begin{tabular}{llccc}
			\toprule
			\textbf{شرایط عملکرد} & \textbf{مجموعه‌داده‌های بلبرینگ} & \textbf{تعداد فایل‌ها} & \textbf{طول عمر بلبرینگ} & \textbf{عنصر خطا} \\
			\midrule
			\multirow{5}{*}{\shortstack[l]{شرایط ۱ \\ (۳۵ هرتز، ۱۲ کیلونیوتن)}} & بلبرینگ ۱\_۱ & ۱۲۳ & ۲ ساعت و ۳ دقیقه & بیرونی \\
			& بلبرینگ ۱\_۲ & ۱۶۱ & ۲ ساعت و ۴۱ دقیقه & بیرونی \\
			& بلبرینگ ۱\_۳ & ۱۵۸ & ۲ ساعت و ۳۸ دقیقه & بیرونی \\
			& بلبرینگ ۱\_۴ & ۱۲۲ & ۲ ساعت و ۲ دقیقه & قفسه \\
			& بلبرینگ ۱\_۵ & ۵۲ & ۵۲ دقیقه & داخلی و بیرونی \\
			\midrule
			\multirow{5}{*}{\shortstack[l]{شرایط ۲ \\ (۵٫۳۷ هرتز، ۱۱ کیلونیوتن)}} & بلبرینگ ۲\_۱ & ۴۹۱ & ۸ ساعت و ۱۱ دقیقه & داخلی \\
			& بلبرینگ ۲\_۲ & ۱۶۱ & ۲ ساعت و ۴۱ دقیقه & بیرونی \\
			& بلبرینگ ۲\_۳ & ۵۳۳ & ۸ ساعت و ۵۳ دقیقه & قفسه \\
			& بلبرینگ ۲\_۴ & ۴۲ & ۴۲ دقیقه & بیرونی \\
			& بلبرینگ ۲\_۵ & ۳۳۹ & ۵ ساعت و ۳۹ دقیقه & بیرونی \\
			\midrule
			\multirow{5}{*}{\shortstack[l]{شرایط ۳ \\ (۴۰ هرتز، ۱۰ کیلونیوتن)}} & بلبرینگ ۳\_۱ & ۲۵۳۸ & ۴۲ ساعت و ۱۸ دقیقه & بیرونی \\
			& بلبرینگ ۳\_۲ & ۲۴۹۶ & ۴۱ ساعت و ۳۶ دقیقه & داخلی، ساچمه، قفسه و بیرونی \\
			& بلبرینگ ۳\_۳ & ۳۷۱ & ۶ ساعت و ۱۱ دقیقه & داخلی \\
			& بلبرینگ ۳\_۴ & ۱۵۱۵ & ۲۵ ساعت و ۱۵ دقیقه & داخلی \\
			& بلبرینگ ۳\_۵ & ۱۱۴ & ۱ ساعت و ۵۴ دقیقه & بیرونی \\
			\bottomrule
		\end{tabular}
	}
\end{table}



نقطه \lr{EOF} در این مجموعه‌داده $10 \times A_H$ در نظر گرفته شده است که $A_H$ بیشینه دامنه سیگنال ارتعاش عمودی و افقی در حالت کاری طبیعی است.


در شکل‌های «\رجوع{شکل:سیگنال‌های ارتعاش افقی و عمودی و RUL بلبرینگ ۱ـ۲}» و «\رجوع{شکل:سیگنال‌های ارتعاش افقی و عمودی و RUL بلبرینگ ۲ـ۲}» نمونه‌ای از داده‌های این مجموعه‌داده آورده شده است.



\شروع{شکل}[ht]
\centerimg{img6_Bearing1_2.pdf}{16cm}
\شرح{سیگنال‌های ارتعاش افقی و عمودی و \lr{RUL} بلبرینگ ۱ـ۲}
\برچسب{شکل:سیگنال‌های ارتعاش افقی و عمودی و RUL بلبرینگ ۱ـ۲}
\پایان{شکل} 
 
 
\شروع{شکل}[ht] 
\centerimg{img7_Bearing2_2.pdf}{16cm} 
\شرح{سیگنال‌های ارتعاش افقی و عمودی و \lr{RUL} بلبرینگ ۲ـ۲}
\برچسب{شکل:سیگنال‌های ارتعاش افقی و عمودی و RUL بلبرینگ ۲ـ۲}
\پایان{شکل}




\زیرقسمت{مجموعه داده \lr{PRONOSTIA}}


\زیرقسمت{مجموعه داده \lr{C-MAPSS}}

































%
%
%
%
%دومین فصل پایان‌نامه به طور معمول به معرفی مفاهیمی می‌پردازد که در پایان‌نامه مورد استفاده قرار می‌گیرند.
%در این فصل به عنوان یک نمونه، نکات کلی در خصوص نحوه‌ی نگارش پایان‌نامه
%و نیز برخی نکات نگارشی به اختصار توضیح داده می‌شوند.
%
%\قسمت{نحوه‌ی نگارش}
%
%\زیرقسمت{پرونده‌ها}
%
%پرونده‌ی اصلی پایان‌نامه در قالب استاندارد\زیرنویس{
%قالب استاندارد از گیت‌هاب به نشانی
%\href{https://github.com/zarrabi/thesis-template}
%{github.com/zarrabi/thesis-template}
%قابل دریافت است.}
%\کد{thesis.tex}  نام دارد.
%به ازای هر فصل از پایان‌نامه، یک پرونده در شاخه‌ی \کد{chapters} ایجاد نموده
%و نام آن را در  \کد{thesis.tex} (در قسمت فصل‌ها) درج نمایید.
%برای مشاهده‌ی خروجی، پرونده‌ی \کد{thesis.tex} را با زی‌لاتک کامپایل کنید.
%مشخصات اصلی پایان‌نامه را می‌توانید در پرونده‌ی \کد{front/info.tex} ویرایش کنید.
%
%\زیرقسمت{عبارات ریاضی}
%
%برای درج عبارات ریاضی در داخل متن از \$...\$ و 
%برای درج عبارات ریاضی در یک خط مجزا از \$\$...\$\$ یا محیط \لر{equation} 
%استفاده کنید. برای مثال عبارت 
%$2x + 3y$
%در داخل متن و عبارت زیر
%\begin{equation}
%\sum_{k=0}^{n} {n \choose k} = 2^n
%\end{equation}
%در یک خط مجزا درج شده است. 
%دقت کنید که تمامی عبارات ریاضی، از جمله متغیرهای تک‌حرفی مانند $x$ و $y$ باید در محیط ریاضی 
%یعنی محصور بین دو علامت \$ باشند. 
%
%
%\زیرقسمت{علائم ریاضی پرکاربرد}
%
%برخی علائم ریاضی پرکاربرد در زیر فهرست شده‌اند. 
%برای مشاهده‌ی دستور  معادل پرونده‌ی منبع را ببینید.
%
%
%\شروع{فقرات}
%\فقره مجموعه‌‌های اعداد: 
%$\IN, \IZ, \IZ^+, \IQ, \IR, \IC$
%\فقره مجموعه:
%$\set{1, 2, 3}$
%\فقره دنباله‌:
%$\seq{1, 2, 3}$
%\فقره سقف و کف:
%$\ceil{x}, \floor{x}$
%\فقره اندازه و متمم:
%$\card{A}, \setcomp{A}$
%\فقره همنهشتی:
%$a \iequiv{n} 1$
%یا
%$a \equiv 1 \imod{n}$ 
%%\فقره شمردن (عاد کردن):
%%$3 \divs n, 2 \ndivs n$
%\فقره ضرب و تقسیم:
%$\times, \cdot, \div$
%\فقره سه‌نقطه‌:
%$1, 2, \dots, n$
%\فقره کسر و ترکیب:
%${n \over k}, {n \choose k}$
%\فقره اجتماع و اشتراک:
%$A \cup (B \cap C)$
%\فقره عملگرهای منطقی:
%$\neg p \vee (q \wedge r)$
%
%\فقره پیکان‌ها:
%$\rightarrow, \Rightarrow, \leftarrow, \Leftarrow, \leftrightarrow, \Leftrightarrow$
%\فقره عملگرهای مقایسه‌ای:
%$\not=, \le, \not\le, \ge, \not\ge$
%\فقره عملگرهای مجموعه‌ای:
%$\in, \not\in, \setminus, \subset, \subseteq, \subsetneq, \supset, \supseteq, \supsetneq$
%
%\فقره جمع و ضرب چندتایی:
%$\sum_{i=1}^{n} a_i, \prod_{i=1}^{n} a_i$
%\فقره اجتماع و اشتراک چندتایی:
%$\bigcup_{i=1}^{n} A_i, \bigcap_{i=1}^{n} A_i$
%\فقره برخی نمادها:
%$\infty, \emptyset, \forall, \exists, \triangle, \angle, \ell, \equiv, \therefore$
%\پایان{فقرات}
%
%
%\زیرقسمت{لیست‌ها}
%
%برای ایجاد یک لیست‌ می‌توانید از محیط‌های «فقرات» و «شمارش» همانند زیر استفاده کنید.
%
%\begin{multicols}{2}
%\شروع{فقرات}
%\فقره مورد اول
%\فقره مورد دوم
%\فقره مورد سوم
%\پایان{فقرات}
%
%\شروع{شمارش}
%\فقره مورد اول
%\فقره مورد دوم
%\فقره مورد سوم
%\پایان{شمارش}
%
%\end{multicols}
%
%
%\زیرقسمت{درج شکل}
%
%یکی از روش‌های مناسب برای ایجاد شکل استفاده از نرم‌افزار \لر{LaTeX Draw} و سپس
%درج خروجی آن به صورت یک فایل \کد{tex} درون متن 
%با استفاده از دستور  \کد{fig} یا \کد{centerfig} است.
%شکل~\رجوع{شکل:پوشش رأسی} نمونه‌ای از اشکال ایجادشده با این ابزار را نشان می‌دهد.
%
%
%\شروع{شکل}[ht]
%\centerfig{cover.tex}{.9}
%\شرح{یک گراف و پوشش رأسی آن}
%\برچسب{شکل:پوشش رأسی}
%\پایان{شکل}
%
%\bigskip
%همچنین می‌توانید با استفاده از نرم‌افزار \lr{Ipe} شکل‌های خود را مستقیما
%به صورت \لر{pdf} ایجاد نموده و آن‌ها را با دستورات \کد{img} یا  \کد{centerimg} 
%درون متن درج کنید. برای نمونه، شکل~\رجوع{شکل:گراف جهت‌دار} را ببینید.
%
%
%\شروع{شکل}[ht]
%\centerimg{strip}{6.5cm}
%\شرح{نمونه شکل ایجادشده توسط نرم‌افزار \lr{Ipe}}
%\برچسب{شکل:گراف جهت‌دار}
%\پایان{شکل}
%
%
%\زیرقسمت{درج جدول}
%
%برای درج جدول می‌توانید با استفاده از دستور  «جدول»
%جدول را ایجاد کرده و سپس با دستور  «لوح»  آن را درون متن درج کنید.
%برای نمونه جدول~\رجوع{جدول:عملگرهای مقایسه‌ای} را ببینید.
%
%\vspace{1.5em}
%
%\شروع{لوح}[ht]
%\تنظیم‌ازوسط
%\شرح{عملگرهای مقایسه‌ای}
%
%\شروع{جدول}{|c|c|}
%\خط‌پر 
%\سیاه عملگر & \سیاه عنوان \\ 
%\خط‌پر \خط‌پر 
%\کد{<} & کوچک‌تر \\ 
%\کد{>} & بزرگ‌تر \\
%\کد{==} &  مساوی \\ 
%\کد{<>} & نامساوی \\ 
%\خط‌پر
%\پایان{جدول}
%
%\برچسب{جدول:عملگرهای مقایسه‌ای}
%\پایان{لوح}
%
%
%
%\زیرقسمت{درج الگوریتم}
%
%برای درج الگوریتم می‌توانید از محیط «الگوریتم» استفاده کنید.
%یک نمونه در الگوریتم~\رجوع{الگوریتم: پوشش رأسی حریصانه} آمده است.
%
%\شروع{الگوریتم}{پوشش رأسی حریصانه}
%\ورودی گراف $G=(V, E)$
%\خروجی یک پوشش رأسی از $G$
%
%\دستور قرار بده $C = \emptyset$  % \توضیحات{مقداردهی اولیه}
%\تاوقتی{$E$ تهی نیست}
%%\اگر{$|E| > 0$}
%%	\دستور{یک کاری انجام بده}
%%\پایان‌اگر
%\دستور یال دل‌‌خواه $uv \in E$ را انتخاب کن
%\دستور رأس‌های $u$ و $v$ را به $C$ اضافه کن
%\دستور تمام یال‌های واقع بر $u$ یا $v$ را از $E$ حذف کن
%\پایان‌تاوقتی
%\دستور $C$ را برگردان
%\پایان{الگوریتم}
%
%
%\زیرقسمت{محیط‌های ویژه}
%
%برای درج مثال‌ها، قضیه‌ها، لم‌ها و نتیجه‌ها به ترتیب از محیط‌های
%«مثال»، «قضیه»، «لم» و «نتیجه» استفاده کنید.
%برای درج اثبات قضیه‌ها و لم‌ها  از محیط «اثبات» استفاده کنید.
%
%تعریف‌های داخل متن را با استفاده از دستور «مهم» به صورت \مهم{تیره‌} نشان دهید.
%تعریف‌های پایه‌ای‌تر را درون محیط «تعریف» قرار دهید.
%
%\شروع{تعریف}[اصل لانه‌کبوتری]
%اگر $n+1$ کبوتر یا بیش‌تر درون  $n$ لانه قرار گیرند، آن‌گاه لانه‌ای 
%وجود دارد که شامل حداقل دو کبوتر است.
%\پایان{تعریف}
%
%
%
%
%\قسمت{برخی نکات نگارشی}
%
%این فصل حاوی برخی نکات ابتدایی ولی بسیار مهم در نگارش متون فارسی است. 
%نکات گردآوری‌شده در این فصل به‌ هیچ‌ وجه کامل نیست، 
%ولی دربردارنده‌ی حداقل مواردی است که رعایت آن‌ها در نگارش پایان‌نامه ضروری به نظر می‌رسد.
%
%\زیرقسمت{فاصله‌گذاری}
%
%\شروع{شمارش}
%
%\فقره 
%علائم سجاوندی مانند نقطه، ویرگول، دونقطه، نقطه‌ویرگول، علامت سؤال و علامت تعجب % (. ، : ؛ ؟ !) 
%بدون فاصله از کلمه‌ی پیشین خود نوشته می‌شوند، ولی بعد از آن‌ها باید یک فاصله‌ قرار گیرد. مانند: من، تو، او.
%\فقره 
%علامت‌های پرانتز، آکولاد، کروشه، نقل قول و نظایر آن‌ها بدون فاصله با عبارات داخل خود نوشته می‌شوند، ولی با عبارات اطراف خود یک فاصله دارند. مانند: (این عبارت) یا \{آن عبارت\}.
%\فقره 
%دو کلمه‌ی متوالی در یک جمله همواره با یک فاصله از هم جدا می‌شوند، ولی اجزای یک کلمه‌ی مرکب باید با نیم‌فاصله\زیرنویس{«نیم‌فاصله» فاصله‌‌ای مجازی است که در عین جدا کردن اجزای یک کلمه‌ی مرکب از یک‌دیگر، آن‌ها را نزدیک به هم نگه می‌دارد. معمولاً برای تولید این نوع فاصله در صفحه‌کلید‌های استاندارد از ترکیب Shift+Space استفاده می‌شود.}‌‌
% از هم جدا شوند. مانند: کتاب درس، محبت‌آمیز، دوبخشی.
% \فقره 
% اجزای فعل‌های مرکب با فاصله از یک‌دیگر نوشته می‌شوند، مانند: تحریر کردن، به سر آمدن.
%\پایان{شمارش}
%
%
%\زیرقسمت{ حروف}
%
%\شروع{شمارش}
%
%\فقره 
%در متون فارسی به جای حروف «ك» و «ي» عربی باید از حروف «ک» و «ی» فارسی استفاده شود. همچنین به جای اعداد عربی مانند ٥ و ٦ باید از اعداد فارسی مانند ۵ و ۶ استفاده نمود. 
%برای این کار، توصیه می‌شود صفحه‌کلید‌ فارسی استاندارد\زیرنویس{\href{http://persian-computing.ir/download/Iranian_Standard_Persian_Keyboard_(ISIRI_9147)_(Version_2.0).zip}{صفحه‌کلید فارسی استاندارد برای ویندوز}، تهیه‌شده توسط بهنام اسفهبد} را بر روی سیستم خود نصب کنید.
%\فقره 
%عبارات نقل‌قول‌شده یا مؤکد باید درون علامت نقل قولِ «» قرار گیرند، نه ''``. مانند: «کشور ایران».
%\فقره 
%کسره‌ی اضافه‌ی بعد از «ه» غیرملفوظ به صورت «ه‌ی» یا «هٔ» نوشته می‌شود. مانند: خانه‌ی علی، دنباله‌ی فیبوناچی.
%
%        تبصره‌: اگر «ه» ملفوظ باشد، نیاز به «‌ی» ندارد. مانند: فرمانده دلیر، پادشه خوبان. 
%
%\فقره 
%پایه‌های همزه در کلمات، همیشه «ئـ» است، مانند: مسئله و مسئول، مگر در مواردی که همزه ساکن است که در این ‌صورت باید متناسب با اعراب حرف پیش از خود نوشته شود. مانند: رأس، مؤمن. 
%
%\پایان{شمارش}
%
%
%\زیرقسمت{جدانویسی}
%
%\شروع{شمارش}
%
%
%\فقره 
%علامت استمرار، «می»، توسط نیم‌فاصله از جزء‌ بعدی فعل جدا می‌شود. مانند: می‌رود، می‌توانیم.
%\فقره 
%شناسه‌های «ام»، «ای»، «ایم»، «اید» و «اند» توسط نیم‌فاصله، و شناسه‌ی «است» توسط فاصله از کلمه‌ی پیش از خود جدا می‌شوند. مانند: گفته‌ام، گفته‌ای، گفته است.
%\فقره 
%علامت جمع «ها» توسط نیم‌فاصله از کلمه‌ی پیش از خود جدا می‌شود. مانند: این‌ها، کتاب‌ها.
%\فقره 
%«به» همیشه جدا از کلمه‌ی بعد از خود نوشته می‌شود، مانند: به‌ نام و به آن‌ها، مگر در مواردی که «بـ» صفت یا فعل ساخته است. مانند: بسزا، ببینم.
%\فقره 
%«به» همواره با فاصله از کلمه‌ی بعد از خود نوشته می‌شود، مگر در مواردی که «به» جزئی از یک اسم یا صفت مرکب است. مانند: تناظر یک‌به‌یک، سفر به تاریخ. 
%%\پایان{شمارش}
%%
%%
%%\زیرقسمت{جدانویسی مرجح}
%%
%%\شروع{شمارش}
%
%%\فقره 
%%اجزای اسم‌ها، صفت‌ها، و قیدهای مرکب توسط نیم‌فاصله از یک‌دیگر جدا می‌شوند. مانند: دانش‌جو، کتاب‌خانه، گفت‌وگو، آن‌گاه، دل‌پذیر.
%%
%%        تبصره: اجزای منتهی به «هاء ملفوظ» را می‌توان از این قانون مستثنی کرد. مانند: راهنما، رهبر. 
%
%\فقره 
%علامت صفت برتری، «تر»، و علامت صفت برترین، «ترین»، توسط نیم‌فاصله از کلمه‌ی پیش از خود جدا می‌شوند. 
%مانند: سنگین‌تر، مهم‌ترین.
%
%        تبصره‌: کلمات «بهتر» و «بهترین» را می‌توان از این قاعده مستثنی نمود. 
%
%\فقره 
%پیشوندها و پسوندهای جامد، چسبیده به کلمه‌ی پیش یا پس از خود نوشته می‌شوند. مانند: همسر، دانشکده، دانشگاه.
%
%        تبصره‌: در مواردی که خواندن کلمه دچار اشکال می‌شود، می‌توان پسوند یا پیشوند را جدا کرد. مانند: هم‌میهن، هم‌ارزی. 
%
%\فقره 
%ضمیرهای متصل چسبیده به کلمه‌ی پیش‌ از خود نوشته می‌شوند. مانند: کتابم، نامت، کلامشان. 
%
%\پایان{شمارش}
%
