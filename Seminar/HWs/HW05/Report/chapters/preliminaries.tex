
\فصل{مفاهیم اولیه}\label{basic}


\قسمت{عمر باقی‌مانده مفید}
مانند انسان‌ها، همه دستگاه‌ها و قطعات نیز عمری دارند و برای پایش سلامت دستگاه نیاز است که بتوانیم از عمر باقی‌مانده قطعه مطلع باشیم.

عمر یک قطعه را می‌توان به‌وسیله یک تابع خطی که آن را تابع \lr{RUL} می‌نامیم «شکل \رجوع{شکل:تابع تقریب زننده عمر باقی مانده یک دستگاه}» تقریب بزنیم.



\شروع{شکل}[ht]
\centerimg{img1.pdf}{10cm}
\شرح{تابع تقریب زننده عمر باقی مانده یک دستگاه}
\برچسب{شکل:تابع تقریب زننده عمر باقی مانده یک دستگاه}
\پایان{شکل}


محور عمودی در شکل «\رجوع{شکل:تابع تقریب زننده عمر باقی مانده یک دستگاه}» نشان‌دهنده میزان سلامت دستگاه و محور افقی نشان‌دهنده زمان است که معمولاً برحسب دقیقه بیان می‌شود.

همه دستگاه‌ها زمانی که در آستانه بروز خطا و خرابی قرار می‌گیرند رفتار غیرعادی از خودشان نشان می‌دهند. از جمله این رفتارها می‌توان به نوسانات غیرطبیعی، افزایش دمای دستگاه، افزایش سروصدا\پانویس{Noise} در دستگاه اشاره نمود.


برای مثال در \مرجع{wei2024conditional} سیگنال ارتعاشات یک بلبرینگ به‌عنوان یکی از اصلی‌ترین قطعات صنعتی از ابتدای شروع به کار تا زمان بروز اولین تخریب\پانویس{First Prediction Time} و تخریب کامل\پانویس{End of Life} در مدت‌زمان ۶۹ روز جمع‌آوری شده است. «شکل \رجوع{شکل:سیگنال ارتعاش بلبرینگ از ابتدای فعالیت تا تخریب کامل}»




\شروع{شکل}[ht]
\centerimg{img2.png}{11cm}
\شرح{سیگنال ارتعاش بلبرینگ از ابتدای فعالیت تا تخریب کامل \مرجع{wei2024conditional}}
\برچسب{شکل:سیگنال ارتعاش بلبرینگ از ابتدای فعالیت تا تخریب کامل}
\پایان{شکل}





\زیرقسمت{اولین زمان خرابی}
اولین زمانی را که دستگاه دچار نوسانات شدید می‌شود را به‌عنوان اولین زمان شروع فرایند تخریب در نظر می‌گیریم و آن را «\lr{FPT}» می‌نامیم.


\زیرقسمت{عمر پایانی دستگاه}
با افزایش دامنه نوسانات ثبت شده از دستگاه، تخریب دستگاه بیشتر شده و دستگاه گرم‌تر می‌شود. از این فرایند به‌عنوان یک بازخورد\پانویس{Feedback} مثبت یاد می‌شود که افزایش گرما، نوسانات را بیشتر کرده و نوسانات بیشتر نیز گرمای دستگاه را افزایش می‌دهد. با تشدید هرچه بیشتر نوسانات، دستگاه به پایان عمر خود نزدیک‌تر شده و درنهایت از کار می‌افتد. زمان از کار فتادن نهایی دستگاه را به‌عنوان زمان پایان زندگی «\lr{EOF}» تعریف می‌کنیم.


و درنهایت سیگنال \lr{RUL} به‌صورت تفاضل این دوزمان تعریف می‌شود:

\شروع{equation}
T_{RUL} = T_{EOF} - T_{FPT}
\پایان{equation}

این سیگنال از جنس زمان است و مقدار \lr{RUL} در این باز زمانی از رابطه زیر پیروی می‌کند:

\شروع{equation}
RUL(t) = -t
\پایان{equation}


برای مثال سیگنال \lr{RUL} برای شکل «\رجوع{شکل:سیگنال ارتعاش بلبرینگ از ابتدای فعالیت تا تخریب کامل}» به صورت زیر محاسبه می‌شود:

\شروع{شکل}[ht]
\centerimg{img3.png}{12cm}
\شرح{سیگنال \lr{RUL} شکل \رجوع{شکل:سیگنال ارتعاش بلبرینگ از ابتدای فعالیت تا تخریب کامل}، \مرجع{wei2024conditional}}
\برچسب{شکل:تابع تقریب زننده عمر باقی مانده یک دستگاه}
\پایان{شکل}






\قسمت{داده‌ها}
برای پیش‌بینی عمر باقی‌مانده مفید، چندین مجموعه‌داده\پانویس{Dataset} وجود دارد که در ادامه آنها را معرفی و بررسی می‌کنیم.

\زیرقسمت{مجموعه داده \lr{XJTU-SY}}
این مجموعه‌داده\زیرنویس{می‌توانید این مجموعه‌داده را از اینجا دریافت کنید: \href{https://biaowang.tech/xjtu-sy-bearing-datasets/}{\textcolor{magenta}{\texttt{biaowang.tech/xjtu-sy-bearing-datasets/}}}} شامل داده‌های ثبت شده از ۱۵ بلبرینگ است که با انجام آزمایش‌های تخریب سریع، دچار تخریب شده‌اند. \مرجع{wang2018hybrid} 



این مجموعه‌داده توسط سیستمی که در شکل «\رجوع{شکل:بستر تهیه مجموعه‌داده XJTU-SY}» نشان‌داده‌شده است، متشکل از یک موتور القایی جریان متناوب\پانویس{Alternating Current}، یک کنترل‌کننده سرعت موتور، یک محور\پانویس{Shaft} پشتیبان، دو بلبرینگ پشتیبان (بلبرینگ‌های سنگین) و یک سیستم بارگذاری هیدرولیک تشکیل شده است.

این بستر آزمون برای انجام آزمایش‌های تخریب تسریع‌شده بلبرینگ‌، تحت شرایط مختلف عملیاتی (نیروی شعاعی و سرعت چرخشی مختلف) طراحی شده است. نیروی شعاعی توسط سیستم بارگذاری هیدرولیک تولید شده و به محفظه بلبرینگ‌های آزمایش شده اعمال می‌شود و سرعت چرخش نیز توسط کنترلر سرعت موتور القایی \lr{AC} تنظیم و نگه داشته می‌شود.








\شروع{شکل}[ht]
\centerimg{img4.png}{11cm}
\شرح{بستر تهیه مجموعه‌داده \lr{XJTU-SY}، \مرجع{wang2018hybrid}}
\برچسب{شکل:بستر تهیه مجموعه‌داده XJTU-SY}
\پایان{شکل}




بلبرینگ‌های مورداستفاده در این آزمایش از نوع \lr{LDK UER204} هستند که پارامترهای دقیق آنها در جدول «\رجوع{جدول:پارامتر‌های بلبرینگ‌های آزمایش شده}» آورده شده است.


\شروع{لوح}[ht]
\تنظیم‌ازوسط
\شرح{پارامتر‌های بلبرینگ‌های آزمایش شده}

\شروع{جدول}{c c c c}
\خط‌پر 
\خط‌پر
\سیاه پارامتر & \سیاه مقدار & \سیاه پارامتر & \سیاه مقدار \\ 
\خط‌پر \خط‌پر 
قطر مسیر بیرونی & \lr{mm} ۸۰٫۳۹ & قطر مسیر داخلی &  \lr{mm} ۳۰٫۲۹ \\ 
قطر متوسط بلبرینگ &  \lr{mm} ۵۵٫۳۴ & قطر توپ &  \lr{mm} ۹۲٫۷ \\ 
تعداد توپ‌ها & ۸ & زاویه تماس &  \textdegree ۰ \\ 
بار استاتیک & \lr{kN} ۶۵٫۶ & بار دینامیک & \lr{kN} ۸۲٫۱۲ \\ 
\خط‌پر
\خط‌پر
\پایان{جدول}

\برچسب{جدول:پارامتر‌های بلبرینگ‌های آزمایش شده}
\پایان{لوح}




این آزمایش، تحت ۳ شرط عملیاتی مختلف انجام شده است و هر ۵ بلبرینگ موجود در این آزمایش تحت این سه شرط عملیاتی قرار گرفته‌اند. این شرایط عملیاتی شامل موارد زیر هستند:

\شروع{فقرات}
\فقره ۲۱۰۰ دور در دقیقه\پانویس{RPM} (۳۵ هرتز) و بار دینامیکی ۱۲ کیلونیوتن
\فقره ۲۲۵۰ دور در دقیقه (۵٫۳۷ هرتز) و بار دینامیکی ۱۱ کیلونیوتن
\فقره ۲۴۰۰ دور در دقیقه (۴۰ هرتز) و بار دینامیکی ۱۰ کیلونیوتن

\پایان{فقرات}


برای جمع‌آوری سیگنال‌های ارتعاشی بلبرینگ‌های آزمایش شده، همان‌طور که در شکل «\رجوع{شکل:بستر تهیه مجموعه‌داده XJTU-SY}» نشان‌داده‌شده است، دو شتاب‌سنج از نوع 
۳۳\lr{C}۳۵۲\lr{PCB}
 در زاویه ۹۰ درجه بر روی محفظه بلبرینگ‌های آزمایش شده قرار داده شده است، یعنی یکی بر روی محور افقی و دیگری بر روی محور عمودی نصب شده است.


همچنین فرکانس نمونه‌برداری بر روی ۶٫۲۵ کیلوهرتز تنظیم شده است. همانطور که در شکل «\رجوع{شکل:تنظیمات نمونه‌برداری برای سیگنال‌های ارتعاشی}» نشان داده شده است، در مجموع ۳۲۷۶۸ نقطه داده (به مدت ۲۸٫۱ ثانیه) برای هر نمونه‌برداری ثبت می‌شوند و دوره نمونه‌برداری برابر با 1 دقیقه است.


\شروع{شکل}[ht]
\centerimg{img5.pdf}{12cm}
\شرح{تنظیمات نمونه‌برداری برای سیگنال‌های ارتعاشی}
\برچسب{شکل:تنظیمات نمونه‌برداری برای سیگنال‌های ارتعاشی}
\پایان{شکل}




برای هر نمونه‌برداری، داده‌های به‌دست‌آمده در یک فایل \texttt{csv} ذخیره شده است که در آن ستون اول، سیگنال‌های ارتعاشی افقی و ستون دوم سیگنال‌های ارتعاشی عمودی را شامل می‌شود. جدول «\رجوع{اطلاعات مجموعه‌داده XJTU-SY}» اطلاعات دقیق هر بلبرینگ آزمایش شده، شامل تعداد فایل‌های \texttt{csv}، عمر بلبرینگ و عنصر خرابی را فهرست می‌کند.



\begin{table}[h!]
	\centering
	\caption{اطلاعات مجموعه‌داده \lr{XJTU-SY}}
	\label{اطلاعات مجموعه‌داده XJTU-SY}
	\resizebox{\textwidth}{!}{
		\begin{tabular}{llccc}
			\toprule
			\toprule
			\textbf{شرایط عملکرد} & \textbf{مجموعه‌داده‌های بلبرینگ} & \textbf{تعداد فایل‌ها} & \textbf{طول عمر بلبرینگ} & \textbf{عامل خطا} \\
			\midrule
			\toprule
			\multirow{5}{*}{\shortstack[l]{شرایط ۱ \\ (۳۵ هرتز، ۱۲ کیلونیوتن)}} & بلبرینگ ۱\_۱ & ۱۲۳ & ۲ ساعت و ۳ دقیقه & بیرونی \\
			& بلبرینگ ۱\_۲ & ۱۶۱ & ۲ ساعت و ۴۱ دقیقه & بیرونی \\
			& بلبرینگ ۱\_۳ & ۱۵۸ & ۲ ساعت و ۳۸ دقیقه & بیرونی \\
			& بلبرینگ ۱\_۴ & ۱۲۲ & ۲ ساعت و ۲ دقیقه & قفسه \\
			& بلبرینگ ۱\_۵ & ۵۲ & ۵۲ دقیقه & داخلی و بیرونی \\
			\midrule
			\multirow{5}{*}{\shortstack[l]{شرایط ۲ \\ (۵٫۳۷ هرتز، ۱۱ کیلونیوتن)}} & بلبرینگ ۲\_۱ & ۴۹۱ & ۸ ساعت و ۱۱ دقیقه & داخلی \\
			& بلبرینگ ۲\_۲ & ۱۶۱ & ۲ ساعت و ۴۱ دقیقه & بیرونی \\
			& بلبرینگ ۲\_۳ & ۵۳۳ & ۸ ساعت و ۵۳ دقیقه & قفسه \\
			& بلبرینگ ۲\_۴ & ۴۲ & ۴۲ دقیقه & بیرونی \\
			& بلبرینگ ۲\_۵ & ۳۳۹ & ۵ ساعت و ۳۹ دقیقه & بیرونی \\
			\midrule
			\multirow{5}{*}{\shortstack[l]{شرایط ۳ \\ (۴۰ هرتز، ۱۰ کیلونیوتن)}} & بلبرینگ ۳\_۱ & ۲۵۳۸ & ۴۲ ساعت و ۱۸ دقیقه & بیرونی \\
			& بلبرینگ ۳\_۲ & ۲۴۹۶ & ۴۱ ساعت و ۳۶ دقیقه & داخلی، ساچمه، قفسه و بیرونی \\
			& بلبرینگ ۳\_۳ & ۳۷۱ & ۶ ساعت و ۱۱ دقیقه & داخلی \\
			& بلبرینگ ۳\_۴ & ۱۵۱۵ & ۲۵ ساعت و ۱۵ دقیقه & داخلی \\
			& بلبرینگ ۳\_۵ & ۱۱۴ & ۱ ساعت و ۵۴ دقیقه & بیرونی \\
			\bottomrule
			\bottomrule
		\end{tabular}
	}
\end{table}



نقطه \lr{EOF} در این مجموعه‌داده $10 \times A_H$ در نظر گرفته شده است که $A_H$ بیشینه دامنه سیگنال ارتعاش عمودی و افقی در حالت کاری طبیعی است.


در شکل‌های «\رجوع{شکل:سیگنال‌های ارتعاش افقی و عمودی و RUL بلبرینگ ۱ـ۲}» و «\رجوع{شکل:سیگنال‌های ارتعاش افقی و عمودی و RUL بلبرینگ ۲ـ۲}» نمونه‌ای از داده‌های این مجموعه‌داده آورده شده است.



\شروع{شکل}[ht]
\centerimg{img6_Bearing1_2.pdf}{16cm}
\شرح{سیگنال‌های ارتعاش افقی و عمودی و \lr{RUL} بلبرینگ ۱ـ۲}
\برچسب{شکل:سیگنال‌های ارتعاش افقی و عمودی و RUL بلبرینگ ۱ـ۲}
\پایان{شکل} 
 
 
\شروع{شکل}[ht] 
\centerimg{img7_Bearing2_2.pdf}{16cm} 
\شرح{سیگنال‌های ارتعاش افقی و عمودی و \lr{RUL} بلبرینگ ۲ـ۲}
\برچسب{شکل:سیگنال‌های ارتعاش افقی و عمودی و RUL بلبرینگ ۲ـ۲}
\پایان{شکل}






















\زیرقسمت{مجموعه داده \lr{PRONOSTIA}}
همانند مجموعه‌داده \lr{XJTY-SY} این مجموعه‌داده\زیرنویس{می‌توانید این مجموعه‌داده را از اینجا دریافت کنید: \\ \href{https://github.com/Lucky-Loek/ieee-phm-2012-data-challenge-dataset.git}{\textcolor{magenta}{\texttt{github.com/Lucky-Loek/ieee-phm-2012-data-challenge-dataset.git}}}} نیز، داده‌های ثبت شده از بلبرینگ توسط بستر آزمایشی نشان‌داده‌شده در شکل «\رجوع{شکل:بستر تهیه مجموعه‌داده PRONOSTIA}» است و امکان انجام آزمایش از ابتدای کار بلبرینگ تا زمان خرابی و شکست کامل را فراهم می‌آورد. \مرجع{nectoux2012pronostia}





\شروع{شکل}[ht]
\centerimg{img6.png}{11cm}
\شرح{بستر تهیه مجموعه‌داده \lr{PRONOSTIA}، \مرجع{nectoux2012pronostia}}
\برچسب{شکل:بستر تهیه مجموعه‌داده PRONOSTIA}
\پایان{شکل}



در این آزمایش، شرایط عملیاتی با اندازه‌گیری‌های لحظه‌ای ۱) نیروی شعاعی\پانویس{Radial Force} اعمال‌شده بر بلبرینگ، ۲) سرعت چرخش شفتی که بلبرینگ را جابه‌جا می‌کند و ۳) گشتاور اعمال‌شده به بلبرینگ تعیین می‌شوند. هر یک از این سه اندازه‌گیری که به‌صورت آنالوگ انجام شده است، با فرکانسی برابر با ۱۰۰ هرتز به‌دست‌آمده است.


در این مجموعه داده، از دو نوع حسگر\پانویس{Sensor} برای تشخیص و توصیف عملیات تخریب استفاده شده است.
حسگر‌های:
\شروع{فقرات}
\فقره ارتعاش
\فقره دما
\پایان{فقرات}



همانند مجموعه‌داده \lr{XJTU-SY}، حسگرهای ارتعاش در این مجموعه آزمایشی نیز شامل دو شتاب‌سنج هستند که با زاویه ۹۰ درجه نسبت به هم قرار دارند (اولی به صورتی عمودی و دومی به‌صورت افقی). این دو شتاب‌سنج به‌صورت شعاعی روی مسیر خارجی بلبرینگ قرار گرفته‌اند.


در این مجموعه آزمایشی علاوه بر حسگرهای ارتعاش، حسگرهای دما نیز وجود دارد. حسگر دما یک پروب \lr{RTD} از جنس پلاتینیم است که در داخل سوراخی نزدیک به حلقه خارجی بلبرینگ قرار می‌گیرد.


حسگرهای شتاب، نمونه‌ها را با فرکانس ۶٫۲۵ کیلوهرتز و حسگر دما با فرکانس ۱۰ هرتز ثبت می‌کنند. یعنی سیگنال‌های ارتعاشی ۲۵۶۰ نمونه ($\frac{1}{10}$ ثانیه) در هر ۱۰ ثانیه ضبط می‌شوند و سیگنال‌های دما، ۶۰۰ نمونه هر دقیقه ضبط می‌شوند.


\شروع{شکل}[ht]
\centerimg{img8.pdf}{12cm}
\شرح{تنظیمات نمونه‌برداری سیگنال‌ها در این آزمایش}
\برچسب{شکل:تنظیمات نمونه‌برداری سیگنال‌ها در این آزمایش}
\پایان{شکل}


در تهیه این مجموعه داده، برای جلوگیری از انتشار آسیب به کل بستر آزمایش، آزمایش‌ها زمانی متوقف شدند که دامنه سیگنال ارتعاش از $20g$ فراتر رفت.


شکل «\رجوع{شکل:سیگنال خام تهیه شده در این آزمایش و بلبرینگ تخریب شده}» مثالی است از آنچه می‌توان بر روی اجزای بلبرینگ قبل و بعد از یک آزمایش مشاهده کرد و همچنین یک سیگنال خام ارتعاش جمع‌آوری‌شده در طول یک آزمایش کامل را نشان می‌دهد.


\شروع{شکل}[ht]
\centerimg{img7.png}{15cm}
\شرح{سیگنال خام تهیه شده در این آزمایش و بلبرینگ تخریب شده، \مرجع{nectoux2012pronostia}}
\برچسب{شکل:سیگنال خام تهیه شده در این آزمایش و بلبرینگ تخریب شده}
\پایان{شکل}



\مهم{ذکر این نکته الزامی است که بلبرینگ‌ها رفتارهای بسیار متفاوتی را در زمان تخریب نشان می‌دهد که منجر به تفاوت در زمان آزمایش و تخریب می‌شود.}



این آزمایش نیز تحت سه شرط عملیاتی مختلف انجام شده است:
\شروع{فقرات}
\فقره ۱۸۰۰ دور در دقیقه و بار دینامیکی ۴ کیلونیوتن
\فقره ۱۶۵۰ دور در دقیقه و بار دینامیکی ۲٫۴ کیلونیوتن
\فقره ۱۵۰۰ دور در دقیقه و بار ۵ کیلونیوتن
\پایان{فقرات}


داده‌های جمع‌آوری شده از این آزمایش به‌صورت جدول «\رجوع{تقسیم بندی داده‌های جمع‌آوری شده در مجموعه داده PRONOSTIA}» تقسیم‌بندی می‌شود

\begin{table}[h]
	\caption{تقسیم بندی داده‌های جمع‌آوری شده در مجموعه داده \lr{PRONOSTIA}}
	\label{تقسیم بندی داده‌های جمع‌آوری شده در مجموعه داده PRONOSTIA}
	\centering
	\begin{tabular}{c c c c}
		\hline
		\hline
		مجموعه داده‌ها & شرایط عملیاتی ۱ & شرایط عملیاتی ۲ & شرایط عملیاتی ۳ \\
		\hline
		\hline
		مجموعه آموزشی & بلبرینگ۱\_1 & بلبرینگ۲\_1 & بلبرینگ۳\_1 \\
		& بلبرینگ۱\_2 & بلبرینگ۲\_2 & بلبرینگ۳\_2 \\
		\hline
		مجموعه آزمون & بلبرینگ۱\_3 & بلبرینگ۲\_3 & بلبرینگ۳\_3 \\
		& بلبرینگ۱\_4 & بلبرینگ۲\_4 & \\
		& بلبرینگ۱\_5 & بلبرینگ۲\_5 & \\
		& بلبرینگ۱\_6 & بلبرینگ۲\_6 & \\
		& بلبرینگ۱\_7 & بلبرینگ۲\_7 & \\
		\hline
		\hline
	\end{tabular}
\end{table}





هر دو مجموعه داده‌های آموزشی و آزمایشی حاوی فایل‌هایی \texttt{csv} ارتعاش به نام \texttt{acc\_xxxxx.csv} و فایل‌های دما به نام \texttt{temp\_xxxxx.csv} هستند.

همه فایل‌های \texttt{csv} ضبط شده از قالب زیر تبعیت می‌کنند:
\شروع{لوح}[ht]
\تنظیم‌ازوسط
\شرح{مشخصات داده‌های ضبط شده در فایل‌های \texttt{csv}}

\شروع{جدول}{ c c c c c c c}
\خط‌پر 
\خط‌پر
\سیاه ستون & \سیاه ۱ & \سیاه ۲ & \سیاه ۳ & \سیاه ۴ & \سیاه ۵ & \سیاه ۶ \\ 
\خط‌پر \خط‌پر 
سیگنال ارتعاش & ساعت & دقیقه &  ثانیه & میکروثانیه & حسگر افقی & حسگر عمودی \\ 
سیگنال دما & ساعت & دقیقه &  ثانیه & $x.0$ ثانیه & حسگر \lr{RTD} & \\ 
\خط‌پر
\خط‌پر
\پایان{جدول}

\برچسب{جدول:مشخصات داده‌های ضبط شده در فایل‌های csv}
\پایان{لوح}


















\زیرقسمت{مجموعه داده \lr{C-MAPSS}}
برخلاف دو مجموعه‌داده قبلی، این مجموعه‌داده\زیرنویس{می‌توانید این مجموعه‌داده را از اینجا دریافت کنید:\\ \href{https://data.nasa.gov/Aerospace/CMAPSS-Jet-Engine-Simulated-Data/ff5v-kuh6/about_data}{\textcolor{magenta}{\texttt{data.nasa.gov/Aerospace/CMAPSS-Jet-Engine-Simulated-Data/ff5v-kuh6/about\_data}}}}، داده‌های تهیه شده از موتور هواپیمای جت از ابتدای کارکرد تا زمان تخریب است. \مرجع{saxena2008damage}

این مجموعه‌داده، شامل چندین سری زمانی چندمتغیره است. هر مجموعه‌داده به زیرمجموعه‌های آموزشی و آزمایشی تقسیم می‌شود. داده‌ها از موتورهای متفاوتی تهیه شده است. هر موتور با درجات مختلفی از سایش اولیه و تغییرات تولیدی که برای کاربر ناشناخته است، شروع به کار می‌کند. این سایش و تغییرات به‌عنوان وضعیت طبیعی در نظر گرفته می‌شوند و به‌عنوان شرایط خطا تلقی نمی‌شوند.

در این مجموعه‌داده، سه تنظیمات عملیاتی وجود دارد که تأثیر قابل‌توجهی بر عملکرد موتور دارند و جمع‌آوری داده‌ها بر اساس این تنظیمات انجام شده است. بر خلاف دو مجموعه‌داده‌ قبلی، داده‌های این مجموعه به نویز حسگرها آلوده شده است.


\شروع{شکل}[ht]
\centerimg{img9.png}{16cm}
\شرح{چند سیگنال تصادفی خام جمع‌آوری شده توسط حسگرهای در موتور از مجموعه داده آموزشی \lr{FD002}، \مرجع{wang2019remaining}}
\برچسب{شکل:چند سیگنال تصادفی خام جمع‌آوری شده توسط حسگرهای در موتور از مجموعه داده آموزش}
\پایان{شکل}



موتور در ابتدای هر سری زمانی به طور عادی عمل می‌کند و در یک نقطه از سری دچار خطا می‌شود. در مجموعه آموزشی، خطا به‌شدت افزایش می‌یابد تا زمانی که سیستم به طور کامل تخریب شود. اما در مجموعه آزمایشی، سری زمانی کمی قبل از خرابی سیستم خاتمه می‌یابد.

داده‌های جمع‌آوری شده توسط ۲۶ حسگر، به‌صورت یک فایل متنی فشرده‌شده با فرمت \texttt{txt} ارائه شده است؛ بنابراین هر فایل دارای ۲۶ ستون است که هر کدام از این ستون‌ها به‌صورت زیر معرفی می‌شود:

\شروع{شمارش} 

\فقره شماره واحد
\فقره زمان (برحسب تعداد چرخه\پانویس{Cycle})
\فقره تنظیمات عملیاتی ۱
\فقره تنظیمات عملیاتی ۲
\فقره تنظیمات عملیاتی ۳
\فقره مقدار اندازه‌گیری شده توسط حسگر شماره ۱
\فقره مقدار اندازه‌گیری شده توسط حسگر شماره ۲
\فقره مقدار اندازه‌گیری شده توسط حسگر شماره ۳
\فقره...
\فقره مقدار اندازه‌گیری شده توسط حسگر شماره ۲۶

\پایان{شمارش}

در جدول «\رجوع{جدول:اطلاعات زیرمجموعه‌ها در مجموعه‌داده C-MAPSS}» اطلاعات دقیقی از این مجموعه‌داده آورده شده است.




\شروع{لوح}[ht]
\تنظیم‌ازوسط
\شرح{اطلاعات زیرمجموعه‌ها در مجموعه‌داده \lr{C-MAPSS}}

\شروع{جدول}{c c c c c}
\خط‌پر
\خط‌پر
& \سیاه \lr{FD001} & \سیاه \lr{FD002} & \سیاه \lr{FD003} & \سیاه \lr{FD004} \\ 
\خط‌پر \خط‌پر 
تعداد موتورها در مجموعه آموزش & ۱۰۰ & ۲۶۰ &  ۱۰۰ & ۲۴۹ \\ 
تعداد موتورها در مجموعه آزمون & ۱۰۰ & ۲۵۹ &  ۱۰۰ & ۲۴۸ \\ 
تعداد شرایط عملیاتی & ۱ & ۶ &  ۱ & ۶ \\ 
تعداد حالت‌های خطا & ۱ & ۱ &  ۲ & ۲ \\ 
\خط‌پر
\خط‌پر
\پایان{جدول}

\برچسب{جدول:اطلاعات زیرمجموعه‌ها در مجموعه‌داده C-MAPSS}
\پایان{لوح}













\قسمت{شبکه عصبی ترنسفرمر}
برای اولین‌بار در سال ۲۰۱۷ مفهومی به نام مکانیزم توجه\پانویس{Attention} در مقاله‌ای تحت عنوان « \lr{\textit{``Attention is All You Need''}} » توسط شرکت گوگل با کاربرد در زمینه پردازش زبان طبیعی\پانویس{Natural Language Processin} معرفی شد \مرجع{vaswani2017attention}.

در این مقاله، ساختار جدیدی از یک شبکه عصبی نیز معرفی شد که بر پایه مکانیزم توجه کار می‌کند و توانست خیلی زود جایگزین مناسبی برای مدل‌های سنتی قدیمی و ساختارهای \lr{RNN} شود. در ادامه به بررسی جزئی معماری این ساختار می‌پردازیم.




\زیرقسمت{ساختار کلی شبکه}
شبکه ترنسفرمر از یک ساختار رمزگذار\پانویس{Encoder} - رمزگشا\پانویس{Decoder} پیروی می‌کند. شکل «\رجوع{شکل:ساختار کلی شبکه ترنسفرمر}»



\شروع{شکل}[ht]
\centerimg{img10-transformer.pdf}{10cm}
\شرح{ساختار کلی شبکه ترنسفرمر}
\برچسب{شکل:ساختار کلی شبکه ترنسفرمر}
\پایان{شکل}







\زیرقسمت{ورودی تعبیه شده}
برای توضیح این قسمت از شبکه، فرض می‌شود که می‌خواهیم یک جمله ناقص را به‌عنوان ورودی به شبکه بدهیم و شبکه باید آن را تکمیل کند. برای مثال فرض شود که جمله ورودی ما به‌صورت زیر است:

\begin{center}
`` زمانی که شما درس می‌خوانید... ''
\end{center}




اولین مرحله در تمامی الگوریتم‌های یادگیری ماشین که ورودی آن‌ها کلمه است، تبدیل کلمات به بردارهای عددی است. وظیفه این مرحله با بلوک ورودی تعبیه شده است. «شکل \رجوع{شکل:تبدیل کلمات به بردار}»


\شروع{شکل}[ht]
\centerimg{img11-word2vec.pdf}{10cm}
\شرح{تبدیل کلمات به بردار}
\برچسب{شکل:تبدیل کلمات به بردار}
\پایان{شکل}





\زیرقسمت{تعبیه موقعیتی}
برخلاف سایر شبکه‌ها مانند \lr{RNN} و \lr{LSTM} که ورودی‌ها به صورتی ترتیبی و پشت‌سرهم وارد می‌شوند، در شبکه ترنسفرمر، ورودی‌ها به‌صورت موازی باهم و هم‌زمان وارد لایه تعبیه شده می‌شوند؛ بنابراین نیاز است که ترتیب و جایگاه ورودی‌ها نسبت به یکدیگر مشخص باشد. ترتیب ورودی‌ها برحسب دو تابع سینوسی و کسینوسی برحسب رابطه‌های زیر تعیین می‌شود.



\شروع{equation}\برچسب{فرمول:ایندکس زوج}
PE_{(pos, 2i)}=sin(\frac{pos}{1000^{\frac{2i}{d}}})
\پایان{equation}



\شروع{equation}\برچسب{فرمول:ایندکس فرد}
PE_{(pos, 2i+1)}=cos(\frac{pos}{1000^{\frac{2i}{d}}})
\پایان{equation}


ورودی‌هایی با نمایه\پانویس{Index} زوج، از رابطه «\رجوع{فرمول:ایندکس زوج}» و نمایه فرد از رابطه «\رجوع{فرمول:ایندکس فرد}» استفاده می‌کنند.

در روابط تعبیه موقعیتی، مقدار $pos$ برابر است با جایگاه کلمه، مقدار $d$ برابر است با طول بردار تعبیه شده (در مثال ما $d=5$) و مقدار $i$ برابر است با جایگاه‌های هر بردار.


با رسم روابط تعبیه موقعیتی، خروجی ای مانند شکل «\رجوع{شکل:خروجی گرافیکی بلوک تعبیه موقعیتی به ازای d=48 و i=96}» حاصل می‌شود.

\شروع{شکل}[ht]
\centerimg{img12.pdf}{16cm}
\شرح{خروجی گرافیکی بلوک تعبیه موقعیتی به‌ازای $d=48$ و $i=96$}
\برچسب{شکل:خروجی گرافیکی بلوک تعبیه موقعیتی به ازای d=48 و i=96}
\پایان{شکل}



سپس مقادیر ساخته شده توسط روابط لایه تعبیه موقعیتی، با بردار‌های ورودی جمع می‌شوند. «شکل \رجوع{شکل:جمع بردار‌های ورودی با بردار‌های موقعیتی تعبیه شده}»

\شروع{شکل}[ht]
\centerimg{img13-input_positional.pdf}{16cm}
\شرح{جمع بردار‌های ورودی با بردار‌های موقعیتی تعبیه شده}
\برچسب{شکل:جمع بردار‌های ورودی با بردار‌های موقعیتی تعبیه شده}
\پایان{شکل}


و درنهایت می‌توان عملکرد دو بلوک ۱) ورودی تعبیه شده و ۲) تعبیه موقعیتی را به‌صورت شکل «\رجوع{شکل:عملیات ورودی تعبیه شده و تعبیه موقعیتی}» نمایش داد.

\شروع{شکل}[ht]
\centerimg{img14-final_positional_embedding.pdf}{10cm}
\شرح{عملیات ورودی تعبیه شده و تعبیه موقعیتی}
\برچسب{شکل:عملیات ورودی تعبیه شده و تعبیه موقعیتی}
\پایان{شکل}





\زیرقسمت{توجه چند سر}
مکانیزم توجه، به مدل کمک می‌کند تا به ورودی‌های مهم توجه بیشتری شود. بلوک توجه چند سر\پانویس{Multi-Head Attention} متشکل است از چندین لایه توجه به خود\پانویس{Self-Attention}. وظیفه این لایه به‌دست‌آوردن و درک وابستگی‌های مهم در ورودی است. برای مثال فرض شود ورودی شبکه جمله زیر است:


\begin{center}
	`` شب‌هنگام که \textcolor{red}{ماه} را در آسمان دیدم، به یاد \textcolor{red}{ماه} رویت افتادم. ''
\end{center}

در این جمله دو بار کلمه ``ماه'' به‌کاررفته است که اولین آن به معنی قمر آسمانی و دومین آن به معنی زیبایی است.

مدل چگونه باید تشخصی دهد که کدام ``ماه'' به معنی قمر و کدام به معنی زیبایی است؟ 

لایه توجه به خود در اینجا به کمک می‌آید و وابستگی کلمه ``ماه'' را با سایر کلمات جمله بررسی می‌کند و بر اساس نقشه توجهی که به دست می‌آورد می‌تواند تشخصی دهد که هرکدام از ``ماه'' ها به چه معناست.



عملیاتی که برای به‌دست‌آوردن نقشه‌های توجه انجام می‌شود، در شکل «\رجوع{شکل:لایه توجه به خود}» خلاصه می‌شود.



\شروع{شکل}[ht]
\centerimg{img15-attention.pdf}{13cm}
\شرح{لایه توجه به خود}
\برچسب{شکل:لایه توجه به خود}
\پایان{شکل}


در ابتدا از بردار‌های به‌دست‌آمده از لایه‌های تعبیه شده، ۳ نمونه می‌سازیم. بردار‌های به‌دست‌آمده برای هر کلمه، به یک لایه شبکه عصبی خطی با تعداد نورون‌های دلخواه وارد می‌شود. «شکل \رجوع{شکل:لایه خطی داخل}»




\شروع{شکل}[ht]
\centerimg{img16-linear.pdf}{8cm}
\شرح{لایه خطی}
\برچسب{شکل:لایه خطی داخل}
\پایان{شکل}


پس از آموزش شبکه خطی و به دست آمدت بهترین وزن‌ها برای شبکه، خروجی سه شبکه خطی را به ترتیب ماتریس‌های پرس‌وجو\پانویس{Query} ،(Q) کلید\پانویس{Key} (K) و مقدار\پانویس{Value} (V) می‌نامیم. معیار توقف آموزش شبکه خطی، میزان شباهت بین ماتریس‌ها است که از رابطه «\رجوع{رابطه:شباهت}» می‌توان آن را به دست آورد:


\شروع{equation}\برچسب{رابطه:شباهت}
Attention(Q, K, V)=softmax(\frac{Q \cdot K^T}{\sqrt{d_K}}) \cdot V
\پایان{equation}



و در نهایت فیلتر توجه به‌دست‌آمده را در ماتریس \lr{V} ضرب می‌کنیم و مقدار فیلترها به دست می‌آید.

شکل «\رجوع{شکل:فرایند اعمال فیلتر توجه به تصویر}» مثالی از یک فیلتر توجه و درنهایت ضرب فیلتر در ورودی و ایجاد تصویر نهایی را نمایش می‌دهد.




\شروع{شکل}[ht]
\centerimg{img17.png}{15cm}
\شرح{فرایند اعمال فیلتر توجه به تصویر}
\برچسب{شکل:فرایند اعمال فیلتر توجه به تصویر}
\پایان{شکل}



مراحل توضیح داده شده برای به‌دست‌آوردن صرفاً یک فیلتر توجه است. برای به‌دست‌آوردن بهترین نتیجه در خروجی، نیاز است که فیلرهای توجه مختلفی داشته باشیم. یا به عبارتی دیگر توجه‌های چند سر داشته باشیم و توجه‌ها را در همه جای جمله (یا تصویر) به دست آوریم. با قراردادن چندلایه از لایه توجه به خود در کنار هم «شکل \رجوع{شکل:لایه توجه چند سر}» می‌توان لایه توجه چند سر را ایجاد نمود و فیلتر‌های توجه مختلفی را تولید کرد. «شکل \رجوع{شکل:فیلتر‌های توجه مختلف در تصویر}»




\شروع{شکل}[ht]
\centerimg{img18.png}{15cm}
\شرح{فیلتر‌های توجه مختلف در تصویر}
\برچسب{شکل:فیلتر‌های توجه مختلف در تصویر}
\پایان{شکل}



\شروع{شکل}[ht]
\centerimg{img19-Multihead_attention.pdf}{16cm}
\شرح{لایه توجه چند سر}
\برچسب{شکل:لایه توجه چند سر}
\پایان{شکل}








\زیرقسمت{لایه روبه‌جلو}
لایه رو‌یه‌جلو متشکل است از دو لایه خطی که در قسمت‌های قبل توضیح داده شد و یک لایه تابع فعالساز\پانویس{Activation Function} رلو\پانویس{ReLU} «شکل \رجوع{شکل:لایه روبه‌جلو}».


\شروع{شکل}[ht]
\centerimg{img20-feedforward.pdf}{4cm}
\شرح{لایه روبه‌جلو}
\برچسب{شکل:لایه روبه‌جلو}
\پایان{شکل}




\زیرقسمت{لایه خطی رمزگشا}
پس از تشکیل فیلترهای توجه در بلوک رمزگذار، ۲ نمونه مشابه از آن ساخته می‌شود. این دو نمونه ماتریس‌های \lr{Q} و \lr{K} برای بلوک رمزگشا خواهند بود و ماتریس سوم که ماتریس \lr{V} است، همانند توضیحات قسمت‌های قبل به‌صورت مستقیم وارد بلوک رمزگذار شده و درنهایت این سه ماتریس به‌صورت هم‌زمان وارد به لایه توجه چند سر موجود در بلوک رمزگذار می‌شوند.


\زیرقسمت{لایه خطی رمزگذار}
درنهایت در بلوک رمزگذار یک لایه خطی وجود دارد که وظیفه طبقه‌بندی کلمات ورودی را دارد. تعداد نورون‌های این لایه برابر است با تعداد کلمات موجود در کلمه. وظیفه این لایه مشخص‌کردن بهترین خروجی و خروجی برنده که می‌تواند به‌عنوان کلمه بعدی برای تکمیل جمله پیشنهاد داده شود است.


برای مثال اگر جمله‌ای که از ابتدا به‌عنوان‌مثال بیان کردیم را به شبکه بدهیم می‌توانیم انتظار داشته باشیم که یکی از خروجی‌های شبکه به‌صورت زیر تولید شود:


\شروع{center}
`` زمانی که شما درس می‌خوانید، \textcolor{red}{بیشترین تمرکز را دارید.}''
\پایان{center}