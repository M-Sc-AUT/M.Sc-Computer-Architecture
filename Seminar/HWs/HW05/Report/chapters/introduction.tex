
\فصل{مقدمه}

پیش‌بینی عمر مفید باقی‌مانده\پانویس{\lr{Remaining Useful Life}} (یا به اختصار \lr{RUL}) یکی از مباحث کلیدی در حوزه مدیریت سلامت و پیش‌بینی\پانویس{\lr{Prognostics and Health Management}} (\lr{PHM}) ابزارها و تجهیزات صنعتی است. \lr{RUL} به مدت زمانی اشاره دارد که یک دستگاه یا ابزار قبل از رسیدن به نقطه خرابی و از کار افتادن نهایی، می‌تواند به طور مؤثر کار کند. این حوزه پژوهشی با استفاده از تکنیک‌های مختلف و پیشرفته در تلاش است تا به صنایع کمک کند تا بهره‌وری و کارایی خود را افزایش دهند و هزینه‌های ناشی از تعمیر و نگهداری غیرضروری را کاهش دهند.







\قسمت{تعریف مسئله}

در صنایع مختلف، از جمله خودروسازی، هوافضا، نفت و گاز و تولیدات صنعتی، تجهیزات و ماشین‌آلات به طور مداوم تحت شرایط کاری سخت و پیچیده قرار دارند. هرگونه خرابی ناگهانی این تجهیزات می‌تواند به وقفه‌های غیرمنتظره در تولید منجر شود که علاوه بر خسارات مالی، ممکن است اثرات زیان‌باری بر کیفیت محصول نهایی و رضایت مشتریان داشته باشد. بنابراین، نیاز است که وضعیت فعلی سلامت دستگاه‌ها به طور مستمر پایش شود و زمان خرابی دستگاه و عمر مفید باقی‌مانده آن با دقت بالایی پیش‌بینی شود. این پیش‌بینی نیازمند استفاده از تحلیل داده‌های حسگرها، مدل‌سازی ریاضی و الگوریتم‌های یادگیری ماشین\پانویس{\lr{Machine Learning}} و عمیق\پانویس{\lr{Deep Learning}} است.






\قسمت{اهمیت موضوع}

اهمیت پیش‌بینی دقیق \lr{RUL} در صنعت به دلیل تاثیر مستقیم آن بر بهره‌وری، کارایی و کاهش هزینه‌های تعمیر و نگهداری غیرضروری، به‌خوبی شناخته شده است. تکنیک‌های پیشرفته در حوزه هوش مصنوعی\پانویس{\lr{Artificial Intelligence}} و تحلیل داده‌های بزرگ\پانویس{\lr{Big Data Analytics}} این امکان را فراهم کرده‌اند که داده‌های جمع‌آوری شده از تجهیزات به صورت بلادرنگ\پانویس{Real Time} تحلیل شوند و مدل‌های پیش‌بینی \lr{RUL} با دقت بالاتری ارائه شوند. این پیشرفت‌ها، به شرکت‌ها این امکان را می‌دهد که فرآیندهای نگهداری و تعمیرات خود را بهینه‌سازی کنند و تصمیم‌گیری‌های بهتری در زمینه مدیریت دارایی‌های خود انجام دهند.




\قسمت{ادبیات موضوع}

اکثر دانشگاه‌های معتبر قالب استانداردی برای تهیه‌ی پایان‌نامه در اختیار دانشجویان خود قرار می‌دهند.
این قالب‌ها عموما مبتنی بر نرم‌افزارهای متداول حروف‌چینی نظیر لاتک و مایکروسافت ورد هستند.

 لاتک\پانویس{\LaTeX} یک نرم‌افزار متن‌باز قوی برای حروف‌چینی متون علمی است.\مرجع
 {knuth1984texbook, lamport1985LaTeX} 
در این نوشتار از نرم‌افزار حروف‌چینی زی‌تک\پانویس{\XeTeX} 
 و افزونه‌ی زی‌پرشین\پانویس{\XePersian}
 استفاده شده است.


\قسمت{اهداف پژوهش}

این پژوهش بر توسعه یک راه‌حل شتابدهی سخت‌افزاری بر بستر \lr{FPGA} پیش‌بینی  \lr{RUL} با استفاده از شبکه عصبی مصنوعی که وظیفه آن آموزش و یادگیری توالی و درنهایت پیش‌بینی آن است تمرکز دارد. با استفاده از قابلیت پردازش موازی \lr{FPGA} و همچنین توان مصرفی بسیار پایین آن، هدف ما افزایش کارایی و مقیاس‌پذیری سیستم‌های پیش‌بینی \lr{RUL} به‌ویژه برای دستگاه‌های دوار است.






\قسمت{ساختار پایان‌نامه}

این پایان‌نامه در پنج فصل به شرح زیر ارائه می‌شود.
%فصل دوم به بیان مفاهیم اولیه  می‌پردازد.
نکات اولیه‌ی نگارشی و نحوه‌ی نگارش پایان‌نامه در محیط لاتک در  فصل دوم به اختصار اشاره شده است. 
فصل سوم به مطالعه و بررسی کارهای پیشین مرتبط با موضوع این پایان‌نامه می‌پردازد.
در فصل چهارم، نتایج جدیدی که در این پایان‌نامه به‌دست آمده است، ارائه می‌شود.
فصل پنجم به جمع‌بندی کارهای انجام شده در این پژوهش و ارائه‌ی پیشنهادهایی برای انجام کارهای آتی خواهد پرداخت.