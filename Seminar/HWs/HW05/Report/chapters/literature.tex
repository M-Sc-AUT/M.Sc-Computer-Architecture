
\فصل{کارهای پیشین}\label{litreture}



ماشین‌های دوار به طیف وسیعی از ماشین‌ها اطلاق می‌شود که حول یک محور می‌چرخند، مانند توربین‌ها، ژنراتورها، پمپ‌ها، کمپرسورها و موتورها \مرجع{yu2023rolling}. این ماشین‌ها در بسیاری از زمینه‌ها مانند تولید برق، نفت و گاز، حمل‌ونقل کاربردهای گسترده‌ای دارند \مرجع{lei2013review}. بلبرینگ\پانویس{Bearing} به‌عنوان یکی از مهم‌ترین ابزارهای صنعتی، با تسهیل چرخش شفت از طریق یک رابط کم اصطکاک بین شفت و محفظه آن و کاهش سایش و پارگی اجزای ماشین، نقش مهمی در تضمین عملکرد روان و کارآمد ماشین‌های دوار دارد \مرجع{hoang2019survey}. بااین‌حال، بلبرینگ‌ها در طول زمان به دلیل بار زیاد، روانکاری ناکافی و آلودگی، تخریب می‌شوند \مرجع{liu2020review}. تخریب بلبرینگ‌ها می‌تواند منجر به لرزش و نویز بیش از حد، بازده کم و مصرف انرژی بالا شود و اگر کنترل نشود، می‌تواند منجر به خرابی‌های فاجعه‌بار شود که باعث تعمیرات پرهزینه و ازکارافتادن ماشین‌آلات شود \مرجع{neupane2020bearing} بنابراین؛ نظارت و پیش‌بینی وضعیت سلامت و عمر مفید باقی‌مانده بلبرینگ‌ها برای کاهش زمان ازکارافتادن دستگاه بسیار مهم است.


 به‌طورکلی روش‌هایی پیش‌بینی \lr{RUL} را می‌توان به ۳ دسته تقسیم کرد:
 \شروع{فقرات}
 
 \فقره روش‌های مبتنی بر مدل
 \فقره روش‌های مبتنی بر داده
 \فقره روش‌های ترکیبی
 
 \پایان{فقرات}


\قسمت{روش‌های مبتنی بر مدل}
روش‌های مبتنی بر مدل، نیازمند مدل‌سازی دقیق دینامیک مسئله و اطلاع از جزئیات دقیق ابزار صنعتی مورد بررسی است. ساختار تجهیزات صنعتی در مقیاس‌های بزرگ با روابط غیرخطی بین سیستم‌ها و قطعات مختلف پیچیده می‌شود که اغلب ارائه یک مدل دقیق از آن کار بسیار دشواری است. روش‌های مبتنی بر مدل‌سازی فیزیکی شامل مدل‌های ارائه شده در \مرجع{li2000stochastic}، \مرجع{oppenheimer2002physically} و \مرجع{choi2007spall} است. در هر سه مقاله مدل‌سازی سیستم به‌صورت فیزیکی ارائه شده است؛ اما با نتوانسته‌اند پیش‌بینی قابل‌قبولی از \lr{RUL} ارائه دهند.


در \مرجع{tobon2011hidden}، لی و همکاران، با استفاده از ارائه یک مدل احتمالاتی مبتنی بر روابط مارکوف\پانویس{Markov} که بخشی از آنها را در ادامه آورده‌ایم:


\شروع{equation}
RUL_{\text{حد بالا}} = \sum_{i=\text{وضعیت فعلی}}^{N} \left[ \mu \left( D(s_i) \right) + cf \cdot \sigma \left( D(s_i) \right) \right]
\پایان{equation}





\شروع{equation}
RUL_{\text{میانگین}} = \sum_{i=\text{وضعیت فعلی}}^{N} \mu \left( D(s_i) \right)
\پایان{equation}






\شروع{equation}
RUL_{\text{حد پایین}} = \sum_{i=\text{وضعیت فعلی}}^{N} \left[ \mu \left( D(s_i) \right) - cf \cdot \sigma \left( D(s_i) \right) \right]
\پایان{equation}


سعی در مدل‌سازی دقیق یک سیستم مکانیکی، برای پیش‌بینی عمر باقی‌مانده مفید داشتند. نتایج ارائه شده در این مقاله «شکل \رجوع{شکل:نتایج ارائه شده در tobon2011hidden}» نشان می‌دهد که نویسندگان نتوانسته‌اند مقدار خطای \lr{RMS} را تا حد خوبی کاهش دهند.



\شروع{شکل}[ht]
\centerimg{img39.png}{12cm}
\شرح{نتایج ارائه شده \مرجع{tobon2011hidden}}
\برچسب{شکل:نتایج ارائه شده در tobon2011hidden}
\پایان{شکل}

دلیل این موضوع را می‌توان، پیچیدگی روابط برای مدل‌سازی و عدم امکان مدل‌سازی دقیق مسئله دانست که همان‌طور نیز که قبلاً بیان شد در همه مقالاتی که دسته روش‌های مبتنی بر مدل قرار می‌گیرند این ضعف دیده می‌شود.

در این مقاله، برای مدل‌سازی روابط حاکم بر مسئله، از ماشین حالت\پانویس{State Machine} استفاده شده است. نمودارهای ارائه شده در این مقاله ضعف را نشان می‌دهند که ماشین حالت انتخاب مناسبی برای این مسئله نمی‌باشد. چرا که سیستم ارائه شده نمی‌تواند با حداکثر سرعت خودکار کند و به‌صورت برخط پیش‌بینی را انجام دهد. این ضعف خودش را در دستگاه‌های بلادرنگ\پانویس{Real-Time} نشان می‌دهد.



\شروع{شکل}[ht]
\centerimg{img40.png}{15cm}
\شرح{نمودار پیش‌بینی شده \مرجع{tobon2011hidden}}
\برچسب{شکل:نمودار پیش‌بینی شده tobon2011hidden}
\پایان{شکل}


\قسمت{روش‌های مبتنی بر داده}
به دلیل پیچیدگی روابط و شرایط موجود در مسئله و دشواری تحلیل کامل سیستم در روش‌های مبتنی بر مدل، روش‌های مبتنی بر تحلیل داده‌ها مورد اقبال قرار گرفته‌اند. هدف در روش‌های مبتنی بر داده، ایجاد یک رابطه نگاشت بین ورودی و خروجی، بدون داشتن دانش قبلی نسبت به سیستم است. این روش‌ها به طور گسترده‌ای برای پیش‌بینی \lr{RUL} بلبرینگ‌ها مورداستفاده قرار گرفته‌اند، زیرا می‌توانند پیش‌بینی را با استفاده از داده‌های نظارت بر وضعیت، بدون دانش و شناخت قبلی نسبت به سیستم انجام دهند. روش‌های مبتنی بر داده را می‌توان به دودسته گروه‌بندی کرد:

\شروع{فقرات}

\فقره روش‌های مبتنی بر یادگیری ماشین سنتی
\فقره روش‌های مبتنی بر یادگیری عمیق

\پایان{فقرات}
	



\زیرقسمت{روش‌های مبتنی بر یادگیری ماشین سنتی}
روش‌های یادگیری ماشین سنتی شامل رگرسیون بردار پشتیبان\پانویس{Spport Vector Regression} (\lr{SVR}) \مرجع{islam2021data}، جنگل تصادفی\پانویس{Random Forect} (\lr{RF}) \مرجع{Zhang2016Lim}، یادگیری افراطی ماشین\پانویس{Extreme Learning Machine} (\lr{ELM}) \مرجع{Liu2018cheng}، فرایند گاوسی\پانویس{Gaussian Process} (\lr{GP}) \مرجع{zhou2022remaining}، یادگیری گروهی\پانویس{Ensemble Learning} (\lr{EL}) \مرجع{shi2021remaining}، مدل مارکوف\پانویس{Markov Model} (\lr{MM}) \مرجع{wang2020probabilistic} و موارد دیگر است.



به‌عنوان‌مثال، وانگ و همکاران \مرجع{li2022novel} یک رویکرد رگرسیون برداری چند پشتیبانی را برای به‌دست‌آوردن پارامترهای مدل فرعی بهینه برای پیش‌بینی \lr{RUL} بلبرینگ‌ها معرفی کرد. آن‌ها از مدلی مانند شکل «\رجوع{شکل:مدل ارائه شده مبتنی بر بردار‌های پشتیبان}» استفاده کردند.

\شروع{شکل}[ht]
\centerimg{img21.png}{16cm}
\شرح{مدل ارائه شده مبتنی بر بردار‌های پشتیبان در \مرجع{li2022novel}}
\برچسب{شکل:مدل ارائه شده مبتنی بر بردار‌های پشتیبان}
\پایان{شکل}


پس از به‌دست‌آوردن پارامترهای مدل بهینه، یک سازوکار به‌روزرسانی وزن خودکار برای ارزیابی مناسب‌بودن هر مدل فرعی برای عملکرد پیش‌بینی قوی‌تر پیشنهاد شد. در این کار از مجموعه‌داده \lr{CMAPASS} برای ارزیابی عملکرد پیش‌بینی روش ارائه‌شده استفاده شد و نتایج ارائه شده «شکل \رجوع{شکل:نتایج ارائه شده در li2022novel}» نشان می‌دهد که داد که \lr{RMSE} روش ارائه شده، ۱۴٫۹۸ به‌دست‌آمده است. نتایج این مقاله نشان می‌دهد که در کاهش خطای پیش‌بینی نسبت به روش‌های مبتنی بر مدل بهبود بسیاری داشته‌اند. در این مقاله ویژگی‌ها توسط انسان به‌واسطه رابطه «\رجوع{معادله:معادله ویژگی}» انتخاب شده است. همین امر می‌تواند به هنگام تغییر مجموعه‌داده‌ها و یا استفاده از داده‌های پرت\پانویس{Outlier} می‌تواند باعث شود که فضای ویژگی‌های ورودی دستخوش تغییراتی بشود. این چالش به‌عنوان یکی از اصلی‌ترین مشکلات این مقاله معرفی می‌شود.




\شروع{equation}\برچسب{معادله:معادله ویژگی}
x_i^{\text{میانگین}} =
\begin{cases} 
	\frac{1}{2i - 1} (x_1 + \cdots + x_{2i-1}), & 1 \leq i \leq \frac{(s - 1)}{2} \\[10pt]
	\frac{1}{s} (x_i - \frac{(s-1)}{2} + \cdots + x_i + \frac{(s-1)}{2}), & \frac{(s-1)}{2} < i < L - \frac{(s-1)}{2} \\[10pt]
	\frac{1}{2(L - i) + 1} (x_{2i - L} + \cdots + x_L), & L - \frac{(s-1)}{2} \leq i \leq L
\end{cases}
\پایان{equation}

 
 
\شروع{شکل}[ht]
\centerimg{img22.png}{12cm}
\شرح{نتایج ارائه شده \مرجع{li2022novel}}
\برچسب{شکل:نتایج ارائه شده در li2022novel}
\پایان{شکل}





منگ و همکاران \مرجع{meng2020remaining} مدل مارکوف خاکستری را با نظریه طیف فراکتال\پانویس{Fractal Spectrum} ادغام کرد تا مسیر تخریب بلبرینگ‌ها را پیش‌بینی کند. برای انجام پیش‌بینی در این مقاله از مجموعه‌داده \lr{PRONOSTIA} استفاده شده است. روش ارائه شده با ویژگی ذرات مورفولوژی تعمیم‌یافته مقایسه شده است که نشان داده روش ارائه شده می‌تواند \lr{RMSE} را تا ۴ درصد کاهش دهد. نتایج این روش در شکل «\رجوع{شکل:نتایج ارائه شده در meng2020remaining}» آورده شده است.



\شروع{شکل}[ht]
\centerimg{img23.png}{10cm}
\شرح{نتایج ارائه شده در \مرجع{meng2020remaining}}
\برچسب{شکل:نتایج ارائه شده در meng2020remaining}
\پایان{شکل}

در این مقاله نیز همانند مقاله قبل، استخراج ویژگی توسط پژوهشگران پژوهش انجام شده است. در این کار از فرمول «\رجوع{فرمول:ویژگی meng2020remaining}» به‌عنوان ویژگی ورودی مسئله در نظر گرفته شده است.


\شروع{equation}\برچسب{فرمول:ویژگی meng2020remaining}
r(X, Y)=\frac{Cov(X, Y)}{\sqrt{Var[X]Var[Y]}}
\پایان{equation}


همانند مقاله قبل، نقدی که به این مقاله وارد است در راستای انتخاب ویژگی و قابلیت تعمیم شبکه به انتخاب بهترین ويژگی برای دستیابی به بهترین پیش‌بینی است.



وانگ و همکاران در \مرجع{wang2022feature} روشی مبتنی بر ترکیب ویژگی‌های استخراجی در حوزه زمان و فرکانس را بر روی شبکه \lr{LSTM} ارائه کردند. ویژگی‌های زمانی و فرکانسی مورداستفاده در این مقاله در شکل «\رجوع{شکل:ویژگی‌های استفاده شده در wang2022feature}» آورده شده است.


\شروع{شکل}[ht]
\centerimg{img24.png}{16cm}
\شرح{ویژگی‌های استفاده شده در \مرجع{wang2022feature}}
\برچسب{شکل:ویژگی‌های استفاده شده در wang2022feature}
\پایان{شکل}

برخلاف دو مقاله قبلی، ایده ترکیب‌کردن ویژگی‌های حوزه زمانی و فرکانسی، مورداستفاده در این مقاله تا حد بسیار خوبی میزان دقت پیش‌بینی را افزایش داده است. در این مقاله از مجموعه‌داده \rl{PRONOSTIA} برای آموزش و ارزیابی استفاده شده است و نتایج ارائه شده نشان از کاهش مقدار \lr{RMSE} دارد. «شکل \رجوع{شکل:ویژگی‌های استفاده شده در wang2022feature}» اما همچنان مشکل اصلی یعنی انتخاب ویژگی‌ها توسط ناظر انسانی پا برجاست. برای مثال اگر سیگنال مورداستفاده در حوزه فرکانس طیف وسیع و گسترده‌ای داشته باشد، ممکن است نتوانیم بهترین ویژگی را برای آن انتخاب نماییم. ضمن آن که استفاده از ویژگی‌های حوزه فرکانسی بار محاسباتی زیادی را بر سیستم متحمل می‌شود (به‌خصوص در سیستم های بلادرنگ) و شخصاً معتقد هستم که اگر می‌خواهیم از مدل‌های یادگیری ماشین سنتی برای این مسئله استفاده کنیم، تا جایی که می‌شود بهتر است با ترکیب ویژگی‌های زمانی سعی در افزایش دقت پیش‌بینی داشت.




\شروع{شکل}[ht]
\centerimg{img25.png}{14cm}
\شرح{نتایج ارائه شده در \مرجع{wang2022feature}}
\برچسب{شکل:ویژگی‌های استفاده شده در wang2022feature}
\پایان{شکل}







\زیرقسمت{روش‌های مبتنی بر یادگیری عمیق}
همان‌طور که در نقد مقالات حوزه یادگیری ماشین سنتی بارها بیان شد، عدم توانایی این الگوریتم‌ها پر انتخاب هوشمندانه بهترین ویژگی‌ها برای مسئله است همچنین روش‌های یادگیری ماشین سنتی در یادگیری روابط پیچیده و غیرخطی مؤثر نیستند. برای رفع این موضوع، در سال‌های اخیر، روش‌های یادگیری عمیق به دلیل توانایی آنها در آزاد گذاشتن شبکه در انتخاب بهترین ویژگی‌ها از ورودی خام\پانویس{Raw}، یادگیری الگوهای پیچیده و مدیریت حجم زیادی از داده‌ها، به طور فزاینده‌ای برای پیش‌بینی \lr{RUL} محبوب شده‌اند. این روش‌های یادگیری عمیق را می‌توان به‌طورکلی به دو زیر گروه تقسیم کرد:

\شروع{فقرات}

\فقره شبکه‌های باز رخدادی\پانویس{Recurrent}
\فقره شبکه‌های غیر باز رخدادی
\پایان{فقرات}


تکنیک‌های یادگیری عمیق غیر باز رخدادی شامل شبکه‌های عصبی مصنوعی\پانویس{Artificial Neural Network} \مرجع{wang2021remain}
شبکه عصبی کانولوشنی\پانویس{Convolutional Neural Network} \مرجع{wang2021rul}، رمزگذار های خودکار\پانویس{Autoencoders} \مرجع{xu2022rul}، شبکه عصبی گرافی\پانویس{Graph Neural Network} \مرجع{wei2023model,wei2024state} و مدل‌های تولیدی مانند شبکه مولد متخاصم\پانویس{Generative Adversarial Network} \مرجع{lu2022joint} هستند.


به‌عنوان‌مثال، زو و همکاران \مرجع{zhu2018estimation} یک \lr{CNN} چند مقیاسی «شکل \رجوع{شکل:ساختار ارائه شده zhu2018estimation}» را برای استخراج ویژگی‌های عمیق از داده‌های پایش وضعیت پیشنهاد کرد، و ویژگی‌های استخراج‌شده با ویژگی‌های زمان - فرکانس برای پیش‌بینی \lr{RUL} بلبرینگ‌ها ترکیب شدند. برای استخراج ویژگی های حوزه زمانی از یک شبکه عصبی \lr{CNN} استفاده شده است. همچنین پس از بردن سیگنال ارتعاش زمانی به حوزه فرکانس مشاهده می‌شود که طیف غالب سیگنال‌های فرکانسی در بازه [$f_0$، ۰] است. «شکل \رجوع{شکل:طیف فرکانسی سیگنال های ارتعاشی. zhu2018estimation}»





\شروع{شکل}[ht]
\centerimg{img41.png}{10cm}
\شرح{طیف فرکانسی سیگنال های ارتعاشی. \مرجع{zhu2018estimation}}
\برچسب{شکل:طیف فرکانسی سیگنال های ارتعاشی. zhu2018estimation}
\پایان{شکل}






در این مقاله، از مجموعه‌داده \lr{PRONOSTIA} برای ارزیابی مدل استفاده شده است. نتایج نشان داد که \lr{CNN} چند مقیاسی ارائه شده عملکرد پیش‌بینی را افزایش می‌دهد. نتایج این کار در شکل «\رجوع{شکل:نتایج ارائه شده zhu2018estimation}» آورده شده است.


\شروع{شکل}[ht]
\centerimg{img26.png}{15cm}
\شرح{ساختار ارائه شده در \مرجع{zhu2018estimation}}
\برچسب{شکل:ساختار ارائه شده zhu2018estimation}
\پایان{شکل}




\شروع{شکل}[ht]
\centerimg{img27.png}{15cm}
\شرح{نتایج ارائه شده در \مرجع{zhu2018estimation}}
\برچسب{شکل:نتایج ارائه شده zhu2018estimation}
\پایان{شکل}

نقد‌هایی که بر این مقاله وارد است به شرح زیر ارائه می‌شود:
\شروع{شمارش}
\فقره در این مقاله از کانولوشن دو - بعدی برای استخراج ویژگی‌ها استفاده شده است. درصورتی‌که سیگنال ورودی مال سیگنالی تک‌بعدی است. بهتر بود برای استخراج ویژگی از شبکه کانولوشنی تک‌بعدی استفاده شود. در صورت استفاده از شبکه کانولوشنی تک‌بعدی، به‌مراتب حجم و بار محاسباتی کاهش پیدا می‌کرد و کیفیت ویژگی‌های استخراج شده نیز افزایش پیدا می‌کرد.

\فقره در این مقاله طیف فرکانسی گسترده‌ای سیگنال در استخراج ویژگی‌های حوزه فرکانس دور ریخته می‌شود. این کار توسط نویسندگان این‌طور توجیه می‌شود که قسمت‌های مهم سیگنال ورودی در بازه فرکانسی [$f_0$، ۰] است و طیف فرکانسی موجود بعدازاین فرکانس بی‌اهمیت است. در نقد این جمله می‌توان این‌طور بیان کرد که ذات سیگنال ورودی، مبتنی بر زمان است و کوچک‌ترین اتفاق درگذشته‌ی دور بلبرینگ در خرابی احتمالی آن در آینده تأثیرگذار است؛ بنابراین سیگنال‌های ضبط شده در ابتدای کار بلبرینگ که فرکانس به‌مراتب کمتری نسبت به زمان تخریب بلبرینگ دارند نیز مهم هستند و نمی‌توان برای پیش‌بینی دقیق از آن‌ها صرف‌نظر کرد.
\پایان{شمارش}




خو و همکاران در \مرجع{xu2022rul} یک رمزگذار خودکار کانولوشنی برای استخراج ویژگی‌ها از داده‌های جمع‌آوری‌شده از بلبرینگ‌های مستهلک شده ارائه کرد، «شکل \رجوع{شکل:مدل ارائه شده xu2022rul}» و یک تابع مقیاس‌بندی شاخص سلامت برای کاهش مقیاس ویژگی‌های استخراج‌شده به کار گرفته‌اند. رمزگذارها در استخراج ویژگی از داده‌های سری زمانی به‌خوبی عمل می‌کنند و ترکیب آن با شبکه \lr{CNN} افزایش کیفیت ویژگی‌های استخراج شده از داده‌های ورودی را امید می‌دهد.

در این مقاله نیز از مجموعه‌داده \lr{PRONOSTIA} برای آموزش و ارزیابی مدل استفاده شده است. نتایج آن‌ها نشان می‌دهد «شکل \رجوع{شکل:نتایج ارائه شده xu2022rul}» که روش ارائه شده در پیش‌بینی \lr{RUL} و ارزیابی مراحل تخریب بلبرینگ‌ها نسبت به مقالات بررسی شده درگذشته کارآمدتر است. 


\شروع{شکل}[ht]
\centerimg{img28.png}{15cm}
\شرح{مدل ارائه شده در \مرجع{xu2022rul}}
\برچسب{شکل:مدل ارائه شده xu2022rul}
\پایان{شکل}



\شروع{شکل}[ht]
\centerimg{img29.png}{15cm}
\شرح{نتایج ارائه شده در \مرجع{xu2022rul}}
\برچسب{شکل:نتایج ارائه شده xu2022rul}
\پایان{شکل}

مشکل درک وابستگی‌های طولانی‌مدت همچنان در این مدل نیز وجود دارد و رمزگذارهای خودکار نیز قادر به درک وابستگی‌های طولانی‌مدت نیستند بنابراین نمی‌توانند بهترین پیش‌بینی را انجام دهند.



یانگ و همکاران برای غلبه بر وابستگی‌های طولانی‌مدت، در \مرجع{yang2022bearing} از یک \lr{GNN} برای تخمین \lr{RUL} بلبرینگ‌ها استفاده کردند. در این مدل، گره\پانویس{Node} های مختلف باهم در ارتباط هستند و می‌توانند وابستگی‌ها را بیشتر از مدل‌های پیشین درک کنند. اما مشکلی که در این مدل‌ها وجود دارد، روابط پیچیده نسبت به سایر مدل‌هاست. مدل‌های گرافی ذاتاً پیچیدگی بیشتری نسبت به مدل‌های معمول مانند \lr{CNN} ها و سایر مدل‌های بررسی شده پیش‌ازاین دارند و همین موضوع می‌تواند برای کاربرد ما که در نهایت نیاز به پیاده‌سازی بر روی \lr{FPGA} داریم چالش‌ساز باشد.


درحالی‌که بسیاری از مطالعات انجام شده، استفاده از شبکه‌های عصبی غیرتکراری را برای پیش‌بینی \lr{RUL} مورد بررسی قرار داده‌اند، اما به دلایل بیان شده در قسمت نقد هر مقاله، مانند ضعف مدل‌ها در درک وابستگی‌های طولانی‌مدت این روش‌ها اغلب در برخورد با داده‌های سری زمانی مؤثر نیستند. در مقابل، الگوریتم‌های یادگیری عمیق مکرر مانند شبکه عصبی باز رخدادی \مرجع{liu2021enhanced}، حافظه کوتاه‌مدت\پانویس{Long Short-Term Memory} \مرجع{zhu2023hybrid,zhang2019bearing}، شبکه بازگشتی دروازه‌ای\پانویس{Gated recurrent network} \مرجع{ni2022data} و نسخه‌های دوطرفه آنها مانند \lr{LSTM} دوطرفه \مرجع{luo2022convolutional} و \lr{GRU} دوطرفه نشان‌داده‌شده است که در پیش‌بینی \lr{RUL} بلبرینگ‌ها مؤثرتر هستند. 





به‌عنوان‌مثال، ما و همکاران در \مرجع{ma2020deep} یک \lr{LSTM} کانولوشنی عمیق «شکل \رجوع{شکل:ساختار ارائه شده ma2020deep}» برای پیش‌بینی \lr{RUL} معرفی کرد. در این مقاله برای استخراج ویژگی از شبکه \lr{CNN} و برای پیش‌بینی از سلول‌های \lr{LSTM} استفاده شده است. در این مقاله نیز از ویژگی‌های زمان - فرکانسی به‌صورت توأم استفاده شده است که مدل را قادر به درک و حفظ وابستگی‌های طولانی‌مدت در داده‌ها که منجر به پیش‌بینی‌ای بادقت بالا می‌شود، می‌سازد. نتایج عددی ارائه شده در این مقاله «شکل \رجوع{شکل:نتایج ارائه شده ma2020deep}»، نشان می‌دهد که \lr{LSTM} کانولوشنی پیشنهادی بهتر از سلول \lr{CNN} و \lr{LSTM} عمیق است.



سلول‌های \lr{LSTM} یک بازخورد\پانویس{Feedback} دارند که این مکانیزم می‌تواند مدل را قادر سازد که سری‌های زمانی را پیش‌بینی کرد؛ اما این بازخورد می‌تواند مشکل‌ساز نیز باشد. اگر کاربرد ما طوری باشد که نیاز داشته باشیم داده‌ها به‌صورت بلادرنگ پردازش شود، این بازخورد به گلوگاه\پانویس{Bottleneck} مدل تبدیل می‌شود و نمی‌توان داده‌ها را به‌صورت کامل موازی پردازش نمود. در این مقاله نیز این مشکل وجود دارد.




\شروع{شکل}[ht]
\centerimg{img29.png}{16cm}
\شرح{ساختار ارائه شده در \مرجع{ma2020deep}}
\برچسب{شکل:ساختار ارائه شده ma2020deep}
\پایان{شکل}



\شروع{شکل}[ht]
\centerimg{img31.png}{8cm}
\شرح{نتایج ارائه شده در \مرجع{ma2020deep}}
\برچسب{شکل:نتایج ارائه شده ma2020deep}
\پایان{شکل}





ژانگ و همکاران \مرجع{zhang2022parallel} یک روش یادگیری عمیق ترکیبی موازی «شکل \رجوع{شکل:ساختار ارائه شده zhang2022parallel}» را توسعه داد که یک \lr{CNN} یک‌بعدی و \lr{GRU} دوطرفه را برای پیش‌بینی‌های \lr{RUL} ترکیب می‌کند و امکان استخراج موازی ویژگی‌های مکانی و زمانی از داده‌ها را فراهم می‌کند. در این مقاله به دلیل استفاده از \lr{GRU} دوطرفه، امکان ثبت وابستگی‌های طولانی‌مدت را بهتر و بیشتر از کارهای قبل فراهم می‌آورد؛ اما همچنان مشکل سرعت و موازی‌سازی شبکه به‌صورت کامل به دلیل وجود بازخورد در این شبکه‌ها وجود دارد.

نتایج ارائه شده «شکل \رجوع{شکل:نتایج ارائه شده zhang2022parallel}» در مقاله نشان می‌دهد که این روش بهبود خوبی نسبت به سایر مقاله‌های بررسی شده در این کار داشته است.



\شروع{شکل}[ht]
\centerimg{img32.png}{14cm}
\شرح{ساختار ارائه شده در \مرجع{zhang2022parallel}}
\برچسب{شکل:ساختار ارائه شده zhang2022parallel}
\پایان{شکل}



\شروع{شکل}[ht]
\centerimg{img33.png}{14cm}
\شرح{نتایج ارائه شده در \مرجع{zhang2022parallel}}
\برچسب{شکل:نتایج ارائه شده zhang2022parallel}
\پایان{شکل}



 هان و همکاران در \مرجع{han2021remaining}، خود کدگذار پشته‌ای\پانویس{Stacked Autoencoder} و \lr{RNN} را برای پیش‌بینی \lr{RUL} ترکیب کردند. خود کدگذار برای ترکیب ویژگی‌ها در شاخص‌های سلامتی استفاده شد و سپس \lr{RNN} برای پیش‌بینی‌شده مدل پیاده‌سازی شده است. ویژگی‌ها به دلیل آنکه باهم ترکیب می‌شوند، ویژگی‌های مفیدتری را برای مدل می‌سازند؛ اما وجود شبکه \lr{RNN} در مدل برای پیش‌بینی شبکه را از نظر سرعت پیش‌بینی کند می‌کند.
 
 
 
 
 
 اگرچه شبکه‌های \lr{RNN} به طور گسترده برای پیش‌بینی‌های \lr{RUL} مورداستفاده قرار گرفته‌اند، اما به‌طورکلی در ثبت و درک وابستگی‌های بلندمدت در داده‌ها مؤثر نیستند و نمی‌توانند وابستگی‌های مهم را در داده‌های طولانی‌مدت مثل داده‌های سری زمانی ثبت کنند. همچنین این شبکه‌ها در فرایند آموزش دچار محوشدگی گرادیان\پانویس{Gradient Vanishing} و فراموشی می‌شوند که این امر در نتیجه‌هایی پیش‌بینی تأثیرگذار است.
 
 
 
برای رفع این چالش، مکانیزم‌های توجه، به‌ویژه مکانیزم توجه به خود\پانویس{Self-Attention} \مرجع{cao2021novel}، به طور گسترده استفاده شده است. مکانیزم توجه در شبکه‌های مختلفی استفاده شده است، اما در میان شبکه‌هایی که از سازوکار توجه به خود استفاده می‌کنند، ترنسفرمر یکی از قدرتمندترین و مؤثرترین شبکه‌ها است. معماری ترنسفرمر توجه را با ویژگی‌های دیگری مانند رمزگذاری موقعیتی و شبکه‌های عصبی پیش‌خور ترکیب می‌کند تا همبستگی غیرخطی در داده‌ها را آشکار کند، در نتیجه عملکرد پیش‌بینی را به طور قابل‌توجهی بهبود می‌بخشد.




برای مثال، سو و همکاران در \مرجع{su2021end} از رمزگذار ترنسفرمر برای پیش‌بینی \lr{RUL} بلبرینگ‌ها استفاده کردند. «شکل \رجوع{شکل:معماری ارائه شده su2021end}» روش پیشنهادی آنها شامل دو مرحله بود که در مرحله اول ویژگی‌های سطح پایین را با استفاده از مکانیزم استخراج ویژگی استخراج می‌کرد و مرحله دوم از رمزگذار ترنسفرمر برای تخمین \lr{RUL} استفاده می‌کرد. در این مقاله از دو مجموعه‌داده عمومی \lr{FEMTO} و \lr{XJTU-SY} برای تأیید اثربخشی روش ارائه شده استفاده شده و نتایج نشان داده است که رمزگذار ترنسفرمر منجر به افزایش عملکرد پیش‌بینی می‌شود. «شکل \رجوع{شکل:خروجی پیش‌بینی ارائه شده su2021end}»



\شروع{شکل}[ht]
\centerimg{img34.png}{14cm}
\شرح{معماری ارائه شده در \مرجع{su2021end}}
\برچسب{شکل:معماری ارائه شده su2021end}
\پایان{شکل}



\شروع{شکل}[ht]
\centerimg{img35.png}{12cm}
\شرح{خروجی پیش‌بینی ارائه شده در \مرجع{su2021end}}
\برچسب{شکل:خروجی پیش‌بینی ارائه شده su2021end}
\پایان{شکل}



دینگ و همکاران در \مرجع{ding2022convolutional} یک ترنسفرمر - کانولوشن را معرفی کرد که عملکرد کانولوشن و مکانیسم خودتوجهی را برای تخمین \lr{RUL} بلبرینگ‌ها باهم ادغام می‌کند. عملیات کانولوشن برای آشکارکردن وابستگی‌های محلی در داده‌ها استفاده شده است، و مکانیسم توجه به خود برای آشکارکردن وابستگی‌های سراسری داده‌ها به کار گرفته شده است. معماری این مدل در شکل «\رجوع{شکل:معماری ارائه شده ding2022convolutional}» آورده شده است. نتایج ارائه شده در این مقاله «\رجوع{شکل:نتیجه ارائه شده ding2022convolutional}» نیز نشان از بهبود عملکرد و کاهش خطای پیش‌بینی نسبت به سایر روش‌ها دارد.




\شروع{شکل}[ht]
\centerimg{img36.png}{8cm}
\شرح{معماری ارائه شده در \مرجع{ding2022convolutional}}
\برچسب{شکل:معماری ارائه شده ding2022convolutional}
\پایان{شکل}



\شروع{شکل}[ht]
\centerimg{img37.png}{10cm}
\شرح{نتیجه ارائه شده در \مرجع{ding2022convolutional}}
\برچسب{شکل:نتیجه ارائه شده ding2022convolutional}
\پایان{شکل}




ژانگ و همکاران در \مرجع{zhang2023integrated} یک مدل ترنسفرمر جدید «شکل \رجوع{شکل:ساختار ارائه شده zhang2023integrated}» برای پیش‌بینی \lr{RUL} پیشنهاد کرد که در آن یک مکانیزم چند سر پراکنده توجه به خود برای بهبود کارایی محاسباتی معرفی شد. نتایج تجربی ارائه شده در این مقاله، نشان داده می‌دهد که روش پیشنهادی از ترنسفرمر معمولی و سایر روش‌های مبتنی بر داده بهتر عمل می‌کند.


\شروع{شکل}[ht]
\centerimg{img38.png}{16cm}
\شرح{ساختار ارائه شده در \مرجع{zhang2023integrated}}
\برچسب{شکل:ساختار ارائه شده zhang2023integrated}
\پایان{شکل}