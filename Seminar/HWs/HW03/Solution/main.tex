\documentclass[a4paper,10pt]{article}

\usepackage{cite}
\usepackage[hidelinks]{hyperref}
\usepackage{graphicx}
\usepackage{xcolor}
\usepackage{xepersian}
\usepackage[super]{nth}

\settextfont{XB Yas}


\title{\textbf{گزارش تمرین سوم سمینار}}
\author{\href{https://github.com/rezaAdinepour}{رضا آدینه پور}\\
دانشگاه صنعتی امیرکبیر (پلی‌تکنیک تهران) \\
تهران، ایران \\
\texttt{\href{mailto:adinepour@aut.ac.ir}{adinepour@aut.ac.ir}}
}


\date{\today}



\begin{document}
	
	\begin{figure}[t!]
		\centering
		\includegraphics[width=0.35\textwidth]{Logo/aut-fa2.png}
	\end{figure}
	
	\maketitle
	
	
	
	
	
	
	\section{مقاله \textcolor{blue}{\cite{article2}}}
	در این مقاله، یک روش ارزیابی فرسایش بلبرینگ بر مبنای شبکه عصبی بازگشتی \lr{LSTM (Long Short-Term Memory)} پیشنهاد شده است که می‌توان آن را به سه گام زیر تقسیم کرد:
	
	\begin{enumerate}
		\item انتخاب شاخص فرسایش
		\item تشخیص خودکار بروز خطا در بلبرینگ
		\item دسته‌بندی مراحل فرسایش
	\end{enumerate}

در گام اول، یک شاخص بنام "آنتروپی موج"\footnote{waveform entropy} برای نمایش وضعیت فعلی بلبرینگ پیشنهاد شده است و مدل شبیه‌سازی فرسایش بر مبنای مکانیزم پاسخ ارتعاشی بلبرینگ برای انتخاب و ویژگی‌های قوی به عنوان ورودی مدل ارزیابی ساخته شده است. در گام دوم، یک روش تشخیص خودکار بروز خطا ارائه شده است تا وضعیت نرمال بلبرینگ با مدت زمان طولانی از وضعیت خرابی تشخیص داده شود. در گام سوم، فرایند فرسایش به مراحل مختلف تقسیم شده و مدل شبکه عصبی بازگشتی LSTM برای تشخیص سطوح فرسایش ایجاد شده است.

در نتیجه این تحقیق، شاخص جدیدی به نام آنتروپی موج ارائه شده که وضعیت فعلی بلبرینگ را بهبود می‌بخشد و مدل شبکه عصبی LSTM-RNN به خوبی قادر به ارزیابی وضعیت فرسایش بلبرینگ است. این مقاله نشان می‌دهد که مدل پیشنهادی به طور موثر می‌تواند وضعیت‌های فرسایش بلبرینگ را تشخیص دهد. با این حال، مقاله به برخی مشکلات همچون ناتوانی در تخمین جزئیات تغییرات تصادفی در فرآیند فرسایش و تقسیم ساده فرآیند فرسایش به مراحل زمانی اشاره کرده و اشاره دارد که این مسائل در آینده باید مورد بررسی و تحقیقات بیشتری قرار گیرند.

% --------------------------------------------------------------------------------------------------------

	\section{مقاله \textcolor{blue}{\cite{article39}}}
در این مقاله، یک روش پیش‌بینی عمر مفید با استفاده از شبکه عصبی عمیق بازگشتی \lr{(DLSTM)} براساس سیگنال‌های دنباله‌ای زمانی چند حسگر ارائه شده است. این مدل DLSTM از سیگنال‌های نظارتی چند حسگری برای پیش‌بینی دقیق عمر مفید استفاده می‌کند و توانایی کشف وابستگی‌های طولانی مخفی بین سیگنال‌های دنباله‌ای زمانی حسگرها را از طریق ساختار یادگیری عمیق دارد. با استفاده از روش جستجوی گرید، ساختار و پارامترهای DLSTM با الگوریتم تخمین لحظه تطبیقی به‌طور کارآمد تنظیم می‌شوند تا پیش‌بینی دقیق و قوی انجام شود. دو مجموعه داده مختلف مربوط به موتورهای توربوفن در این تحقیق برای اثبات عملکرد مدل DLSTM به‌کار گرفته شده است و نتایج آزمایشات نشان می‌دهد که این مدل عملکرد رقابتی‌ای نسبت به روش‌های پیشین گزارش شده در ادبیات و سایر مدل‌های شبکه عصبی دارد.

در کل، ایده‌های اصلی این مقاله به شرح زیر است: اولاً، یک مدل DLSTM جدید برای پیش‌بینی دقیق عمر مفید ساخته شده است که برخی تلاش‌ها برای بهینه‌سازی ساختار و پارامترها انجام شده است. دوماً، مدل DLSTM پیشنهادی سیگنال‌های نظارتی چند حسگری را برای بهبود عملکرد پیش‌بینی عمر مفید ترکیب می‌کند، که قادر به درک وابستگی‌های طولانی مخفی بین سیگنال‌های دنباله‌ای زمانی حسگرها از طریق ساختار یادگیری عمیق است. سوماً، این روش پیشنهادی برای سناریوهای چند حسگری مناسب است که در بخش آزمایش تأیید شده است.

% --------------------------------------------------------------------------------------------------------

	\section{مقاله \textcolor{blue}{\cite{article47}}}
در این مقاله، یک روش نوآورانه برای پیش‌بینی عمر مفید با نام \lr{Dual Aspect Self-attention based on Transformer (DAST)} ارائه شده است. این روش، یک ساختار رمزگذار-رمزگشا مبتنی بر توجه به خود\footnote{Self attention} بدون هیچ ماژول RNN/CNN ایجاد می‌کند. DAST از دو رمزگذار تشکیل شده است که به صورت موازی برای استخراج ویژگی‌های حسگرهای مختلف و گام‌های زمانی اقدام می‌کنند. این روش تنها بر اساس توجه به خود، از اثربخشی بیشتری در پردازش دنباله‌های طولانی داده‌های نگهداری مبتنی بر شرایط برخوردار است و توانایی یادگیری تطبیقی برای تمرکز بر بخش‌های مهم‌تر ورودی را دارد. نتایج آزمایشات بر دو مجموعه داده متداول موتورهای توربوفن نشان می‌دهد که این روش به‌طور قابل توجهی از روش‌های پیش‌بینی عمر مفید مطرح در حوزه پیشرفت کرده است.

ایده‌های اصلی این مقاله به شرح زیر است:
\begin{enumerate}
	\item ارائه یک روش نوآورانه پیش‌بینی عمر مفید عمیق بر مبنای معماری ترانسفورمر.
	\item بر اساس توجه به خود، این روش قادر به پرداخت خودکار به ویژگی‌های مهم بدون نیاز به دانش حوزه است و برای پردازش داده‌های طولانی مبتنی بر شرایط نگهداری، موثرتر از روش‌های مبتنی بر RNN/CNN است. 
	\item وزن‌های حسگرها و گام‌های زمانی مختلف که توسط مدل یادگیری می‌شوند، برای پرسنل نگهداری قابل تفسیر و روشن است، به‌طوری‌که آن‌ها می‌توانند استراتژی‌های نگهداری بهتری را تدوین کنند.
\end{enumerate}

% --------------------------------------------------------------------------------------------------------

	\section{مقاله \textcolor{blue}{\cite{article48}}}
در این مقاله، یک رویکرد جدید برای پیش‌بینی شکست بلبرینگ‌های المان لغزشی ارائه شده است. این پیش‌بینی عمر مفید باقی‌مانده\footnote{RUL} بر پایه یک پیش‌بینی‌گر \lr{Long Short-Term Memory (LSTM)} در چارچوب \lr{Generative Adversarial Network (GAN)} آموزش دیده می‌شود. پیش‌بینی‌گر LSTM با استفاده از مشاهدات فعلی و گذشته یک شاخص سلامتی معین به عنوان ورودی، مسیر فرسایش آینده را پیش‌بینی می‌کند و سپس عمر مفید باقی‌مانده (RUL) را استخراج می‌کند. این رویکرد دارای ویژگی‌های منحصر به فردی است که شامل تعریف آستانه شکست بلبرینگ بر اساس استاندارد سازمان بین‌المللی استاندارد‌ها (ISO)، استفاده از تکنیک افزایش داده‌ها بر مبنای GAN جهت بهبود دقت و قویت پیش‌بینی RUL در مواقعی است که مدل یادگیری عمیق به حجم محدودی از داده‌های آموزش دسترسی دارد، و یک رویکرد آموزش مشترک که تضمین می‌کند پیش‌بینی‌گر LSTM هم داده‌های اصلی و هم داده‌های مصنوعی تولید شده را یاد بگیرد تا مسیرهای فرسایش را بهتر شناسایی کند.

در مجموع، این رویکرد پیشنهادی دو مرحله مهم در فرایند آفلاین دارد:

1) آماده‌سازی داده که شامل استخراج ویژگی‌های فرسایش از داده‌های ارتعاش در دامنه سرعت آنها می‌شود 2) آموزش مدل که به ابتدا پیش‌آموزش \lr{GAN-LSTM predictor} بر اساس داده‌های آموزش اصلی انجام می‌دهد، سپس ژنراتور و تشخیص‌دهنده را با استفاده از شبکه GAN-LSTM پیش‌آموزش داده و در نهایت \lr{GAN-LSTM predictor} ژنراتور و تشخیص‌دهنده را به‌صورت مشترک آموزش می‌دهد. بر اساس نتایج یک مطالعه تصاویر متقابل پنج‌گانه، این رویکرد نشان می‌دهد که ادغام پیش‌بینی‌گر LSTM با GAN منجر به کاهش خطای میانگین پیش‌بینی RUL به میزان 29٪ نسبت به مدل ساده LSTM بدون پیاده‌سازی GAN می‌شود.

% --------------------------------------------------------------------------------------------------------
	
	\section{مقاله \textcolor{blue}{\cite{article40}}}
در این مقاله، یک رویکرد نوآورانه برای پیش‌بینی خرابی با استفاده از شبکه‌های Generative ارائه شده است که در زمینهٔ \lr{Prognostics and Health Management (PHM)} مورد استفاده قرار می‌گیرد. یکی از چالش‌های اساسی در PHM، پیش‌بینی دقیق خرابی‌های ناگهانی در تجهیزات است. در سال‌های اخیر، راه‌حل‌ها برای پیش‌بینی خرابی از مدل‌های فیزیکی پیچیده به الگوریتم‌های یادگیری ماشینی که از داده‌های تولیدشده توسط تجهیزات بهره‌مند می‌شوند، تکامل یافته‌اند. اما مشکلات پیش‌بینی خرابی چالش‌های خاصی را ایجاد می‌کنند که استفاده مستقیم از الگوریتم‌های سنتی دسته‌بندی و پیش‌بینی غیرعملی می‌سازد. این مقاله با ارائه الگوریتمی نوآورانه برای پیش‌بینی خرابی با استفاده از شبکه‌های Generative این چالش‌ها را مدیریت می‌کند. GAN-FP از دو شبکه GAN برای همزمان تولید نمونه‌های آموزشی و ساخت یک شبکه استنتاج که برای پیش‌بینی خرابی در نمونه‌های جدید قابل استفاده است، استفاده می‌کند.

در این پژوهش، یک الگوریتم نوین با نام GAN-FP ارائه شده که از سه ماژول مختلف تشکیل می‌شوند: (1) در یک ماژول، نمونه‌های واقعی از خرابی و عدم‌خرابی با استفاده از infoGAN تولید می‌شوند. (2) در ماژول دیگر، هدف اصلی از دست دادن وزن برای آموزش شبکه استنتاج با استفاده از نمونه‌های واقعی از خرابی و عدم‌خرابی است. در این طراحی، این شبکه استنتاج وزن‌های لایه‌های اولیه را با شبکه تشخیص‌دهنده infoGAN به اشتراک می‌گذارد. (3) در ماژول سوم، شبکه استنتاج با استفاده از یک GAN دوم بهبود می‌یابد که هدف آن تضمین سازگاری بین نمونه‌های تولیدشده و برچسب‌های متناظر توسط شبکه استنتاج است. این رویکرد می‌تواند برای مسائل دیگر دسته‌بندی نامتوازن نیز مورد استفاده قرار گیرد. ارزیابی تجربی بر روی چندین مجموعه داده مرجع نشان می‌دهد که GAN-FP به طریق قابل توجهی نتایج بهتری نسبت به رویکردهای موجود، از جمله زیر نمونه‌برداری SMOTE و ADASYN تلفیق وزنی دارد.

% --------------------------------------------------------------------------------------------------------
	
	\section{مقاله \textcolor{blue}{\cite{article5}}}
در این مقاله، یک روش جدید در زمینه نگهداری مبتنی بر وضعیت \lr{Condition-Based Maintenance} به‌منظور کاهش هزینه‌های تعمیر و نگهداری در صنعت ارائه شده است. با انجام نگهداری دوره‌ای مرتب، مشکل تعمیر اصلاحی می‌تواند به حداقل برسد، اما این روش به بهترین شکل ممکن عمل نمی‌کند. در این راستا، این مقاله یک روش پیشنهادی برای پردازش داده‌های جمع‌آوری‌شده از یک سیستم ارتعاشی که یک مدل موتور را شبیه‌سازی می‌کند، ارائه می‌دهد. با استفاده از این داده‌ها، یک مجموعه داده‌ی ساختارمند برای آموزش و آزمون یک شبکه عصبی مصنوعی ایجاد شده است که قادر به پیش‌بینی شرایط آینده تجهیزات و هشدار دهی در مورد زمان احتمالی خرابی می‌باشد.

افزایش دقت در پیش‌بینی زمان خرابی تجهیزات و فرآیندها در صنعت هوشمند به تصمیم‌گیری در مورد نگهداری کمک کرده و هزینه‌ها و فشار کاری را کاهش می‌دهد. روش پیشنهادی در این مقاله شامل پردازش داده‌های ارتعاشی جمع‌آوری‌شده از یک مدل دستگاه است که به شبیه‌سازی یک سیستم واقعی می‌پردازد. این مقاله به ارائه مجموعه داده برای آموزش ANN قابل پیش‌بینی خرابی پرداخته و نتایج آموزش و آزمون نشان داده‌اند که ANN مدل MLP با الگوریتم یادگیری بازگشتی بهتر از تکنیک‌های دیگر مانند درخت رگرسیون و جنگل تصادفی می‌باشد. مقاله به پیشنهادات برای تحقیقات آتی نیز اشاره کرده و امکان استفاده از سیستم‌های تشخیص خرابی و پیش‌بینی با ANN را در زمینه برنامه‌ریزی نگهداری در صنعت پیشنهاد کرده است.

% --------------------------------------------------------------------------------------------------------

	\section{مقاله \textcolor{blue}{\cite{article45}}}
در این مقاله، نویسندگان به RUL در تجهیزات صنعتی با یک رویکرد نوآورانه می‌پردازند. آن‌ها یک معماری عمیق بر مبنای رمزگذار ترانسفورمر معرفی می‌کنند که از موفقیت آن در یادگیری دنباله بهره می‌برد. مدل پیشنهادی، بر خلاف شبکه‌های عصبی کانولوشنی، با محدودیت از اندازه کرنل مواجه نمی‌شود و این امر تضمین می‌کند که یک فیلد تأثیر کامل برای تمام مراحل زمانی فراهم باشد. در مقایسه با شبکه‌های عصبی بازگشتی، مدل پیشنهادی به شکل کارآمد از محاسبات موازی بهره می‌برد و سرعت محاسباتی را افزایش می‌دهد. نتایج آزمایشات بر مجموعه داده‌های فرسایش موتور توربوفن نشان می‌دهد که عملکرد مدل پیشنهادی بهتر یا مقایسه‌پذیر با روش‌های موجود دیگر است.

ایده‌های اصلی شامل معرفی یک مدل بر مبنای رمزگذار ترانسفورمر برای تخمین RUL، بررسی واحد کانولوشنی با گیت برای بهبود ادغام محتواهای محلی و انجام با موفقیت آزمایشات بر روی مجموعه داده‌های فرسایش موتور هستند. نویسندگان برجسته‌سازی مهارت مدل در درک وابستگی‌های کوتاه و بلند مدت و تأکید بر کارآیی محاسباتی آن را مورد تاکید قرار می‌دهند. به طور کلی، معماری عمیق پیشنهادی در زمینه تخمین RUL، به عنوان یک راهکار مؤثر و نوآورانه، برجسته می‌شود.

% --------------------------------------------------------------------------------------------------------
	
	\section{مقاله \textcolor{blue}{\cite{article43}}}
	
مقاله مورد بحث توسعه یک مدل شبکه عصبی جدید به نام SA-ConvLSTM برای پیش‌بینی عمر مفید باقی‌مانده\footnote{RUL} بلبرینگ‌های چرخان است. این مدل از عملگرهای کانولوشن برای کاهش اضافی شبکه و افزایش قابلیت مدل‌سازی غیرخطی بهره برده است. نتایج آزمایشی بر روی دیتاست PRONOSTIA نشان داد که SA-ConvLSTM نسبت به روش‌های پیش‌بینی سنتی از نظر سرعت همگرایی و دقت پیش‌بینی مزایایی دارد.
	
این مطالعه بر اهمیت مدل SA-ConvLSTM در حل محدودیت‌های شبکه‌های عصبی LSTM سنتی، به ویژه در زمینه پیش‌بینی RUL بلبرینگ‌های چرخان تأکید می‌کند. با ادغام عملگرهای کانولوشن و یک ماژول خودتوجه، این مدل پیشنهادی نسبت به روش‌های سنتی سرعت همگرایی و دقت پیش‌بینی بهتری ارائه می‌دهد. تأیید آزمایشی و مطالعات موردی نشان داد که SA-ConvLSTM در پیش‌بینی RUL اثربخشی دارد و به عنوان یک رویکرد امیدبخش برای کاربردهای مهندسی عملی، مانند مدیریت سلامتی بلبرینگ‌های گیربکس توربین‌های بادی، مطرح است.

% --------------------------------------------------------------------------------------------------------
	
	\section{مقاله \textcolor{blue}{\cite{article38}}}
در این مقاله، یک روش جدید برای تشخیص خطا در دستگاه‌های مختلف به نام DDTLN (شبکه یادگیری انتقالی عمیق) ارائه شده است. برای حل مسائلی مانند هماهنگی توزیع و انتقال دانش بین دامنه هدف و دامنه منبع، بسیاری از روش‌های تطبیق دامنه ارائه شده‌اند. اما بیشتر آنها تنها به هماهنگی توزیع‌های حاشیه‌ای توجه می‌کنند و یادگیری ویژگی تمایزدهنده را در دو دامنه نادیده می‌گیرند. برای بهبود هماهنگی توزیع و تطابق توزیع‌های حاشیه‌ای و شرطی دو دامنه، یک مکانیزم بهبود یافته تطبیق توزیع مشترک (IJDA) ارائه شده است. علاوه بر این، یک مکانیزم بهبود یافته تطابق توزیع شرطی ساخته شده است. برای ارتقاء یادگیری ویژگی و یادگیری ویژگی‌های قابل تفکیک‌تر، یک تابع تلفیقی جدید به نام \lr{I-Softmax loss} ارائه شده که مانند تابع اصلی Softmax قابل بهینه‌سازی است و توانایی طبقه‌بندی قوی‌تری دارد. این روش DDTLN با استفاده از مکانیزم IJDA و \lr{I-Softmax loss} برای انجام تشخیص انتقال خطا پیاده‌سازی شده است و نتایج آزمایشی نشان می‌دهد که در مقایسه با روش‌های معمول دیگر، عملکرد بهتری در تشخیص خطا دارد.

از دیگر ویژگی‌های اصلی این مقاله می‌توان به ارائه مکانیزم CDA بهبود یافته برای هماهنگی بهتر توزیع‌های احتمال شرطی واقعی دو دامنه، معیار بهبود یافته برای اندازه‌گیری فاصله توزیع از دو نظر میانگین و کوواریانس (ترکیب \lr{MMD} و \lr{CORAL})، و تابع تلفیقی \lr{I-Softmax} با حاشیه انعطاف‌پذیر اشاره کرد. آزمایشات نشان می‌دهند که DDTLN با این ویژگی‌ها می‌تواند در وظایف تشخیص خطا در شش دستگاه مختلف با دقت میانگین بیش از 90٪ عمل کند.

% --------------------------------------------------------------------------------------------------------

	\section{مقاله \textcolor{blue}{\cite{article14}}}
در این مقاله، یک روش تشخیص خطا برای بلبرینگ با دقت تشخیص بالا و کارایی زمانی ارائه شده است. این روش بر اساس استخراج خودکار ویژگی هوشمندانه و شبکه عصبی ترنسفرمر استفاده می‌کند. در ابتدا، الگوریتم SPBO برای انتخاب انعطاف‌پذیر پارامترها، شامل تعداد نرون‌های لایه مخفی، ضریب پراکندگی استفاده می‌کند. از شبکه اتوانکودر تصحیح‌کننده نویز (DAE) برای تعیین ساختار بهینه‌ی شبکه اتوانکودر SDAE استفاده می‌شود. سپس، شبکه بهینه‌شده  \lr{SPBO-SDAE } برای استخراج ویژگی از داده اصلی با ابعاد بالا استفاده می‌شود. در نهایت، برای اعتبارسنجی عملکرد مدل پیشنهادی در شرایط پیچیده، با افزودن نویز گوسی به داده اصلی، عملکرد تشخیص خطا از طریق چهار مجموعه داده بررسی می‌شود و نتایج نشان می‌دهند که در مقایسه با روش‌های یادگیری کم‌عمق و عمیق موجود، این روش دارای مزایای قابل توجهی در عملکرد کلی، دقت تشخیص خطا و کارایی زمانی است.


نتایج آزمایش نشان می‌دهند که میزان مشارکت ویژگی‌ها می‌تواند به بیش از 85٪ برسد و با افزایش ابعاد ویژگی‌ها به 120 و 140، افزایش کمی در میزان مشارکت رخ می‌دهد و کارایی محاسباتی کاهش می‌یابد. در نهایت، مقاله به پیشنهاد استفاده از مکانیزم خودتوجه چند‌سر (multi-head) و بهینه‌سازی شبکه عصبی فیدفوروارد در مدل تبدیل‌دهنده برای بهبود عملکرد در آینده اشاره دارد.

% --------------------------------------------------------------------------------------------------------

	\section{مقاله \textcolor{blue}{\cite{article9}}}
در این مقاله، سیستم‌های نگهداری پیشگو با قابلیت کاهش قابل ملاحظه هزینه‌های نگهداری هواپیما و افزایش ایمنی با شناسایی مشکلات نگهداری قبل از اینکه جدی شوند، مورد بحرانی قرار دارند. با این وجود، توسعه چنین سیستم‌هایی به دلیل کمبود داده‌های عموماً برچسب‌گذاری شده‌ی حسگرهای دنباله‌های زمانی چندمتغیره (MTS) محدود شده است. مجموعه داده استفاده شده در این مقاله شامل بیش از 7500 پرواز با بیش از 11500 ساعت اطلاعات ثبت‌شده در هر ثانیه از 23 پارامتر حسگر است. با استفاده از این مجموعه داده، نشان داده شده است که روش‌های شبکه عصبی بازگشتی (RNN) برای گرفتن روابط زمانی دور مناسب نیستند و یک معماری جدید به نام \lr{Convolutional Multiheaded Self Attention (Conv-MHSA)} ارائه می‌دهیم که در عملکرد طبقه‌بندی از بار محاسباتی بیشتری برخوردار است. همچنین نشان می‌دهیم که تقویت‌های الهام‌گرفته از تصویر مانند cutout، mixup و cutmix می‌توانند برای کاهش اورفیتینگ و بهبود عمومیت در طبقه‌بندی MTS مفید باشند.

نشان داده می‌شود که مجموعه داده \lr{NGAFID-MC} به عنوان یک بنچمارک چالشی ارزش ارزیابی روش‌های مختلف MTS را دارد. نویسندگان اظهار می‌کنند که این مجموعه داده نقاط داده یا طول دنباله بیشتری نسبت به دیگر مجموعه‌ها دارد و هیچ مجموعه داده دیگری با همزمان بیشترین نقاط داده و طول دنباله را ندارند. همچنین، هیچ مجموعه داده MTS دیگری وجود ندارد که یک سیستم پویا را که به طور عمده در یک محیط غیرکنترل شده و ناپایدار تغییر ردیابی کند. این مقاله به علاوه نشان می‌دهد که این مجموعه داده حاوی روابط زمانی دور است که روش‌های قبلی طبقه‌بندی MTS با آن مشکل دارند. نویسندگان امیدوارند که چالش‌آور بودن این مجموعه داده منجر به ایجاد و بهبود روش‌های بهتر برای طبقه‌بندی MTS شود.

% --------------------------------------------------------------------------------------------------------

	\section{مقاله \textcolor{blue}{\cite{article10}}}
در این مقاله، مشکلات پیچیدگی‌های ذاتی در خطوط تولید دارویی در تسهیلات صنعتی مدرن بررسی شده و نیاز به تشخیص دقیق و به موقع حوادث ناشناخته در یک خط تولید به ویژه احتمال وقوع مشکلات زنجیره تولید را ضروری می‌سازد. در این مقاله، یک مدل مبتنی بر یادگیری عمیق به نام ManuTrans ارائه شده است که برای نظارت بر داده‌های حسگری به صورت Real-time و استخراج وضعیت یک خط تولید دارو و پیش‌بینی زمان بعدی وقوع عیب طراحی شده است. این رویکرد از قابلیت مدل‌های ترانسفورمر عمیق برای استخراج هم‌بستگی‌های بلند و کوتاه مدت و الگوها در داده‌های توالی بهره می‌برد و با لایه خروجی خطی، همزمان عملیات طبقه‌بندی و رگرسیون را هم انجام می‌دهد.

مدل ارائه شده در اینجا برای مدیریت تعمیر و نگهداری پیشگیرانه در خطوط تولید دارو طراحی شده است. این مدل قادر به ارزیابی وضعیت سلامت خط تولید، پیش‌بینی زمان وقوع عیب بعدی در چرخه‌های تولید و همچنین پیش‌بینی شدت چنین عیبی می‌باشد. مدل بر اساس ترانسفورمرها و به ویژه بخش انکودر استوار است و بر روی داده‌های سیگنال خام نرمال‌شده کار می‌کند. این مدل در مقایسه با طبقه‌بند‌ها و رگرسورهای مبتنی بر LSTM و SVM و ARIMA عملکرد بهتری را ارائه داده و تقریباً در تمام وظایف از آنها بهتر جلو زده است.

% --------------------------------------------------------------------------------------------------------

	\section{مقاله \textcolor{blue}{\cite{article15}}}
در این مقاله، یک روش نوآورانه برای تشخیص خطاها در ماشین‌آلات دوار به نام شبکه کانولوشن ترانسفورمر (TCN) معرفی شده است. این روش از یک ساختار ترکیبی شامل یک انکودر ترانسفورمر و یک شبکه عصبی کانولوشن (CNN) استفاده می‌کند. ابتدا داده‌های سیگنال به قطعات ثابت تقسیم شده و این قطعات به عنوان ورودی به انکودر ترانسفورمر استفاده می‌شود. سپس یک شبکه CNN با یک لایه طبقه‌بند ساخته می‌شود تا الگوها را تجزیه و تحلیل کرده و دسته‌بندی کند. این مدل TCN ابتدا پیش‌آموزش داده شده و سپس با استفاده از استراتژی یادگیری انتقالی بهینه می‌شود. نتایج آزمایشات نشان می‌دهد که این روش نه تنها عملکرد مناسبی در تشخیص خطاهای ماشین‌آلات دوار دارد بلکه به طور قابل توجهی از روش‌های پیشین پیشرفت داشته است.

در این کار، دو نوآوری اصلی معرفی می‌شود. ۱) ارائه یک مدل نوآورانه TCN از طریق معماری انکودر ترانسفورمر و CNN است که برای تشخیص خطاها از این معماری برای پردازش سیگنال‌های یک بعدی استفاده می‌کند. ۲) این روش امکان تشخیص خطاها با استفاده از تعداد محدودی نمونه آموزشی مرتبط با شرایط و امکانات مختلف را فراهم می‌کند. این امکان با استفاده از استراتژی یادگیری انتقالی حاصل می‌شود که به مدل TCN امکان آموزش بهینه در دامنه هدف را می‌دهد و به سرعت به وظایف جدید با دخالت حداقلی انسان تطبیق می‌یابد.

% --------------------------------------------------------------------------------------------------------

\section{مقاله \textcolor{blue}{\cite{article11}}}
در این مقاله، یک چارچوب جدید و یکپارچه معرفی شده است که از قدرت شبکه‌های عصبی ترانسفورمر و الگوریتم‌های یادگیری تقویتی عمیق (DRL) برای بهینه‌سازی اقدامات نگهداری استفاده می‌کند. رویکرد این مقاله از مدل ترانسفورمر برای به دقت زیاد پیش‌بینی عمر مفید باقی‌مانده (RUL) تجهیزات استفاده می‌کند تا الگوهای زمانی پیچیده در داده‌های حسگر را به طور موثر درک کرده و در نتیجه، بهینه‌سازی عملکردهای نگهداری را فراهم آورد. این روش بر روی دیتاست \lr{NASA C-MPASS} تست شده است و بهبودهای قابل توجهی در دقت پیش‌بینی RUL و بهینه‌سازی اقدامات نگهداری نشان می‌دهد.

% --------------------------------------------------------------------------------------------------------

\section{مقاله \textcolor{blue}{\cite{article1}}}
در این مقاله، یک روش جامع بر اساس سیگنال های ارتعاشی\footnote{Vibration} ارائه شده است. داده‌های جمع‌آوری شده از 30 پمپ صنعتی در یک کارخانه شیمیایی در طی 2.5 سال برای اعتبارسنجی این مفهوم استفاده شده است. به این منظور، معیارهای مشتق شده از داده‌های ارتعاشی تا 7 روز به وسیله الگوریتم معتبر و سریع \lr{Random Forest} پیش‌بینی شده‌اند. عملکرد مدل با یک تکنیک استقامت استاندارد مقایسه شده است.


در این راستا، نو‌آوری های اصلی مقاله به‌ صورت زیر است:
\begin{enumerate}
	\item نمونه‌هایی از کارکرد خوب معیارهای نظارت ارتعاشی با استفاده از حسگرهای کم هزینه در محیط صنعتی در طول ۲٫۵ سال نشان داده شده است.
	\item مراحل پیش‌پردازش داده‌ها برای رسیدن به یک مدل پیش‌بینی معتبر و با دقت بالا و سریع نشان داده شده است.
	\item خرابی‌های واقعی ای که در کارخانه‌ها رخ داده است که می‌شد با استفاده از این مدل وقوع آن را جلو گیری کرد بررسی شده است
\end{enumerate}

% --------------------------------------------------------------------------------------------------------

\section{مقاله \textcolor{blue}{\cite{article4}}}
در صنایع کاربردی، برای افزایش بهره‌وری ماشین‌آلات از نظارت بر تجهیزات بهره‌گیری می‌شود تا خطرات خرابی غیرمنتظره و قطعیات ناشی از آنها کاهش یابد. ارتعاشات ماشین، به عنوان روشی مهم و گسترده برای تشخیص و پیش‌بینی خرابی‌ها، مورد نظر قرار می‌گیرد. این مطالعه با هدف تشخیص اشکال در تجهیزات چرخان از تجزیه و تحلیل ارتعاشات استفاده می‌کند. در یک آزمایش نظارت بر وضعیت موتور، سرعت عملیاتی آن توسط یک درایو موتور AC کنترل می‌شود و ارتعاش موتور اندازه‌گیری و نظارت می‌شود. با تحلیل داده‌های ارتعاش با استفاده از نرم‌افزارهای طیف‌سنجی و MATLAB وضعیت موتور تعیین و فرکانس طبیعی مشخصه با نوع خرابی یا حالت خرابی مرتبط است.

نتایج حاصل از آزمایشات نشان می‌دهند که تجزیه و تحلیل ارتعاشات در نگهداری پیش‌بینانه موثر است. با مقایسه داده‌های ارتعاش برای هر شرایط خرابی با موتور سالم، الگوهای مرتبط با هر حالت خرابی مشخص می‌شوند. این نتایج نشان می‌دهند که چگونه از طریق تجزیه و تحلیل ارتعاشات، می‌توان خرابی‌های مختلف در سیستم‌های مکانیکی چرخان را تشخیص داد و با پیش‌بینی روندهای ارتعاشی، میزان نگهداری و هزینه آن را به حداقل رساند.

% --------------------------------------------------------------------------------------------------------

\section{مقاله \textcolor{blue}{\cite{article8}}}
در این مقاله، یک روش نوآورانه تحلیل داده‌ به نام \lr{functional MLP} برای تخمین RUL ارائه شده است. این روش با نگاهی به داده‌های سری زمانی از چندین تجهیز به عنوان یک نمونه از فرآیندهای تصادفی پیوسته در طول زمان می‌پردازد. این روش تجزیه و تحلیل داده‌ها را به گونه‌ای به انجام می‌دهد که همگرایی داخل همان تجهیز و نوسانات تصادفی در سری‌های زمانی سنسورهای تجهیزات مختلف را در مدل لحاظ می‌کند. همچنین، این روش این امکان را دارد که رابطه بین RUL و متغیرهای سنسوری در طول زمان متغیر باشد. نتایج اعمال این روش بر داده‌های بنچمارک \lr{NASA C-MAPSS} نشان‌دهنده عملکرد خوب این روش نسبت به روش های دیگر در این زمینه می‌باشد.

% --------------------------------------------------------------------------------------------------------

\section{مقاله \textcolor{blue}{\cite{article36}}}
در این مقاله، یک روش پیش‌پردازش بر مبنای تبدیل دیسکرت اورتونورمال استوکول (DOST) برای تصویربرداری ارتعاشی به عنوان گامی اولیه پیشنهاد شده است تا سناریوهای مستقل از بار و سرعت چرخش را برای سیگنال‌های مختلف با شرایط سلامتی متفاوت پشتیبانی کند. برای هر شرایط سلامتی، ویژگی‌ها به سادگی از الگوی سلامت تولید شده می‌توانند استخراج شوند. برای اتوماسیون فرآیند انتخاب ویژگی، یک رویکرد یادگیری انتقالی (TL) بر مبنای شبکه عصبی CNN برای تشخیص نیز معرفی شده است. یادگیری انتقالی به یک مدل تأسیس‌شده این امکان را می‌دهد که از دانش ویژگی‌های بدست‌آمده تحت یک مجموعه شرایط کاری از طریق لایه‌های پنهان برای تشخیص اشکالی که در شرایط کاری دیگر رخ می‌دهد، استفاده کند. استفاده از مجموعه داده بلبرینگ دانشگاه \lr{Case Western Reserve} به روش پیشنهادی دقت طبقه‌بندی میانگین ۹۹٫۸٪ و به ویژه ۹۹٫۹۹٪ برای شرایط سالم (HC)، ۹۹٫۹۵٪ برای خرابی میله داخلی (IRF)، ۹۹٫۹۶٪ برای خرابی توپ (BF)، ۹۹٫۶۸٪ برای خرابی میله خارجی در موقعیت 12 (ORF@12)، ۹۹٫۹۳٪ برای خرابی میله خارجی در موقعیت 3 (ORF@3) و ۹۹٫۸۹٪ برای خرابی میله خارجی در موقعیت 6 (ORF@6) داده شده است. در این مقاله، رویکرد پیشنهادی با شبکه‌های عصبی مصنوعی معمولی (ANNs)، ماشین‌های بردار پشتیبان (SVMs)، CNNهای سلسله‌مراتبی و اتوانکودرهای عمیق مقایسه شده است و در دقت تحت تمام شرایط کاری، این رویکرد پیشنهادی بهتر عمل کرده است.

% --------------------------------------------------------------------------------------------------------

\section{مقاله \textcolor{blue}{\cite{article22}}}
در این مقاله، یک الگوریتم تشخیص خطا در بلبرینگ موتورهای صنعتی بر مبنای تصمیم‌گیری حل تداخل چندمحلی (MLMF-CR) ارائه شده است تا اطلاعات چندمنبعی و تنوع‌های مختلف را به‌طور کامل یکپارچه سازی کرده و تداخل‌های اطلاعات چندمنبعی را به‌طور مناسب حل کند. روش پیشنهادی بر اساس \lr{Bi-LSTM} است و ابتدا از کدگذار کاهنده نویز برای تمیز کردن سیگنال ارتعاش و جریان بلبرینگ موتور صنعتی استفاده می‌کند تا به ترتیب سیگنال بازسازی ارتعاش و سیگنال بازسازی جریان را به‌دست آورد. سپس، سیگنال بازسازی ارتعاش و سیگنال بازسازی جریان را به مدل تشخیص خطا محلی بر مبنای \lr{Bi-LSTM} می‌دهد و اطلاعات مشخصه سیگنال بازسازی در هر وضعیت خطا به‌طور عمیق استخراج می‌شود تا تصمیم تشخیص محلی شکل گیرد. در نهایت، تصمیم‌های تشخیص محلی با استفاده از تئوری شواهد \lr{D-S} جمع‌آوری می‌شوند تا نتایج تشخیص نهایی حاصل شود. آزمایشات با ارزیابی دقت، امتیاز \lr{PI}، زمان آموزش و نرخ مثبت غلط انجام شد است که برتری روش پیشنهادی را نشان می‌دهد. با این حال بیان شده است که، هنوز برخی نکات قابل بهبود وجود دارد که در آینده توسعه داده خواهد شد.

% --------------------------------------------------------------------------------------------------------

\section{مقاله \textcolor{blue}{\cite{article23}}}
در این مقاله، یک سیستم اندازه‌گیری مبتنی بر پردازش دیجیتال سیگنال (DSP) برای تجزیه و تحلیل ارتعاشات در دستگاه‌های چرخشی طراحی و اجرا شده است. سیگنال‌های ارتعاش به صورت آنلاین جمع‌آوری و پردازش می‌شوند تا یک نظارت مداوم بر وضعیت دستگاه فراهم شود. در صورت وجود خطا، سیستم قابلیت تشخیص خطا را با قابلیت اطمینان بالا دارد. این مقاله به طور دقیق رویکرد انجام شده برای ایجاد مدل‌های خطا و بدون خطا همراه با راهکارهای سخت‌افزاری و نرم‌افزاری انتخاب شده را شرح می‌دهد. تست‌های انجام شده بر روی موتورهای القایی سه فاز کوچک به عالی بودن سرعت در تشخیص خطا، کاهش نرخ هشدارهای غلط، و عملکرد تشخیصی بسیار خوب اشاره دارد.

% --------------------------------------------------------------------------------------------------------











	
	
%	\cite{article1} \cite{article2} \cite{article3} \cite{article4}
%	\cite{article5} \cite{article6} \cite{article7} \cite{article8}
%	\cite{article9} \cite{article10} \cite{article11} \cite{article12}
%	\cite{article13} \cite{article14} \cite{article15} \cite{article16}
%	\cite{article17} \cite{article18} \cite{article19} \cite{article20}
%	\cite{article21} \cite{article22} \cite{article23} \cite{article24}
%	\cite{article25} \cite{article26} \cite{article27} \cite{article28} 
%	\cite{article29} \cite{article30} \cite{article31} \cite{article32} 
%	\cite{article33} \cite{article34} \cite{article35} \cite{article36} 
%	\cite{article37} \cite{article38} \cite{article39} \cite{article40}
%	\cite{article41} \cite{article42} \cite{article43} \cite{article44}   
%	\cite{article45} \cite{article46}
	
	
	
	\newpage
	
	\begin{latin}
		\bibliographystyle{IEEEtran}
		\bibliography{refs}
	\end{latin}
	
\end{document}  