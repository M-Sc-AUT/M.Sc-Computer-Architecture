\section{ پردازش تصویر دوبعدی}

% اگر تصویری در صفحە ی کانونی یک عدسی (ایدە آل) قرار گیرد، تبدیل فوریە ی دوبˀعدی آن در صفحە ی کانونی طرف دیگر
% عدسی تشکیل می شود. با توجˁه به شکل های صفحە ی بعد، فرض کنید در صفحە ی کانونی عدسی تصویری که در سمت
% چپ نشان داده شده به عنوان ورودی قرار دارد (نواحی سفید مقدار واحد و نواحی سیاه مقدار صفر دارند). در صفحە ی
% کانونی طرف دیگر این عدسی، یک صفحە ی دوبعدی غیرشفّاف با ناحیە ای شفاف (که با رنگ سفید نشان داده شده) طبق
% شکل قرار گرفته است. این صفحه، خود در فاصله کانونی عدسی می باشد که در طرف دیگرش قرار گرفته است. صفحە ی
% تصویر نهائی مورد نظر در فاصلە ی کانونی در طرف دیگر عدسی (در منتهاالیه سمت راست شکل) قرار دارد. با فرض ایدە آل
% بودن اجزای تشکیل دهندە ی این سامانه، به طور تقریبی شکل تصویری را که در صفحە ی تصویر نهائی تشکیل می شود، برای
% شکل های ورودی داده شده ترسیم نمائید (مقادیر غیرصفر را با سفید و مقادیر صفر را با سیاه نشان دهید).

\begin{figure}[h]
    \caption{قسمت اول}
    \centering
    \includegraphics*[width=0.5\linewidth]{pics/q5_1.png}
\end{figure}

\begin{qsolve}[]
    تصویری که به فیلتر میرسد، تبدیل فوریه ورودی ما است، ابتدا این تبئیل فوریه را حساب میکنیم.
    \begin{center}
        \includegraphics*[width=0.8\linewidth]{pics/q5_1_ans.png}
        \captionof{figure}{تبدیل فوریه ورودی}
        \label{q5_1}
    \end{center}

    حال اگر فیلتر پایین گذر ما به اندازه کافی محدود کننده باشد، یعنی فرکانس قطع آن پایین تر از فرکانس خطوط باشد، 
    شکل نهایی ما تبدیل به شکلی مانند شکل b در شکل \ref*{q5_1} میشود، اگر نه به طور محو تر خطوط عمودی را میبینیم.

    \begin{center}
        \includegraphics*[width=4.3cm,angle=90]{pics/q1_1_ans_2_1.png}
        \includegraphics*[width=4.3cm,angle=90]{pics/q1_1_ans_2_2.png}
        \captionof{figure}{سمت راست فیلتر با فرکانس قطع پایین و سمت چپ با فرکانس قطع بالا}
    \end{center}
\end{qsolve}

\begin{figure}[h]
    \caption{قسمت دوم}
    \centering
    \includegraphics*[width=0.5\linewidth]{pics/q5_2.png}
\end{figure}

\begin{qsolve}[]
    تصویری که به فیلتر میرسد، تبدیل فوریه ورودی ما است، ابتدا این تبئیل فوریه را حساب میکنیم.
    \begin{center}
        \includegraphics*[width=0.8\linewidth]{pics/q5_2_ans_1.png}
        \captionof{figure}{تبدیل فوریه ورودی}
        \label{q5_2}
    \end{center}
    حال جواب شکل \ref*{q5_2} را با یک فیلتر ایده آل فیلتر میکنیم، این فیلتر در حوزه مکان شبیه به 
    تبدیل فوریه قسمت c در شکل \ref*{q5_2} میباشد. 

    پس انگار شکل اولیه ما با یک sinc, کانوالو میشود که شکل نهایی خروجی تقریبا به شکل زیر است.

    \begin{center}
        \includegraphics*[width=4cm]{pics/q5_2_ans_2.png}
        \captionof{figure}{شمای تقریبی خروجی}
    \end{center}
\end{qsolve}