\section{سوال دوم}

فرض کنید شما یکی از کارمندان AMD می‌باشید، از آنجایی که Yeild پردازنده‌های تولید شده بسیار پایین است،‌ همکار شما پیشنهاد می‌کند، که ‫با‬ ‫تولید‬ ‫نسخه‬های ‬‫متعدد‬ ‫از‬ ‫یک‬ ‫تراشه‬ ‫با‬ ‫تعداد‬ ‫هسته‬‫های ‬‫متفاوت‬ ‫ممکن‬ ‫است‬ ‫تراشه‬‫های ‬‫ارزان‬ ‫تری را بتوان تولید کرد. به‌عنوان مثال می‌توان ،Phonix8 ،Phonix4 ،Phonix2 Phonix1 که به‌ترتیب دارای ۸، ۴، ۲ و ۱ هسته هستند را به فروش برسانید.‬ اگر هر ۸ هسته سالم باشند، به‌عنوان Phonix8 به فروش می‌رسد. تراشه‌های با ۴ تا ۷ هسته سالم به‌عنوان Phonix4 و تراشه‌های با ۲ یا ۳ هسته سالم به‌عنوان Phonix2 به‌فروش می‌رسند. برای ساده‌تر شدن Yield یک هسته را معادل Yeild تراشه ای که ۱٫۸ مساحت تراشه اصلی Phonix است درنظر بگیرید. ‫سپس‬ ‫آن‬ ‫را‬ ‫به‬‫عنوان‬ یک‬ ‫احتمال مستقل از یک هسته سالم درنظر بگیرید. Yield را به ازای هر پیکره‌بندی بدون درنظر گرفتن تعداد هسته‌ها محاسبه کنید.‬
\begin{qsolve}
فرمول احتمال بی‌نقص بودن تراشه به‌صورت زیر است:
\begin{equation}
	\#combinations = (0.87)^N \times (1 - 0.87)^{8-N}
\end{equation}
مقاپیر بدست آمده به‌صورت زیر است: (مقادیر توسط اسکریپت پایتون نوشته شده محاسبه شده است که می‌توانید آن را از \href{https://github.com/rezaAdinepour/M.Sc-AUT/tree/main/Advanced%20Computer%20Architecture/HWs/Theory/HW01/Solution/Calc}{اینجا} دریافت کنید)

\begin{latin}
	\begin{center}
		\begin{tabular}{||c c c||} 
			\hline
			\#defect-free & \#combinations & \#Probability \\ [0.5ex] 
			\hline\hline
			0 & 1 & 3.28211672e-01 \\ 
			\hline
			1 & 8 & 4.90431233e-02 \\ 
			\hline
			2 & 28 & 7.32828280e-03 \\
			\hline
			3 & 56 & 1.09503076e-03 \\
			\hline
			4 & 70 & 1.63625286e-04 \\
			\hline
			5 & 56 & 2.44497554e-05 \\
			\hline
			6 & 28 & 3.65341173e-06 \\
			\hline
			7 & 8 & 5.45912098e-07 \\
			\hline
			8 & 1 & 8.15730721e-08 \\ [1ex] 
			\hline
		\end{tabular}
	\end{center}
\end{latin}


\end{qsolve}


\begin{enumerate}
	\item Yield برای یک هسته سالم برای ،Phonix4 ،Phonix2 Phonix1 چقدر است؟
	\begin{qsolve}
فرمول برای پردازنده تک هسته ای به‌صورت زیر است:
		\begin{equation}
			Yield = \frac{1}{(1 + (0.04 \times 0.25))^{14}} = \mathcolorbox{yellow}{0.87}
		\end{equation}
		\begin{equation}
			Yield_{Phonix^4} = (0.39+0.21+0.06+0.01)=\mathcolorbox{yellow}{057}
		\end{equation}
		\begin{equation}
			Yield_{Phonix^2} = (0.001+0.0001)=\mathcolorbox{yellow}{0.0011}
		\end{equation}
		\begin{equation}
			Yield_{Phonix^1} = \mathcolorbox{yellow}{0.000004}
		\end{equation}
	\end{qsolve}
	
	
	
	
	\item با توجه به قسمت قبل،‌ کدام تراشه ها ارزش بسته‌بندی و فروش دارند؟ چرا؟
	\begin{qsolve}
		با توجه به قسمت قبل، بسته‌بندی و فروش Phonix4 ارزشمند است. Phonix2 و Phonix1 به قدری احتمال وقوع کمی دارند که اصلا ارزش اقتصادی برای فروش آنها نیست.
	\end{qsolve}
	
	
	
	
	\item اگر قبلا در تولید Phonix8 به ازای هر تراشه ۲۰ دلار هزینه داشتیم، با فرض اینکه هزینه اضافی ای برای از رده خارج شدن نداشته باشیم، هزینه تراشه‌های جدید Phonix چقدر خواهد بود؟
	
	\begin{qsolve}
دیتا مسئله برای محاسبه قیمت پردازنده‌های جدید کم است. فقط می‌توان گفت که هزینه ۲۰ دلار به صورت زیر محاسبه شده است:
		\begin{equation}
			\$20=\frac{wafer \ size}{odd \ dpw \times 0.28}
		\end{equation}
	\end{qsolve}
	
	
	
	
	
	
	
	
	
	
	
	
	\item شما درحال حاضر برای هر Phonix8 سالم، به ازای هر تراشه، ۳۰ دلار سود می‌کنید و هر تراشه Phonix4 را به قیمت ۲۵ دلار می‌فروشید. اگر قیمت قیمت خرید تراشه‌های Phonix4 را کاملا سود درنظر بگیرید، و سود تراشه Phonix4 را به نسبت تعداد تولبد شده در هر تراشه Phonix8 اعمال کنید، چقدر سود شما در تراشه Phonix8 است؟ از Yield محاسبه شده در قسمت اول استفاده کنید.
	\begin{qsolve}
	ابتدا محاسبه می‌کنیم به ازای هر تراشه Phonix8 چند تراشه Phonix4 تولید میشود:
	
	به ازای هر تراشه ،Phonix8 ۱٫۷۲ تراشه Phonix4 تولید میشود. بنابراین هزینه تولید تراشه‌های جدید به صورت زیر محاسبه می‌شود:
	
	\begin{equation}
		\$30+1.73\times\$25=\mathcolorbox{yellow}{\$73.25}
	\end{equation}
	\end{qsolve}
	
	
	
	
	
	
	
\end{enumerate}