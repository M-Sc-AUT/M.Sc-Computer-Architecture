\section{سوال اول}

\lr{Overlapped Parity}
را برای داده‌های ۶ بیتی توسعه دهید.

\begin{qsolve}
	برای حل این سوال از کد همینگ استفاده می‌کنیم.
	
	ابتدا باید تعداد بیت‌های \lr{Parity} مورد نیاز برای داده‌های ۶ بیتی را محاسبه کنیم. طبق فرمول زیر داریم:
	
	\begin{latin}
		$$ D+P+1 \le 2^P  \xRightarrow{D=6} 7+P\le2^P $$
		
		$$ \text{if } P=3: 7 + 3 \not \leq 2^3 \rightarrow 10 \not \leq 8$$
		$$ \text{if } P=4: 7 + 4 \le  2^4 \rightarrow 11 \leq 16$$
	\end{latin}
	
	
	بنابراین برای ۶ بیت داده، حداقل ۴ بیت \lr{Parity} نیاز است.
	
	بیت‌های توازن را در موقعیت‌های توانی از ۲ قرار می‌دهیم (موقعیت‌های ۱، ۲، ۴، ۸) و سایر موقعیت‌ها را به بیت‌های داده اختصاص می‌دهیم. در نتیجه، موقعیت‌ها به این شکل خواهند بود:
	
	\begin{latin}
		\begin{center}
			\begin{tabular}{|c|c|c|c|c|c|c|c|c|c|c|}
				\hline
				Position & 1 & 2 & 3 & 4 & 5 & 6 & 7 & 8 & 9 & 10 \\
				\hline
				Parity/Data & P1 & P2 & D1 & P4 & D2 & D3 & D4 & P8 & D5 & D6 \\
				\hline
			\end{tabular}
		\end{center}
	\end{latin}
	
	
	بیت‌های توازن به‌صورت زیر محاسبه می‌شوند:
	
	\begin{itemize}
		\item \( P1 \): بیت‌های ۱، ۳، ۵، ۷، ۹ و ۱۱
		\item \( P2 \): بیت‌های ۲، ۳، ۶، ۷ و ۱۰
		\item \( P4 \): بیت‌های ۴، ۵، ۶ و ۷
		\item \( P8 \): بیت‌های ۸، ۹ و ۱۰
	\end{itemize}
	
	
	برای مثال اگر داده ۶ بیتی ما 101011 باشد، بیت‌های توازن به شکل زیر محاسبه می‌شوند:
	
	\begin{align*}
		P1 &= 0 \\
		P2 &= 1 \\
		P4 &= 1 \\
		P8 &= 0
	\end{align*}
\end{qsolve}



\begin{qsolve}
	
	بنابر این \lr{CodeWord} حاصل به‌صورت زیر تشکیل می‌شود:
	
	\begin{align*}
		0111010011
	\end{align*}
	
	
	 همچنین بیت‌های توازن و داده به‌صورت جداگانه عبارتند از:
	\begin{itemize}
		\item \textbf{داده}: \(101011\)
		\item \textbf{بیت‌های توازن }: \(0110\)
	\end{itemize}
	
	
	در جدول زیر چند حالت از \(2^6 = 64\) ترکیب داده‌های ۶ بیتی و بیت‌های توازن مربوط به آن‌ها را آورده‌ایم:
	
	\begin{latin}
		\begin{center}
		\begin{tabular}{|c|c|c|c|}
			\hline
			\textbf{Row} & \textbf{Data} & \textbf{Parity} \\
			\hline\hline
			1 & 000000 & 0000 \\
			2 & 000001 & 0110 \\
			3 & 000010 & 1100 \\
			4 & 000011 & 1010 \\
			5 & 000100 & 1111 \\
			6 & 000101 & 1001 \\
			7 & 000110 & 0011 \\
			8 & 000111 & 0101 \\
			9 & 001000 & 0111 \\
			10 & 001001 & 0001 \\
			11 & 001010 & 1011 \\
			12 & 001011 & 1101 \\
			13 & 001100 & 1000 \\
			14 & 001101 & 1110 \\
			15 & 001110 & 0100 \\
			16 & 001111 & 0010 \\
			\vdots & \vdots & \vdots \\
			64 & 111111 & 1001 \\
			\hline
		\end{tabular}
	\end{center}
	\end{latin}
	
	
	
\end{qsolve}



