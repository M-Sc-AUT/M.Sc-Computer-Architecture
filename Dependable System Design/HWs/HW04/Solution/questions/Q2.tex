\section{سوال دوم}

پنج مثال از کاربردهایی ارائه دهید که در آن‌ها استفاده از افزونگی «آماده‌به‌کار سرد» و «آماده‌به‌کار گرم» را توصیه می‌کنید. پاسخ خود را با دلایل مناسب توجیه کنید.

\begin{qsolve}
	
	\begin{itemize}
		\item 
		\textbf{مثال‌های آماده به کار گرم:}\\
		\begin{enumerate}
			\item 
			سیستم‌های مخابراتی حیاتی: در شبکه‌های مخابراتی که توقف سیستم ممکن است منجر به قطعی‌های بزرگ و اختلالات گسترده در ارتباطات شود، استفاده از «آماده‌به‌کار گرم» ترجیح داده می‌شود. در این سیستم‌ها، تجهیزات یدکی همیشه در حالت فعال هستند و در صورت خرابی بلافاصله می‌توانند جایگزین شوند، بنابراین زمان خرابی به حداقل می‌رسد.
			
			
			
			\item 
			سرورهای بانکداری و مالی: در سرورهای مالی که عملیات لحظه‌ای و پیوسته از اهمیت بالایی برخوردار است، از افزونگی «آماده‌به‌کار گرم» استفاده می‌شود. به دلیل اهمیت بالای حفظ داده‌ها و جلوگیری از توقف خدمات، سرورها همیشه آماده هستند تا در صورت خرابی سرور اصلی، بدون تأخیر به کار گرفته شوند. در این سیستم‌ها، زمان خرابی نباید وجود داشته باشد.
		\end{enumerate}
	\end{itemize}
\end{qsolve}



%		\begin{enumerate}
%	\item [۳.]
%	



\begin{qsolve}
	\begin{itemize}
		\item [ ]
		\begin{enumerate}
			\item [3.]
			مراکز داده حیاتی: در مراکز داده‌ای که داده‌ها به صورت لحظه‌ای پردازش می‌شوند و از دست رفتن حتی چند ثانیه اطلاعات می‌تواند خسارات زیادی به بار آورد، استفاده از «آماده‌به‌کار گرم» توصیه می‌شود. در این سیستم‌ها، هرگونه اختلال در عملکرد سرورها می‌تواند با تغییر سریع به سرور یدکی جلوگیری شود تا عملکرد مداوم سیستم تضمین گردد.
		\end{enumerate}
		
		\item 
		\textbf{مثال‌های آماده به کار سرد:}\\
		\begin{enumerate}
			\item 
			سیستم‌های کنترل صنعتی: در سیستم‌های کنترل صنعتی مانند کارخانه‌های تولیدی، که عملیات به طور متناوب انجام می‌شود و نیازی به عملکرد بی‌وقفه نیست، از «آماده‌به‌کار سرد» استفاده می‌شود. در این حالت، سیستم یدکی تنها در صورت خرابی سیستم اصلی فعال می‌شود. این رویکرد مقرون‌به‌صرفه است و هزینه‌های انرژی را کاهش می‌دهد.
			
			\item 
			سیستم‌های ماهواره‌ای: در سیستم‌های ماهواره‌ای، مصرف انرژی بسیار اهمیت دارد. به دلیل محدودیت‌های انرژی در فضا، استفاده از افزونگی «آماده‌به‌کار سرد» توصیه می‌شود. در این حالت، تجهیزات یدکی در حالت خاموش قرار می‌گیرند و فقط در صورت خرابی تجهیزات اصلی فعال می‌شوند. این رویکرد باعث صرفه‌جویی در انرژی می‌شود، هرچند زمان بیشتری برای فعال‌سازی یدک نیاز است.
		\end{enumerate}
	\end{itemize}
\end{qsolve}