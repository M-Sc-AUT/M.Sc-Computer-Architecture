\section{سوال چهارم}


پیکربندی نشان داده شده در شکل زیر به نام افزونگی سه‌گانه‌-دوگانه \lr{(triple-duplex redundancy)} شناخته می‌شود. در این پیکربندی، شش ماژول یکسان که در سه جفت گروه‌بندی شده‌اند، به صورت موازی عمل می‌کنند. در هر جفت، نتایج محاسبات با استفاده از یک مقایسه‌گر مقایسه می‌شود. اگر نتایج همخوانی داشته باشند، خروجی مقایسه‌گر در رأی‌گیری شرکت می‌کند. در غیر این صورت، جفت ماژول‌ها معیوب اعلام شده و سوئیچ آن‌ها را از سیستم حذف می‌کند. یک رأی‌دهنده آستانه‌ای که قادر به تطبیق با کاهش تعداد ورودی‌ها است، استفاده می‌شود.

زمانی که اولین جفت دوگانه از رأی‌گیری حذف می‌شود، به عنوان یک مقایسه‌گر عمل می‌کند. وقتی که جفت دوم حذف می‌شود، سیگنال ورودی خود را مستقیماً به خروجی منتقل می‌کند. چنین پیکربندی چند خطای ماژول را می‌تواند تحمل کند؟

\begin{figure}[h]
	\centering
	\includegraphics*[width=0.6\linewidth]{pics/img2.png}
\end{figure}


\begin{qsolve}
	
\end{qsolve}