%Source: page 31 of book (PDF)
\section{سوال اول}

مداری منطقی با ۳۲۰۰ خط دارای ۲۰ خطای \lr{stuck-at} غیرقابل تشخیص است. مجموعه آزمونی که برای تست تولید این مدار طراحی شده است، قادر به شناسایی ۶۲۵۲ خطای \lr{stuck-at} تک‌خطی در مدار می‌باشد. بررسی کنید که آیا پوشش خطاهای حاصل‌شده، به حد نصاب صنعتی 99\% پوشش شناسایی خطاهای قابل تشخیص می‌رسد یا خیر.




\begin{qsolve}
	از آنجا که برای هر خط دو خطای ممکن \lr{stuck-at-0} و \lr{stuck-at-1} می‌تواند رخ دهد، تعداد کل خطاهای ممکن در مدار ۶۴۰۰ است. پوشش شناسایی خطاهای قابل تشخیص به صورت زیر محاسبه می‌شود:
	
	\[
	C = \frac{\text{تعداد خطاهای تشخیص داده شده}}{\text{تعداد خطاهای قابل تشخیص}} = \frac{6252}{6400 - 20} = 0.9799
	\]
	
	بنابر این طبق مقدار به‌دست آمده، می‌توان نتیجه گرفت که مجموعه آزمونی که برای تست تولید این مدار طراحی شده است، با حدنصاب صنعتی تطابق ندارد.
\end{qsolve}



