\section{سوال سوم}

۵ مثال با استفاده از \lr{ChatGPT} بزنید و در هرکدام \lr{Fault} و \lr{Error} و \lr{Failure} را مشخص کنید.


\begin{qsolve}[]
	\begin{enumerate}
		\item 
		\textbf{خرابی سرور ابری \lr{(Cloud Server)}}\\
		فرض کنید شما یک سرویس ابری برای میزبانی از چندین وب‌سایت و اپلیکیشن دارید که مشتریان زیادی از آن استفاده می‌کنند.
		\begin{itemize}
			\item \lr{\textbf{:Fault}}\\
			یکی از درایوهای ذخیره‌سازی در سرور ابری که میزبان داده‌های حیاتی وب‌سایت‌ها است، دچار خرابی می‌شود. این مشکل سخت‌افزاری ممکن است به دلیل مشکلات فیزیکی یا مصرف بالای منابع ایجاد شده باشد.
			
			\item \lr{\textbf{:Error}}\\
			به دلیل خرابی درایو، سیستم ابری نمی‌تواند به فایل‌های وب‌سایت‌ها و داده‌های ذخیره‌شده به درستی دسترسی پیدا کند. کاربران با مشکلاتی مانند کندی بارگذاری یا نمایش خطاهای عدم دسترسی به داده‌ها مواجه می‌شوند.
			
			\item \lr{\textbf{:Failure}}\\
			در نهایت، سرور ابری از کار می‌افتد و تمامی وب‌سایت‌ها و اپلیکیشن‌هایی که به این سرور وابسته هستند، غیرفعال می‌شوند. این مشکل تا زمانی که درایو معیوب تعویض شود و داده‌ها بازیابی شوند، باقی می‌ماند. 
		\end{itemize}
		
		
		\item 
		\textbf{نقص در الگوریتم هوش مصنوعی}\\
		فرض کنید شما در حال توسعه یک سیستم هوش مصنوعی برای تشخیص سرطان در تصاویر پزشکی هستید و این سیستم به مدل‌های پیچیده یادگیری عمیق وابسته است.
		
		\begin{itemize}
			\item \lr{\textbf{:Fault}}\\
			یکی از پارامترهای آموزش شبکه عصبی (مثل نرخ یادگیری یا تعداد لایه‌ها) به اشتباه پیکربندی شده است و این باعث می‌شود که مدل نتواند به درستی بهینه شود.
			
			\item \lr{\textbf{:Error}}\\
			به دلیل پیکربندی نادرست، مدل هوش مصنوعی خروجی‌های اشتباه تولید می‌کند. به جای تشخیص صحیح سرطان در تصاویر، نرخ خطا بسیار بالا است و بسیاری از تشخیص‌ها نادرست می‌شوند.
			
			\item \lr{\textbf{:Failure}}\\
			در نهایت، سیستم هوش مصنوعی شما عملاً بی‌فایده می‌شود زیرا نتایج نادرست تولید می‌کند. این به معنای شکست کامل پروژه است و نیاز به بازبینی و اصلاح پارامترهای مدل دارید.
		\end{itemize}
		
		
		\item 
		\textbf{پایگاه داده توزیع‌شده \lr{(Distributed Database)} در یک شرکت بزرگ}\\
		فرض کنید شما یک سیستم پایگاه داده توزیع‌شده برای یک شرکت بین‌المللی مدیریت می‌کنید که اطلاعات حیاتی مشتریان و تراکنش‌ها را ذخیره می‌کند.
		
		\begin{itemize}
			\item \lr{\textbf{:Fault}}\\
			یکی از نودهای پایگاه داده به دلیل خرابی شبکه محلی نمی‌تواند با دیگر نودهای سیستم ارتباط برقرار کند. این نقص در ارتباط شبکه رخ داده است و عملکرد نود را مختل کرده است.
			
			\item \lr{\textbf{:Error}}\\
			به دلیل این نقص، پایگاه داده توزیع‌شده نمی‌تواند به درستی داده‌ها را همگام‌سازی کند. کاربران ممکن است هنگام ثبت سفارش یا انجام تراکنش‌های مالی با خطا مواجه شوند.			

		\end{itemize}
	\end{enumerate}
\end{qsolve}





\begin{qsolve}
	\begin{enumerate}
		\item [ ]
		\begin{itemize}
			\item \lr{\textbf{:Failure}}\\
			در نهایت، سیستم پایگاه داده قادر به ارائه خدمات به مشتریان نمی‌باشد و تمامی تراکنش‌های جدید متوقف می‌شوند. تا زمانی که مشکل نود معیوب برطرف نشود، سیستم از کار افتاده است و شرکت دچار اختلال جدی می‌شود.
			
		\end{itemize}
		
		
		\item [4.]
		\textbf{سیستم کنترل خودکار هواپیما}\\
		فرض کنید شما در حال توسعه یک سیستم کنترل خودکار برای هواپیماهای مسافربری هستید که از حسگرهای مختلف برای جمع‌آوری داده‌ها و کنترل پرواز استفاده می‌کند.
		
		\begin{itemize}
			\item \lr{\textbf{:Fault}}\\
			یکی از حسگرهای هواپیما که اطلاعات مربوط به سرعت هوا را ارائه می‌دهد، دچار نقص شده و دیگر قادر به ارسال داده‌های صحیح نیست. این نقص می‌تواند به دلیل مشکلات مکانیکی یا الکترونیکی باشد.
			
			\item \lr{\textbf{:Error}}\\
			سیستم کنترل خودکار به دلیل داده‌های نادرست از حسگر سرعت، تصمیمات اشتباهی برای کنترل پرواز می‌گیرد. به عنوان مثال، ممکن است سرعت هوا را کمتر یا بیشتر از حد واقعی تشخیص دهد و دستورات اشتباهی برای تغییر ارتفاع یا سرعت ارسال کند.
			
			\item \lr{\textbf{:Failure}}\\
			در نهایت، سیستم کنترل خودکار به طور کامل از کار می‌افتد و نیاز به دخالت انسانی برای کنترل هواپیما وجود دارد. اگر این مشکل به موقع شناسایی نشود، ممکن است خطرات جدی برای ایمنی پرواز ایجاد شود.
		\end{itemize}
		
		
		
		\item [5.]
		\textbf{نقص در سیستم امنیت سایبری یک سازمان}\\
		فرض کنید شما مسئول یک سیستم امنیت سایبری پیشرفته برای یک سازمان بزرگ دولتی هستید که اطلاعات محرمانه را مدیریت می‌کند.
		
		\begin{itemize}
			\item \lr{\textbf{:Fault}}\\
			یکی از قوانین فایروال سازمان به اشتباه پیکربندی شده و اجازه دسترسی به برخی ترافیک‌های غیرمجاز از طریق شبکه را می‌دهد. این نقص می‌تواند به دلیل خطای انسانی در پیکربندی یا نقص در نرم‌افزار امنیتی باشد.
			
			\item \lr{\textbf{:Error}}\\
			به دلیل این نقص در پیکربندی، یک نفوذگر سایبری موفق به ارسال درخواست‌های غیرمجاز به شبکه می‌شود و برخی از داده‌های حساس سازمانی را به دست می‌آورد. این خطای امنیتی به اطلاعات نادرست و نفوذپذیری شبکه منجر می‌شود.
			
			\item \lr{\textbf{:Failure}}\\
			در نهایت، سیستم امنیت سایبری به طور کامل از کار می‌افتد و نفوذگر موفق به دسترسی به اطلاعات محرمانه سازمان می‌شود. سازمان دچار آسیب جدی می‌شود و این نقص باید به سرعت رفع شود تا از وقوع آسیب‌های بیشتر جلوگیری شود.
		\end{itemize}
		
	\end{enumerate} 
\end{qsolve}