\section{سوال اول}

۵ مثال از دنیای روزمره بزنید و در هرکدام \lr{Fault} و \lr{Error} و \lr{Failure} را مشخص کنید.

\begin{qsolve}[]
	\textbf{تمامی موارد با دیدگاه آقای جانسون پاسخ داده شده است.}
	\begin{enumerate}
		\item 
		\textbf{سفر با اتومبیل به یک شهر دیگر}\\
		فرض کنید شما به همراه خانواده در حال سفری طولانی با خودرو هستید. در طول مسیر، هنگامی که با سرعت بالا در حال رانندگی بودید، ناگهان شیعی را در وسط جاده مشاهده می‌کنید و چون سرعت شما بالابود با آن برخورد می‌کنید و به ادامه مسیر خود می‌پردازید. در ادامه مسیر، ناگهان خودرو خاموش می‌شود.
		\begin{itemize}
			\item \lr{\textbf{:Fault}}\\
			برخورد با مانع وسط جاده به‌عنوان \lr{Fault} در نظر گرقته می‌شود.

			\item \lr{\textbf{:Error}}\\
			سوراخ‌شدن رادیات خودرو در اثر برخورد با مانع و تخلیه تدریجی آب رادیات و بالارفتن آمپر دمای خودرو به‌عنوان \lr{Error} در این سناریو در نظر گرفته می‌شود.
			
			\item \lr{\textbf{:Failure}}\\
			درنهایت خاموش‌شدن خودرو که علت آن دمای بالای خودرو بوده است به‌عنوان \lr{Failure} معرفی می‌شود.
		\end{itemize}
		
		
		\item 
		\textbf{ماشین لباسشویی}\\
		فرض کنید در یک روز طوفانی در خانه مشغول شستشوی لباس‌ها هستید و ماشین لباسشویی خودکار را روشن کرده‌اید تا چرخه شستشو انجام شود. ناگهان ماشین لباسشویی خاموش می‌شود.
		
		\begin{itemize}
			\item \lr{\textbf{:Fault}}\\
			برخورد یک صاعقه به پست برق منطقه شما، موجب افزایش ولتاژ لحظه‌ای برق می‌شود.
			
			\item \lr{\textbf{:Error}}\\
			این افزایش ولتاژ پست، به خطوط انتقال برق به منازل نیز منتقل می‌شود و ولتاژی بالای ۲۲۰ ولت به منابع مصرف‌کننده در خانه وارد می‌شود.
			
			\item \lr{\textbf{:Failure}}\\
			در اثر افزایش ولتاژ ورودی به لباسشویی، محافظ برق موجود بر سر راه لباس‌شویی، برای حفاظت از ماشین لباسشویی و عدمم واردشدن آسیب به آن، برق ورودی به لباس‌شویی را قطع کرده و لباسشویی خاموش می‌شود.
		\end{itemize}
		
		
		\item 
		\textbf{لیگ برتر انگلیس}\\
		‫در‬ ‫فصل‬ ‫‪۲۰۱۴‬‬ ‫لیگ‬ ‫برتر‬ ‫انگلیس‪،‬‬ ‫در‬ ‫بازی‬ ‫چلسی‬ ‫و‬ ‫لیورپول‪،‬‬ ‫استیون‬ ‫جرارد‬، کاپیتان تیم لیورپول، در یک ضد حمله که از جانب تیم چلسی رخ داده بود، ‫به هنگام دفاع از حمله کننده، لیز‬‫خورد‬ و بر زمین افتاد. این اتفاق‬ ‫باعث‬ ‫شد‬ ‫تیم‬ ‫لیورپول‬ ‫گل‬ ‫دریافت‬ ‫کند‬ ‫و‬ ‫همین‬ ‫گل‬ ‫منجر‬ ‫شد‬ ‫لیورپول‬ ‫قهرمانی‬ ‫آن‬ ‫فصل‬ ‫را‬ ‫از‬ ‫دست‬ ‫بدهد‪.‬‬	
	\end{enumerate}
\end{qsolve}





\begin{qsolve}[]
	\begin{enumerate}
		\item [ ]
		\begin{itemize}
			\item \lr{\textbf{:Fault}}\\
			لیزخوردن بازیکن لیورپول.
			
			\item \lr{\textbf{:Error}}\\
			‫لو‬‫رفتن‬ ‫توپ‬ ‫و‬ ‫ایجاد‬ ‫موقعیت‬ ‫برای‬ ‫تیم‬ ‫چلسی‪.‬‬
			
			\item \lr{\textbf{:Failure}}\\
			‫گل‫خوردن‬ ‫تیم‬ ‫لیورپول‬ ‫و‬ ‫از‬ ‫دست‬ ‫رفتن‬ ‫قهرمانی‬ ‫لیگ‬ ‫برتر‬ ‫انگلیس‪.‬‬
		\end{itemize}
		
		
		
		
		\item [4.]
		\textbf{آتش سوزی مهتابی آزمایشگاه}\\
		در آزمایشگاه مشغول مطالعه و تحقیق بودیم که بوی بد سوختن پلاستیک را حس کردیم. در مدت کوتاهی دود غلیظی آزمایشگاه را فرا گرفت و با صدای بلند یکی از مهتابی ها منفجر شد.
		
		\begin{itemize}
			\item \lr{\textbf{:Fault}}\\
			ایجاد نوسان در برق آزمایشگاه.
			
			\item \lr{\textbf{:Error}}\\
			که منجر به بالا رفتن دمای مهتابی و ذوب شدن کابل های متصل به آن شده.
			
			\item \lr{\textbf{:Failure}}\\
			و درنهایت موجب به آتش گرفتن و منفجر شدن مهتابی شده است.
		\end{itemize}
		
		
		
		\item [5.]
		\textbf{فرمول ۱}\\
		در فصل ۲۰۲۱-۲۰۲۲ مسابقات اتومبیل‌رانی فرمول ۱، در مسابقه آخر که قرار بود قهرمان از بین \lr{Lewis Hamilton} و \lr{Max Verstappen} انتخاب شود، در دور نهایی مسابقه، هنگامی که \lr{Lewis Hamilton} از رقیبش پیش بود، ناگهان سرعت اتومبیل کاهش پیدا می‌کند و غریب از او پیشی می‌گیرد و درنهایت \lr{Hamilton} در جایگاه دوم قرار می‌گیرد و قهرمانی فصل را از دست می‌دهد.
		
		\begin{itemize}
			\item \lr{\textbf{:Fault}}\\
			با بررسی‌های فنی متوجه شدند که در دور پایانی مسابقه، اختلالی الکترونیکی در سیستم فرمان اتومبیل \lr{Hamilton} ایجاد شده است.
			
			\item \lr{\textbf{:Error}}\\
			در اتومبیل‌های \lr{F1} تعویض دنده به‌صورت دستی توسط راننده و با استفاده از فرمان انجام می‌شود و ازآنجایی‌که در سیستم فرمان اختلال ایجاد شده بوده است، راننده نمی‌توانسته دنده عوض کند و همین باعث کاهش سرعت وی شده است.
			
			\item \lr{\textbf{:Failure}}\\
			در نهایت \lr{Hamilton} جایگاه قهرمانی را از دست داد.
		\end{itemize}
		
	\end{enumerate}
	
\end{qsolve}