\documentclass[12pt]{article}
\usepackage{amsmath, amssymb, geometry}
\geometry{a4paper, margin=1in}

\begin{document}
	
	% Header Information
	\begin{center}
		\Large\textbf{Dependable System Design - Fall 2024} \\
		\large\textbf{Homework 12} \\
		\normalsize\textbf{Reza Adinepour}
	\end{center}
	
	\vspace{1cm}
	
	% Title
	\begin{center}
		\large\textbf{Conditional Reliability Analysis}\\
	\end{center}
	
	\section*{Problem Statement}
	Evaluate the conditional probability of the system being operational during the time interval between \([a, t]\), given that the system was functional at time 0, considering repairs. Repairs before time \(a\) are acceptable, but the system must remain operational during the interval \([a, t]\) with no repairs allowed.
	
	
	\section*{Definitions}
	\begin{itemize}
		\item \textbf{Reliability Function:} The reliability function, \(R(t)\), is the probability that the random time to failure \(T\) exceeds \(t\):
		\begin{equation}
			R(t) = P(T > t) = 1 - F(t),
		\end{equation}
		where \(F(t)\) is the cumulative distribution function (CDF).
		
		\item \textbf{Conditional Reliability Function:} The conditional reliability function, \(m(x)\), is the probability that the system survives an additional time \(x\), given it has already survived up to time \(t\):
		\begin{equation}
			m(x) = P(T > t + x \mid T > t) = \frac{R(t + x)}{R(t)}.
		\end{equation}
		
		\item \textbf{Final Conditional Probability:} The conditional probability of the system being operational during \([a, t]\), given it was functional at time 0:
		\begin{equation}
			P(T > t \text{ in } [a, t] \mid T > 0) = \frac{R(t)}{R(0)}.
		\end{equation}
	\end{itemize}
	
	\section*{Solution}
	\subsection*{1. Define the Required Probability}
	We need:
	\begin{equation}
		P(T > t \text{ in } [a, t] \mid T > 0) = P(T > t \mid T > a) \cdot P(T > a \mid T > 0).
	\end{equation}
	
	\subsection*{2. Use the Conditional Reliability Function}
	From the definition of the conditional reliability function:
	\begin{equation}
		P(T > t \mid T > a) = \frac{R(t)}{R(a)}.
	\end{equation}
	
	The probability of surviving up to \(a\), given survival at time 0, is:
	\begin{equation}
		P(T > a \mid T > 0) = \frac{R(a)}{R(0)}.
	\end{equation}
	
	\subsection*{3. Combine Results}
	Multiplying these probabilities:
	\begin{align}
		P(T > t \text{ in } [a, t] \mid T > 0) &= P(T > t \mid T > a) \cdot P(T > a \mid T > 0) \\
		&= \frac{R(t)}{R(a)} \cdot \frac{R(a)}{R(0)} \\
		&= \frac{R(t)}{R(0)}.
	\end{align}
	\newpage
	
	
	
	
	\section*{Numerical Example}
	Suppose the reliability function of a system is given by:
	\begin{equation}
		R(t) = e^{-\lambda t},
	\end{equation}
	where \(\lambda = 0.1\) is the failure rate. Let \(a = 5\) and \(t = 10\), and we want to calculate the conditional probability that the system is operational during \([5, 10]\) given it was operational at \(t = 0\).
	
	\subsection*{1. Compute \(R(t)\) and \(R(0)\)}
	\begin{align}
		R(0) &= e^{-0.1 \cdot 0} = 1, \\
		R(10) &= e^{-0.1 \cdot 10} = e^{-1} \approx 0.3679.
	\end{align}
	
	\subsection*{2. Compute the Conditional Probability}
	Using the formula:
	\begin{equation}
		P(T > t \text{ in } [a, t] \mid T > 0) = \frac{R(t)}{R(0)}.
	\end{equation}
	Substitute the values:
	\begin{align}
		P(T > 10 \text{ in } [5, 10] \mid T > 0) &= \frac{R(10)}{R(0)} \\
		&= \frac{0.3679}{1} \\
		&= 0.3679.
	\end{align}
	
	Thus, the conditional probability that the system remains operational during \([5, 10]\), given it was operational at \(t = 0\), is approximately \(36.79\%\).
	
	
\end{document}
