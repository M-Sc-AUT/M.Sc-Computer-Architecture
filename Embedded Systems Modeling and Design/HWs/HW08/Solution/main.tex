\documentclass[12pt]{article}
\usepackage{blindtext}
\usepackage[en,bordered]{uni-style}
\usepackage{uni-math}
\usepackage{graphicx,wrapfig}
\usepackage{amsmath}
\usepackage{amssymb}
\usepackage{array}
\usepackage{enumitem}



\title{\href{https://github.com/M-Sc-AUT/M.Sc-Computer-Architecture/tree/main/Embedded Systems Modeling and Design}{\textcolor{black}{Embedded Systems}}}
\prof{\href{https://scholar.google.com/citations?user=2RN0Y2YAAAAJ&hl=en}{\textcolor{black}{Prof. Sedighi}}}
\subtitle{Chapter 13 - Invariants and Temporal Logic}
\subject{Homework 8}
\info{
    \begin{tabular}{lr}
        \href{https://github.com/M-Sc-AUT/M.Sc-Computer-Architecture/tree/main/Embedded Systems Modeling and Design}{\textcolor{black}{Reza Adinepour}} & ID: 402131055\\
    \end{tabular}
    }
    \date{\today}
    % \usepackage{xepersian}
    % \settextfont{Yas}
    \usepackage{uni-code}
    
    \begin{document}
\maketitlepage
\maketitlestart







\section{Question 2}
Consider the following state machine:

\begin{center}
	\includegraphics*[width=0.4\linewidth]{images/img1}
	\captionof{figure}{State machine of Q2}
\end{center}

(Recall that the dashed line represents a default transition.) For each of the fol
lowing LTL formulas, determine whether it is true or false, and if it is false, give a
counterexample:

\begin{enumerate}
	\item[(a)] \(x \implies \mathbf{Fb}\)
	\item[(b)] \(\mathbf{G}(x \implies \mathbf{F}(y = 1))\)
	\item[(c)] \((\mathbf{G}x) \implies \mathbf{F}(y = 1)\)
	\item[(d)] \((\mathbf{G}x) \implies \mathbf{GF}(y = 1)\)
	\item[(e)] \(\mathbf{G}((b \land \neg x) \implies \mathbf{FG}c)\)
	\item[(f)] \(\mathbf{G}((b \land \neg x) \implies \mathbf{G}c)\)
	\item[(g)] \((\mathbf{GF}\neg x) \implies \mathbf{FG}c\)
\end{enumerate}


\begin{qsolve}
	(Recall that the dashed line represents a default transition.) For each of the following LTL formulas, determine whether it is true or false, and if it is false, give a counterexample:
	
	
	
	\begin{tabular}{rlcl}
		(a) & \(x \implies \mathbf{Fb}\) & & T \\
		(b) & \(\mathbf{G}(x \implies \mathbf{F}(y = 1))\) & & F \\
		(c) & \((\mathbf{G}x) \implies \mathbf{F}(y = 1)\) & & F \\
		(d) & \((\mathbf{G}x) \implies \mathbf{GF}(y = 1)\) & & F \\
		(e) & \(\mathbf{G}((b \land \neg x) \implies \mathbf{FG}c)\) & & T \\
		(f) & \(\mathbf{G}((b \land \neg x) \implies \mathbf{G}c)\) & & F \\
		(g) & \((\mathbf{GF}\neg x) \implies \mathbf{FG}c\) & & F \\
	\end{tabular}
	
	\vspace{1cm}
	
	\begin{enumerate}[label=(\alph*)]
		\item 
		\[
		\begin{array}{c}
			x / 1 \quad \text{True} \\
			a \implies b \implies c
		\end{array}
		\]
		\item 
		\[
		\begin{array}{c}
			x / 1 \quad \text{True} \\
			a \implies b \implies c
		\end{array}
		\]
		\item 
		\[
		\begin{array}{c}
			x / 1 \quad \text{True} \\
			a \implies b \implies c
		\end{array}
		\]
		\item 
		\[
		\begin{array}{c}
			x / 1 \quad \text{True} \\
			a \implies b \implies c
		\end{array}
		\]
		\item 
		\[
		\begin{array}{c}
			b \land \neg x / \\
			b \implies b \ldots
		\end{array}
		\]
		\item 
		\[
		\begin{array}{c}
			\neg x / \\
			a \implies a \ldots
		\end{array}
		\]
	\end{enumerate}
\end{qsolve}

\vfil
\clearpage








\section{Question 4}
This problem is concerned with specifying in linear temporal logic tasks to be per
formed by a robot. Suppose the robot must visit a set of n locations $l_1, l_2, ..., l_n$ .
Let $p_i$ be an atomic formula that is \textit{t}rue if and only if the robot visits location $l_i$.

Give LTL formulas specifying the following tasks:

\begin{enumerate}
	\item[(a)] The robot must eventually visit at least one of the $n$ locations.
	\item[(b)] The robot must eventually visit all $n$ locations, but in any order.
	\item[(c)] The robot must eventually visit all n locations, in the order $l_1, l_2,..., l_n$.
\end{enumerate}

\begin{qsolve}
	\begin{enumerate}
		\item[(a)] 
		$\mathbf{F}p_1 \lor \mathbf{F}p_2 \lor \mathbf{F}p_3 \lor \ldots \lor \mathbf{F}p_n$
		
		\item 
		$\mathbf{F}p_1 \land \mathbf{F}p_2 \land \mathbf{F}p_3 \land \ldots \land \mathbf{F}p_n$
		
		\item 
		$\mathbf{F}(p_n \land \ldots \mathbf{F}(p_3 \land \mathbf{F}(p_2 \land \mathbf{F}p_1)))$
	\end{enumerate}
\end{qsolve}

\vfil







\vspace*{\fill}
\begin{center}
	\makeendpage

\end{center}
\vfill % equivalent to \vspace{\fill}
\clearpage




\end{document}