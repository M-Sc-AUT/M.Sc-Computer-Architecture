\documentclass[12pt]{article}
\usepackage{blindtext}
\usepackage[en,bordered]{uni-style}
\usepackage{uni-math}
\usepackage{graphicx,wrapfig}
\usepackage{amsmath}
\usepackage{amssymb}
\usepackage{array}
\usepackage{algorithm}
\usepackage{algpseudocode}
\usepackage{enumitem}



\title{\href{https://github.com/M-Sc-AUT/M.Sc-Computer-Architecture/tree/main/Embedded Systems Modeling and Design}{\textcolor{black}{Embedded Systems}}}
\prof{\href{https://scholar.google.com/citations?user=2RN0Y2YAAAAJ&hl=en}{\textcolor{black}{Prof. Sedighi}}}
\subtitle{Chapter 16 - Quantitative Analysis}
\subject{Homework 11}
\info{
    \begin{tabular}{lr}
        \href{https://github.com/M-Sc-AUT/M.Sc-Computer-Architecture/tree/main/Embedded Systems Modeling and Design}{\textcolor{black}{Reza Adinepour}} & ID: 402131055\\
    \end{tabular}
    }
    \date{\today}
    % \usepackage{xepersian}
    % \settextfont{Yas}
    \usepackage{uni-code}
    
    \begin{document}
\maketitlepage
\maketitlestart







\section{Question 2}
Consider the program given below:

\begin{lstlisting}[language=C, keywordstyle=\color{blue}\bfseries, commentstyle=\color{green}, basicstyle=\ttfamily\small, numbers=left, numberstyle=\tiny, stepnumber=1, numbersep=5pt]
void testFn(int *x, int flag) 
{
	while (flag != 1) 
	{
		flag = 1;
		*x = flag;
	}
	if (*x > 0)
		*x += 2;
}
\end{lstlisting}

In answering the questions below, assume that \texttt{x} is not \texttt{NULL}.

\begin{enumerate}
	\item [(a)]
	Draw the control-flow graph of this program. Identify the basic blocks with
	unique IDs starting with 1.
	\begin{qsolve}
		\begin{center}
			\includegraphics*[width=0.2\linewidth]{images/Q2.pdf}
			\captionof{figure}{Control-flow graph of Q3}
		\end{center}
	\end{qsolve}
	
	Note that we have added a dummy source node, numbered 0, to represent the
	entry to the function. For convenience, we have also introduced a dummy sink
	node, although this is not strictly required.
	
	\item [(b)]
	Is there a bound on the number of iterations of the while loop? Justify your
	answer.
	\begin{qsolve}
		Yes, given that if the condition is true, \texttt{flag} will be set to 1, the \texttt{while} loop will therefore iterate at most once.
	\end{qsolve}
	
	
	\item [(c)]
	How many total paths does this program have? How many of them are feasi
	ble, and why?
	\begin{qsolve}
		"The program has 4 paths (two for the while loop and two for the if statement). If the program enters the while loop, the value of \texttt{*x} will be set to 1, thus one of the if conditions after the while loop will not be reachable. Therefore, it can be concluded that there are 3 reachable states."
	\end{qsolve}
	
	
	\item [(d)]
	Write down the system of flow constraints, including any logical flow con
	straints, for the control-flow graph of this program.
	\begin{qsolve}
			$$ x_0 = 1 $$
			$$x_1 = 2$$
			$$x_1 = d_{12} + d_{13}$$
			$$x_2 = 1$$
			$$x_2 = d_{12} = d_{21}$$
			$$x_3 = d_{13} = d_{34} + d_{35}$$
			$$x_4 = d_{34} = d_{45}$$
			$$x_5 = d_{35} + d_{45}$$
	\end{qsolve}
	
	
	\item [(e)]
	Consider running this program uninterrupted on a platform with a data cache.
	Assume that the data pointed to by \texttt{x} is not present in the cache at the start of
	this function.
	
	For each read/write access to \texttt{*x}, argue whether it will be a cache hit or miss.
	Now, assume that \texttt{*x} is present in the cache at the start of this function. Iden
	tify the basic blocks whose execution time will be impacted by this modified
	assumption.
	\begin{qsolve}
		The topic of write-allocate versus no-write-allocate is not mentioned, so the more common write-allocate is considered. In this case, if the while loop executes, the first miss will occur at \texttt{ID=2}, and after that, it will be a hit. If the while loop does not execute, the first miss will occur at \texttt{ID=3}, and after that, it will be a hit. If a block containing \texttt{*x} is in memory, the two mentioned \texttt{IDs} will execute faster.
	\end{qsolve}
\end{enumerate}
\vfil
\clearpage






\section{Question 3}
Consider the function \texttt{check\_password} given below that takes two arguments: a
user ID uid and candidate password \texttt{pwd} (both modeled as \texttt{ints} for simplicity).
This function checks that password against a list of user IDs and passwords stored
in an array, returning 1 if the password matches and 0 otherwise.





\begin{enumerate}
	\item [(a)]
	Draw the control-flow graph of the function \texttt{check\_password}. State the num
	ber of nodes (basic blocks) in the CFG. (Remember that each conditional
	statement is considered a single basic block by itself.)
	
	Also state the number of paths from entry point to exit point (ignore path
	feasibility).
	
	\begin{qsolve}
		
	\end{qsolve}
	
	
	
	
	\item [(b)]
	Suppose the array \texttt{all\_pwds} is sorted based on passwords (either increasing
	or decreasing order). In this question, we explore if an external client that calls
	\texttt{check\_password} can \textit{infer anything about the passwords} stored in \texttt{all\_pwds}
	by repeatedly calling it and \textit{recording the execution time} of \texttt{check\_ password}.
	Figuring out secret data from “physical” information, such as running time, is
	known as a \textit{side-channel attack}.
	In each of the following two cases, what, if anything, can the client infer about
	the passwords in \texttt{all\_pwds}?
	\begin{itemize}
		\item The client has exactly one (uid, password) pair present in \texttt{all\_pwds}
		\item The client has NO (uid, password) pairs present in in \texttt{all\_pwds}
	\end{itemize}
	Assume that the client knows the program but not the contents of the array
	\texttt{all\_pwds}
\end{enumerate}




\begin{qsolve}

\end{qsolve}









\vspace*{\fill}
\begin{center}
	\makeendpage

\end{center}
\vfill % equivalent to \vspace{\fill}
\clearpage




\end{document}