\documentclass[12pt]{article}
\usepackage{blindtext}
\usepackage[en,bordered]{uni-style}
\usepackage{uni-math}
\usepackage{graphicx,wrapfig}
\usepackage{amsmath}
\usepackage{amssymb}
\usepackage{array}
\usepackage{enumitem}



\title{\href{https://github.com/M-Sc-AUT/M.Sc-Computer-Architecture/tree/main/Embedded Systems Modeling and Design}{\textcolor{black}{Embedded Systems}}}
\prof{\href{https://scholar.google.com/citations?user=2RN0Y2YAAAAJ&hl=en}{\textcolor{black}{Prof. Sedighi}}}
\subtitle{Chapter 14 - Equivalence and Refinement}
\subject{Homework 9}
\info{
    \begin{tabular}{lr}
        \href{https://github.com/M-Sc-AUT/M.Sc-Computer-Architecture/tree/main/Embedded Systems Modeling and Design}{\textcolor{black}{Reza Adinepour}} & ID: 402131055\\
    \end{tabular}
    }
    \date{\today}
    % \usepackage{xepersian}
    % \settextfont{Yas}
    \usepackage{uni-code}
    
    \begin{document}
\maketitlepage
\maketitlestart







\section{Question 1}
In Figure 14.6 are four pairs of actors. For each pair, determine whether

\begin{itemize}
	\item \textit{A} and \textit{B} are type equivalent,
	\item \textit{A} is a type refinement of \textit{B},
	\item \textit{B} is a type refinement of \textit{A}, or
	\item none of the above.
\end{itemize}

\begin{center}
	\includegraphics*[width=0.6\linewidth]{images/img1}
	\captionof{figure}{Four pairs of actors whose type refinement relationships are explored in Exercise 1}
\end{center}


\begin{qsolve}
	\begin{enumerate}
		\item [(a)] B is a type refinement of A
		\item [(b)] B is a type refinement of A
		\item [(c)] None
		\item [(d)] B is a type refinement of A
	\end{enumerate}
	
\end{qsolve}
\vfil
\clearpage








\section{Question 3}
The state machine in Figure 14.7 has the property that it outputs at least one 1
between any two 0’s. Construct a two-state nondeterministic state machine that
simulates this one and preserves that property. Give the simulation relation. Are
the machines bisimilar?



\begin{qsolve}
	The simulation relation for the machine created in the figure below can be written as follows:
	
	$$ \{ (0, a), (1, b), (2, b), (3, b) \} $$
	
	\begin{center}
		\includegraphics*[width=0.6\linewidth]{images/Q3//Q3.pdf}
		\captionof{figure}{Solution of Q3}
	\end{center}
	
	The created machine has non-deterministic states that cannot be generated by machine 14.7. Therefore, the bisimilar relation does not hold.
	
\end{qsolve}
\vfil
\clearpage






\section{Question 5}
Consider the state machine in Figure 14.10. Find a bisimilar state machine with
only two states, and give the bisimulation relation.

\begin{center}
	\includegraphics*[width=0.5\linewidth]{images/img2}
	\captionof{figure}{A machine that has more states than it needs}
\end{center}


\begin{qsolve}
	We can write simulation relation as bellow:
	$$ S_{a,b}=\{ (A,AB), (B,CD), (C,AB), (D,CD)  \} $$
	
	\begin{center}
		\includegraphics*[width=0.5\linewidth]{images/Q5//Q5.pdf}
		\captionof{figure}{Solution of Q5}
	\end{center}
	
\end{qsolve}
\vfil
\clearpage






\vspace*{\fill}
\begin{center}
	\makeendpage
	All of this figures, draw with \texttt{ipe}. You can download this software here:\\
	\href{https://ipe.otfried.org/}{\textcolor{magenta}{\texttt{ipe.otfried.org}}}

\end{center}
\vfill % equivalent to \vspace{\fill}
\clearpage




\end{document}