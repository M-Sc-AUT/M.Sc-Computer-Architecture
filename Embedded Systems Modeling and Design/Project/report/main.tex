
% ----------------------------------------------------------------------
% Set the document class
% ----------------------------------------------------------------------
\documentclass[12pt	]{article}
\usepackage{multirow}
\usepackage{matlab-prettifier}



% ----------------------------------------------------------------------
% Define external packages, language, margins, fonts, new commands 
% and colors
% ----------------------------------------------------------------------
\usepackage[utf8]{inputenc} % Codification
\usepackage[english]{babel} % Writing idiom

\usepackage[export]{adjustbox} % Align images
\usepackage{amsmath} % Extra commands for math mode
\usepackage{amssymb} % Mathematical symbols
\usepackage{anysize} % Personalize margins
    \marginsize{2cm}{2cm}{2cm}{2cm} % {left}{right}{above}{below}
\usepackage{appendix} % Appendices
\usepackage{cancel} % Expression cancellation
\usepackage{caption} % Captions

    \DeclareCaptionFont{newfont}{\fontfamily{cmss}\selectfont}
    \captionsetup{labelfont={bf, newfont}}
\usepackage{cite} % Citations, like [1 - 3]
\usepackage{color} % Text coloring
\usepackage{fancyhdr} % Head note and footnote
    \pagestyle{fancy}
    \fancyhf{}
    \fancyhead[L]{\footnotesize \fontfamily{cmss}\selectfont Embedded Systems} % Left of Head note
    \fancyhead[R]{\footnotesize \fontfamily{cmss}\selectfont CE5439} % Right of Head note
    \fancyfoot[L]{\footnotesize \fontfamily{cmss}\selectfont CE Dep.} % Left of Footnote
    \fancyfoot[C]{\thepage} % Center of Footnote
    \fancyfoot[R]{\footnotesize \fontfamily{cmss}\selectfont AUT} % Right of Footnote
    \renewcommand{\footrulewidth}{0.4pt} % Footnote rule
\usepackage{float} % Utilization of [H] in figures
\usepackage{graphicx} % Figures in LaTeX
\usepackage[colorlinks = true, plainpages = true, linkcolor = blue, urlcolor = blue, citecolor = blue, anchorcolor = blue]{hyperref}
\usepackage{indentfirst} % First paragraph
\usepackage[super]{nth} % Superscripts
\usepackage{siunitx} % SI units
\usepackage{subcaption} % Subfigures
\usepackage{titlesec} % Font
    \titleformat{\section}{\fontfamily{cmss}\selectfont\Large\bfseries}{\thesection}{1em}{}
    \titleformat{\subsection}{\fontfamily{cmss}\selectfont\large\bfseries}{\thesubsection}{1em}{}
    \titleformat{\subsubsection}{\fontfamily{cmss}\selectfont\normalsize\bfseries}{\thesubsubsection}{1em}{}
    \fancyfoot[C]{\fontfamily{cmss}\selectfont\thepage}

% Random text (not needed)
\usepackage{lipsum}
\usepackage{duckuments}

% New and re-newcommands
\newcommand{\sen}{\operatorname{\sen}} % Sine function definition
\newcommand{\HRule}{\rule{\linewidth}{0.5mm}} % Specific rule definition
\renewcommand{\appendixpagename}{\LARGE \fontfamily{cmss}\selectfont Appendices}

% Colors
\definecolor{istblue}{RGB}{3, 171, 230}
\definecolor{dkgreen}{rgb}{0,0.6,0}
\definecolor{gray}{rgb}{0.5,0.5,0.5}

% Image path
\graphicspath{ {./Images/} }

\usepackage[most]{tcolorbox}

%%%%%%%%%%%%%%%%%%%%%%%%%%%%%%%%%%%%%%%%%% Solution box setting %%%%%%%%%%%%%%%%%%%%%%%%%%%%%%%%%%%%%%%%%%
\newtcbtheorem{Problem}{\bfseries Problem}{enhanced,drop shadow={black!50!white},
	coltitle=black,
	top=0.3in,
	attach boxed title to top left=
	{xshift=1.5em,yshift=-\tcboxedtitleheight/2},
	boxed title style={size=small,colback=pink}
}{summary}

\newtcolorbox[auto counter]{summary}[1][]{title={\bfseries Problem~\thetcbcounter},enhanced,drop shadow={black!50!white},
	coltitle=black,
	top=0.3in,
	attach boxed title to top left=
	{xshift=1.5em,yshift=-\tcboxedtitleheight/2},
	boxed title style={size=small,colback=pink},#1}
	
%%%%%%%%%%%%%%%%%%%%%%%%%%%%%%%%%%%%%%%%%%%%%%%%%%%%%%%%%%%%%%%%%%%%%%%%
%                                 Document                             %
%%%%%%%%%%%%%%%%%%%%%%%%%%%%%%%%%%%%%%%%%%%%%%%%%%%%%%%%%%%%%%%%%%%%%%%%
\begin{document}

% ----------------------------------------------------------------------
% Cover
% ----------------------------------------------------------------------
\begin{center}
    \begin{figure}
        \vspace{-1.0cm}
        \centering
        \includegraphics[scale = 0.35]{Images/AUT_logo.png} % IST logo
    \end{figure}
    \mbox{}\\[2.0cm]
    \textsc{\Huge \textbf{Embedded Systems Modeling and Design}}\\[1.0cm]
    \textsc{\LARGE Instructor: \href{https://scholar.google.com/citations?user=2RN0Y2YAAAAJ&hl=en}{\textcolor{black}{Prof. Mehdi Sedighi}}}\\[2.5cm]
    \textsc{\LARGE Amirkabir University of Technology} \\%\\[1.0cm]
    \textsc{(Tehran polytechnic)}
    \HRule\\[0.4cm]
    {\large \bf {\fontfamily{cmss}\selectfont Design and Modeling of an Intelligent Automotive Airbag System \& Petri Net-Based Modeling of an Elevator System} }\\[0.2cm]
    \HRule\\[1.5cm]
\end{center}

\begin{flushleft}
    \textbf{\fontfamily{cmss}\selectfont Authors:}
\end{flushleft}

\begin{center}
    \begin{minipage}{0.5\textwidth}
        \begin{flushleft}
            \href{https://rezaadinepour.github.io/}{\textcolor{black}{Reza Adinepour}}\\
        \end{flushleft}
    \end{minipage}%
    \begin{minipage}{0.5\textwidth}
        \begin{flushright}
            \href{mailto:adinepour@aut.ac.ir}{\texttt{adinepour@aut.ac.ir}}
        \end{flushright}
    \end{minipage}
\end{center}

\vspace{1em}

    
\begin{center}
    \bigskip \bigskip \bigskip \bigskip
    \large \bf \fontfamily{cmss}\selectfont Spring 2024
\end{center}

\thispagestyle{empty}

\setcounter{page}{0}

\newpage

% ----------------------------------------------------------------------
% Contents
% ----------------------------------------------------------------------
\tableofcontents

\newpage

% ----------------------------------------------------------------------
% Body
% ----------------------------------------------------------------------

% ------------------Section 1--------------------
\section{Airbag System}
Imagine you want to design a smart car airbag system. This system consists of 4 impact sensors located at the front, rear, right, and left sides of the vehicle, 2 airbags placed at the front and left side for the driver, a speed sensor, a movement direction sensor, and a driver distance sensor, each of which will be described further.





\subsection{Project Description}
The front sensor activates if the vehicle's speed exceeds 40 $\frac{Km}{h}$ and, in the event of a collision risk, inflates the airbag in front of the driver. The rear sensor activates if the vehicle is moving forward at less than \textit{30} $\frac{Km}{h}$ or moving backward at more than \textit{10} $\frac{Km}{h}$, inflating the airbag in front of the driver. The side sensors are not dependent on speed and, in the event of a collision risk from either side of the vehicle, inflate the left side airbag for the driver. If more than one sensor is activated, it is possible for both airbags to inflate, but in such a case, priority will clearly be given to the front airbag for the driver.

Additionally, there is another sensor that measures the distance between the driver and the steering wheel. If this distance is less than \textit{30 cm}, the airbag should only inflate halfway to avoid harming the driver. This must be completed within \textit{30 ms} after detecting an imminent collision. If the distance is more than \textit{30 cm} but less than \textit{40 cm}, the airbag should inflate to $\frac{3}{4}$ of its capacity within \textit{40 ms}. If the distance is more than \textit{40 cm}, the airbag should fully inflate within \textit{50 ms}. The side airbag should fully inflate within \textit{60 ms}.






\subsection{Project Detail}
\begin{enumerate}
	\item
	First, choose one of the types of MoCs (Models of Computation) covered in this course that you think is most suitable for describing, modeling, implementing, and evaluating this system. Fully explain your reasons for your choice. Note that this question does not have a single correct answer. Therefore, your answer will be evaluated based on the validity of your reasoning and logic, not on a specific correct answer. 
	
	\item 
	Now, design and describe this system based on the MoC you chose in the previous step. 
	
	\item 
	The next step is always modeling and then implementation. For modeling, assume that this system will be implemented on a processor that, in addition to controlling the airbags, also controls the interior climate of the car and the lights inside and outside the vehicle. Naturally, for regulating the interior temperature, a thermal sensor is required to measure the temperature inside the vehicle and adjust the airflow temperature based on the driver's desired temperature. Therefore, the same driver distance sensor used for airbags can be used for regulating the intensity of warm or cold air inside the vehicle. Do these new assumptions affect your choice in step 1? Explain. If needed, revise the description provided in step 2.
	
	\item 
	Based on the assumptions and considerations in step 3, what type of implementation would you propose for your designed system (e.g., process-based, thread-based, interrupt-based)? Fully explain your reasons for your choice. Also, draw a flowchart of your proposed implementation. 
	
	\item 
	Estimate the WCET (Worst-Case Execution Time) for your program. For this, assume that each access to a sensor or actuator takes 1 ms and each atomic instruction takes 10 nanoseconds. To avoid bouncing, each sensor is read at least 3 times and at most 5 times, with the first 3 consistent readings being selected. Other tasks that the processor needs to handle will take up to 90\% of the processor's computational power. (If you think more assumptions are needed to solve this problem, specify them and proceed with solving the problem.)
	
	\item 
	Can you guarantee that your design is real-time? In other words, can you ensure that the timing constraints mentioned above will be met? Provide a comprehensive explanation. 
\end{enumerate}






\subsection{Choose The Best MoC}








\subsection{Design and Describe Model}
\subsection{implementation}
\subsection{WCET Estimation}
\subsection{Real-time Checking}











\newpage

\section{Elevator System}


\subsection{Project Description}
You have probably had a frustrating experience using the elevator in the east wing of the faculty. The goal in designing this elevator was that by pressing a button again (either on the floors or in the cabin), the command to move to that floor would be canceled. However, the timing of this second press (whether before starting to move towards that floor or during the movement) was not properly considered. Additionally, the designer intended that if the elevator could not move, the call request would be canceled (timeout). Unfortunately, this part of the system does not function correctly in practice either.



\subsection{Project Detail}
In this part of the project, provide a complete description of the correct behavior of the faculty's elevator using a Petri net. In other words, suppose that while retaining the features of the existing elevator, you want to provide an accurate description to fix its current issues. You can base your description on a simple elevator system using a Petri net that you have seen in the course. However, keep in mind that the number of floors in this problem is known (4 floors in the faculty), and the current features of the elevator (including cancellation by pressing again and timeout) must remain. For simplicity, there is no need to perform a qualitative analysis and evaluation of the description. However, if you do perform this analysis and evaluation, you will receive additional points.

Note: \textcolor{red}{This provides a good opportunity for students who did not manage to earn good marks in other parts of the course to make up for this weakness through the project's bonus sections.}












\newpage
% ----------------------------------------------------------------------
% References
% ----------------------------------------------------------------------
\bibliographystyle{plain}
\bibliography{refs}

\end{document}