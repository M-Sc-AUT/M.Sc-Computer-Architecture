\فصل{نتیجه‌گیری و جمع‌بندی}

با توجه به مقالات بررسی شده، و همچنین با در نظر گرفتن سرعت بالای نفوذ ابزارهای \lr{IoT} به زندگی انسان که در اکثر موارد به منبع توان پایدار و دسترسی نداشته‌اند، استفاده از روش‌های مختلف در جهت کاهش توان مصرفی سیستم‌های پردازشی و غیر پردازشی در ابزارهای \lr{IoT} الزامی به نظر می‌رسد. در این میان پیشرفت‌های فراوانی در حوزه مخابرات کم‌ توان و همچنین ساخت سیستم‌های پردازشی با میزان توان استاتیک پایین انجام شده است. اما در خصوص نحوه انجام محاسبات و پردازش‌های داخل سیستم، با ظهور \lr{AI} و نیاز به انجام برخی پردازش‌های سنگین، در داخل پردازنده با توان پایین، نیاز به اصلاح روش‌های کنونی پردازش احساس می‌گردد. در برخی موارد این امر با اضافه نمودن بخش‌های شتاب‌دهنده اختصاصی و یا همه منظوره در داخل تراشه انجام پذیرفته است اما بازهم فاصله زیادی با توان هدف برای انجام یک پردازش مشخص وجود دارد. در این خصوص برخی پژوهش‌ها در زمینه سیستم‌های \lr{Event-Driven} در جریان است که بتوان با کمک آن‌ها توان مصرفی را تا حد ممکن کاهش داد. در خصوص منابع توان و همچنین ذخیره‌سازی توان نیز پیشرفت‌هایی صورت گرفته است، اما سرعت این پیشرفت‌ها و همچنین هزینه اجرای انجام آن‌ها هنوز تا استفاده نهایی در محصولات تجاری فاصله زیادی دارد و اکنون اکثر محصولات روزمره با توان پردازشی بالا نیازمند شارژ روزانه هستند. ترکیب ایجاد پردازش‌های کم‌ مصرف و قدرت پردازش و روش‌های انتقال اطلاعات کم‌مصرف و با قابلیت اطمینان بالا، منابع توان با ظرفیت مناسب و قابلیت برداشت توان از محیط راه را برای ایجاد سیستم‌های \lr{IoT} هوشمندتر بدون نیاز به شارژ مجدد و با قابلیت کارکرد تا چندین سال بدون نیاز به تعمیر و یا تعویض را فراهم می‌آورد. در صورت وجود این موارد زندگی بشر با کمک این لوازم متحول خواهد شد و راه برای استفاده بهتر از منابع در دسترس بشر هموارتر خواهد شد.