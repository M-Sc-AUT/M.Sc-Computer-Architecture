
\فصل{مقدمه}\label{فصل۱:مقدمه}


\قسمت{تعریف مسئله}
ابزارهای \lr{IoT}\پانویس{Internet of Things} و تکنولوژی‌های وابسته به آن در حال پیشرفت سریع و ورود به زندگی روزمره بشر هستند. سرعت این امر به قدری بالا است که در آینده نزدیک تقریباً تمامی لوازم به شبکه اینترنت متصل خواهند بود و مفهومی با نام \lr{IoET}\پانویس{Internet of Every Things} در زندگی بشر ایجاد خواهد شد. این مسئله در شکلی زیر نمایش داده شده است.

\شروع{شکل}[ht]
\centerimg{img1}{11cm}
\شرح{برآورد توان در سیستم‌های \lr{IoT} \مرجع{Capra2019}}
\برچسب{شکل:برآورد توان در سیستم‌های iot}
\پایان{شکل}



نیازهای کلی این لوازم از دیدگاه طراحی از زوایای مختلف قابل بررسی می‌باشند، اما به طور کلی می‌توان موارد زیر را به صورت خلاصه بیان کرد:

\شروع{فقرات}

\فقره سیستم پردازش
\فقره روش‌های انتقال اطلاعات
\فقره تامین توان مورد نیاز

\پایان{فقرات}


در تمامی موارد ذکر شده استفاده از روش‌هایی جهت بهینه سازی در راستای افزایش کارایی و در دسترس بودن سیستم انجام پذیرفته است. این موضوع به دلیل رشد کندتر قطعات با قابلیت ذخیره انرژی مانند ابرخازن‌ها\پانویس{‫‪Supercapacitor‬‬} و باتری‌ها با سرعت کمتری انجام شده است. لذا یکی از مهمترین مسائل در سیستم‌های \lr{IoT} خصوصاً نمونه‌های بدون دسترسی مستقیم به شبکه برق، تامین پایدار توان مصرفی آن‌ها می‌باشد. این موضوع از جهات دیگری نیز قابل بررسی است، به عنوان مثال با رشد کاربرد سیستم‌های \lr{IoT} و کاربرد وسیع آن‌ها، در صورت وجود توان مصرفی بالا و نیاز به تعویض سریع باتری‌ها، مشکلات تولیدی و زیست محیطی فراوانی ایجاد خواهد گردید. همچنین قابلیت اطمینان چنین سیستم‌هایی به دلیل مشکل تامین توان پایدار مورد نیاز بسیار پایین خواهد بود.



\قسمت{اهمیت پژوهش}
بدون شک، بحث توان در سیستم‌های \lr{IoT} از اهمیت ویژه‌ای برخوردار است. با توجه به رشد روزافزون فناوری‌های اینترنت اشیا و نیاز مبرم به دستگاه‌های کم‌مصرف\پانویس{Low Power} و خودمختار\پانویس{Autonomous}، استفاده از منابع انرژی محیطی برای تأمین انرژی این دستگاه‌ها نقش حیاتی دارد. این امر نه تنها به کاهش هزینه‌های عملیاتی و افزایش طول عمر مفید\پانویس{Remaining Useful Life} شبکه‌های حسگر بی‌سیم کمک می‌کند، بلکه باعث کاهش اثرات زیست‌محیطی ناشی از استفاده از باتری‌های سنتی می‌شود. پژوهش در این زمینه می‌تواند به توسعه راهکارهای نوآورانه برای افزایش بهره‌وری انرژی، بهبود پایداری و کارایی سیستم‌های \lr{IoT} و در نهایت ارتقای کیفیت زندگی انسان‌ها منجر شود.



\قسمت{اهداف پژوهش}
در این نوشته سعی می‌گردد که در ابتدا مسائل موجود در سیستم‌های \lr{IoT} که مرتبط با توان مصرفی هستند مورد بررسی کوتاهی قرار گیرد و سپس راه‌حل های موجود برای هر مورد معرفی گردند. سپس به مسئله اصلی تامین توان مصرفی سیستم‌های \lr{IoT} و قابل حمل با استفاده از تکنیک‌های برداشت انرژی از محیط پرداخته می‌شود و با مقایسه روش‌های موجود و بهره‌وری هر یک نتایج حاصله ارائه می‌گردد. در انتها نیز به چند روش جدیدتر تامین توان با استفاده از برداشت انرژی از محیط پرداخته می‌شود. برخی راهکارهای پیشنهادی و نمونه‌های عملی حاصل از تحقیق در این خصوص نیز ارائه می‌گردد.


	
\قسمت{ساختار پژوهش}
اینن پژوهش در ۵ فصل انجام شده است. در فصل \ref{فصل۱:مقدمه} به مقدمه و اهمیت موضوع پژوهش پرداخته شده است. در فصل \ref{فصل۲:مفاهیم اولیه} به مفاهیم اولیه و پیش‌نیاز ها پرداخته شده است. در ادامه در فصل \ref{کارهای پیشین} پژوهش به بررسی کار‌های پیشین انجام شده در این زمینه پرداخت شده است. در فصل \ref{بررسی و مقایسه مقالات} به بررسی دقیق و جزئی مقالات مطالعه شده در این پژوهش پرداخته شده است و در فصل پایانی، جمع‌بندی و نتیجه گیری پژوهش ارائه شده است.