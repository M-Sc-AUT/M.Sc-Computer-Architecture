\فصل{برداشت توان از محیط}\label{برداشت توان از محیط}

\قسمت{مقدمه}
یک روش برای تامین توان سیستم‌های \lr{IoT} غیر متصل به شبکه برق استفاده از سیستم برداشت توان از محیط می‌باشد. در این روش با توجه به وجود انرژی غیر الکتریکی در محیط و تبدیل این انرژی به روش‌های گوناگون به انرژی الکتریکی، می‌توان یک سیستم \lr{IoT} بدون نیاز به تغذیه را طراحی نمود.

در تمامی سیستم‌های با قابلیت برداشت توان از محیط، یک بخش مربوط به مدیریت توان و تبدیل سطوح وجود دارد. طراحی این قسمت باتوجه به نوع سیستم‌ برداشت توان از محیط و سطح ولتاژ و جریان آن و همچنین توان مورد نیاز سیستم استفاده کننده توان متفاوت است اما به طور کلی می‌توان چنین سیستمی را مطابق شکل «\رجوع{شکل:بلوک دیاگرام یک سیستم برداشت توان از محیط}» نمایش داد:


\شروع{شکل}[ht]
\centerimg{img11}{10cm}
\شرح{بلوک دیاگرام یک سیستم برداشت توان از محیط \مرجع{Elahi2020}}
\برچسب{شکل:بلوک دیاگرام یک سیستم برداشت توان از محیط}
\پایان{شکل}


طبق شکل «\رجوع{شکل:بلوک دیاگرام یک سیستم برداشت توان از محیط}» با توجه به اینکه سطح ولتاژ دریافتی از محیط و پایداری آن معمولاً مناسب تغذیه مدار پردازنده و باقی قسمت‌های مدار نیست در ابتدا با استفاده از یک مبدل \lr{DC/DC} سطح ولتاژ به سطح مناسب تبدیل می‌گردد و سپس توسط یک رگلاتور ولتاژ، به ولتاژ پایدار و مناسب برای استفاده باقی مدار تبدیل می‌گردد. \رجوع{Grossi2021}

در میان قسمت مبدل \lr{DC/DC} و رگلاتور، ممکن است یک عنصر قابل شارژ با ظرفیت بالا وجود داشته باشد (تقریباً در تمامی سیستم‌ها یک خازن برای پایدارسازی در برابر تغییرات وجود دارد)، که ممکن است از نوع باری قابل شارژ و یا ابرخازن باشد. با توجه به وجود یا عدم وجود این بخش دو دسته

\شروع{شمارش}
\فقره دارای عنصر ذخیره‌ساز 
\فقره بدون عنصر ذخیره‌ساز 
\پایان{شمارش}

در سیستم‌های برداشت توان از محیط تعریف می‌گردند. سیستم‌های بدون عنصر ذخیره‌ساز در کاربردهای بسیار کم‌توان ‌زمان و یا قیمت کم مورد استفاده قرار می‌گیرند.

در صورت برداشت توان به صورت \lr{AC} از محیط بایستی توجه داشت که در قسمت یکسوساز ورودی سیستم با توجه به نوع روش برداشت، پهنای باند سیگنال آن از چند هرتز تا چندین گیگاهرتز می‌تواند باشد و لذا برای هر مدل برداشت می‌بایست طراحی منحصربه فردی برای قسمت یکسوساز وجود داشته باشد. \رجوع{Grossi2021}


به عنوان مثال در برداشت توان از محیط با استفاده از حرکت افراد، فرکانس ولتاژ ایجاد شده در حد چند هرتز می‌باشد اما در خصوص برداشت توان از سیستم \lr{RF} این فرکانس برابر با چندین گیگاهرتز می‌باشد.


برخی از روش‌ها در برداشت توان از محیط ولتاژ را به صورت \lr{DC} ایجاد می‌نماید که در این حالت احتیاجی به قسمت یکسوساز ورودی نمی‌باشد. به عنوان مثال می‌توان به پنل‌های خورشیدی\پانویس{Solar Panel} اشاره نمود که خروجی آن‌ها به صورت \lr{DC} می‌باشد.


در خصوص قسمت \lr{DC/DC} انواع مختلفی از مبدل‌های سوئیچینگ\پانویس{Switching Regulator} وجود دارند که مفصلا در \رجوع{Briones2019} به آن‌ها اشاره شده است. در شکل «\رجوع{شکل:مبدل‌های سوئیچینگ}» انواع مختلف مبدل‌های سوئیچینگ آورده شده است که موارد \lr{Buck}، \lr{Boost} و \lr{Buck\_Boost} پرکاربردتر آنها هستند. به طور کلی می‌توان به این نکته اشاره کرد که مبدل‌های \lr{Buck} نسبت به مبدل‌های \lr{Boost} و \lr{Buck\_Boost} کارایی بالاتری را دارا می‌باشند و لذا بایستی در صورت امکان خروجی قسمت برداشت توان از محیط ولتاژ بالاتری را از ولتاژ مورد نیاز مدار داشته باشد تا بتوان با استفاده از مبدل \lr{Buck} از بهره‌وری بیشتری آن استفاده نمود.


\شروع{شکل}[ht]
\centerimg{img12}{6cm}
\شرح{مبدل‌های سوئیچینگ \مرجع{Briones2019}}
\برچسب{شکل:مبدل‌های سوئیچینگ}
\پایان{شکل}


استفاده از رگلاتورهای خطی\پانویس{Linear Regulator} معمول، به غیر از موارد با جریان نشتی بسیار پایین و مصرف کم مدار به دلیل اتلاف بالا توصیه نمی‌گردد.




\قسمت{روش‌های برداشت توان از محیط}
با توجه به محیط کاری یک سیستم و توان در دسترس از پارامترهای مختلف محیطی انواع مختلفی از روش‌های برداشت توان از محیط وجود دارد که هر یک مزایا و معایب مربوط به خود را دارا هستند و هیچ یک از روش‌ها به‌طور مطلق برتری کامل به روش‌های دیگر نداشته و هر یک بنا به شرایط کاری سیستم می‌توانند برای تامین تغذیه سیستم به کار روند.

منابع مختلف جهت برداشت توان از محیط در شکل زیر آورده شده‌اند:

\شروع{شکل}[ht]
\centerimg{img12}{6cm}
\شرح{منابع مختلف برداشت توان \مرجع{Maamer2019}}
\برچسب{شکل:منابع مختلف برداشت توان}
\پایان{شکل}

در جدول زیر یه مقایسه از پارامتر‌های مختلف هریک از روش‌ها گرد آوری شده است و برای هریک میزان چگالی توان و همچنین میزان توان معمول قابل برداشت آورده شده است:



