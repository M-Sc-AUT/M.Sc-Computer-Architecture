
\فصل{مقدمه}\label{فصل۱:مقدمه}


\قسمت{تعریف مسئله}
طبقه‌بندی\پانویس{\lr{Classification}} یکی از مسائل اصلی در حوزه یادگیری ماشین\پانویس{\lr{Machine Learning}} است که هدف آن تخصیص ورودی‌ها به یکی از دسته‌های از پیش تعریف‌شده می‌باشد. شبکه‌های عصبی پیچشی\پانویس{\lr{Convolutional Neural Network}} (\lr{CNN}) به دلیل توانایی بالای خود در استخراج ویژگی‌های سلسله‌مراتبی از داده‌های خام، در بسیاری از مسائل طبقه‌بندی، از جمله شناسایی تصاویر عملکرد بسیار خوبی داشته‌اند. مسئله طبقه‌بندی ارقام دست‌نویس به عنوان یک مسئله مرجع، نقش مهمی در نشان دادن توانایی شبکه‌های عصبی در پردازش داده‌های بصری دارد و به طور گسترده برای ارزیابی روش‌ها و مدل‌های مختلف استفاده می‌شود.

\شروع{شکل}[ht]
\centerimg{img0}{11cm}
\شرح{مسئله طبقه‌بندی \مرجع{cat_dog_classification}}
\برچسب{شکل:مسئله طبقه‌بندی}
\پایان{شکل}

با این حال، اجرای مدل‌های \lr{CNN} در کاربردهای عملی چالش‌هایی مانند پیچیدگی محاسباتی بالا و نیاز به منابع سخت‌افزاری کارآمد را به همراه دارد. در حالی که \lr{GPU‌}ها به دلیل توان عملیاتی بالا گزینه‌ای مناسب برای آموزش و استنتاج\پانویس{Inference} مدل‌ها هستند، مصرف انرژی بالا و محدودیت‌های آن‌ها در کاربردهای نهفته\پانویس{\lr{Embedded}} و محیط‌هایی با منابع محدود، آن‌ها را برای برخی کاربردها نامناسب می‌سازد. در مقابل، \lr{FPGA}‌ها با قابلیت پردازش موازی، مصرف انرژی کمتر و قابلیت بازپیکربندی\پانویس{\lr{Reconfigurability}}، گزینه‌ای ایده‌آل برای پیاده‌سازی مدل‌های \lr{CNN} در کاربردهایی هستند که نیاز به پردازش بی‌درنگ\پانویس{\lr{Real-Time}} و بهره‌وری بالا\پانویس{\lr{High Performance}} دارند.

\شروع{شکل}[ht]
\centerimg{img1}{11cm}
\شرح{مسئله طبقه‌بندی \مرجع{gpus_vs_fpgas}}
\برچسب{شکل:GPU یا FPGA}
\پایان{شکل}




\قسمت{اهمیت پژوهش}
پیاده‌سازی شبکه‌های عصبی پیچشی بر روی \lr{FPGA} نه تنها به دلیل چالش‌های فنی موجود در ترکیب یادگیری عمیق با سخت‌افزارهای نهفته اهمیت دارد، بلکه از جنبه‌های کاربردی نیز تأثیر بسزایی دارد. این پروژه امکان استفاده از مدل‌های یادگیری عمیق در محیط‌هایی با منابع محدود و نیاز به مصرف انرژی کم، مانند دستگاه‌های \lr{IoT}، سیستم‌های صنعتی بی‌درنگ و تجهیزات پزشکی قابل حمل\پانویس{\lr{Portable}} را فراهم می‌کند. علاوه بر این، \lr{FPGA}‌ها به دلیل انعطاف‌پذیری در طراحی و تطبیق‌پذیری با کاربردهای متنوع، می‌توانند بستری مناسب برای توسعه سامانه‌های هوشمند با کارایی بالا باشند. نتایج این پژوهش می‌تواند راهگشای کوچکی برای پژوهشگران و مهندسان در کاهش هزینه‌های طراجی، بهبود سرعت پردازش و افزایش بهره‌وری سیستم‌های مبتنی بر یادگیری عمیق باشد.


\قسمت{اهداف پژوهش}
هدف اصلی این پژوهش، شتاب‌دهی سخت‌افزاری فاز استنتاج\پانویس{\lr{Inference}} شبکه‌های عصبی پیچشی با بهینه‌سازی مصرف توان و انرژی است. با توجه به نیاز روزافزون به پردازش سریع و کارآمد داده‌ها در کاربردهای بی‌درنگ و نهفته، استفاده از \lr{FPGA} به عنوان بستری مناسب برای تحقق این هدف در اولویت قرار گرفته است. این پروژه به دنبال دستیابی به معماری سخت‌افزاری است که علاوه بر ارائه سرعت بالا در پردازش، مصرف انرژی را به حداقل برساند و قابلیت پیاده‌سازی در محیط‌های محدود به منابع مانند سیستم‌های \lr{IoT}، دستگاه‌های قابل حمل و کاربردهای صنعتی را فراهم کند.

	
\قسمت{ساختار پژوهش}
این پژوهش در ۴ فصل انجام شده است. در فصل \ref{فصل۱:مقدمه} به مقدمه و اهمیت موضوع پژوهش پرداخته شده است. در فصل \ref{فصل۲:مفاهیم اولیه} به مفاهیم اولیه و پیش‌نیاز ها پرداخته شده است. در ادامه در فصل \ref{طراحی و شبیه‌سازی} پژوهش به بررسی معماری‌های مختلف و طراحی بهترین معماری با هدف کمینه کردن تاخیر و منابع مصرفی و درنهایت شبیه‌سازی و تست معماری پرداخته شده است.