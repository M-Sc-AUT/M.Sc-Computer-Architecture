\فصل{نتایج}

در این فصل به بررسی نتایج به‌دست آمده از سنتز‌های مختلف می‌پردازیم. مدل توضیح داده شده در فصل قبل را در سه حالت مختلف سنتز کرده و خروجی به‌دست آمده را از نظر تاخیر و منابع مصرفی با هم مقایسه می‌کنیم.




\قسمت{بدون بهینه‌سازی}
در این قسمت از هیچ \lr{Pragma}ای برای بهینه‌سازی کد استفاده نشده است. منابع مصرفی و تاخیر ها برای این حالت به‌صورت شکل «\ref{شکل:خروجی سنتز بهینه نشده}» گزارش می‌شود.


\شروع{شکل}[ht]
\centerimg{synthesis_first.png}{15cm}
\شرح{خروجی سنتز بهینه نشده}
\برچسب{شکل:خروجی سنتز بهینه نشده}
\پایان{شکل}


در این حالت تاخیر در بدترین (بیشترین) حالت نسبت به دوحالت دیگر قرار دارد اما منابع مصرفی (\lr{LUT} ها، \lr{Flip Flop} ها و ... در کمترین مقدار خود قرار دارند.





\قسمت{نیمه بهینه}
در این حالت صرفا با استفاده از دو پراگما:

\begin{latin}
	\begin{enumerate}
		\item \texttt{\#pragma HLS PIPELINE}
		\item \texttt{\#pragma HLS UNROLL}
	\end{enumerate}
\end{latin}


حلقه‌ها را \lr{Unroll}می‌کنیم و ساختاری \lr{Pipeline} طور در محاسبات حلقه‌ها ایجاد کنیم. در این حالت، میزان منابع مصرفی نسبت به حالت قبل، افزایش داشته است اما تاخیر کلی مدار کاهش چشمگیری پیدا کرده است. خروجی این حالت به‌صورت شکل «\ref{شکل:خروجی سنتز نیمه بهینه}» گزارش می‌شود.


\شروع{شکل}[ht]
\centerimg{synthesis_second.png}{15cm}
\شرح{خروجی سنتز نیمه بهینه}
\برچسب{شکل:خروجی سنتز نیمه بهینه}
\پایان{شکل}



\قسمت{بهینه‌سازی کامل}
درنهایت در آخرین گام بهینه‌سازی، پراگما‌های دیگری مثل:

\begin{latin}
	\begin{enumerate}
		\item \texttt{\#pragma HLS ARRAY\_PARTITION variable=X complete}
		\item \texttt{\#pragma HLS ARRAY\_PARTITION variable=X block factor=4 dim=1}
	\end{enumerate}
\end{latin}

را به طراحی اضافه می‌کنیم تا با شکستن ورودی ماژول‌های مختلف (که عمدتا ماتریس‌های بزرگی هستند) با اجزاء کوچکتر، سریعتر محاسبات را انجام دهند اما با انجام این کار به‌صورت حسی نیز افزایش منابع مصرفی توجیح می‌شود. خروجی‌های ارائه شده در شکل «\ref{شکل:خروجی سنتز بهینه کامل}» این حرف را تایید می‌کند. منابع مصرفی به ضدت افزایش پیدا کرده است. و در مقابل آن تاخیر نه‌تنها بهتر نشده است، بلکه مقدار بسیار کمی افزایش یافته است.

این پدیده را اینطور می‌توان توجیح نمود که ابزار سنتز درتلاش برای سنتز مدار و \lr{Schedule} کردن مدار تولیدی بوده است اما به‌دلیل منابع\پانویس{\lr{Resource}} محدود در این مدل \lr{FPGA} انتخاب شده\زیرنویس{در این طراحی از \lr{FPGA} ای با پارت‌نامبر \texttt{xc7z010-clg400-1} استفاده شده است.}
ابزار نتوانسته است به درستی فرآیند سنتز، جایابی\پانویس{Placement} و مسیریابی\پانویس{Routing} را به‌درستی انجام دهد؛ بنابراین تاخیر کمی نسبت به حالت قبل زیاد شده است.


\شروع{شکل}[ht]
\centerimg{synthesis_third.png}{15cm}
\شرح{خروجی سنتز بهینه کامل}
\برچسب{شکل:خروجی سنتز بهینه کامل}
\پایان{شکل}



توضیحات این سه قسمت در جدول زیر خلاصه می‌شود: \\


\begin{table}[ht]
	\centering
	\caption{اطلاعات مربوط به معماری‌های مختلف}
	\resizebox{0.8\textwidth}{!}{ % Resize the table to 80% of the text width
		\begin{tabular}{ c c c c c c }
			\hline
			\textbf{مدل} & \textbf{تاخیر (\lr{ns})} & \textbf{تعداد \lr{LUT}} & \textbf{مبزان بهبود تاخیر} & \textbf{میزان افزایش منابع} \\ \hline \hline
			\textbf{بدون بهینه‌سازی} & ۶۵۰٫۴ & ۷۹۲۹۷ & − & −  \\ \hline
			\textbf{نیمه بهینه} & ۳۰۰٫۲ & ۸۴۱۷۱ & ٪ ۰۲٫۲ & ٪ ۰۶٫۱ \\ \hline
			\textbf{بهینه کامل} &۳۶۰٫۲ & ۴۵۳۰۴۹ & ٪ ۹۷٫۱ & ٪ ۷۱٫۵ \\ \hline\hline
		\end{tabular}
	}
\end{table}


ذکر این نکته الزامی است که؛ واژه بهینه کامل و نیمه بهینه از بابت مصرف \lr{pragma} ها استفاده شده است و نه از بابت خروجی منابع مصرفی و تاخیر. با توجه به گزارشات به‌دست آمده، مدل نیمه بهینه عملکرد بهتری را از دو مدل دیگر هم از نظر تاخیر و هم از نظر منابع مصرفی داشته است.


