\فصل{نتیجه‌گیری و جمع‌بندی}


این پژوهش به بررسی و پیاده‌سازی شبکه‌های عصبی \lr{CNN} بر روی سخت‌افزار \lr{FPGA} با هدف بهبود سرعت پردازش و کاهش مصرف انرژی در فاز استنتاج پرداخته است. شبکه‌های عصبی پیچشی به دلیل توانایی بالا در استخراج ویژگی‌های سلسله‌مراتبی از داده‌های خام، به‌ویژه در مسائل طبقه‌بندی تصاویر، عملکرد بسیار خوبی دارند. با این حال، اجرای این مدل‌ها در کاربردهای عملی با چالش‌هایی مانند پیچیدگی محاسباتی بالا و نیاز به منابع سخت‌افزاری کارآمد مواجه است. در این راستا، \lr{FPGA}‌ها به دلیل قابلیت پردازش موازی، مصرف انرژی کمتر و انعطاف‌پذیری در طراحی، گزینه‌ای ایده‌آل برای پیاده‌سازی مدل‌های \lr{CNN} در کاربردهای بی‌درنگ و محیط‌های با منابع محدود محسوب می‌شوند.


در این پژوهش، ابتدا معماری مناسب برای شبکه عصبی پیچشی با توجه به معیارهای دقت، خطا و حجم پارامترها انتخاب شد. سپس، مدل انتخاب‌شده با استفاده از مجموعه داده \lr{MNIST} آموزش داده شد و پارامترهای آن برای پیاده‌سازی سخت‌افزاری ذخیره گردید. در فاز سخت‌افزاری، کدهای مربوط به لایه‌های مختلف شبکه (مانند کانولوشن، \texttt{Pooling}، \texttt{Fully Connected} و \texttt{Flatten}) به زبان \texttt{C} نوشته شده و با استفاده از ابزار \texttt{HLS} بر روی \lr{FPGA} پیاده‌سازی شدند. در نهایت، مدل سنتز شده و عملکرد آن از نظر تاخیر و منابع مصرفی مورد ارزیابی قرار گرفت.


نتایج نشان داد که استفاده از تکنیک‌های بهینه‌سازی مانند \lr{Pipeline} و \lr{Unroll} می‌تواند به طور چشمگیری تاخیر پردازش را کاهش دهد، هرچند که این بهبود با افزایش منابع مصرفی همراه است. در این پژوهش، مدل نیمه بهینه که از ترکیب این تکنیک‌ها استفاده کرده بود، بهترین عملکرد را از نظر تعادل بین تاخیر و منابع مصرفی ارائه داد. این مدل توانست فاز استنتاج را برای ۱۰۰ تصویر در مدت زمان ۴۵٫۴ میلی‌ثانیه انجام دهد، در حالی که همین فرآیند در فاز نرم‌افزاری ۹۷٫۳۹ میلی‌ثانیه طول کشید. همچنین، دقت شبکه در تشخیص ارقام دست‌نویس به ۱۰۰٪ برای ۱۰۰ تصویر اول و ۹۹٪ برای ۵۰۰ تصویر رسید.

در مجموع، این پژوهش نشان داد که پیاده‌سازی شبکه‌های عصبی پیچشی بر روی \lr{FPGA} می‌تواند به عنوان یک راه‌حل کارآمد برای کاربردهای بی‌درنگ و محیط‌های با منابع محدود مورد استفاده قرار گیرد. با این حال، انتخاب استراتژی‌های بهینه‌سازی مناسب و تعادل بین تاخیر و منابع مصرفی از جمله چالش‌های اصلی در این زمینه است که نیاز به بررسی‌های بیشتر دارد. نتایج این پژوهش می‌تواند به عنوان پایه‌ای برای توسعه سیستم‌های هوشمند با کارایی بالا و مصرف انرژی کم در آینده مورد استفاده قرار گیرد.