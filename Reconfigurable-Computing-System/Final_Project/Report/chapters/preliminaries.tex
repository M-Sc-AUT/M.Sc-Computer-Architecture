
\فصل{مفاهیم اولیه}\label{فصل۲:مفاهیم اولیه}

\قسمت{معماری کلی سیستم‌های \lr{IoT} با نگرش به مصرف توان}
به طور کلی سیستم‌های \lr{IoT} اطلاعاتی را از محیط برداشت و پس از پردازش اولیه (و یا به‌صورت خام در برخی موارد) نتایج را از طریق یک کانال ارتباطی به خدمات‌دهنده ارسال می‌کنند و سپس با توجه به کاربرد سیستم ممکن است در پاسخ عملی را نیز در محیط انجام دهد. در حقیقت یک سیستم \lr{IoT} یک سیستم نهفته\پانویس{Embedded} می‌باشد که به شبکه اینترنت متصل است و از طریق آن، به رد و بدل اطلاعات با خدمات‌دهنده می‌پردازد. چنین سیستم‌هایی کاربردهای بالقوه فراوانی را در زمینه‌های مختلف دارند که می‌توان به موارد زیر به عنوان برخی از کاربردهای گسترده آن اشاره نمود:

\شروع{فقرات}

\فقره لوازم متصل سلامتی شخصی
\فقره کشاورزی هوشمند متصل
\فقره سیستم‌های مدیریت هوشمند ساختمان
\فقره سیستم‌های متصل مدیریت مصرف منابع (مانند آب و برق و گاز)
\فقره ذخیره‌سازی تامین هوشمند
\فقره خودروهای متصل
\فقره صنایع متصل

\پایان{فقرات}

لایه‌های معماری یک سیستم \lr{IoT} در شکل زیر آمده است.

\شروع{شکل}[ht]
\centerimg{img2}{11cm}
\شرح{معماری یک سیستم \lr{IoT} \مرجع{ElHakim2018}}
\برچسب{شکل:معماری یک سیستم‌ iot}
\پایان{شکل}



آنچه در این پژوهش مورد بحث و بررسی قرار می‌گیرد قسمت مربوط به \lr{Physical Device} ها در سیستم \lr{IoT} می‌باشد. در \lr{Physical Device} با توجه به محدودیت‌های حجم، وزن، قیمت و کاربری آن، توان در دسترس برای سیستم می‌تواند محدود باشد، لذا استفاده بهینه از توان در هر بخش بسیار مورد توجه است. 


در شکل زیر به انواع مختلف سیستم‌های \lr{IoT} مورد استفاده با توجه به میزان منابع و همچنین مصرف توان آنها اشاره شده است:
	
\شروع{شکل}[ht]
\centerimg{img3}{12cm}
\شرح{طبقه‌بندی سیستم‌های \lr{IoT} از نظر توان و منابع \مرجع{Taivalsaari2018}}
\برچسب{شکل:سیستم‌های iot از نظر توان}
\پایان{شکل}


موارد آورده شده در قسمت ۱ و ۲ شکل دارای مصرف توان بسیار پایین هستند و از تکنیک‌های مختلف جهت کاهش مصرف استفاده می‌کنند و در لایه ارتباطی نیز از روش‌های مبتنی بر ارتباطات با توان پایین استفاده می‌نمایند. \مرجع{Taivalsaari2018}

استفاده از پردازنده‌های با توان پایین و همچنین مدارات تغذیه با جریان نشتی\پانویس{Quiescent Current} حالت خاموشی بسیار کم نیز از الزامات طراحی چنین سیستم‌هایی می‌باشد. \مرجع{Paidimarri2017}


سخت‌افزار\پانویس{Firmware} مورد استفاده در این لوازم بایستی برای حداکثر کاهش مصرف توان بهینه شده باشد که یکی از مهمترین بخش‌های سیستم می‌باشد و می‌تواند کارایی یک سخت‌افزار خوب را در صورت عدم کارکرد صحیح به شدت تحت تاثیر قرار دهد. در ادامه نکات استخراج شده از تحقیقات مختلف در خصوص هر بخش به صورت جداگانه مورد بررسی قرار می‌گیرد.


\قسمت{انتخاب سیستم پردازشی}
در خصوص پردازنده سیستم، می‌بایست نکاتی از قبیل قابلیت انجام فعالیت‌های مورد نظر با سرعت مناسب و توان مصرفی پایین برای اجرای هر بخش از کد توسط طراح رعایت گردند. در سیستم‌های \lr{IoT} با توان بسیار پایین، به دلیل اینکه زمان کاری سیستم در حالت \lr{idle} می‌باشد، کم بودن توان \lr{Static} در پردازنده ضروری است. \مرجع{Cheour2020} به عنوان مثال در پردازنده‌های مدرن کم مصرفی همچون خانواده \texttt{STM32L552xx} توان مصرفی در حالت‌های مختلف کاهش توان به صورت زیر است.


\شروع{شکل}[ht]
\centerimg{img4}{13cm}
\شرح{مشخصات تراشه \texttt{STM32L552xx} \مرجع{ST2020}}
\برچسب{شکل:مشخصات تراشه STM32L552xx}
\پایان{شکل}


همان‌گونه که مشاهده می‌شود این مقدار بسیار نسبت به پردازنده‌ها و میکروکنترلرهای قدیمی‌تر کاهش یافته‌است که بهبود پروسه تولید، کاهش سایز ترانزیستورها، بهبود پیاده‌سازی‌های سخت‌افزاری، کاهش ولتاژ کاری و قابلیت کار در محدوده فرکانسی وسیع سیستم از دلایل این امر می‌باشد. \مرجع{Hennessy2017}

همچنین پشتیبانی از مواردی همچون \lr{DVFS}\پانویس{Dynamic Voltage Frequency Scaling}، وجود حالت‌های مختلف \lr{Power}، وجود کامپایلر با دید به محدودیت توان می‌تواند بسیار در مصرف توان محصول نهایی موثر باشد. \مرجع{Cheour2020}

نکته بسیار مهم در انتخاب یک پردازنده در سیستم های نهفته مورد استفاده در کاربردهای \lr{IoT} زمان خروج پایین از حالت های با مصرف کاهش یافته به حالت عملکرد کامل می‌باشد.

در سیستم های مورد بررسی در این پژوهش فرض بر این است که نیازی به \lr{MMU}\پانویس{Memory Management Unit} جهت مدیریت حافظه نمی‌باشد و توان میکروکنترلرهای فاقد این واحد را برای ایجاد سفت‌افزار، بدون \lr{RTOS}\پانویس{Real-Time Operating System} و یا \lr{RTOS}های ساده ای مانند \lr{FreeRTOS} استفاده نمود. دلیل این امر پیچیدگی زیاد و وجود این بخش در تحلیل و بررسی حالت‌های مختلف بوجود آمده در سیستم می‌باشد.

پردازنده انتخابی بایستی قابلیت قطع تغذیه و کلاک بخش‌هایی که در فعالیت انجامی مورد استفاده نیستند را دارا باشد. به عنوان مثال در صورت عدم نیاز به بخش \lr{ADC} بتوان کلاک و توان تحویلی به آن را در میکروکنترلر قطع نمود. در اکثر میکروکنترلرها این امر با قطع کلاک صورت می‌پذیرد اما در مواردی نیز قطع کامل توان نیز وجود دارد.

دو روش قطع کلاک و قطع توان به ترتیب در اشکال زیر نمایش داده شده‌اند. \مرجع{Capra2019}


\شروع{شکل}[ht]
\centerimg{img5}{6cm}
\شرح{روش قطع کلاک}
\برچسب{شکل:روش قطع کلاک}
\پایان{شکل}

\شروع{شکل}[ht]
\centerimg{img6}{8cm}
\شرح{روش قطع توان}
\برچسب{شکل:روش قطع توان}
\پایان{شکل}

در صورتی که کاربری مورد نظر دارای زمان های \lr{idle} طولانی باشد بایستی میکروکنترلر و یا پردازنده انتخابی دارای مدهای خواب عمیق باشد تا در این صورت بتوان به حداکثر میزان کاهش مصرف دست یافت. 

در این حالت معمولاً پردازنده توسط یک تحریک خارجی و یا زمانبندی \lr{RTC}\پانویس{Real Time Clock} از این حالت خارج شده و پس از انجام عمل مورد نیاز که بایستی تا حد ممکن کوتاه باشد، مجدداً به حالت خواب عمیق بازمی‌گردد.

مسئله مهم دیگر در انتخاب یک پردازنده مناسب وجود حافظه کافی با قابلیت مصرف کم انرژی می‌باشد. 

نسل‌های جدید حافظه مانند \lr{ReRAM}\پانویس{Resistive RAM}، \lr{FeRAM}\پانویس{Ferroelectric RAM} ، \lr{PCM RAM}\پانویس{Phase-Change RAM} و \lr{MRAM}\پانویس{Magnetic RAM} نیز در این خصوص با دارا بودن مصرف کم و قابلیت نگهداری اطلاعات بدون نیاز به منبع بسیار در کاهش مصرف موثر خواهند بود. \مرجع{Capra2019}



در صورتی که کاریری مورد نظر دارای زمان های \lr{idle} طولانی باشد بایستی میکروکنترلر و یا پردازنده انتخابی دارای مدهای خواب عمیق باشد تا در این صورت بتوان به حداکثر میزان کاهش مصرف دست یافت.

در این حالت معمولاً پردازنده توسط یک تحریک خارجی و یا زمانبندی \lr{RTC}\textsuperscript{1} از این حالت خارج شده و پس از انجام عمل مورد نیاز که بایستی تا حد ممکن کوتاه باشد، مجدداً به حالت خواب عمیق بازمی‌گردد.

مسئله مهم دیگر در انتخاب یک پردازنده مناسب وجود حافظه کافی با قابلیت مصرف کم انرژی می‌باشد. 

نسل‌های جدید حافظه مانند \lr{ReRAM}، \lr{FeRAM}، \lr{PCRAM}، \lr{MRAM}\textsuperscript{5} نیز در این خصوص با دارا بودن مصرف کم و قابلیت نگهداری اطلاعات بدون نیاز به منبع بسیار در کاهش مصرف موثر خواهند بود.\lr{[1]}



در صورتی که کاریری مورد نظر دارای زمان های \lr{idle} طولانی باشد بایستی میکروکنترلر و یا پردازنده انتخابی دارای مدهای خواب عمیق باشد تا در این صورت بتوان به حداکثر میزان کاهش مصرف دست یافت. 

در این حالت معمولاً پردازنده توسط یک تحریک خارجی و یا زمانبندی \lr{RTC}\textsuperscript{1} از این حالت خارج شده و پس از انجام عمل مورد نیاز که بایستی تا حد ممکن کوتاه باشد، مجدداً به حالت خواب عمیق بازمی‌گردد.

مسئله مهم دیگر در انتخاب یک پردازنده مناسب وجود حافظه کافی با قابلیت مصرف کم انرژی می‌باشد. 

نسل‌های جدید حافظه مانند \lr{ReRAM}، \lr{FeRAM}، \lr{PCRAM}، \lr{MRAM}\textsuperscript{5} نیز در این خصوص با دارا بودن مصرف کم و قابلیت نگهداری اطلاعات بدون نیاز به منبع بسیار در کاهش مصرف موثر خواهند بود.\lr{[1]}


وجود یا عدم وجود قسمت‌های سخت‌افزاری برای عملکردهای خاص همچون شتابدهی به رمزنگاری و رمزگشایی نیز می‌تواند در عملکرد کلی سیستم از نظر مصرف توان بسیار موثر باشد. \مرجع{Baldanzi2019}


\قسمت{مدارات تغذیه}
مدارات مربوط به بخش تغذیه نقش عمده‌ای در مصرف توان \lr{Static} مدار دارند. تکنیک‌های \lr{Power\_Gating} مشابه آنچه در میکروکنترلرها شاهد آن بودیم در سطح مدار نیز می‌تواند برای قطع توان به قسمت‌هایی که مورد نیاز نمی‌باشد مورد استفاده قرار گیرد. همچنین مدارات تنظیم ولتاژ مورد استفاده می‌بایست از نوع با جریان نشتی بسیار پایین انتخاب گردند \پانویس{Paidimarri2017}. در صورتی که بتوان از اتصال مستقیم باتری به مدار استفاده کرد این روش به دلیل عدم وجود توان تلفاتی در رگلاتور بیشترین عمر باتری را خواهد داشت.

همچنین رگلاتورهای خطی و غیرخطی با جریان نشتی حالت خاموشی بسیار پایین وجود دارند که در شکل‌های بعدی به دو نمونه از آنها اشاره شده است:

\شروع{شکل}[ht]
\centerimg{img8}{10cm}
\شرح{تراشه \texttt{TPS783}}
\برچسب{شکل:تراشه TPS783}
\پایان{شکل}



\شروع{شکل}[ht]
\centerimg{img9}{10cm}
\شرح{تراشه \texttt{TPS62743}}
\برچسب{شکل:تراشه TPS62743}
\پایان{شکل}



\قسمت{مدارات ارتباطی}

قسمت ارتباطات یکی از قسمت‌هایی است که در هنگام کارکرد بسته به روش ارتباطی انتخابی، توان به نسبت بالایی را مورد استفاده قرار می‌دهد. انتخاب این بخش می‌تواند بر کارایی سیستم تاثیر فراوانی داشته باشد. این تاثیر از دو جهت توان مصرفی و دسترسی مناسب به اطلاعات سیستم، قابل بررسی است و معمولاً یک سیستم مخابراتی با برد بلند توان مصرفی بیشتری نسبت به سیستم مشابه با برد کمتر را دارد، لذا می‌بایست در طراحی سیستم موارد مربوط به برد بهینه مدار مخابراتی و همچنین نحوه عملکرد آن مانند ارتباط مستقیم نقطه به نقطه، ارتباط به صورت \lr{Mesh} نوع مدولاسیون و ... مورد توجه قرار گیرد.

بخش عمده‌ای از توان در بخش سیستم مخابراتی در زمانی سیستم در حالت گیرندگی قرار دارد استفاده می‌گردد. در صورتی که بتوان الگوریتم ارتباطی را به صورتی ایجاد نمود که تنها نیاز به روشن بودن گیرنده رادیویی در مدت کوتاهی باشد و یا به طور کلی نیازی به استفاده از این حالت نباشد، توان مصرفی تا حد زیادی کاهش خواهد یافت. علت این امر نامعلوم بودن زمان ارسال توسط فرستنده در شبکه می‌باشد. استفاده از تکنیک‌های گیرنده‌های خودبیدار\پانویس{Wake-Up-Ratio} شونده نیز از روش‌های موثر در کاهش مصرف این بخش می‌باشد. \مرجع{Mangal2019}





\قسمت{تامین توان و ذخیره‌سازی انرژی}
تامین توان سیستم می‌تواند به صورت متصل و یا غیر متصل به شبکه برق باشد. در حالت غیر متصل جهت تامین توان مصرفی سیستم، استفاده از باتری با ظرفیت بالا و یا برداشت انرژی از محیط می‌تواند مورد استفاده قرار گیرد. در صورت انتخاب باتری برای تامین توان بدون منبع خارجی جهت شارژ، بایستی توجه داشت که جریان دشارژ خود به خودی باتری یک عامل مهم در کاهش زمان مفید عملکرد سیستم است، لذا برای حل این مشکل بایستی از انواعی از باتری که جریان دشارژ خود به خودی پایینی دارند استفاده نمود.

عمر مفید باتری نیز نکته مهم دیگری در انتخاب باتری خصوصاً در سیستم‌های با تامین توان خارجی ناپایدار و شارژ های متعدد است. این امر در سیستم‌های با برداشت توان از محیط که با توجه به نوع سیستم مورد استفاده، دائماً باتری بین حالت شارژ و دشارژ سوئیچ می‌نماید بسیار مهم می‌باشد زیرا که این امر می‌تواند باعث کاهش عمر مفید باتری و یا کاهش ظرفیت آن گردد. در این گونه موارد استفاده از ابر خازن‌ها نیز می‌تواند مفید باشد \مرجع{Kjellby2018}.




\قسمت{سفت افزار}
مواردی همچون نحوه عملکرد پردازنده در سیستم در هنگام خروج از حالت خواب عمیق میزان زمان خواب عمیق و نحوه خروج از آن، چگونگی به‌کارگیری مدارات مختلف پردازش داده‌ها، ارتباط با سرور و بسیاری موارد دیگر توسط سفت‌افزار یک سیستم مدیریت می‌گردد.

نحوه مدیریت \lr{Process} ها و \lr{Task} ها نیز در سفت‌افزار بنا بر وجود و یا عدم وجود یک \lr{RTOS} به روش‌های مختلف انجام می‌پذیرد. نکته مهم در خصوص سفت‌افزار وابستگی شدید توان مصرفی به نحوه انجام یک عمل می‌باشد. به گونه ای که ممکن است یک کد نامناسب در سفت‌افزار باعث عدم عملکرد صحیح سیستم گردد و با ورود سیستم به حالت \lr{Deadlock} باعث عدم پاسخ سیستم و تخلیه باتری گردد. در این حالت، به کارگیری \lr{Watchdog} می‌تواند کمک کننده باشد لذا در انتخاب میکروکنترلر بایستی موردی انتخاب گردد که دارای \lr{Watchdog} سخت‌افزاری باشد. بایستی توجه داشت که اکثر میکروکنترلرهای کنونی دارای این واحد می‌باشند اما تفاوت آن‌ها در قابلیت‌ها و درجه استقلال اجرایی می‌باشد. استفاده و یا عدم استفاده صحیح از یک بخش مدار توسط سفت‌افزار می‌تواند باعث ایجاد توان تلفاتی در مدار گردد. به عنوان مثال روشن نمودن قسمت‌های بدون استفاده در میکروکنترلر و یا عدم استفاده از واحد شتاب‌دهنده سخت‌افزاری \مرجع{Baldanzi2019} می‌تواند مثالی از این موارد باشد.

پس از بررسی اولیه بخش‌های مختلف به بررسی دقیق‌تر بحث توان با رویکرد برداشت توان از محیط پرداخته می‌شود و انواع مختلف انجام این روش و همچنین مزایا و معایب هر یک مورد بررسی قرار می‌گیرد.