
\فصل{مفاهیم اولیه}\label{فصل۲:مفاهیم اولیه}

\قسمت{شبکه عصبی \lr{CNN}}
شبکه‌های عصبی پیچشی (\lr{Convolutional Neural Networks} یا \lr{CNN}) نوعی از شبکه‌های عصبی مصنوعی هستند که به طور خاص برای پردازش داده‌های با ساختار شبکه‌ای مانند تصاویر طراحی شده‌اند. این شبکه‌ها به دلیل توانایی بالای خود در شناسایی الگوها و ویژگی‌ها، به طور گسترده در مسائلی مانند طبقه‌بندی تصاویر \پانویس{\lr{Image Classification}} و تشخیص اشیاء\پانویس{\lr{Object Detection}} استفاده می‌شوند. \lr{CNN‌}ها از معماری سلسله‌مراتبی\پانوس{\lr{Hierarchical}} برای استخراج ویژگی‌ها از داده‌های ورودی استفاده می‌کنند و قادرند ویژگی‌های سطح پایین (مانند لبه‌ها) تا ویژگی‌های سطح بالا (مانند اشکال پیچیده) را به طور خودکار شناسایی کنند.



\شروع{شکل}[ht]
\centerimg{img2}{11cm}
\شرح{ساختار یک شبکه \lr{CNN} \مرجع{convolutional_neural_network}}
\برچسب{شکل:ساختار یک CNN}
\پایان{شکل}





به طور کلی سیستم‌های \lr{IoT} اطلاعاتی را از محیط برداشت و پس از پردازش اولیه (و یا به‌صورت خام در برخی موارد) نتایج را از طریق یک کانال ارتباطی به خدمات‌دهنده ارسال می‌کنند و سپس با توجه به کاربرد سیستم ممکن است در پاسخ عملی را نیز در محیط انجام دهد. در حقیقت یک سیستم \lr{IoT} یک سیستم نهفته\پانویس{Embedded} می‌باشد که به شبکه اینترنت متصل است و از طریق آن، به رد و بدل اطلاعات با خدمات‌دهنده می‌پردازد. چنین سیستم‌هایی کاربردهای بالقوه فراوانی را در زمینه‌های مختلف دارند که می‌توان به موارد زیر به عنوان برخی از کاربردهای گسترده آن اشاره نمود:

\شروع{فقرات}

\فقره لوازم متصل سلامتی شخصی
\فقره کشاورزی هوشمند متصل
\فقره سیستم‌های مدیریت هوشمند ساختمان
\فقره سیستم‌های متصل مدیریت مصرف منابع (مانند آب و برق و گاز)
\فقره ذخیره‌سازی تامین هوشمند
\فقره خودروهای متصل
\فقره صنایع متصل

\پایان{فقرات}

لایه‌های معماری یک سیستم \lr{IoT} در شکل زیر آمده است.

\شروع{شکل}[ht]
\centerimg{img2}{11cm}
\شرح{معماری یک سیستم \lr{IoT} }
\برچسب{شکل:معماری یک سیستم‌ iot}
\پایان{شکل}
