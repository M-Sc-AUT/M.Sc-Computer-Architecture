
% -------------------------------------------------------
%  Common Styles and Formattings
% -------------------------------------------------------


\usepackage{amssymb,amsmath}
\usepackage[colorlinks,linkcolor=blue,citecolor=blue]{hyperref}
\usepackage[usenames,dvipsnames]{pstricks}
\usepackage{graphicx,subfigure,wrapfig}
\usepackage{geometry}
\usepackage[mathscr]{euscript}
\usepackage{multicol}
\usepackage{multirow}
\usepackage[figureposition=bottom,tableposition=top,font={small,bf},labelfont=bf]{caption}
\usepackage{setspace}
\usepackage{xcolor}
\usepackage{listings}
\usepackage{color}
\usepackage{amsmath}
\usepackage{booktabs}
\usepackage[utf8]{inputenc}
\usepackage{fourier} 
\usepackage{array}
\usepackage{makecell}
\usepackage{soul} % For highlighting
\soulregister{\texttt}{1} % Make \hl compatible with \texttt
\setlength{\columnsep}{10pt} % Adjust column spacing

\usepackage[localise=on,extrafootnotefeatures]{xepersian}
\usepackage[noend]{algpseudocode}
\input{styles/alg-fa}




\definecolor{codegreen}{rgb}{0,0.6,0}
\definecolor{codegray}{rgb}{0.5,0.5,0.5}
\definecolor{codeblue}{RGB}{49,49,255}
\definecolor{codeorange}{rgb}{1,0.49,0}
\definecolor{backcolour}{rgb}{0.95,0.95,0.96}
\lstdefinestyle{mystyle}{
	backgroundcolor=\color{backcolour},
	commentstyle=\color{codegray},
	keywordstyle=\color{codeorange},
	numberstyle=\small\color{codegray},
	stringstyle=\color{codegreen},
	basicstyle=\ttfamily\footnotesize,
	moredelim=*[s][\colorIndex]{[}{]},
	literate=*{:}{:}1,
	breakatwhitespace=false,
	breaklines=true,
	%captionpos=b,
	keepspaces=true,
	numbers=left,
	numbersep=5pt,
	showspaces=false,
	showstringspaces=false,
	showtabs=false,
	tabsize=8,
	xleftmargin=3pt,
	framexbottommargin=1pt,
	mathescape=true,
	frame=tb,
	label={lst:code},
}
\lstset{style = mystyle}
\renewcommand\lstlistingname{Source Code}
\renewcommand\lstlistlistingname{Source Code}
% python
\lstdefinestyle{python-style}
{
	language=Python,
	basicstyle=\footnotesize\ttfamily,
	commentstyle=\color{codegray},
	morekeywords={self},              % Add keywords here
	keywordstyle=\color{codeblue},
	emph={MyClass,__init__},          % Custom highlighting
	emphstyle=\color{codeorange},    % Custom highlighting style
	stringstyle=\color{codegreen},
	frame=tb,                         % Any extra options here
	showstringspaces=false,
	label={lst:code:python},
	framexbottommargin=1pt,
}
\definecolor{darkgreen}{HTML}{006400}


% VHDL
\definecolor{light-gray}{gray}{0.96}
\definecolor{pageheading-gray}{gray}{0.2}
\definecolor{dark-gray}{gray}{0.45}
\definecolor{dark-green}{rgb}{0.245,0.121,0.0}

\lstdefinelanguage{VHDL}{
	morekeywords=[1]{
		ALL,ARCHITECTURE,ABS,AND,ASSERT,ARRAY,AFTER,ALIAS,%
		ACCESS,ATTRIBUTE,BEGIN,BODY,BUS,BLOCK,BUFFER,CONSTANT,CASE,%
		COMPONENT,CONFIGURATION,DOWNTO,ELSE,ELSIF,END,ENTITY,EXIT,%
		FUNCTION,FOR,FILE,GENERIC,GENERATE,GUARDED,GROUP,IF,IN,INOUT,IS,%
		INERTIAL,IMPURE,LIBRARY,LOOP,LABEL,LITERAL,LINKAGE,MAP,MOD,NOT,%
		NOR,NAND,NULL,NEXT,NEW,OUT,OF,OR,OTHERS,ON,OPEN,PROCESS,PORT,%
		PACKAGE,PURE,PROCEDURE,POSTPONED,RANGE,REM,ROL,ROR,REPORT,RECORD,%
		RETURN,REGISTER,REJECT,SIGNAL,SUBTYPE,SLL,SRL,SLA,SRA,SEVERITY,%
		SELECT,THEN,TYPE,TRANSPORT,TO,USE,UNITS,UNTIL,VARIABLE,WHEN,WAIT,%
		WHILE,XOR,XNOR,%
		DISCONNECT,ELIF,WITH}, % Arnaud Tisserand
	sensitive=f,% 1998 Gaurav Aggarwal
	morestring=[d]{"}%
	morekeywords=[2]{
		std_logic_vector,std_logic,std_logic_1164,numeric_std,numeric_signed,numeric_unsigned,numeric_bit,math_real,math_complex,
		unsigned,integer,
		rising_edge,to_integer,resize
	},
	morecomment=[l]--
}
\lstset{
	language=VHDL,
	basicstyle=\footnotesize\ttfamily,
	%basicstyle=\scriptsize\ttfamily, % font size for code
	%basicstyle=\tiny\ttfamily,
	lineskip= -1pt, % space between code lines (default: 0pt)
	%numbers=left,
	numberstyle=\tiny,
	%stepnumber=2,
	numbersep=5pt,
	aboveskip={0pt},
	tabsize=4,
	extendedchars=true,
	breaklines=true,
	classoffset=0,
	morecomment=[s][\textit]{my_lab}{el}, % make the label word italics
	otherkeywords={<=,:=, shared,=>,'}, % add <=, shared
	keywordstyle=\color{blue},
	keywordstyle = [2]\color{STD}\bfseries,
	commentstyle=\color[RGB]{10,140,20}, % green color
	frame=b, % turn on end line
	rulecolor=\color{dark-gray}, % end line color
	%captionpos=b, 
	belowcaptionskip=2pt,  
	stringstyle=\color{red}\ttfamily,
	morestring=[b]',
	showspaces=false,
	showtabs=false,
	xleftmargin=3pt,
	framexleftmargin=3pt,
	framexrightmargin=0pt,
	framexbottommargin=1pt,
	backgroundcolor=\color{light-gray},
	showstringspaces=false}








% -------------------- Page Layout --------------------


%\newgeometry{top=3cm,right=3cm,left=2.5cm,bottom=3cm,footskip=1.25cm}
\newgeometry{margin=1in,bottom=1.1in,footskip=.4in}

\renewcommand{\baselinestretch}{1.4}
\linespread{1.6}
\setlength{\parskip}{0.45em}

%\fancyhf{}
%\rhead{\leftmark}
%\lhead{\thepage}


% -------------------- Fonts --------------------

\settextfont[
Scale=1.09,
Extension=.ttf, 
Path=styles/fonts/,
BoldFont=Yas,
ItalicFont=YAS IT,
BoldItalicFont=YAS BDIT
]{Yas}

\setdigitfont[
Scale=1.09,
Extension=.ttf, 
Path=styles/fonts/,
BoldFont=YAS BD,
ItalicFont=YAS IT,
BoldItalicFont=YAS BDIT
]{Yas}

\defpersianfont\sayeh[
Scale=1,
Path=styles/fonts/
]{XB Kayhan Pook}


% -------------------- Styles --------------------


\SepMark{-}
\renewcommand{\labelitemi}{$\small\bullet$}



% -------------------- Environments --------------------


\newtheorem{قضیه}{قضیه‌ی}[chapter]
\newtheorem{لم}[قضیه]{لم}
\newtheorem{ادعا}[قضیه]{ادعای}
\newtheorem{مشاهده}[قضیه]{مشاهده‌ی}
\newtheorem{نتیجه}[قضیه]{نتیجه‌ی}
\newtheorem{مسئله}{مسئله‌ی}[chapter]
\newtheorem{تعریف}{تعریف}[chapter]
\newtheorem{مثال}{مثال}[chapter]


\newenvironment{اثبات}
	{\begin{trivlist}\item[\hskip\labelsep{\em اثبات.}]}
	{\leavevmode\unskip\nobreak\quad\hspace*{\fill}{\ensuremath{{\square}}}\end{trivlist}}

\newenvironment{alg}[2]
	{\begin{latin}\settextfont[Scale=1.0]{Times New Roman}
	\begin{algorithm}[t]\caption{#1}\label{algo:#2}\vspace{0.2em}\begin{algorithmic}[1]}
	{\end{algorithmic}\vspace{0.2em}\end{algorithm}\end{latin}}


% -------------------- Titles --------------------


\renewcommand{\listfigurename}{فهرست شکل‌ها}
\renewcommand{\listtablename}{فهرست جدول‌ها}
\renewcommand{\bibname}{\rl{{مراجع}\hfill}} 


% -------------------- Commands --------------------


\newcommand{\IN}{\ensuremath{\mathbb{N}}} 
\newcommand{\IZ}{\ensuremath{\mathbb{Z}}} 
\newcommand{\IQ}{\ensuremath{\mathbb{Q}}} 
\newcommand{\IR}{\ensuremath{\mathbb{R}}} 
\newcommand{\IC}{\ensuremath{\mathbb{C}}} 

\newcommand{\set}[1]{\left\{ #1 \right\}}
\newcommand{\seq}[1]{\left< #1 \right>}
\newcommand{\ceil}[1]{\left\lceil{#1}\right\rceil}
\newcommand{\floor}[1]{\left\lfloor{#1}\right\rfloor}
\newcommand{\card}[1]{\left|{#1}\right|}
\newcommand{\setcomp}[1]{\overline{#1}}
\newcommand{\provided}{\,:\,}
\newcommand{\divs}{\mid}
\newcommand{\ndivs}{\nmid}
\newcommand{\iequiv}[1]{\,\overset{#1}{\equiv}\,}
\newcommand{\imod}[1]{\allowbreak\mkern5mu(#1\,\,\text{پیمانه‌ی})}

\newcommand{\poly}{\mathop{\mathrm{poly}}}
\newcommand{\polylog}{\mathop{\mathrm{polylog}}}
\newcommand{\eps}{\varepsilon}

\newcommand{\lee}{\leqslant}
\newcommand{\gee}{\geqslant}
\renewcommand{\leq}{\lee}
\renewcommand{\le}{\lee}
\renewcommand{\geq}{\gee}
\renewcommand{\ge}{\gee}

\newcommand{\مهم}[1]{\textbf{#1}}
\renewcommand{\برچسب}{\label}

\newcommand{\REM}[1]{}
\renewcommand{\حذف}{\REM}
\newcommand{\لر}{\lr}
\newcommand{\کد}[1]{\lr{\tt #1}}
\newcommand{\پاورقی}[1]{\footnote{\lr{#1}}}



% -------------------- Dictionary --------------------


\newcommand{\dicalphabet}[1]{
\begin{minipage}{\columnwidth}
	\centerline{\noindent\textbf{\large #1 }}
	\vspace{.5em}
\end{minipage}
\nopagebreak[4]
}

\newcommand{\dic}[2]{\noindent  #2 \dotfill  \lr{#1} \\ }


% ------------------------------ Images and Figures --------------------------

\graphicspath{{figs/}}
\setlength{\intextsep}{0pt}  % for float boxes
\renewcommand{\psscalebox}[1]{}  % for LaTeX Draw

\newcommand{\floatbox}[2]
	{\begin{wrapfigure}{l}{#1}
	\centering #2 \end{wrapfigure}}

\newcommand{\centerfig}[2]
	{\centering\scalebox{#2}{\input{figs/#1}}}

\newcommand{\fig}[3]
	{\floatbox{#3}{\centerfig{#1}{#2}}}

\newcommand{\centerimg}[2]
	{\vspace{1em}\begin{center}\includegraphics[width=#2]{figs/#1}\end{center}\vspace{-1.5em}}

\NewDocumentCommand{\img}{m m o}
	{\begin{wrapfigure}{l}{\IfValueTF{#3}{#3}{#2}}
	\centering\includegraphics[width=#2]{figs/#1}\end{wrapfigure}}


