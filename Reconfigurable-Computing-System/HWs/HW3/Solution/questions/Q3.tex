\section{سوال سوم}



تفاوت‌های اصلی بین سیستم‌های قابل پیکربندی مجدد استاتیک آفلاین و سیستم‌های قابل پیکربندی مجدد پویای زمان-اجرا را با تمرکز بر نحوه تعریف توالی محاسبات و قابلیت‌های پیکربندی مجدد توضیح دهید. یک مقاله را که از این قابلیت‌های \lr{FPGA} استفاده کرده است بررسی کنید و خلاصه آن را در دو پاراگراف گزارش نمایید.


\begin{qsolve}
	تفاوت‌های اصلی بین سیستم‌های قابل پیکربندی مجدد استاتیک آفلاین و سیستم‌های قابل پیکربندی مجدد پویای زمان-اجرا را می‌توان به‌صورت زیر دسته‌بندی نمود:
	
	
	\begin{enumerate}
		\item \textbf{توالی محاسبات:}
		\begin{itemize}
			\item 
			سیستم‌های استاتیک آفلاین: در این سیستم‌ها، توالی محاسبات و پیکربندی‌های مربوط به آن‌ها در زمان کامپایل و پیش از اجرای برنامه تعریف می‌شود. یعنی کل پیکربندی سیستم از پیش مشخص است و نیازی به تغییر آن در زمان اجرا نیست. به عبارت دیگر، توالی محاسبات به صورت ثابت است و تغییر نمی‌کند.
			
			
			\item 
			سیستم‌های پویای زمان-اجرا: در این سیستم‌ها، توالی محاسبات و پیکربندی‌ها می‌توانند در طول زمان اجرا تغییر کنند. به این معنا که در حین انجام پردازش‌ها، ممکن است سیستم به‌طور پویا تغییر کند تا بهترین عملکرد را در پاسخ به شرایط لحظه‌ای فراهم کند. این امکان به سیستم‌ها اجازه می‌دهد که به‌طور مؤثرتر از منابع خود استفاده کنند و به نیازهای مختلف پردازش پاسخ دهند.
		\end{itemize}
		
		
		
		\item \textbf{قابلیت‌های پیکربندی مجدد:}
		\begin{itemize}
			\item 
			سیستم‌های استاتیک آفلاین: پیکربندی سخت‌افزار در این سیستم‌ها در زمان کامپایل و پیش از اجرا تعیین می‌شود. این سیستم‌ها هیچگونه تغییر پیکربندی در زمان اجرا ندارند و همه‌ی تنظیمات از ابتدا مشخص و ثابت هستند.
			
			
			\item 
			سیستم‌های پویای زمان-اجرا: در این سیستم‌ها، پیکربندی سخت‌افزار می‌تواند در حین اجرا تغییر کند. این سیستم‌ها قابلیت پیکربندی مجدد پویا دارند، که این امکان را فراهم می‌کند که سخت‌افزار به‌طور مداوم و در پاسخ به نیازهای پردازشی مختلف تنظیم شود. این ویژگی به ویژه در کاربردهایی که نیاز به انعطاف‌پذیری و کارایی بالا دارند مفید است.
		\end{itemize}
	\end{enumerate}
	
	در ادامه با توجه به اینکه موضوع تمرین عملی پیاده سازی کانولوشن بود، مقاله \cite{ref1} را انتخاب کردم و در ادامه سعی می‌کنم کلیت این کار را در دو پاراگراف شرح دهم.
	
	
	مقاله به بررسی استفاده از پیکربندی مجدد پویا در پردازش تصویر و سیگنال می‌پردازد. این تحقیق به طور خاص روی تخصیص منابع سخت‌افزاری به‌صورت پویا تمرکز دارد، به گونه‌ای که بر اساس سطح نویز در یک لحظه خاص، منابع سخت‌افزاری به حداقل مورد نیاز برای اجرای عملیات پردازش تصویر تخصیص داده می‌شود. در این سیستم‌ها، پیکربندی مجدد می‌تواند سرعت تغییرات پیکربندی و تعداد پیکربندی‌های ممکن را مشخص کند، این رویکرد به الگوریتم‌های پردازش‌های تصویر کمک می‌کند تا با توجه به تغییرات سیگنال تصویر، میزان محاسبات و نیاز به حافظه تغییر کند.
	
	
	در این مقاله به توصیف الگوریتمی برای فیلتر کردن تصویر می‌پردازد. این پیاده‌سازی بر روی \lr{FPGA} مدل \lr{Xilinx 600K Spartan-IIE} انجام شده است. مقاله نشان می‌دهد که سیستم قادر است میزان محاسبات و حافظه مورد استفاده را در مقیاس‌های زمانی ریز (میلی‌ثانیه) و درشت (ثانیه) به طور پویا تنظیم کند. نتایج تجربی نشان می‌دهند که زمان اجرای فیلتر تصویر روی \lr{FPGA} تا ۴۰۰ برابر سریع‌تر از پیاده‌سازی نرم‌افزاری روی پردازنده ۲.۸ گیگاهرتزی \lr{Pentium} است، که این خود نشان‌دهنده مزیت‌های قابل توجه پیکربندی مجدد پویا در \lr{FPGA} برای کاربردهای پردازش تصویر است.
	
	
\end{qsolve}





\begin{latin}
	\begin{thebibliography}{9}
		\bibitem{ref1}
		FPGA Based Hardware Implementation of Image Filter With Dynamic Reconfiguration Architecture \href{https://citeseerx.ist.psu.edu/document?repid=rep1&type=pdf&doi=e064fe8a4e6a3808ee5cc6f53d13be94a3671fc4}{[Link]}
	\end{thebibliography} 
\end{latin}