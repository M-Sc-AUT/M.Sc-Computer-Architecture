\section{سوال چهارم - پروژه عملی}

در این پروژه، با نگاه به پروژه قبلی، بخش کانولوشن، یک سیستم پردازش تصویر طراحی می‌گردد. در این سیستم ورودی مربوط به دیتاست \lr{MNIST} با سایز ۲۸ در ۲۸ بوده و ۳ فیلتر کانولوشن به ابعاد ۳ در ۳ به صورت پشت سر هم بر روی تصویر اعمال می‌شود. مقادیر موجود در ماتریس‌های کانولوشن به صورت تصادفی انتخاب شده و به عنوان ورودی به تابع کانولوشن داده می‌شود (در کد \lr{Fix} نشده باشد) و خروجی با نمونه نرم‌افزاری مورد بررسی قرار می‌گیرد.

برای اطلاعات بیشتر می‌توانید از
\href{https://serokell.io/blog/introduction-to-convolutional-neural-networks}{این لینک}
استفاده کنید.

در گزارش ارسالی علاوه بر شرح مراحل کار با فرض استفاده از \lr{Zyng7010} میزان سرعت و تأخیر اولیه را گزارش نمایید. همچنین با فرض امکان گسترش که برای پردازش موازی چه تعداد از بلوک طراحی شده شما در این \lr{FPGA} قابل به کارگیری به صورت همزمان خواهد بود؟

\begin{qsolve}
	
\end{qsolve}