\section{سوال چهارم}

کاربردهای \lr{CGRA} در حوزه امنیت، یادگیری عمیق و سلامتی (از هر مورد ۱ نمونه کاربرد) را با ذکر مثال (مانند \lr{COBRA} و \lr{Eyeriss}) با ارجاع به منابع شرح دهید.



\begin{qsolve}
	\begin{enumerate}
		\item \textbf{یادگیری عمیق:}
		
		
		یادگیری عمیق، به‌ویژه شبکه‌های عصبی کانولوشنی (\lr{CNN} ها) که از نظر محاسباتی مرتب و منظم هستند، تبدیل به هدفی برای \lr{CGRA}های شده‌اند. در اینجا هدف این است که عمومیت و قابلیت پیکربندی مجدد \lr{CGRA}های سنتی محدود شود تا با الگوهای محاسباتی \lr{CNN} هماهنگ شود و به جای آن، منطق اضافی برای پشتیبانی از عملیات خاص برای بارهای کاری یادگیری عمیق (مثل فشرده‌سازی، \lr{Multi Scaling} و ...) به کار رود. همچنین، این معماری‌ها معمولاً از نمایش‌های عددی کوچکتر (یا ترکیبی) پشتیبانی می‌کنند، چون یادگیری عمیق معمولاً با محاسبات با دقت پایین سازگار است \cite{ref2}.
		
		
		معماری \lr{DT-CGRA} \cite{ref4}، \cite{ref3} طراحی \lr{CGRA} با واحدهای پردازشی نسبتاً درشتی دارد که شامل حداکثر سه دستور ضرب و جمع می‌شود. این واحدهای پردازشی همچنین شامل خطوط تأخیر قابل برنامه‌ریزی هستند تا داده‌های زمانی نزدیک به هم راحت‌تر نقشه‌برداری شوند. داده‌ها در داخل \lr{RC}ها از طریق توکن‌ها همگام‌سازی می‌شوند. پشتیبانی از پترن‌های دسترسی رایج در یادگیری عمیق (مثل \lr{stride}، \lr{type} و ...) از طریق واحدهای \lr{Stream-Buffers} های شخصی‌سازی شده که به‌صورت \lr{VLIW} قابل برنامه‌ریزی هستند، فراهم می‌شود و دسترسی به حافظه خارجی را تولید می‌کنند.
		
		همچنین در مثالی دیگر \lr{Cambricon} \cite{ref1} یک معماری \lr{CGRA} است که به طور خاص برای شتاب‌دهی محاسبات شبکه‌های عصبی طراحی شده است. این معماری به ویژه در دستگاه‌های موبایل و سیستم‌های \lr{embedded} استفاده می‌شود و به دلیل مصرف پایین انرژی و عملکرد بالای آن، در اجرای مدل‌های یادگیری عمیق در زمان واقعی مفید است. \lr{Cambricon} برای پردازش‌های پیچیده مانند تشخیص اشیاء و پردازش زبان طبیعی بهینه‌سازی شده است 
	
	
	
	\item \textbf{امنیت:}
	
	\lr{AES}
	یکی از استانداردهای معروف و پرکاربرد برای رمزگذاری داده‌ها است که در بسیاری از سیستم‌های امنیتی استفاده می‌شود. از آنجا که \lr{AES} نیاز به پردازش‌های ریاضی پیچیده‌ای دارد، استفاده از معماری‌های موازی می‌تواند زمان پردازش را به طور چشمگیری کاهش دهد. در \cite{ref5}،  به عنوان یک معماری قابل برنامه‌ریزی و منعطف معرفی می‌شود که می‌تواند عملیات مختلفی از جمله الگوریتم‌های رمزنگاری را با بهره‌وری انرژی بالا و کارایی بالا پردازش کند. چالش اصلی در طراحی سیستم‌های رمزنگاری سخت‌افزاری، دستیابی به تعادل میان سرعت پردازش بالا و انعطاف‌پذیری است. بسیاری از معماری‌ها یا سرعت بالایی دارند ولی انعطاف‌پذیری کمی دارند، یا برعکس. این محدودیت‌ها باعث می‌شود که بهره‌وری سخت‌افزار پایین باشد و استفاده از آن‌ها در کاربردهای چندگانه \lr{(multi-domain)} با مشکلاتی مواجه شود. برای حل این مشکل، نویسندگان به \lr{CGRA} روی آورده‌اند که به دلیل ویژگی‌های خاص خود، می‌تواند هم سرعت پردازش بالا و هم انعطاف‌پذیری خوبی را فراهم کند. \lr{UECP} بر اساس معماری \lr{CGRA} طراحی شده و با استفاده از دو تکنیک بهینه برای افزایش عملکرد و کارایی سخت‌افزار بهینه شده است:
	\begin{itemize}
		\item 
		آرایه پردازش پایپلاین شده $4 \times 4 $ (\lr{PEA}): این آرایه پردازشی پایپلاین‌شده به \lr{UECP} اجازه می‌دهد تا عملیات‌های رمزنگاری را به صورت موازی و با سرعت بالا انجام دهد.
		
		\item 
		واحد حسابی منطقی قابل تنظیم \lr{:(C-ALU)} \lr{C-ALU} به \lr{UECP} اجازه می‌دهد که عملیات‌های ریاضی و منطقی مختلف مورد نیاز برای الگوریتم‌های رمزنگاری را به طور دینامیک و بسته به نیاز برنامه تنظیم کند. این امکان انعطاف‌پذیری بالایی به سیستم می‌دهد تا بتواند به راحتی برای انواع مختلف الگوریتم‌ها، 
	\end{itemize}
	\end{enumerate}
\end{qsolve}



\begin{qsolve}
	\begin{enumerate}
		\item [ ]
		\begin{itemize}
			\item [ ]
			پیکربندی شود.
		\end{itemize}
		
		
		
		\item [3. ]
		\textbf{سلامت: }
		در زمینه بیوانفورماتیک، معماری‌های \lr{CGRA} برای پردازش داده‌های پیچیده و محاسبات سنگین مورد استفاده قرار می‌گیرند که در تحلیل‌های ژنومیک و مدل‌سازی‌های بیولوژیکی کاربرد دارند. یک مثال از این نوع کاربرد، معماری \lr{CGRA} در پردازش داده‌های ژنومیک و شبیه‌سازی‌های بیولوژیکی است. در \cite{ref6} معماری \lr{CGRA} به‌گونه‌ای طراحی شده است که می‌تواند چهار الگوریتم محبوب بیوانفورماتیک را به طور کامل پردازش کند:
		
		\begin{itemize}
			\item 
			\lr{Needleman-Wunsch} برای هم‌ترازی دنباله‌های پروتئینی.
			
			\item 
			\lr{Smith-Waterman} برای هم‌ترازی دنباله‌های \lr{DNA}.
			
			\item 
			\lr{HMMER} برای شبیه‌سازی مدل‌های مارکوف مخفی (\lr{HMM}) در توالی‌های زیستی.
			
			\item 
			\lr{Maximum Likelihood} برای تحلیل‌های فیلوژنتیکی.
		\end{itemize}
		
		این الگوریتم‌ها معمولاً نیاز به پردازش‌های سنگین و پیچیده دارند که با استفاده از معماری \lr{CGRA} می‌توانند به صورت موازی و بهینه پردازش شوند. این پلتفرم طراحی شده به‌طور خاص برای نیازهای محاسباتی این الگوریتم‌ها بهینه شده است و باعث افزایش کارایی و سرعت در پردازش داده‌ها می‌شود.
	\end{enumerate}
\end{qsolve}




\begin{latin}
	\begin{thebibliography}{9}
		\bibitem{ref1}
		J. Chen et al., "Cambricon: An Efficient and Flexible Architecture for Deep Learning," 2016
		\href{https://reconfigdeeplearning.wordpress.com/wp-content/uploads/2017/02/2016-isca_cambricon-an-instruction-set-architecture-for-neural-networks_cyj.pdf}{[Link]}
		
		
		\bibitem{ref2}
		M. Courbariaux, Y. Bengio, and J.-P. David, ‘‘Training deep neural networks with low precision multiplications,’’ 2014, arXiv:1412.7024.
		\href{http://arxiv.org/abs/1412.7024}{[Link]}
		
		\bibitem{ref3}
		X. Fan, H. Li, W. Cao, and L. Wang, ‘‘DT-CGRA: Dual-track coarse grained reconfigurable architecture for stream applications,’’ in Proc. 26th Int. Conf. Field Program. Log. Appl. (FPL), Aug. 2016, pp. 1–9.
		
		
		\bibitem{ref4}
		X. Fan, D. Wu, W. Cao, W. Luk, and L. Wang, ‘‘Stream processing dual track CGRA for object inference,’’ IEEE Trans. Very Large Scale Integr. (VLSI) Syst., vol. 26, no. 6, pp. 1098–1111, Jun. 2018.
		
		
		\bibitem{ref5}
		A. G. et al., "AES Encryption Accelerator with CGRA Architecture," 2017
		\href{https://www.cs.hiroshima-u.ac.jp/Proceedings23/CANDAR%202023/pdfs/CANDAR2023-1SROm9fDeMyoBrKU1vPn44/067000a189/067000a189.pdf}{[Link]}
		
		
		\bibitem{ref6}
		A Coarse-Grained Reconfigurable Processor for Sequencing and Phylogenetic Algorithms in Bioinformatics
		\href{https://ieeexplore.ieee.org/document/6128576}{[Link]}
		
		
	\end{thebibliography} 
\end{latin}