\section{سوال دوم}


شکل زیر یک \lr{DFG} است که می‌بایست به صورت بهینه بر روی یک \lr{2x2 CGRA} نگاشت شود.


\begin{center}
	\includegraphics*[width=0.5\linewidth]{pics/img1.png}
	\captionof{figure}{\lr{DFG}}
	\label{DFG سوال ۲}
\end{center}


هر گره تنها شامل یک عملیات است و شماره هر گره داخل آن درج شده است.

\begin{enumerate}
	\item نحوه نگاشت خود را شرح دهید.
	\begin{qsolve}
		\begin{center}
			\includegraphics*[width=1\linewidth]{pics/Q2.pdf}
			\captionof{figure}{\lr{Optimal Mapping}}
			\label{Optimal Mapping}
		\end{center}
	\end{qsolve}
\end{enumerate}



\begin{enumerate}
	\item [ ]
	\begin{qsolve}
		با توجه با شکل بالا (سمت راست) باتوجه به اینکه گره شماره ۶ به ۴ گره وابسته است، به‌صورت مستقیم نمی‌توان مسئله را حل نمود. برای حل آن گره‌های ۴ و ۵ را باهم ترکیب می‌کنیم و نام آن را \lr{M} می‌گذاریم. با این روش مسئله را می‌توان به روش \lr{N2N} حل نمود. در روش دوم که در شکل سمت چپ آورده شده است محدودیتی نداریم و به سادگی می‌توان مسئله را حل نمود
	\end{qsolve}
	
	
	\item [2. ]
	مقدار \lr{Initiation Interval} را گزارش نمایید.
	
	\begin{qsolve}
		\begin{enumerate}
			\item 
			برای \lr{N2N}: ۴
			\item 
			برای \lr{HyCUBE}: ۳
		\end{enumerate}
	\end{qsolve}
\end{enumerate}