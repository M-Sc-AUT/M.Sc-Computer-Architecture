\section{سوال پنجم}

در این تمرین هدف طراحی و پیاده‌سازی بخشی از یک سیستم پردازش تصویر بی‌درنگ بر روی \lr{Zynq SoC} است. برای انجام این تمرین بایستی مهارت‌های مربوط به نحوه ارتباط بین بخش \lr{PS} (سیستم پردازنده) و \lr{PL} (منطق قابل برنامه‌ریزی) و همچنین نحوه استفاده از رابط میان آنها به عنوان مثال \lr{AXI} برای ارتباط بین \lr{PS} و \lr{PL} مطرح شده در تمرین قبلی را به خوبی فراگرفته باشید.

هدف ایجاد یک هسته برای پردازش تصویر ورودی و تشخیص لبه به صورت بی‌درنگ است. در این تمرین قسمت هسته پردازشی بایستی طراحی شود که یک تصویر را دریافت و خروجی متناظر تشخیص لبه را ایجاد کند. تشخیص لبه یکی از عملیات پایه در پردازش تصویر است که تغییرات ناگهانی در شدت پیکسل‌ها را شناسایی می‌کند. الگوریتم‌های رایج برای تشخیص لبه شامل فیلتر \lr{Sobel}، \lr{Prewitt} و \lr{Canny} هستند. نمونه خروجی تشخیص لبه در تصویر زیر آورده شده است:

\begin{center}
	\includegraphics*[width=0.8\linewidth]{pics/img1.png}
	\captionof{figure}{تشخیص لبه در تصویر}
	\label{تشخیص لبه در تصویر}
\end{center}


در این تمرین بایستی تصویر از قسمت \lr{PS} برای پردازش به قسمت \lr{PL} ارسال شود و نتایج به قسمت \lr{PS} جهت نمایش بازگشت داده شود. برای شبیه‌سازی می‌توان قسمت \lr{PL} را با داده ورودی از طریق \lr{Testbench} مورد آزمایش قرار داد. برای ورودی، از یک تصویر که شماره دانشجویی شما بر روی آن نوشته شده استفاده نمایید. توضیح کامل نحوه پیاده‌سازی و ایجاد ورودی و خروجی‌ها را در گزارش اضافه کنید و همچنین فایل پروژه خود را با فرمت \lr{ZIP} در سامانه بارگذاری کنید.

برای الگو گرفتن از یک کد نمونه می‌توانید از این
\href{https://github.com/JeffreySamuel/canny_edge_detection_in_FPGA/tree/main}{لینک}
 استفاده نمایید.
 
 همچنین الگو گرفتن از کدهای مشابه با ارجاع به منبع، منع ندارد.

