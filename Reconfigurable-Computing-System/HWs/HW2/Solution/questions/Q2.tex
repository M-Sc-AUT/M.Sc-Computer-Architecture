\section{سوال دوم}
تفاوت‌های اصلی بین خانواده‌های \lr{Cyclone} و \lr{Stratix} را توضیح دهید و ذکر کنید در چه شرایطی استفاده از هر کدام مناسب‌تر است؟ همین مقایسه را در خصوص خانواده \lr{Stratix} و \lr{Virtex} نیز انجام دهید. موارد را در داخل مدارک فنی شرکت‌های تولیدکننده مشخص کرده و محل آنها را در گزارش خود بیاورید.



%\begin{center}
%	\includegraphics*[width=0.8\linewidth]{pics/img4.png}
%	\captionof{figure}{\lr{FPGA} یا \lr{ASIC} ؟ مسئله این است.}
%	\label{اف‌پی‌جی‌ای یا ایسیک؟}
%\end{center}


\begin{qsolve}
	عمده تفاوت بین این دو خانواده، جامعه و کارد مورد استفاده از آن‌هاست که آن هم بده‌دلیل ویژگی‌های خاص هر‌یک از این دو تراشه است. برای مثال تراشه‌های خانواده \lr{Cyclone} برای کاربردهای کم‌هزینه و با مصرف انرژی پایین طراحی شده است. به دلیل طراحی کم‌هزینه و اندازه کوچک‌تر، بیشتر در کاربردهای تجاری و صنعتی با پیچیدگی نه‌چندان زیاد مانند کنترل‌کننده‌ها، سیستم‌های ساده شبکه و دستگاه‌های مصرفی استفاده می‌شود. اما در مقابل تراشه‌های خانواده \lr{Stratix} از نظر عملکرد و منابع، در سطح بالاتری نسبت به \lr{Cyclone} قرار دارد و برای کاربردهای پیچیده و توان پردازشی بالا مناسب است. از \lr{Stratix} بیشتر در کاربردهای پیشرفته مانند پردازش سیگنال دیجیتال، شبکه‌های مخابراتی سرعت‌بالا، محاسبات سنگین، و کاربردهای هوافضا و نظامی استفاده می‌شود.
	
	در ادامه به‌طور دقیق با ارجاع به دیتاشیت‌های شرکت‌های سازنده این دو خانواده از تراشه‌ها را مقایسه می‌کنیم.
	
	
	شرکت \lr{Intel} (\lr{Altera} سابق) اسم بلوک منطقی و قابل برنامه‌ریزی اش را \lr{LAB (Logic Array Block)}
	نام‌گذاری کرده است. ساختار ارتباطات بین این بلوک‌ها برای تراشه های هردوخانواده یکسان و به‌صورت زیر است: (صفحه ۱۰ در \cite{ref1} و \cite{ref2})
	
	\begin{center}
		\includegraphics*[width=0.8\linewidth]{pics/img2.png}
		\captionof{figure}{ساختار ارتباطات میان \lr{LAB} ها}
		\label{ساختار ارتباطات میان LAB ها}
	\end{center}


	ساختار \lr{LUT} ها در این دو خانواده متفاوت هست. در \lr{Cyclone} از \lr{LUT} های ۵ ورودی استفاده شده اما در \lr{Stratix} از \lr{LUT} های ۶ ورودی. (صفحه ۱۶ در \cite{ref1} و صفحه ۱۸ در \cite{ref2})
\end{qsolve}




\begin{qsolve}
	برای \lr{Cyclone} به‌صورت زیر:
	\begin{center}
		\includegraphics*[width=0.8\linewidth]{pics/img4.png}
		\captionof{figure}{ساختار \lr{LUT} ها در \lr{Cyclone}}
		\label{ساختار LUT ها در Cyclone}
	\end{center}
	
	
	و برای \lr{Stratix} نیز به‌صورت زیر:
	\begin{center}
		\includegraphics*[width=0.8\linewidth]{pics/img3.png}
		\captionof{figure}{ساختار \lr{LUT} ها در \lr{Stratix}}
		\label{ساختار LUT ها در Stratix}
	\end{center}
	
	
	
	
	از منظر \lr{Memory} این این دو خانواده به شدت متفاوت از یکدیگر هستند. در \lr{Stratix} بلوک‌های پرسرعت حافظه به‌نام \lr{M20K} داریم در‌صورتی که در \lr{Cyclone} تنها \lr{M10K} داریم. مطابق با جدول شماره ۲-۱ در صفحه ۲۵ \cite{ref2} و صفحه ۲۰ از \cite{ref1} این دو خانواده از نظر مجموع حافظه نیز بسیار متفاوت هستند. به‌طوری که کمترین میزان حافظه در یکی از مدل‌های خانواده \lr{Stratix}،
	۱۶ کیلو بیت است درصورتی که بیشترین حجم حافظه در خانواده \lr{Cyclone} تقریبا ۱۴ کیلوبیت است. تصویر این دو جدول در ادامه آورده شده است.
	
	
\end{qsolve}



\begin{qsolve}
	جافظه‌های موجود در \lr{Cyclone}:
	\begin{center}
		\includegraphics*[width=0.8\linewidth]{pics/img6.png}
		\captionof{figure}{حافظه‌های \lr{Cyclone} \cite{ref1}}
		\label{حافظه‌های Cyclone}
	\end{center}
	
	
	جافظه‌های موجود در \lr{Cyclone}:
	\begin{center}
		\includegraphics*[width=0.8\linewidth]{pics/img5.png}
		\captionof{figure}{حافظه‌های \lr{Stratix} \cite{ref2}}
		\label{حافظه‌های Stratix}
	\end{center}
	
	
		از نظر تعداد ضرب‌کننده‌ها نیز در تراشه‌های خانواده \lr{Stratix} بیشتر هستند این مورد در جدول صفحه ۴۲ در \cite{ref1} و صفحه ۴۸ در \cite{ref2} آورده شده است.
\end{qsolve}



\begin{qsolve}
	مقایسه این دو خانواده از نظر تعداد ضرب‌کننده:
	\begin{center}
		\includegraphics*[width=0.8\linewidth]{pics/img8.png}
		\captionof{figure}{ضرب‌کننده‌های \lr{Stratix} \cite{ref2}}
		\label{ضرب‌کننده‌های Cyclone}
	\end{center}
\end{qsolve}


\begin{qsolve}
	\begin{center}
		\includegraphics*[width=0.8\linewidth]{pics/img7.png}
		\captionof{figure}{ضرب‌کننده‌های \lr{Stratix} \cite{ref2}}
		\label{ضرب‌کننده‌های Stratix}
	\end{center}
	
	تراشه‌های خانواده \lr{Virtex} از شرکت \lr{Xilinx} را می‌توان از نظر سرعت و توان پردازشی، هم‌رده تراشه‌های \lr{Stratix} قرار داد. که در ادامه برخی از ویژگی‌های این خانواده را با خانواده \lr{Stratix} مقایسه می‌کنیم.
	
	در تراشه‌های شرکت \lr{Xilinx} نام بلوک‌های قابل برنامه‌ریزی \lr{CLB} است و ساختار آن تقریبا مشابه است با خانواده \lr{Stratix} 
	
	\begin{center}
		\includegraphics*[width=0.5\linewidth]{pics/img9.png}
		\captionof{figure}{ساختار \lr{CLB} ها در \lr{Virtex}}
		\label{ساختار CLB ها در Virtex}
	\end{center}
	
\end{qsolve}





\begin{qsolve}
	ساختار درونی هر \lr{CLB} به‌صورت زیر است:
	
	\begin{center}
		\includegraphics*[width=1\linewidth]{pics/img10.png}
		\captionof{figure}{ساختار درونی \lr{CLB} ها در \lr{Virtex} \cite{ref3}}
		\label{ساختار درونی CLB ها در Virtex}
	\end{center}
	
	
	مطابق با توضیحات موجود در صفحه ۱۱ سند \cite{ref3} ورودی‌های \lr{LUT} ها در این خانواده نیز ۶ عددی هستند. همچنین از نظر مقدار منابع موجود در هر \lr{CLB} تراشه‌های این خانواده به‌صورت زیر تقسیم‌بندی می‌شوند:
\end{qsolve}



\begin{qsolve}
	\begin{center}
		\includegraphics*[width=1\linewidth]{pics/img11.png}
		\captionof{figure}{منابع موجود در هر \lr{CLB} \cite{ref3}}
		\label{منابع موجود در هر CLB}
	\end{center}
	
	همانطور که مشخص است، تعداد منابع اعم از تعداد \lr{Slice}ها، تعداد \lr{LUT}ها و مقدار حافظه تقریبا با تراشه‌های خانواده \lr{Stratix} مشابه است و در مواردی بیشتر است.
	
	
	مطابق با جدول ارائه شده در صفحه ۱۳ سند \cite{ref4} تراشه‌های این خانواده به‌طور میانگین بیش‌از ۵۰۰ بلوک محاسبات سریع \lr{DSP48} را دارند.
	
	
	\begin{center}
		\includegraphics*[width=1\linewidth]{pics/img12.png}
		\captionof{figure}{تعداد بلوک‌های \lr{DSP48} موجود در هر تراشه خانواده \lr{Virtex} \cite{ref4}}
		\label{تعداد بلوک‌های DSP48 موجود در هر تراشه خانواده Virtex}
	\end{center}
	
	که برخلاف تراشه‌های خانواده \lr{Stratix} این بلوک می‌تواند ضرب سریع، جمع و عملیات‌های منطقی را انجام دهد. هر \lr{Slice} از این بلوک به‌صورت زیر است:
\end{qsolve}


\begin{qsolve}
	\begin{center}
		\includegraphics*[width=1\linewidth]{pics/img13.png}
		\captionof{figure}{ساختار درونی بلوک \lr{DSP48}}
		\label{درون بلوک DSP48}
	\end{center}
	
	و درنهایت برای جمع‌بندی می‌توان گفت اگر در کاربردی نیاز به سرعت و محاسبات خیلی سریع نداشته باشیم و منابع زیادی هم نیاز نداشته باشیم، تراشه‌های خانواده \lr{Cyclone} با قیمت مناسب و توان مصرفی کمتر نسبت به دو خانواده دیگر انتخاب مناسبی است. اما اگر نیازمند توان پردازشی بالا و منابع زیادی باشیم می‌توانیم از یکی از خانواده‌های \lr[Stratix] و یا \lr{Virtex} استفاده کنیم. از انجایی که این دو خانواده تا حد زیادی شبیه به هم هستند انتخاب میان این دو تراشه کاملا به کاربرد و نیاز‌های ما از تراشه بستگی دارد.
\end{qsolve}
 


\begin{latin}
	\begin{thebibliography}{9}
		\bibitem{ref1}
		Cyclone® V Device Overview \href{https://www.intel.com/content/www/us/en/docs/programmable/683694/current/cyclone-v-device-overview.html}{[Link]}
		
		\bibitem{ref2}
		Stratix® V Device Handbook
		Volume 1: Device Interfaces and Integration \href{https://www.intel.com/content/www/us/en/docs/programmable/683694/current/cyclone-v-device-overview.html}{[Link]}
		
		\bibitem{ref3}
		7 Series FPGAs
		Configuration
		User Guide \href{https://docs.amd.com/v/u/en-US/ug470_7Series_Config}{[Link]}
		
		\bibitem{ref4}
		Virtex-6 FPGA
		DSP48E1 Slice
		User Guide \href{https://docs.amd.com/v/u/en-US/ug369}{[Link]}
	\end{thebibliography} 
\end{latin}