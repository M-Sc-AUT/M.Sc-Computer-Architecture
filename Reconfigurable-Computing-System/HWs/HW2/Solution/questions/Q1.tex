\section{سوال اول}

با ذکر دلیل بیان کنید جملات زیر صحیح هستند یا خیر.

\begin{enumerate}
	\item 
	خانواده \lr{Cyclone} نسبت به \lr{Stratix} مصرف انرژی کمتری دارد.
	\begin{qsolve}
		\textbf{\textcolor{darkgreen}{درست.}}\\
		تراشه‌های خانواده \lr{Cyclone} برای کاربردهایی طراحی شده که نیاز به مصرف انرژی کمتر و هزینه پایین‌تری دارند. از طرف دیگر تراشه‌های خانواده \lr{Stratix} برای کاربردهای پیشرفته‌تر و پیچیده‌تر مانند پردازش سیگنال دیجیتال، پردازش داده‌های سنگین طراحی شده است که توان محاسباتی بیشتری را برای انجام می‌طلبد و درنتیجه انرژی مصرفی آن نیز بیشتر است.
	\end{qsolve}
	
	
	
	\item 
	معماری کلی تراشه‌های برنامه‌پذیر از تولیدکننده‌ای به تولیدکننده دیگر کاملاً متفاوت است.
	\begin{qsolve}
		\textbf{\textcolor{red}{نادرست.}}\\
		معماری کلی ساخت تراشه‌های برنامه‌پذیر در شرکت‌های مختلف به‌طور کامل نسبت به دیگیری متفاوت نیست. شرکت‌های مختلف بخش‌های کلی تراشه‌را که عمدتا شامل بلوک‌های منطقی و سوئیچ‌ها می‌باشند را (تقریبا) ثابت و مشابه نگه ‌می‌دارند.
	\end{qsolve}
	
	
	
	\item 
	مدل‌های \lr{Cyclone} تولیدی شرکت \lr{Intel} دارای هسته پردازنده \lr{ARM} هستند.
	\begin{qsolve}
		\textbf{\textcolor{darkgreen}{درست.}}\\
		مدل \lr{Cyclone V } دارای هسته \lr{ARM-Cortex-A9} است.
	\end{qsolve}
	
	
	
	\item 
	بلوک‌های منطقی قابل پیکربندی (\lr{CLB}) در خانواده اسپارتان دارای \lr{slice}های مشابه هستند.
	\begin{qsolve}
		\textbf{\textcolor{darkgreen}{درست.}}\\
		در این خانواده \lr{Slice}ها شامل منابعی مانند \lr{LUT} و فلیپ‌فلاپ ها هستند که ساختار مشابهی با هم دارند. البته اگر تعداد \lr{LUT} های هر \lr{Slice} را به‌عنوان جزئی متمایز کننده درنظر نگیریم.
	\end{qsolve}
	
	\item 
	برای ارتباط دو سیستم مبتنی بر اسپارتان \lr{LX25} با سرعت بالا می‌توان از رابط \lr{Gigabyte} استفاده کرد.
	\begin{qsolve}
		\textbf{\textcolor{red}{نادرست.}}\\
		در \lr{FPGA} های سری \lr{Spartan LX}، از جمله \lr{Spartan-3 LX25}، رابط‌های با سرعت بالا مانند \lr{Gigabit Ethernet} به صورت داخلی وجود ندارند. این سری از \lr{FPGA} ها به طور خاص برای کاربردهای کم‌هزینه و با پیچیدگی پایین طراحی شده‌اند و معمولاً برای ارتباطات با سرعت بالا مناسب نیستند، زیرا فاقد \lr{Transceiver} های پرسرعت هستند.
	\end{qsolve}
	
	
	
	
	\item 
بلوک \lr{URAM} در اسپارتان قابل پیکربندی به صورت دسترسی تک کاناله و دوکاناله است.
	\begin{qsolve}
		\textbf{\textcolor{red}{نادرست.}}\\
		در خانواده اسپارتان، بلوک \lr{URAM} ای وجود ندارد. این بلوک‌هادر خانواده‌های پیشرفته‌ای مانند \lr{UltraScale} و \lr{UltraScale+} وجود دارد.
	\end{qsolve}
	
	
	
	
	\item 
خانواده \lr{Artix-7} دارای بیش از ۷۰۰ ضرب‌کننده سخت‌افزاری است.
	\begin{qsolve}
		\textbf{\textcolor{darkgreen}{درست.}}\\
		\lr{FPGA}
		های موجود در این خانواده‌همگی دارای حدودا ۷۰۰۰ (و بیشتر) ضرب‌کننده سخت‌افزاری هستند.
	\end{qsolve}
	
	
	
	
	\item 
بلوک‌های \lr{MLAB} در \lr{Cyclone} برای پیاده‌سازی \lr{FIFO} مناسب نیست.
	\begin{qsolve}
		\textbf{\textcolor{darkgreen}{درست.}}\\
	این بلوک‌ها به طور خاص برای پیاده‌سازی ساختارهای حافظه با تأخیر کم و توان مصرفی پایین طراحی شده‌اند و می‌توانند در ساختارهای \lr{FIFO} از آن‌ها استفاده نمود اما بستگی به اندازه \lr{FIFO} نیز دارد. چرا که برای \lr{FIFO}هایی با اندازه بزرگ معمولا از بلوک‌های \lr{M9K} و \lr{M10K} استفاده می‌شود. اما برای \lr{FIFO} های کوچک مناسب است.
	\end{qsolve}
	
	
	\item 
معماری \lr{FPGA}ها برای داده‌های پردازشی با سایز مختلف مناسب نیست و برای این منظور \lr{GPU}ها کاربرد بیشتری دارند.

	\begin{qsolve}
		\textbf{\textcolor{darkgreen}{درست.}}\\
		در \lr{FPGA} ها معمولا چون قرار است به ازای یک پردازش و محاسبه خاص، واحد سخت‌افزاری طراحی شود، هرچقدر که اندازه ها فیکس باشد از نظر سرعت پردازش و توان مصرفی بهتر عمل می‌کند. اما \lr{GPU} ها ساخته‌شده اند تا محاسبات برداری را به‌صورت موازی انجام بدهند بنابراین برای کاربردهایی که نیاز به تغییر سایز داده و موازی‌سازی بسیار بالایی دارند (مانند شبکه‌های عصبی عمیق یا پردازش تصویر پیچیده)، \lr{GPU}ها معمولاً کاربرد بیشتری دارند.
	\end{qsolve}
	
	
	\item 
	در \lr{Stratix 10} از معماری \lr{LUT} قابل شکستن استفاده شده است که قادر به تامین دو \lr{LUT} با ۳ ورودی و یک \lr{LUT} با ۴ ورودی با ورودی‌های مستقل هستند.
	
	\begin{qsolve}
		\textbf{\textcolor{darkgreen}{درست.}}\\
		براساس اطلاعات موجود در دیتاشیت این خانواده، این امکان وجود دارد.
	\end{qsolve}
	
\end{enumerate}