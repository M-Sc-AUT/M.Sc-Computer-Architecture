\section{سوال دوم}
در یک سیستم ایمنی مرتبط با خودرو نیاز به طراحی یک سیستم ایمنی با قابلیت اطمینان بالا می‌باشد که بایستی دارای امکان به‌روزرسانی الگوریتم ایمنی نیز باشد. همچنین زمان عملکرد سیستم نیز بایستی به صورت \lr{Hard Real-time} باشد. برای طراحی این سیستم در صورت نمونه‌سازی و در صورتی که ۱ میلیون نسخه از آن نیاز باشد استفاده از چه نوع بستر پردازشی را پیشنهاد می‌نمایید؟ برای انجام محاسبات، هزینه‌های مربوط به ساخت معماری پیشنهادهای خود را از اینترنت استخراج نمایید.



\begin{qsolve}
	
	همیشه یکی از مهمترین پاسخ‌ها در ابتدای هر طراحی انتخاب پلتفرم برای آن است. به طوری که آقای \lr{Rajeev Jayaraman} در \textcolor{blue}{[1]} توضیحات مفصلی در این مورد می‌دهد که مطابق با این سوال از برخی از پاسخ‌های ایشان استفاده می‌کنیم.
	
	
\begin{center}
	\includegraphics*[width=0.8\linewidth]{pics/img4.png}
	\captionof{figure}{\lr{FPGA} یا \lr{ASIC} ؟ مسئله این است.}
	\label{اف‌پی‌جی‌ای یا ایسیک؟}
\end{center}

در ابتدا باید این مورد را عنوان کرد که انتخاب پلتفرم برای طراحی به معیار های زیادی بستگی دارد که این سوال به چند مورد از آن‌ها یعنی قابلیت به‌روزرسانی الگوریتم (بازپیکربندی)، تیراژ ساخت، سرعت بالا (\lr{Real-Time}) اشاره کرده است. انتخاب پلتفرم طراحی به همین موارد بسنده نمی‌کند و به موارد دیگری همچون میزان سرعت پردازش مورد نیاز (میزان موازی سازی) برای آن اپلیکیشن خاص، توان مصرفی نیز نیاز است توجه کنیم.

با توجه به موارد گفته شده و فرضیات محدود مسئله، پاسخ بسیار دقیقی را نمی‌توان برای این سوال مطرج کرد اما با درنظر گرفتن یک‌سری فرضیات آن‌را تحلیل می‌کنیم.

از آنجایی که می‌خواهیم عملکرد سیستم به‌صورت \lr{Hard Real-Time} باشد، یه پردازنده‌ای نیاز داریم که یا به‌صورت \lr{FPGA} ها قایلیت موازی سازی بالایی داشته باشد یا مانند \lr{CPU} ها از فرکانس کلاک بالایی برخوردار باشد تا به کاربر احساس عملکرد \lr{Real-Time} بودن را بدهد.

موضوع دومی که مورد توجه ما قرار می‌گیرد بحث توان مصرفی است. چون کاربرد ما خودرو است نمی‌توانیم از \lr{GPU} استفاده کنیم (به دلیل توان مصرفی زیاد) می بایست از تراشه های \lr{Low Power} استفاده نماییم. چون نیاز داریم که بتوانیم الگوریتممان را در زمان‌های مختلف آپدید کنیم، می‌بایست از تراشه ای استفاده کنیم که بتوان این قابلیت را برای ما فراهم کند.

\end{qsolve}


\begin{qsolve}
	 بسته به نوع الگوریتمی که قرار است پیاده سازی شود می‌توان بین \lr{CPU} و \lr{FPGA} انتخاب نمود. اگر به موازی‌سازی های بالا در اپلیکیشنمان نیاز داشته باشیم انتخاب ما باید \lr{FPGA} باشد، درهیر این صورت می‌توان از \lr{CPU} های مرسوم نیز استفاده نمود.
	
	موضوع بعد بحث هزینه است. بر اساس آن‌که قرار است از این تراشه به تعداد یک میلیون قطعه ساخته شود شاید به‌نظر برسد که \lr{FPGA} به‌صرفه نباشد و هزینه آن بسیار گران بشود. اما باید دید که آیا در این \lr{Trade-Off} بین هزینه و سرعت پردازش بالا کدام یک بیشتر به نفع ما و کاربرد ماست.
	
	در مقابل این‌ها \lr{ASIC} ها قرار می‌گیرد که از نظر توان پردازشی، توان مصرفی و سرعت بسار خوب هستند و در تیراژ بالا بسیار ارزان تر از \lr{FPGA} ها و \lr{CPU} ها در‌می‌آید اما قابلیت آپدیت الگوریتم در آن‌ها وجود ندارد.
	
	مگر آن‌که در زمان ساخت از شرکت سازنده درخواست کنیم که یک واحد پردازشی \lr{Reconfigurable} به‌صورت \lr{On Chip} درون \lr{ASIC} ما بگذارند. اینطوری هم در قیمت برای ما به‌صرفه است و هم قابلیت آپدیت کردن الگوریتم را برای ما فراهم می‌کند.
	
	طبق گفته \lr{Rajeev Jayaraman} نمودار تحلیل هزینه‌های \lr{ASIC} در مقایسه با \lr{FPGA} به شکل زیر است. 
	
	\begin{center}
		\includegraphics*[width=0.8\linewidth]{pics/img5.png}
		\captionof{figure}{نمودار هزینه‌های \lr{FPGA} و \lr{ASIC}}
		\label{نمودار هزینه‌های اف‌پی‌جی‌ای و ایسیک}
	\end{center}
	
	
	مقادیر هزینه و واحدها از نمودار حذف شده‌اند زیرا این مقادیر بسته به فناوری پردازش استفاده شده و با گذشت زمان متفاوت هستند. \lr{ASIC‌}ها دارای هزینه‌های مهندسی غیرقابل تکرار (\lr{NRE}) بسیار بالایی هستند که ممکن است به میلیون‌ها دلار برسند، در حالی که هزینه واقعی هر تراشه ممکن است تنها چند سنت باشد. در مورد \lr{FPGA}ها، هیچ هزینه \lr{NRE} وجود ندارد. ما فقط هزینه تراشه \lr{FPGA} را پرداخت می‌کنیم و پولی هم بابت استفاده از نرم‌افزار‌های مربوطه آن نمی‌پردازیم :))) بنابراین، هزینه کل برای \lr{ASIC‌}ها به دلیل هزینه‌های \lr{NRE} بسیار بالا شروع می‌شود، اما شیب آن کمتر است. به این معنی که نمونه‌سازی \lr{ASIC‌}ها در مقادیر کم بسیار پرهزینه است، اما در حجم‌های بالا، هزینه هر واحد بسیار کاهش می‌یابد. در مورد \lr{FPGA}‌ها، هزینه تراشه نسبتاً بالاتر است، بنابراین در حجم‌های زیاد، نسبت به \lr{ASIC}‌ها هزینه بیشتری دارد.
	
	بنابر این می‌توان محاسبات تخمینی زیر را نیز برای یک طراحی مشابه بر روی \lr{FPGA}	و \lr{ASIC} انجام داد.
\end{qsolve}


\begin{qsolve}
	\begin{enumerate}
		\item 
		\textbf{برای \lr{:FPGA}}
		\begin{itemize}
			\item 
			فرض شود یک \lr{FPGA} به قیمت ۵۰ دلار برای هر واحد داریم و قصد تولید ۱ میلیون نسخه را داریم
			
			\item 
			هزینه کل = تعداد نسخه‌ها $\times$ هزینه هرواحد
			
			\item 
			هزینه کل = $50 \times 1000000 = 50000000 $
			
			\item 
			هزینه \lr{NRE} = صفر
			
		\end{itemize}
		
		\item 
		\textbf{برای \lr{:ASIC}}
		\begin{itemize}
			\item 
			فرض شود هزینه \lr{NRE} برای \lr{ASIC} دو میلیون دلار باشد و هزینه‌ی تولید هر واحد \lr{ASIC} پس از پرداخت هزینه‌های \lr{NRE}، ۵ دلار باشد.
		\end{itemize}
		
		\begin{itemize}
			\item 
			هزینه کل = هزینه \lr{NRE} + (تعداد نسخه‌ها $\times$ هزینه هرواحد)
			
			\item 
			هزینه کل = $2000000 + (5 \times 1000000 ) = 7000000$ 
		\end{itemize}
	\end{enumerate}
	
	
		بنابر این برای ۱ میلیون نسخه، هزینه \lr{FPGA} حدود ۷ برابر بیشتر از هزینه تمام شده \lr{ASIC} است.
\end{qsolve}

 


\begin{latin}
	\begin{thebibliography}{9}
		\bibitem{ref1}
		Rajeev Jayaraman, Xilinx Inc, 2001 \href{https://www.doc.ic.ac.uk/~wl/teachlocal/arch/killasic.pdf}{https://www.doc.ic.ac.uk/~wl/teachlocal/arch/killasic.pdf}
		
	\end{thebibliography} 
\end{latin}