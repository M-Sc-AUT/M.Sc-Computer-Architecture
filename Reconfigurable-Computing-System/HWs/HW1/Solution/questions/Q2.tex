\section{سوال دوم}
در یک سیستم ایمنی مرتبط با خودرو نیاز به طراحی یک سیستم ایمنی با قابلیت اطمینان بالا می‌باشد که بایستی دارای امکان به‌روزرسانی الگوریتم ایمنی نیز باشد. همچنین زمان عملکرد سیستم نیز بایستی به صورت \lr{Hard Real-time} باشد. برای طراحی این سیستم در صورت نمونه‌سازی و در صورتی که ۱ میلیون نسخه از آن نیاز باشد استفاده از چه نوع بستر پردازشی را پیشنهاد می‌نمایید؟ برای انجام محاسبات، هزینه‌های مربوط به ساخت معماری پیشنهادهای خود را از اینترنت استخراج نمایید.



\begin{qsolve}
	
	همیشه یکی از مهمترین پاسخ‌ها در ابتدای هر طراحی انتخاب پلتفرم برای آن است. به طوری که آقای \lr{Rajeev Jayaraman} در \textcolor{blue}{[1]} توضیحات مفصلی در این مورد می‌دهد که مطابق با این سوال از برخی از پاسخ‌های ایشان استفاده می‌کنیم.
	
	
\begin{center}
	\includegraphics*[width=0.8\linewidth]{pics/img4.png}
	\captionof{figure}{\lr{FPGA} یا \lr{ASIC} ؟}
	\label{اف‌پی‌جی‌ای یا ایسیک؟}
\end{center}

	
	
	
	از آنجایی که می‌خواهیم عملکرد سیستم به‌صورت \lr{Hard Real-Time} باشد، در ابتدای کار که نیازمند آن هستیم الگوریتم موردنظرمان را چندین بار تست کنیم تا بهترین نتیجه را در خروجی بگیریم، تراشه های \lr{FPGA}
به‌دلیل \lr{Reconfigurable} بودن و سرعت بالا و اجرای موازی الگوریتم‌ها می‌تواند بهترین انتخاب برای طراحی سیستم مورد نظر باشد.

پس از اطمینان از عملکرد الگوریتم و طراحی نهایی، بهتر است که طراحی مورد نظر در تیراژ بالا بر روی \lr{ASIC} انجام شود. تراشه‌های \lr{ASIC}س به دلیل اینکه \lr{Reconfigurable} نیستند و مستقیما برای یک کاربرد خاص طراحی شده اند، به مراتب سرعت بیشتر، توان مصرفی کمتری نسبت به \lr{FPGA} ها دارند. هزینه اولیه طراحی و توسعه \lr{ASIC} بالاست (می‌تواند در مقیاس میلیون دلار باشد)، اما در تولید انبوه، هزینه هر واحد بسیار پایین می‌آید، که آن را گزینه‌ای مناسب برای تولیدات بزرگ مقیاس می‌کند.




طبق گفته \lr{Rajeev Jayaraman} نمودار تحلیل هزینه‌های \lr{ASIC} در مقایسه با \lr{FPGA} به شکل زیر است. 


	
	
\end{qsolve}


\begin{qsolve}
	\begin{center}
		\includegraphics*[width=0.8\linewidth]{pics/img5.png}
		\captionof{figure}{نمودار هزینه‌های \lr{FPGA} و \lr{ASIC}}
		\label{نمودار هزینه‌های اف‌پی‌جی‌ای و ایسیک}
	\end{center}
	
	
	مقادیر هزینه و واحدها از نمودار حذف شده‌اند زیرا این مقادیر بسته به فناوری پردازش استفاده شده و با گذشت زمان متفاوت هستند. \lr{ASIC‌}ها دارای هزینه‌های مهندسی غیرقابل تکرار (\lr{NRE}) بسیار بالایی هستند که ممکن است به میلیون‌ها دلار برسند، در حالی که هزینه واقعی هر تراشه ممکن است تنها چند سنت باشد. در مورد \lr{FPGA}ها، هیچ هزینه \lr{NRE} وجود ندارد. ما فقط هزینه تراشه \lr{FPGA} را پرداخت می‌کنیم و پولی هم بابت استفاده از نرم‌افزار‌های مربوطه آن نمی‌پردازیم :))) بنابراین، هزینه کل برای \lr{ASIC‌}ها به دلیل هزینه‌های \lr{NRE} بسیار بالا شروع می‌شود، اما شیب آن کمتر است. به این معنی که نمونه‌سازی \lr{ASIC‌}ها در مقادیر کم بسیار پرهزینه است، اما در حجم‌های بالا، هزینه هر واحد بسیار کاهش می‌یابد. در مورد \lr{FPGA}‌ها، هزینه تراشه نسبتاً بالاتر است، بنابراین در حجم‌های زیاد، نسبت به \lr{ASIC}‌ها هزینه بیشتری دارد.
	
	بنابر این می‌توان محاسبات تخمینی زیر را نیز برای یک طراحی مشابه بر روی \lr{FPGA}	و \lr{ASIC} انجام داد.
	
	\begin{enumerate}
		\item 
		\textbf{برای \lr{:FPGA}}
		\begin{itemize}
			\item 
			فرض شود یک \lr{FPGA} به قیمت ۵۰ دلار برای هر واحد داریم و قصد تولید ۱ میلیون نسخه را داریم
			
			\item 
			هزینه کل = تعداد نسخه‌ها $\times$ هزینه هرواحد
			
			\item 
			هزینه کل = $50 \times 1000000 = 50000000 $
		
			\item 
			هزینه \lr{NRE} = صفر
		
		\end{itemize}
		
		
		\item 
		\textbf{برای \lr{:ASIC}}
		\begin{itemize}
			\item 
			فرض شود هزینه \lr{NRE} برای \lr{ASIC} دو میلیون دلار باشد و هزینه‌ی تولید هر واحد \lr{ASIC} پس از پرداخت هزینه‌های \lr{NRE}، ۵ دلار باشد.
			
			\item 
			هزینه کل = هزینه \lr{NRE} + (تعداد نسخه‌ها $\times$ هزینه هرواحد)
			
			\item 
			هزینه کل = $2000000 + (5 \times 1000000 ) = 7000000$ 
		\end{itemize}
	\end{enumerate}
	
	بنابر این برای ۱ میلیون نسخه، هزینه \lr{FPGA} حدود ۷ برابر بیشتر از هزینه تمام شده \lr{ASIC} است.
\end{qsolve}

 


\begin{latin}
	\begin{thebibliography}{9}
		\bibitem{ref1}
		Rajeev Jayaraman, Xilinx Inc, 2001 \href{https://www.doc.ic.ac.uk/~wl/teachlocal/arch/killasic.pdf}{https://www.doc.ic.ac.uk/~wl/teachlocal/arch/killasic.pdf}
		
	\end{thebibliography} 
\end{latin}