\section{سوال اول}

با ذکر دلیل بیان کنید جملات زیر صحیح هستند یا خیر.

\begin{enumerate}
	\item 
	در یک پروژه با زمان محدود بهترین راه جهت پیاده‌سازی الگوریتم پردازشی استفاده از تراشه‌های قابل بازپیکربندی است.
	\begin{qsolve}
		\textcolor{red}{نادرست.}\\
		زیرا در زمان محدود بهترین راه برای پیاده‌سازی یک الگوریتم پردازشی استفاده از پردازنده‌های مرسوم موجود در بازار مانند \lr{CPU} است. چون معمولا زمان طراحی و برنامه‌ریزی برای تراشه‌های قابل‌بازپیکربندی مانند \lr{FPGA} بشتر از \lr{CPU} های مرسوم است.
	\end{qsolve}
	
	
	
	\item 
	طراحی‌های مبتنی بر پردازنده‌های همه منظوره و تراشه‌های خاص منظوره، دو انتهای بردار کارآیی و انعطاف‌پذیری هستند.
	\begin{qsolve}
		\textcolor{red}{نادرست.}\\
		به ترتیب نام‌برده شده، دو انتهای بردار انعطاف‌پذیری و کارآیی هستند.
	\end{qsolve}
	
	
	
	\item 
	معماری قابل بازپیکربندی جهت حل مشکل دسترسی حافظه در کامپیوتر فون نیومن ارائه شده است.
	\begin{qsolve}
		\textcolor{red}{نادرست.}\\
		این دلیل هم در کنار مصرف انرژی زیاد کامپیوتر‌های فن نیومن درست است اما دلیل اصلی ارائه معماری بازپیکربندی نزدیک کردن میزان انعطاف پذیری \lr{ASIC} ها با مصرف انرژی به مراتب کمتر نسبت به کامپیوتر فن نیومن به این نوع کامپیوترها بوده است.
	\end{qsolve}
	
	
	
	\item 
	در کاربردهای فضایی و محیط‌های دارای تشعشعات زیاد، تراشه‌های مبتنی بر \lr{FLASH} بهترین گزینه انتخابی هستند.
	\begin{qsolve}
		\textcolor{red}{نادرست.}\\
		استفاده از حافظه فلش در یک محیط پر تشعشع به دلیل ویژگی‌های غیر فرار بودن و چگالی بالای ذخیره‌سازی می‌تواند گزینه مناسبی برای ذخیره‌سازی طولانی‌مدت داده‌های حجیم باشد. این حافظه‌ها با حفظ داده‌ها در مواقع قطع برق، به ویژه در فضاپیماها و ماهواره‌ها که منابع انرژی محدود است، بسیار مفید هستند. همچنین، نسخه‌های مقاوم در برابر تابش فلش، با وجود هزینه بیشتر، می‌توانند تابش‌های محیطی را تحمل کرده و عملکرد پایدار و
		
		
		
%		 تراشه‌های مبتنی بر \lr{FLASH} در برابر (\lr{Single Event Upset (SEU)}) ها و (\lr{Single Event Latchups (SELs)}) های ناشی از تشعشع آسیب پذیرتر هستند و بیشتر دچار \lr{BitFlip} می‌شوند. بنابراین استفاده از \lr{Flash} در این نوع محیط‌ها توسیه نمی‌شود و بهتر است در \lr{SRAM} استفاده شود.
	\end{qsolve}
	
	\begin{qsolve}[ادامه پاسخ]
	مطمئنی در شرایط سخت فراهم کنند، که آن را به انتخابی اقتصادی و کارآمد برای ذخیره‌سازی داده‌های غیر حساس تبدیل می‌کند.
	\end{qsolve}
	
	
	
	
	\item 
	از تراشه‌های مبتنی بر آنتی‌فیوز به دلیل مقاومت مناسب در برابر دمای بالا در کاربردهای صنعتی استفاده می‌شود.
	\begin{qsolve}
		\textcolor{darkgreen}{درست.}\\
		تراشه‌های مبتنی بر آنتی‌فیوز به دلیل معماری ای که دارند، در برابر شرایط سخت، از جمله دمای بالا، مقاومت بهتری دارند. این تراشه‌ها به دلیل ماهیت فیزیکی فرآیند آنتی‌فیوز که شامل ایجاد یک اتصال دائم و غیرقابل تغییر است، در برابر تغییرات محیطی مانند دما یا تشعشعات نسبت به سایر تکنولوژی‌ها پایدارتر هستند.
		
		البته این سوال با این فرض درست است که در آن کاربرد صنعتی مورد استفاده نیازی به بازپیکره‌بندی نداشته باشیم.
	\end{qsolve}
	
	
	
	
	\item 
	تراشه‌های \lr{CGRA} با دارابودن واحدهای خاص منظوره بیشتر، توان کمتری نسبت به \lr{FPGA}ها دارند.
	\begin{qsolve}
		\textcolor{darkgreen}{درست.}\\
		\lr{CGRA} ها
		به دلیل \lr{Granularity} ‌بزرگتر، معمولاً شامل واحدهای پردازشی بزرگ‌تر و خاص‌منظوره‌تر هستند که می‌توانند برای انجام وظایف خاص بازپیکره‌بندی شوند. اما یکی از مزایای CGRAها نسبت به FPGAها این است که مصرف توان کمتری دارند، زیرا این واحدها برای انجام وظایف مشخص بهینه شده‌اند و نیازی به بازپیکره‌بندی در سطح بسیار ریز (\lr{Boolean level (Fine Grain)}) ندارند.
		
	\end{qsolve}
	
	
	
	
	\item 
	استفاده از \lr{FPGA}ها در مقایسه با تولید یک تراشه خاص باعث کاهش هزینه تولید محصول خواهد شد.
	\begin{qsolve}
		\textcolor{red}{نادرست.}\\
		بستگی به مقدار \lr{Cross-over volume} دارد. اگر ساخت تعداد زیادی آیسی مدنظر باشد، هزینه‌های ساخت \lr{ASIC} در تیراژ بالا کمتر از \lr{FPGA } در می‌آید.
	
	\end{qsolve}
	
	
	
	
	\item 
	یک \lr{ASIC} همواره سریع‌تر از یک \lr{FPGA} دستورات پردازشی سطح بالا را انجام خواهد داد.
	\begin{qsolve}
		\textcolor{darkgreen}{درست.}\\
		\lr{FPGA}
		ها به دلیل ساختار \lr{Reconfigurable} ای که دارند، برای آنکه بتوانند پیاده‌سازی طیف وسیع‌تری از الگوریتم ها و کاربرد‌ها را پوشش دهند، از سرعت پردازش کمتری نسبت به \lr{ASIC} ها که به‌طور ویژه و خاص برای انجام یک کار مشخص به‌صورت \lr{Un-Recunfigurable} دیزاین شده اند دارند.
		
	\end{qsolve}
	
	
	
	\item 
	افزایش تعداد ورودی یک \lr{LUT} همواره باعث افزایش سرعت مدار پیاده‌سازی شده با استفاده از آن خواهد شد.
	\begin{qsolve}
		\textcolor{darkgreen}{درست.}\\
		تاخیر کل \lr{FPGA} به‌عنوان تابعی از اندازه \lr{LUT} ها معرفی می‌شود. با افزایش تعداد ورودی‌های \lr{LUT} ها، تعداد حالات پیاده سازی یک \lr{Logic} یکسان زیاد می‌شود و احتمال آنکه \lr{Placement} بهتری برای آن نسبت به \lr{LUT} های کوچکتر پیدا بشود بیشتر است. بنابر این تاخیر همواره کمتر و درنتیجه سرعت بیشتر می‌شود.
		
		
		کل تأخیر FPGA به عنوان تابعی از اندازه LUT شامل تأخیر مسیریابی است
	\end{qsolve}
	
	
	
	
	\item 
	بلوک‌های \lr{UltraRAM} در کنار بلوک‌های \lr{DSP} برای پیاده‌سازی الگوریتم‌های هوش مصنوعی به کمک \lr{FPGA} خانواده \lr{Zynq} بسیار مناسب هستند.
	\begin{qsolve}
		\textcolor{darkgreen}{درست.}\\
		بلوک‌های \lr{UltraRAM} به عنوان حافظه‌هایی با ظرفیت بالا و دسترسی سریع در \lr{FPGA} های خانواده \lr{Zynq} عمل می‌کنند که می‌توانند حجم زیادی از داده‌ها و وزن‌ها را به سرعت خوانده و برای پردازش توسط بلوک‌های \lr{DSP} آماده کنند. \lr{UltraRAM} ها با ارائه حافظه ای با ظرفیت زیاد و تأخیر کم، نقش کلیدی در ذخیره‌سازی و دسترسی سریع به داده‌های مورد نیاز الگوریتم‌های یادگیری ماشین و شبکه‌های عصبی ایفا می‌کند. همچنین بلوک‌های \lr{DSP} نیز برای انجام عملیات های محاسباتی پیچیده مثل ضرب و جمع که در الگوریتم‌های هوش مصنوعی به وفور استفاده می‌شوند، بهینه شده‌اند. بنابر این در کنار یک حافظه سریع برای انجام محاسبات بسیار مناسب هستند.
		
	\end{qsolve}
	
	
\end{enumerate}