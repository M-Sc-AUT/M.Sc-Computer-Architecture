\section{سوال پنجم}
\textbf{آشنایی اولیه با ابزار ویوادو:} در این درس دانشجویان با استفاده از ابزار ویوادو از شرکت زایلینکس به انجام پروژه‌ها خواهند پرداخت. هدف از انجام پروژه‌ها، آشنایی عملی با طراحی توأم بر روی سیستم‌های قابل بازپیکربندی است. برای این منظور در این بخش در ابتدا دانشجویان می‌بایست نرم‌افزار ویوادو را بر روی سیستم خود نصب کنند. سپس با بررسی لینک زیر در ارتباط با نحوه طراحی توأمان و نحوه کار با ابزار آشنایی لازم را کسب کرده و توضیحات موردنیاز را در ارتباط با این نوع طراحی ارائه دهند.

\begin{latin}
	\begin{itemize}
		\item 
		\texttt{\textcolor{magenta}{\href{https://www.youtube.com/watch?v=_odNhKOZjEo}{Link (I)}}}
		
		\item 
		\texttt{\textcolor{magenta}{\href{https://www.youtube.com/watch?v=AOy5l36DroY&t=0s}{Link (II)}}}
	\end{itemize}
\end{latin}

پروژه مشابه موارد یاد شده در دو ویدئو نیز بایستی به همراه پاسخ تمرین‌ها بارگذاری شود. جهت دانلود نرم‌افزار ویوادو از این 
\href{https://downloadly.ir/software/engineering-specialized/xilinx-vivado-design-suite/}{لینک}
استفاده نمایید. نسخه پیشنهادی ۲۰۲۰.۲ به بعد می‌باشد. به دلیل مشکل احتمالی در فعال‌ساز بهتر است از نسخه ۲۰۲۴ استفاده نشود. \newline \newline


%
%
%
%همانطور که در ویدئو نیز بیان شد، هدف در این قسمت، طراحی \lr{Co-Design} است. بدین منظور، برای طراحی یک گیت \lr{NAND} ساده، گیت \lr{AND} را با استفاده از \lr{Logic Block} های \lr{FPGA} طراحی می‌کنیم و ماژول \lr{NOT} را در هسته پردازشی یعنی \lr{CPU} طراحی کرده و اتصالات بین این دو طراحی را برقرار می‌کنیم.
%
%
%ذکر این نکته الزامی است که در این تمرین ما از بورد \texttt{EBAZ4205} که تراشه موجود بر روی آن \texttt{\lr{Zynq 7000}} است استفاده نمودیم. این آیسی در نرم‌افزار \lr{Vivado} با پارت‌نامبر \texttt{xc7z010clg400-3} شناخته می‌شود. تصویر این برد در «شکل \ref{fig:EBAZ}» آورده شده است:
%
%
%
%\begin{center}
%	\includegraphics*[width=0.7\linewidth]{pics/img2.jpg}
%	\captionof{figure}{بورد مورد استفاده در این تمرین}
%	\label{fig:EBAZ}
%\end{center}
%
%همچنین فایل \lr{Constrain} مربوط به این بورد را می‌توان از 
%\href{https://github.com/xjtuecho/EBAZ4205/blob/master/Development/EBAZ4205.xdc}{اینجا}
%دانلود کرد.
%
%
%در ابتدا پس از نصب نرم‌افزار و انتخاب آیسی، طراحی سمت \lr{PL} را انجام می‌دهیم. در این قسمت صرفا یک گیت \lr{AND} را طراحی می‌کنیم. کد نوشته شده برای ماژول \lr{AND} به‌صورت زیر است:
%
%\begin{latin}
%\begin{lstlisting}[label=case_sens, caption=AND Module for PL]
%library IEEE;
%use IEEE.STD_LOGIC_1164.ALL;
%
%
%entity NAND_gate is
%	Port ( a, b: in std_logic;
%		   y: out std_logic );
%end NAND_gate;
%
%architecture Behavioral of NAND_gate is
%
%begin
%
%	y <= a and b;
%
%end Behavioral;
%
%\end{lstlisting} 
%\end{latin}
%
%پس اط طراحی ماژول \lr{AND} می‌بایست نحوه \lr{Interconnection} سمت \lr{PL} و \lr{PS} را در درون تراشه برقرار کنیم.
%
%بدین منظور از قسمت طراحی دیاگرامی \lr{Vivado} پردازنده \lr{ZYNQ} را انتخاب می‌کنیم و همچنین از ماژول \lr{AND} خودمان نیز یک بلوک می‌سازیم. 
%
%با استفاده از بلوک‌های \texttt{\lr{AXI  GPIO}} می‌توانیم ارتباطات بین \lr{PL} و \lr{PS} را برقرار کنیم.
%
%درنهایت سیستم طراحی شده به‌صورت «شکل \ref{fig:vivado design}» می‌شود. 
%
%\begin{center}
%	\includegraphics*[width=1\linewidth]{pics/img3.png}
%	\captionof{figure}{طراحی \lr{PL} و \lr{PS} انجام شده}
%	\label{fig:vivado design}
%\end{center}
%
%پس از تکمیل شدن طراحی، می‌بایست گیت \lr{NOT} را نیز به‌صورت نرم‌افزاری (به زبان \texttt{c}) طراحی کنیم و سپس طراحی را سنتز نهایی کنیم.
%
%برای انجام این کار، نرم‌افزار \lr{Vitis Clasic} را اجرا می‌کنیم و یک \lr{Application Project} جدید می‌سازیم و فایل \texttt{\lr{.xsa}} ساخته شده در  مرحله قبل را به این پروژه اضافه می‌کنیم.
%
%پس از ایجاد پروژه کد گیت \lr{NOT} را به‌صورت زیر می‌نویسیم:
%
%\begin{latin}
%\begin{lstlisting}[label=case_sens, caption=AND Module for PL]
%#include <stdio.h>
%// #include "platform.h"
%#include "xgpio.h"
%#include "xparameters.h"
%#include "xil_printf.h"
%
%int main()
%{
%	init_platform();
%	
%	XGpio input, output;
%	int a;
%	int y;
%	
%	XGpio_Initialize(&input, XPAR_AXI_GPIO_0_DEVICE_ID);
%	XGpio_Initialize(&output, XPAR_AXI_GPIO_1_DEVICE_ID);
%	
%	XGpio_SetDataDirection(&input, 1, 1);
%	XGpio_SetDataDirection(&output, 1, 0);
%	
%	/* print("debug the code"); */
%	
%	while(1)
%	{
%		a = XGpio_DiscreteRead(&input, 1);
%		
%		if(a == 1)
%		{
%			y = 0;
%		}
%		else
%		{
%			y = 1;
%		}
%		
%		XGpio_DiscreteWrite(&output, 1, y);
%	}
%	
%	cleanup_platform();
%	return 0;
%}
%\end{lstlisting} 
%\end{latin}
%
%
%
%سپس بعد از سنتر کد، با قرار دادن فایل \lr{Bitstream} ایجاد شده در مرحله \lr{PL} به عنوان فایل پروگرام، طراحی \lr{Integrate} شده را بر روی بردمان پروگرم می‌کنیم.
%
%
%
