\section{سوال پنجم}
\textbf{آشنایی اولیه با ابزار ویوادو:} در این درس دانشجویان با استفاده از ابزار ویوادو از شرکت زایلینکس به انجام پروژه‌ها خواهند پرداخت. هدف از انجام پروژه‌ها، آشنایی عملی با طراحی توأم بر روی سیستم‌های قابل بازپیکربندی است. برای این منظور در این بخش در ابتدا دانشجویان می‌بایست نرم‌افزار ویوادو را بر روی سیستم خود نصب کنند. سپس با بررسی لینک زیر در ارتباط با نحوه طراحی توأمان و نحوه کار با ابزار آشنایی لازم را کسب کرده و توضیحات موردنیاز را در ارتباط با این نوع طراحی ارائه دهند.

\begin{latin}
	\begin{itemize}
		\item 
		\texttt{\textcolor{magenta}{\href{https://www.youtube.com/watch?v=_odNhKOZjEo}{Link (I)}}}
		
		\item 
		\texttt{\textcolor{magenta}{\href{https://www.youtube.com/watch?v=AOy5l36DroY&t=0s}{Link (II)}}}
	\end{itemize}
\end{latin}

پروژه مشابه موارد یاد شده در دو ویدئو نیز بایستی به همراه پاسخ تمرین‌ها بارگذاری شود. جهت دانلود نرم‌افزار ویوادو از این 
\href{https://downloadly.ir/software/engineering-specialized/xilinx-vivado-design-suite/}{لینک}
استفاده نمایید. نسخه پیشنهادی ۲۰۲۰.۲ به بعد می‌باشد. به دلیل مشکل احتمالی در فعال‌ساز بهتر است از نسخه ۲۰۲۴ استفاده نشود.