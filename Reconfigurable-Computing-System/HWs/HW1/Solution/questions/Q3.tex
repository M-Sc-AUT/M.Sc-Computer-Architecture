\section{سوال سوم}
می‌خواهیم مدار زیر را یک بار با \lr{LUT}های ۳ ورودی و بار دیگر با \lr{LUT}های ۴ ورودی پیاده‌سازی کنیم به طوری که در هر حالت تعداد \lr{LUT}های مورد استفاده کمینه باشد.


\begin{center}
	\includegraphics*[width=0.6\linewidth]{pics/img1.png}
	\captionof{figure}{مدار مورد نظر}
\end{center}



\begin{qsolve}
	تابع بولی خروجی به‌صورت زیر محاسبه می‌شود:
	$$ f=(A'CD') + (AB'CD) + (A'BD) + (ABC') $$
	
	همچنین جدول درستی این تابع نیز به‌صورت زیر محاسبه می‌شود:
\end{qsolve}


\begin{qsolve}
	\begin{latin}
		\[
		\begin{array}{|c|c|c|c|c|c|c|c|>{\columncolor{yellow}}c|}
			\hline
			A & B & C & D & A' \cdot C \cdot D' & A \cdot B' \cdot C \cdot D & A' \cdot B \cdot D & A \cdot B \cdot C' & f \\
			\hline
			0 & 0 & 0 & 0 & 0 & 0 & 0 & 0 & 0 \\
			0 & 0 & 0 & 1 & 0 & 0 & 0 & 0 & 0 \\
			0 & 0 & 1 & 0 & 1 & 0 & 0 & 0 & 1 \\
			0 & 0 & 1 & 1 & 0 & 0 & 0 & 0 & 0 \\
			0 & 1 & 0 & 0 & 0 & 0 & 0 & 0 & 0 \\
			0 & 1 & 0 & 1 & 0 & 0 & 1 & 0 & 1 \\
			0 & 1 & 1 & 0 & 1 & 0 & 0 & 0 & 1 \\
			0 & 1 & 1 & 1 & 0 & 0 & 1 & 0 & 1 \\
			1 & 0 & 0 & 0 & 0 & 0 & 0 & 0 & 0 \\ 
			1 & 0 & 0 & 1 & 0 & 0 & 0 & 0 & 0 \\
			1 & 0 & 1 & 0 & 0 & 0 & 0 & 0 & 0 \\
			1 & 0 & 1 & 1 & 0 & 1 & 0 & 0 & 1 \\
			1 & 1 & 0 & 0 & 0 & 0 & 0 & 1 & 1 \\
			1 & 1 & 0 & 1 & 0 & 0 & 0 & 1 & 1 \\
			1 & 1 & 1 & 0 & 0 & 0 & 0 & 0 & 0 \\
			1 & 1 & 1 & 1 & 0 & 0 & 0 & 0 & 0 \\
			\hline
		\end{array}
		\]
	\end{latin}
	
	
	از آنجایی که تابع ۴ ورودی است، برای پیاده‌سازی آن با استفاده از \lr{LUT}، به یک \lr{LUT}، ۴ ورودی نیاز داریم. مقادیر خروجی $f$ در سلول‌های \lr{SRAM} ذخیره می‌شوند و به ازای ورودی‌های مختلف، خروجی های متناظر با آن ورودی را مطابق با جدول درستی نوشته شده می‌دهند. مدار طراحی شده به‌صورت زیر است:
	
	\begin{center}
		\includegraphics*[width=0.42\linewidth]{pics/Q3.pdf}
		\captionof{figure}{تابع با \lr{LUT}، ۴ ورودی}
	\end{center}
\end{qsolve}


\begin{qsolve}
	برای طراحی همین تابع با استفاده از \lr{LUT} های ۳ ورودی، جدول درستی را از وسط نصف می‌کنیم و خروجی های متناظر با ورودی‌های \lr{BCD} را به ورودی یک \lr{LUT}، ۳ ورودی می‌دهیم. مطابق با طراحی زیر.
	
	\begin{center}
		\includegraphics*[width=0.42\linewidth]{pics/Q3_b.pdf}
		\captionof{figure}{تابع با \lr{LUT} های ۳ ورودی}
	\end{center}
	
	همچنین می‌توان به‌جای \lr{LUT} آخر، از یک \lr{MUX} دو ورودی استفاده نمود که خط \lr{Select} آن به \lr{A} متصل است.
	
		
	
\end{qsolve}