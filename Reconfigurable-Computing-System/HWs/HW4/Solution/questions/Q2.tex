\section{سوال دوم}




در کلاس درس، مدار زیر را با هدف حداقل کردن تأخیر به صورت دستی روی \lr{LUT}های ۴ ورودی نگاشت کرده‌اید. الگوریتم \lr{FlowMap} را روی این گراف اجرا کنید و مراحل آن را نشان دهید و نتیجه نگاشت را رسم کنید. هر مستطیل نماینده یک گیت است.


\begin{center}
	\includegraphics*[width=0.7\linewidth]{pics/img1.png}
	\captionof{figure}{شکل سوال ۲}
	\label{شکل سوال ۲}
\end{center}


\begin{qsolve}
	\begin{center}
		\includegraphics*[width=0.8\linewidth]{pics/img6.png}
	\end{center}
\end{qsolve}
\newpage

\begin{qsolve}
	\begin{center}
		\includegraphics*[width=0.8\linewidth]{pics/img7.png}
	\end{center}
	
	\begin{center}
		\includegraphics*[width=0.8\linewidth]{pics/img8.png}
	\end{center}
	
	\begin{center}
		\includegraphics*[width=0.8\linewidth]{pics/img9.png}
	\end{center}
\end{qsolve}
\newpage


\begin{qsolve}
	\begin{center}
		\includegraphics*[width=0.8\linewidth]{pics/img10.png}
	\end{center}
	
	\begin{center}
		\includegraphics*[width=0.8\linewidth]{pics/img11.png}
	\end{center}
\end{qsolve}
\newpage

\begin{qsolve}
	\begin{center}
		\includegraphics*[width=0.8\linewidth]{pics/img12.png}
	\end{center}
\end{qsolve}