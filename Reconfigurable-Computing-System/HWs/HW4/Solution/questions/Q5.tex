\section{سوال پنجم}


مفهوم برش \lr{k-feasible} در نگاشت فناوری \lr{FPGA} را توضیح دهید و مزایای استفاده از آن در بهینه‌سازی طراحی را در یک پاراگراف شرح دهید.

\begin{qsolve}
	برش  \lr{$k-$feasible cut} در نگاشت فناوری \lr{FPGA} به مفهومی اشاره دارد که در آن یک گره در شبکه بولین به همراه تمام گره‌های پیشین آن به یک برش تقسیم می‌شود، به‌طوری که تعداد گره‌های موجود در این برش حداکثر $k$ باشد. این مفهوم زمانی کاربرد دارد که طراحی مدار برای \lr{FPGA}های مبتنی بر \lr{LUT} انجام می‌شود، جایی که هر \lr{LUT} می‌تواند حداکثر $k-$ورودی داشته باشد. در این روش، گره‌های موجود در برش به ورودی‌های یک \lr{LUT} نگاشت می‌شوند.
	
	از مزایا مفهوم می‌توان به موارد زیر اشاره نمود:
	
	استفاده از برش $k-$قابل قبول در نگاشت فناوری موجب بهینه‌سازی تأخیر و کاهش عمق مدار می‌شود، زیرا الگوریتم می‌تواند به‌طور موثری مسیرهای بحرانی را شناسایی و نگاشت کند. این روش همچنین به دلیل محدود کردن تعداد گره‌ها در برش، بهره‌وری منابع \lr{FPGA} را افزایش می‌دهد و استفاده بهینه از \lr{LUT}ها را ممکن می‌سازد. به‌علاوه، الگوریتم‌های مبتنی بر \lr{$k-$feasible cut} مانند \lr{FlowMap} به‌طور کارآمد و در زمان چندجمله‌ای عمل کرده و راه‌حل‌های بهینه را با تضمین درستی و عملکرد تولید می‌کنند.
\end{qsolve}