\section{سوال ششم - پروژه عملی}


در ادامه پروژه قبلی دو لایه مخفی کاملاً متصل را به سیستم خود متصل کنید. علاوه بر این یک لایه خروجی با ۱۰ نورون نیز برای خروجی شبکه در نظر گرفته و به شبکه متصل شود.



\begin{enumerate}
	\item عملکرد شبکه کاملاً متصل را به صورت مستقل بررسی کنید.
	\item در صورتی که شبکه مشابه در پایتون آموزش داده شده و وزن‌های آن برای تست شبکه در نظر گرفته شود، ۱۰\% امتیاز بیشتر برای بخش پروژه در نظر گرفته می‌شود.
	\item در صورتی که کل شبکه (شامل لایه‌های کانولووشن و کاملاً متصل) در پایتون آموزش داده شده و وزن‌های آن برای تست شبکه در نظر گرفته شود ۲۰\% امتیاز بیشتر برای بخش پروژه در نظر گرفته می‌شود.
\end{enumerate}

	
	در صورت انجام «۲» یا «۳» نیازی به انجام بخش «۱» نمی‌باشد.
	

\begin{qsolve}
	در این پروژه قصد داریم تا پروژه‌های قبلی را با اضافه نمودن یک لایه \lr{Fully Connected} به آن تکمیل کرده و یک شبکه عصبی \lr{CNN} را پیاده‌سازی کنیم. بدین منظور پروژه را به دو فاز تقسیم می‌کنیم:
	\begin{enumerate}
		\item فاز نرم‌افزاری
		\item فاز سخت‌افزاری
	\end{enumerate}
	
	\paragraph{فاز نرم‌افزاری:}
	
	در مرحله اول ابتدا به‌صورت نرم افزاری شبکه مورد نظر را تعریف کرده و آن را با داده‌های مجموعه داده \lr{MNIST} آموزش می‌دهیم تا از وزن‌های آن در فاز سخت افزاری استفاده کنیم.
	
	بدین منظور شبکه ای با معماری زیر را تعریف می‌کنیم:
	
	\begin{center}
		\includegraphics*[width=0.6\linewidth]{pics/img2.pdf}
		\captionof{figure}{معماری شبکه مورد نظر}
		\label{معماری شبکه مورد نظر}
	\end{center}

	کد نوشته شده برای پیاده سازی شبکه به‌صورت زیر است:
\end{qsolve}

\begin{qsolve}
	
\end{qsolve}
	
\begin{latin}
\begin{lstlisting}[language=Python,caption={Model Definition}]
edge_out = np.sqrt(np.power(pre_x, 2) + np.power(pre_y, 2))
edge_out = (edge_out / np.Max(edge_out)) * 255
\end{lstlisting}
\end{latin}








