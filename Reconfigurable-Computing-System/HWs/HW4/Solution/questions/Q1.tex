\section{سوال اول}



%\textbf{\textcolor{red}{نادرست.}}\\
%\textbf{\textcolor{darkgreen}{درست.}}\\


با ذکر دلیل بیان کنید جملات زیر صحیح هستند یا خیر.
\begin{itemize}
	\item نگاشت فناوری (\lr{technology mapping}) می‌تواند بر اساس نوع شبکه ورودی به دو دسته ترکیبی یا ترتیبی طبقه‌بندی شود.
	
	\begin{qsolve}
		\textbf{\textcolor{darkgreen}{درست.}}\\
		این گزاره درست است. بر اساس آنکه شبکه ورودی چه باشد نیاز است تا مشخص شود که فرایند نگاشت قرار است از چه نوعی باشد تا بر اساس آن منابع را اختصاص دهد.
	\end{qsolve}
	
	
	\item هدف اصلی نگاشت فناوری \lr{FPGA} فقط کمینه‌سازی مساحت اشغال شده توسط جداول جستجو است.
	\begin{qsolve}
		\textbf{\textcolor{red}{نادرست.}}\\
		علاوه بر کمینه‌سازی مساحت (یا تعداد \lr{LUT}) های مصرفی، کمینه‌سازی تاخیر سیگنال ها (افزایش سرعت)، افزایش قابلیت مسریابی (\lr{Routability}) و کاهش توان مصرفی نیز حزء اهداف نگاشت فناوری است.
	\end{qsolve}
	
	
	
	\item نگاشت فناوری \lr{FPGA} عمدتاً از جداول جستجو (\lr{LUT}) برای عملیات خود استفاده می‌کند و فقط شامل نگاشت \lr{LUT} است.
	\begin{qsolve}
		\textbf{\textcolor{darkgreen}{درست.}}\\
		چون در \lr{FPGA}‌ها \lr{Logic Element}ها متشکل است از \lr{LUT}ها، بنابر این در فرایند نگاشت فناوری در \lr{FPGA}ها فقط از \lr{LUT}ها استفاده می‌شود.
	\end{qsolve}
	
	
	
	
	\item شبیه‌سازی پس از چیدمان (\lr{post-layout}) اطلاعات کمتری نسبت به شبیه‌سازی قبل از سنتز ارائه می‌دهد.
	\begin{qsolve}
		\textbf{\textcolor{red}{نادرست.}}\\
		در شبیه‌سازی پس از چیدمان، اطلاعات کامل طرح (شامل طول سیم‌ها، تعداد سوییچ‌های موجود در مسیر)، تأخیرهای دقیق (حداکثر فرکانس کلاک) درنظر گرفته می‌شوند. این اطلاعات در شبیه‌سازی قبل از سنتز وجود ندارد، زیرا شبیه‌سازی قبل از سنتز مبتنی بر توصیف منطقی مدار (\lr{RTL}) است و فاقد اطلاعات فیزیکی دقیق است.
	\end{qsolve}
	
	
	
	
	\item \lr{Chortle-d} برای بهینه‌سازی مساحت طراحی شده است.
	\begin{qsolve}
		\textbf{\textcolor{darkgreen}{درست.}}\\
		این الگوریتم از یک رویکرد مبتنی بر \lr{(Directed Acyclic Graph)} \lr{DAG} استفاده می‌کند تا گره‌های منطقی مدار را به گره‌هایی با اندازه‌های کوچک‌تر (\lr{LUT}های کمتر در \lr{FPGA}) نگاشت کند. در این فرآیند، تمرکز اصلی بر کاهش تعداد گره‌های نهایی (که به معنای کاهش مساحت است) می‌باشد.
	\end{qsolve}
	
	
	
	
	\item الگوریتم نگاشت ترتیبی می‌تواند فلیپ‌فلاپ‌ها را در طول فرآیند نگاشت جابجا کند.
	\begin{qsolve}
		\textbf{\textcolor{darkgreen}{درست.}}\\
		قابلیت جابجایی فلیپ‌فلاپ‌ها در نگاشت ترتیبی (با استفاده از \lr{Retiming}) یکی از ابزارهای اصلی بهینه‌سازی است و امکان طراحی‌های کاراتر و بهینه‌تر را فراهم می‌آورد.
		
	\end{qsolve}
	
	
	
	
	\item الگوریتم \lr{FlowMap} تأخیر سیگنال‌ها را در طراحی‌های نگاشت شده حداقل می‌کند.
	\begin{qsolve}
		\textbf{\textcolor{darkgreen}{درست.}}\\
		\lr{FlowMap} گره‌های یک مدار منطقی را به گره‌های کوچک‌تر (مانند \lr{LUT}ها در \lr{FPGA}) تقسیم می‌کند، به گونه‌ای که طولانی‌ترین مسیر بحرانی (\lr{Critical Path}) کمترین تأخیر ممکن را داشته باشد.
	\end{qsolve}
	
	
	
	\item بهینه‌سازی برای مساحت در نگاشت منجر به کاهش تأخیر نیز می‌شود.
	\begin{qsolve}
		\textbf{\textcolor{red}{نادرست.}}\\
		این گزاره هم می‌تواند درست باشد و هم نادرست. اگر بهینه‌سازی برای مساحت به معنای کاهش تعداد منابع سخت‌افزاری مورد استفاده (مانند تعداد LUTها یا گیت‌ها) باشد گزاره \textcolor{red}{نادرست} است. اما اگر کاهش مساحت به معنای حذف منطق غیرضروری یا ساده‌تر کردن مدار باشد، مسیرهای بحرانی نیز ممکن است کوتاه‌تر شوند، که منجر به کاهش تأخیر می‌شود و گزاره \textcolor{darkgreen}{درست} است.
	\end{qsolve}
	
	
	
	\item کارایی مسیریابی مستقل از جایابی در طراحی‌های \lr{FPGA} است.
	\begin{qsolve}
		\textbf{\textcolor{red}{نادرست.}}\\
		طراحی‌ها درون \lr{FPGA} به شدت به \lr{Placement} وابسته است. تصمیمات مربوط به جایابی می‌توانند تأثیر زیادی بر کارایی مسیریابی داشته باشند.
	\end{qsolve}
	
	
	\item در شبیه‌سازی تبرید (\lr{simulated annealing})، کاهش هزینه همیشه منجر به پذیرش یک حرکت می‌شود.
	\begin{qsolve}
		\textbf{\textcolor{darkgreen}{درست.}}\\
		در شبیه‌سازی تبرید، کاهش هزینه ($\Delta cost < 0$) همیشه منجر به پذیرش حرکت می‌شود، زیرا هدف الگوریتم یافتن نقطه بهینه با کمترین هزینه است. این ویژگی یکی از اصول اساسی این الگوریتم است که به آن اجازه می‌دهد به‌تدریج به سمت بهینه‌سازی حرکت کند.
	\end{qsolve}
	
	
	\item تابع هزینه در \lr{VPR} بر اساس طول مسیر و تراکم می‌باشد.
	\begin{qsolve}
		\textbf{\textcolor{darkgreen}{درست.}}\\
		تابع هزینه در \lr{VPR} یک ترکیب وزن‌دار از طول مسیر و تراکم است. این ترکیب به طراح اجازه می‌دهد که بسته به نیاز، روی کاهش تأخیر یا جلوگیری از تراکم بیشتر تمرکز کند.
	\end{qsolve}
	
\end{itemize}












%\begin{center}
%	\includegraphics*[width=0.5\linewidth]{pics/img1.png}
%	\captionof{figure}{\lr{DFG}}
%	\label{DFG سوال ۲}
%\end{center}