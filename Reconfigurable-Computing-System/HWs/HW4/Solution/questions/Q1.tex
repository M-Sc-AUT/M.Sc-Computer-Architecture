\section{سوال اول}



%\textbf{\textcolor{red}{نادرست.}}\\
%\textbf{\textcolor{darkgreen}{درست.}}\\


با ذکر دلیل بیان کنید جملات زیر صحیح هستند یا خیر.
\begin{itemize}
	\item نگاشت فناوری (\lr{technology mapping}) می‌تواند بر اساس نوع شبکه ورودی به دو دسته ترکیبی یا ترتیبی طبقه‌بندی شود.
	
	\begin{qsolve}
		\textbf{\textcolor{red}{نادرست.}}\\
	\end{qsolve}
	
	
	\item هدف اصلی نگاشت فناوری \lr{FPGA} فقط کمینه‌سازی مساحت اشغال شده توسط جداول جستجو است.
	\begin{qsolve}
		\textbf{\textcolor{red}{نادرست.}}\\
	\end{qsolve}
	
	
	
	\item نگاشت فناوری \lr{FPGA} عمدتاً از جداول جستجو (\lr{LUT}) برای عملیات خود استفاده می‌کند و فقط شامل نگاشت \lr{LUT} است.
	\begin{qsolve}
		\textbf{\textcolor{red}{نادرست.}}\\
	\end{qsolve}
	
	
	
	
	\item شبیه‌سازی پس از چیدمان (\lr{post-layout}) اطلاعات کمتری نسبت به شبیه‌سازی قبل از سنتز ارائه می‌دهد.
	\begin{qsolve}
		\textbf{\textcolor{red}{نادرست.}}\\
	\end{qsolve}
	
	
	
	
	\item \lr{Chortle-d} برای بهینه‌سازی مساحت طراحی شده است.
	\begin{qsolve}
		\textbf{\textcolor{red}{نادرست.}}\\
	\end{qsolve}
	
	
	
	
	\item الگوریتم نگاشت ترتیبی می‌تواند فلیپ‌فلاپ‌ها را در طول فرآیند نگاشت جابجا کند.
	\begin{qsolve}
		\textbf{\textcolor{red}{نادرست.}}\\
	\end{qsolve}
	
	
	
	
	\item الگوریتم \lr{FlowMap} تأخیر سیگنال‌ها را در طراحی‌های نگاشت شده حداقل می‌کند.
	\begin{qsolve}
		\textbf{\textcolor{red}{نادرست.}}\\
	\end{qsolve}
	
	
	
	\item بهینه‌سازی برای مساحت در نگاشت منجر به کاهش تأخیر نیز می‌شود.
	\begin{qsolve}
		\textbf{\textcolor{red}{نادرست.}}\\
	\end{qsolve}
	
	
	\item کارایی مسیریابی مستقل از جایابی در طراحی‌های \lr{FPGA} است.
	\begin{qsolve}
		\textbf{\textcolor{red}{نادرست.}}\\
	\end{qsolve}
	
	
	\item در شبیه‌سازی تبرید (\lr{simulated annealing})، کاهش هزینه همیشه منجر به پذیرش یک حرکت می‌شود.
	\begin{qsolve}
		\textbf{\textcolor{red}{نادرست.}}\\
	\end{qsolve}
	
	
	\item تابع هزینه در \lr{VPR} بر اساس طول مسیر و تراکم می‌باشد.
	\begin{qsolve}
		\textbf{\textcolor{red}{نادرست.}}\\
	\end{qsolve}
	
\end{itemize}












%\begin{center}
%	\includegraphics*[width=0.5\linewidth]{pics/img1.png}
%	\captionof{figure}{\lr{DFG}}
%	\label{DFG سوال ۲}
%\end{center}