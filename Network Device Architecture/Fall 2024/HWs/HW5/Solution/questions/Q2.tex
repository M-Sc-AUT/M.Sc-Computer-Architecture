\section{سوال دوم}

در یک سوییچ با \lr{Speedup} برابر با ۳ و سرعت خط ورودی ۴۰ گیگابیت بر ثانیه:

\begin{enumerate}
	\item 
	حداقل سرعت باس داخلی برای پشتیبانی از ۱۶ پورت چقدر باید باشد؟
	
	\begin{qsolve}
			
			برای محاسبه حداقل سرعت باس داخلی، از فرمول زیر استفاده می‌کنیم:
			\[
			\text{\lr{Internal Bus Speed}} = \text{\lr{Speedup}} \times \text{\lr{Aggregate Input Rate}}
			\]
			که در آن، \lr{\texttt{Aggregate Input Rate}} به صورت زیر محاسبه می‌شود:
			\[
			\text{\lr{Aggregate Input Rate}} = \text{\lr{Number of Ports}} \times \text{\lr{Line Rate}}
			\]
			
			با جایگذاری مقادیر:
			\[
			\text{\lr{Aggregate Input Rate}} = 16 \times 40 \, \text{\lr{Gbps}} = 640 \, \text{\lr{Gbps}}
			\]
			\[
			\text{\lr{Internal Bus Speed}} = 3 \times 640 \, \text{\lr{Gbps}} = 1920 \, \text{\lr{Gbps}}
			\]
			
			بنابراین، حداقل سرعت باس داخلی باید برابر با:
			\[
			\boxed{1920 \, \text{\lr{Gbps}} \, (1.92 \, \text{\lr{Tbps}})}
			\]
			باشد.
	\end{qsolve}
	
	
	\item 
	اگر هر پورت دارای ۱ مگابایت بافر باشد، حداکثر تأخیر بافرینگ چقدر خواهد بود؟
	\begin{qsolve}
		حداکثر تأخیر بافرینگ را می‌توان با استفاده از فرمول زیر محاسبه کرد:
		\[
		\text{\lr{Maximum Buffering Delay}} = \frac{\text{\lr{Buffer Size}}}{\text{\lr{Line Rate}}}
		\]
		
		با توجه به اینکه:
		\[
		\text{\lr{Buffer Size}} = 1 \, \text{\lr{MB}} = 8 \times 10^6 \, \text{\lr{bits}}
		\]
		و
		\[
		\text{\lr{Line Rate}} = 40 \, \text{\lr{Gbps}} = 40 \times 10^9 \, \text{\lr{bits/second}}
		\]
		
		تأخیر بافرینگ به صورت زیر محاسبه می‌شود:
		\[
		\text{\lr{Maximum Buffering Delay}} = \frac{8 \times 10^6}{40 \times 10^9}
		\]
		\[
		\text{\lr{Maximum Buffering Delay}} = 0.2 \, \text{\lr{ms}}
		\]
		
		بنابراین، حداکثر تأخیر بافرینگ برابر با:
		\[
		\boxed{0.2 \, \text{ms}}
		\]
		است.
	\end{qsolve}
\end{enumerate}
