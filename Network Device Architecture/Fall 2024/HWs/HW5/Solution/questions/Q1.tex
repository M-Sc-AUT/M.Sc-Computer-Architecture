\section{سوال اول}

\begin{enumerate}
	\item 
	تفاوت اصلی بین سوییچینگ و مسیریابی چیست؟ چگونه هرکدام در انتقال اطلاعات از یک نقطه به نقطه دیگر عمل می‌کنند؟
	
	\begin{qsolve}
		\begin{itemize}
			\item 
			\textbf{سوئیچینگ:} سوئیچینگ فرآیندی است که در سطح لایه ۲ (لایه پیوند داده) انجام می‌شود و هدف آن انتقال داده‌ها بین دستگاه‌های موجود در یک شبکه محلی (\lr{LAN}) است. سوئیچ‌ها از آدرس‌های \lr{MAC} برای هدایت داده‌ها استفاده می‌کنند.
			
			
			\item 
			\textbf{روش انجام:} سوئیچ، فریم داده را دریافت کرده، آدرس مقصد آن را بررسی می‌کند و داده‌ها را تنها به پورتی ارسال می‌کند که دستگاه مقصد به آن متصل است.
			
			\item 
			\textbf{مسیریابی:} مسیریابی فرآیندی است که در سطح لایه ۳ (لایه شبکه) انجام می‌شود و هدف آن انتقال داده‌ها بین شبکه‌های مختلف است. روترها از آدرس‌های \lr{IP} برای هدایت داده‌ها استفاده می‌کنند.
			
			\item 
			\textbf{روش انجام:} مسیریاب بسته‌ها را بررسی می‌کند، جدول مسیریابی را جستجو می‌کند و بهترین مسیر برای ارسال داده به مقصد را انتخاب می‌کند.
			
			\item 
			\textbf{تفاوت کلیدی:} سوئیچینگ در سطح شبکه محلی (\lr{LAN}) عمل می‌کند و از آدرس‌های \lr{MAC} استفاده می‌کند، در حالی که مسیریابی در سطح شبکه گسترده (\lr{WAN}) یا بین شبکه‌های مختلف عمل کرده و از آدرس‌های \lr{IP} استفاده می‌کند.
		\end{itemize}
	\end{qsolve}
	
	
	\item 
	چه مکانیزم‌هایی برای پشتیبانی از \lr{multicast} در سوییچ‌ها لازم است و چگونه این مکانیزم‌ها به ارسال داده از یک ورودی به چندین خروجی کمک می‌کنند؟
	
	\begin{qsolve}
		برای پشتیبانی از \lr{multicast}، سوئیچ‌ها نیاز به مکانیزم‌های زیر دارند:
		
		\begin{enumerate}
			\item 
			\textbf{جدول عضویت گروه (\lr{Group Membership Table}):}
			\begin{itemize}
				\item 
				سوئیچ باید نگهدارنده جدولی باشد که در آن پورتی که اعضای گروه \lr{multicast} در آن قرار دارند مشخص شده باشد.
				
				\item 
				پروتکل‌هایی مانند \lr{IGMP snooping} می‌توانند در شناسایی عضویت گروه‌ها کمک کنند.
			\end{itemize}
		\end{enumerate}
	\end{qsolve}
\end{enumerate}
	\newpage
	
	
	
	
	
	\begin{enumerate}
	
	\item [ ]
	\begin{qsolve}
		\begin{enumerate}
			\item [(ب)]
			\textbf{پشتیبانی از ارسال انتخابی (\lr{Selective Forwarding}):}
			\begin{itemize}
				\item 
				سوئیچ باید داده‌های \lr{multicast} را تنها به پورتی ارسال کند که عضو گروه مقصد در آن قرار دارد، نه به تمام پورت‌ها.
			\end{itemize}
			
			\item [(ج)]
			\textbf{پشتیبانی از \lr{VLAN}:}
			\begin{itemize}
				\item 
				اگر گروه‌های \lr{multicast} در شبکه‌ای مبتنی بر \lr{VLAN} باشند، سوئیچ باید این ترافیک را به درستی در محدوده \lr{VLAN} مربوطه ارسال کند.
			\end{itemize}
		\end{enumerate}
		
		این مکانیزم‌ها امکان ارسال داده‌ها از یک ورودی به چندین خروجی را بدون تداخل و با بهینه‌سازی پهنای باند فراهم می‌کنند.
	\end{qsolve}
	
	\item 
	تعریف \lr{throughput} و \lr{speedup} در سوییچینگ چیست؟ چگونه \lr{speedup} می‌تواند باعث افزایش \lr{throughput} شود؟
	
	\begin{qsolve}
		\begin{itemize}
			\item 
			\textbf{Throughput:} نرخ موثر انتقال داده از طریق سوئیچ یا شبکه. واحد آن معمولاً بیت در ثانیه است و نشان‌دهنده عملکرد کلی سیستم در ارسال داده‌ها است.
			
			\item 
			\textbf{Speedup:} نسبت بین نرخ پردازش داخلی سوئیچ به نرخ ارسال داده در پورت‌ها.
			
			\item 
			\textbf{فرمول:} $\frac{\text{\lr{Switch Internal Rate}}}{\text{\lr{Port Rate}}}$=\text{\lr{Speedup}}
		\end{itemize}
		
		افزایش \lr{Throughput} با \lr{Speedup}: با افزایش \lr{Speedup}، سوئیچ می‌تواند داده‌ها را سریع‌تر از نرخ پورت‌ها پردازش کند و این باعث کاهش تأخیر و جلوگیری از ازدحام در شبکه می‌شود، که به افزایش \lr{Throughput} منجر می‌گردد.
	\end{qsolve}
	
	
	
	\item 
	تفاوت بین \lr{blocking} و \lr{output contention} در سوییچ‌های مبتنی بر تقسیم فضایی چیست؟ چگونه هرکدام می‌توانند بر عملکرد سوییچ تأثیر بگذارند؟
	\begin{qsolve}
		\begin{itemize}
			\item 
			\textbf{\lr{:Blocking}} زمانی رخ می‌دهد که منابع سوئیچ (مانند پهنای باند یا پورت‌ها) نتوانند به دلیل محدودیت، داده‌ها را از یک ورودی به خروجی موردنظر منتقل کنند.
			
			
			\item 
			\textbf{تأثیر:} بسته ممکن است نتواند به مقصد برسد یا تأخیر زیادی را تجربه کند.
			
			\item 
			\textbf{\lr{:Output Contention}} زمانی رخ می‌دهد که چندین بسته به طور همزمان قصد ارسال به یک خروجی مشترک را دارند. در این حالت، بسته‌ها باید در صف قرار گیرند یا برخی از آن‌ها دور ریخته شوند.
			
			\item 
			\textbf{تأثیر:} ایجاد تأخیر در ارسال یا اتلاف داده.
		\end{itemize}
	\end{qsolve}
	\newpage
	\end{enumerate}
	
	
	
	\begin{enumerate}
	\item [ ]
	\begin{qsolve}
		\textbf{تفاوت:} \lr{Blocking} به محدودیت‌های داخلی سوئیچ مرتبط است، در حالی که \lr{Output Contention} به محدودیت‌های مربوط به تقاضا برای یک خروجی خاص اشاره دارد.
	\end{qsolve}
	
	
	
	
	\item 
	توضیح دهید که تفاوت بین سوییچینگ به صورت \lr{packet-mode} و \lr{cell-mode} چیست؟ مزایا و معایب هرکدام چیست و چگونه به طراحی سوییچ‌های \lr{IP} کمک می‌کنند؟
	\begin{qsolve}
		\begin{enumerate}
			\item 
			\lr{\textbf{:Packet-Mode Switching}}
			\begin{itemize}
				\item 
				بسته‌ها به صورت کامل در سوئیچ پردازش می‌شوند. اندازه بسته‌ها می‌تواند متغیر باشد.
				
				\item 
				مزایا: انعطاف‌پذیری بالا، بهینه برای پروتکل‌های موجود.
				
				\item 
				معایب: ممکن است منجر به تأخیر و ایجاد مشکل در کنترل جریان شود.
			\end{itemize}
			
			
			\item 
			\lr{\textbf{:Cell-Mode Switching}}
			\begin{itemize}
				\item 
				بسته‌ها به قطعات کوچک‌تر (سلول‌ها) با اندازه ثابت تقسیم شده و سپس پردازش می‌شوند.
				
				\item 
				مزایا: پیش‌بینی‌پذیری در زمان انتقال و کاهش تأخیر.
				
				\item 
				معایب: افزایش \lr{overhead} به دلیل قطعه‌بندی.
			\end{itemize}
		\end{enumerate}
		در \lr{Packet-Mode}، سوئیچ‌ها انعطاف‌پذیری بیشتری در پردازش بسته‌ها دارند، اما \lr{Cell-Mode} برای محیط‌هایی با نیاز به زمان انتقال ثابت مناسب‌تر است (مانند \lr{ATM}).
	\end{qsolve}
\end{enumerate}
