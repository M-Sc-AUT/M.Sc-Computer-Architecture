\section{سوال ششم}


پیکربندی مسئله ۴ را با استفاده از معیارهای \lr{Clos} تکرار کنید.


\begin{qsolve}
	\begin{enumerate}[label=\arabic*.]
		\item \textbf{طراحی پیکربندی:}
		
		با توجه به معیارهای \lr{Clos}:
		\[
		n \geq \max(m, \lceil \frac{N}{m} \rceil)
		\]
		که در اینجا:
		\begin{itemize}
			\item \(N = 100\): تعداد کل ورودی‌ها/خروجی‌ها.
			\item \(m = 10\): تعداد خطوط ورودی به هر \lr{crossbar} در مرحله اول.
			\item \(\lceil \frac{N}{m} \rceil = \lceil \frac{100}{10} \rceil = 10\).
		\end{itemize}
		بنابراین:
		\[
		n = 10
		\]
		
		تعداد \lr{crossbar}‌ها در هر مرحله:
		\begin{itemize}
			\item مرحله اول: تعداد \(k_1 = \frac{N}{m} = \frac{100}{10} = 10\).
			\item مرحله میانی: تعداد \(k_2 = n = 10\).
			\item مرحله سوم: تعداد \(k_3 = \frac{N}{m} = \frac{100}{10} = 10\).
		\end{itemize}
		
		\item \textbf{تعداد کل \lr{crosspoints}:}
		
		تعداد \lr{crosspoints} در هر مرحله به صورت زیر محاسبه می‌شود:
		\[
		\text{\lr{Crosspoints in Stage 1}} = k_1 \times (m \times n) = 10 \times (10 \times 10) = 1000
		\]
		\[
		\text{\lr{Crosspoints in Stage 2}} = k_2 \times (n \times n) = 10 \times (10 \times 10) = 1000
		\]
		\[
		\text{\lr{Crosspoints in Stage 3}} = k_3 \times (n \times m) = 10 \times (10 \times 10) = 1000
		\]
		تعداد کل \lr{crosspoints} برابر است با:
		\[
		\text{\lr{Total Crosspoints}} = 1000 + 1000 + 1000 = 3000
		\]
		
		\item \textbf{تعداد اتصالات همزمان ممکن:}
		
		در پیکربندی \lr{Clos} بدون \lr{blocking}، تعداد اتصالات همزمان ممکن برابر با تعداد کل ورودی‌ها است:
		\[
		\text{\lr{Maximum Simultaneous Connections}} = N = 100
		\]
		
		\item \textbf{نمودار پیکربندی:}
		
		نمودار سه‌مرحله‌ای مطابق معیار \lr{Clos} به صورت زیر است:
		
	\end{enumerate}
\end{qsolve}
\newpage

\begin{center}
	\includegraphics*[width=1\linewidth]{pics/img9.pdf}
	\captionof{figure}{پیکربندی سوال ۶}
	\label{پیکربندی سوال ۶}
\end{center}



%\begin{center}
%	\includegraphics*[width=0.5\linewidth]{pics/img1.png}
%	\captionof{figure}{شبکه سوال ۶}
%\end{center}

