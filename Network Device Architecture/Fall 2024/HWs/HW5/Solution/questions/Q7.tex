\section{سوال هفتم}

در یک \lr{Parallel Packet Switch} با $k=3 $ صفحه موازی و سرعت خط ورودی ۱۰۰ گیگابیت بر ثانیه:

\begin{enumerate}
	\item 
	سرعت خط مورد نیاز در هر صفحه را محاسبه کنید.
	
	\begin{qsolve}
		
		سرعت خط مورد نیاز در هر صفحه به صورت زیر محاسبه می‌شود:
		\[
		\text{\lr{Speed per plane}} = \frac{\text{\lr{Input Line Rate}}}{k}
		\]
		با توجه به اینکه:
		\[
		\text{\lr{Input Line Rate}} = 100 \, \text{\lr{Gbps}}, \quad k = 3
		\]
		داریم:
		\[
		\text{\lr{Speed per plane}} = \frac{100}{3} \, \text{\lr{Gbps}} = 33.33 \, \text{\lr{Gbps}}
		\]
		
		بنابراین، سرعت خط مورد نیاز در هر صفحه برابر است با:
		\[
		\boxed{33.33 \, \text{\lr{Gbps}}}
		\]
	\end{qsolve}
	
	\item 
	اگر بسته‌ها ۱۵۰۰ بایتی باشند، فاصله زمانی بین بسته‌های متوالی در هر صفحه چقدر است؟
	\begin{qsolve}
		
		فاصله زمانی بین بسته‌های متوالی (\(\Delta T\)) از رابطه زیر محاسبه می‌شود:
		\[
		\Delta T = \frac{\text{\lr{Packet Size}}}{\text{\lr{Speed per plane}}}
		\]
		با توجه به اینکه:
		\begin{itemize}
			\item اندازه بسته‌ها: 
			\[
			\text{\lr{Packet Size}} = 1500 \, \text{\lr{Bytes}} = 1500 \times 8 = 12000 \, \text{\lr{bits}}
			\]
			\item سرعت خط در هر صفحه: 
			\[
			\text{\lr{Speed per plane}} = 33.33 \, \text{\lr{Gbps}} = 33.33 \times 10^9 \, \text{\lr{bits/sec}}
			\]
		\end{itemize}
		محاسبه:
		\[
		\Delta T = \frac{12000}{33.33 \times 10^9} \, \text{\lr{seconds}}
		\]
		\[
		\Delta T = 0.36 \, \mu\text{\lr{s}}
		\]
		
		بنابراین، فاصله زمانی بین بسته‌های متوالی برابر است با:
		\[
		\boxed{0.36 \, \mu\text{\lr{s}}}
		\]
	\end{qsolve}
\end{enumerate}