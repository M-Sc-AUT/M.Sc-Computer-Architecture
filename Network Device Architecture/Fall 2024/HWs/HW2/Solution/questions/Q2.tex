\section{سوال دوم}

روش‌های مختلفی برای جستجوی آدرس \lr{IP} در جدول جلورانی وجود دارد. هر کدام از این روش‌ها ویژگی‌هایی دارند.


\begin{itemize}
	\item اهداف اصلی یک روش جستجوی آدرس \lr{IP} چیست؟
	
	\begin{qsolve}
		اهداف اصلی یک روش جستجوی آدرس \lr{IP} شامل موارد زیر است:
		\begin{itemize}
			\item \textbf{تطابق سریع و دقیق}: یافتن سریع و دقیق‌ترین تطابق (\lr{Longest Prefix Match}) برای آدرس \lr{IP}، به طوری که بسته به درستی و بدون تأخیر به مقصد هدایت شود.
			\item \textbf{بهره‌وری بالا}: استفاده بهینه از منابع سخت‌افزاری مانند حافظه و پردازشگر، تا روش جستجو بتواند با کمترین هزینه پردازشی و مصرف حافظه کار کند.
			\item \textbf{قابلیت مقیاس‌پذیری}: روش باید بتواند با افزایش اندازه جدول جلورانی و تعداد آدرس‌ها، همچنان به‌خوبی کار کند و عملکرد مطلوبی داشته باشد.
			\item \textbf{پایداری و مقاومت در برابر تغییرات}: امکان تطبیق و بروزرسانی آسان جدول جلورانی با ورود و خروج آدرس‌ها و یا تغییر مسیرها.
		\end{itemize}
	\end{qsolve}
	
	
	\item معیارهای ارزیابی یک روش جستجوی آدرس \lr{IP} کدام‌اند و چگونه می‌توان کارایی یک روش جست‌و‌جوی آدرس \lr{IP} را در شبکه‌های بزرگ ارزیابی کرد؟ (پاسخ خود را با توجه به مواردی مانند کارایی حافظه، زمان جستجو، مقیاس‌پذیری و موارد مشابه دیگر توضیح دهید.)
	
	\begin{qsolve}
		\begin{itemize}
			\item \textbf{کارایی حافظه}:
			\begin{itemize}
				\item میزان استفاده از حافظه یکی از معیارهای مهم برای ارزیابی روش‌های جستجو است. روش‌های کارا از حافظه بهینه استفاده می‌کنند، به‌ویژه در جدول‌های جلورانی بزرگ.
				\item \textbf{روش ارزیابی}: اندازه جدول و ساختار داده‌های ذخیره شده برای هر روش را مقایسه می‌کنیم. روش‌هایی که از ساختارهای داده فشرده‌تری مانند درخت‌ها و یا درخت‌های باینری استفاده می‌کنند، معمولاً کارایی حافظه بالاتری دارند.
			\end{itemize}
			
			\item \textbf{زمان جستجو}:
			\begin{itemize}
				\item زمان لازم برای یافتن تطابق در جدول، معیار مهم دیگری است. در مسیر‌یاب‌ها، این معیار باید تا حد ممکن کمینه باشد.
				\item \textbf{روش ارزیابی}: می‌توان میانگین زمان جستجو یا پیچیدگی زمانی روش‌ها (مثلاً \(\mathcal{O}(\log n)\) یا \(\mathcal{O}(1)\) برای جستجوهای خاص) را بررسی کرد. روش‌های سریع‌تر زمان جستجو را کاهش می‌دهند و عملکرد بهتری دارند.
			\end{itemize}
		\end{itemize}
	\end{qsolve}
	
	\begin{qsolve}[ادامه پاسخ]
		\begin{itemize}
			\item \textbf{مقیاس‌پذیری}:
			\begin{itemize}
				\item روش باید بتواند با افزایش تعداد و طول آدرس‌ها (به خصوص در شبکه‌های بزرگ و پیچیده)، همچنان با سرعت و دقت بالا کار کند.
				\item \textbf{روش ارزیابی}: عملکرد روش‌ها در حجم داده‌های بزرگ و همچنین میزان تأثیر افزایش تعداد آدرس‌ها بر زمان جستجو و حافظه مصرفی بررسی می‌شود.
			\end{itemize}
			
			\item \textbf{انعطاف‌پذیری و بروزرسانی}:
			\begin{itemize}
				\item روش باید در برابر تغییرات آدرس‌ها، افزودن یا حذف آن‌ها انعطاف‌پذیر باشد و امکان بروزرسانی سریع و آسان را فراهم کند.
				\item \textbf{روش ارزیابی}: سرعت و سهولت عملیات درج، حذف و به‌روزرسانی آدرس‌ها در روش بررسی می‌شود. روش‌های با ساختار پویا (مانند ساختارهای درختی) معمولاً در این زمینه عملکرد بهتری دارند.
			\end{itemize}
		\end{itemize}
	\end{qsolve}
	
	
	
	
	\item  دسته‌بندی روش‌های جستجوی آدرس \lr{IP} در جدول جلورانی را با ذکر ویژگی‌های هر روش بیان کنید.
	\begin{qsolve}
		\begin{itemize}
			\item \textbf{روش‌های مبتنی بر درخت (\lr{Trie-Based Methods})}:
			\begin{itemize}
				\item \textbf{ویژگی‌ها}: این روش‌ها از ساختار درختی مانند \lr{Binary Trie} استفاده می‌کنند و آدرس‌های \lr{IP} را به‌صورت بیتی ذخیره می‌کنند. به این صورت که هر سطح درخت نشان‌دهنده یک بیت از آدرس است و مسیر به پایین‌ترین سطح نشان‌دهنده طولانی‌ترین تطابق است.
				\item \textbf{مزایا}: زمان جستجوی نسبتاً سریع و مقیاس‌پذیری خوب.
				\item \textbf{معایب}: در شرایط خاص، ممکن است نیاز به حافظه زیادی داشته باشند و نگهداری و به‌روزرسانی آنها پیچیده باشد.
			\end{itemize}
			
			\item \textbf{روش‌های مبتنی بر هشینگ (\lr{Hashing-Based Methods})}:
			\begin{itemize}
				\item \textbf{ویژگی‌ها}: از جدول‌های هش برای ذخیره پیشوندهای \lr{IP} استفاده می‌کنند. در این روش، جستجو از طریق هش کردن انجام می‌شود.
				\item \textbf{مزایا}: سرعت جستجوی بسیار سریع، به‌ویژه در شرایطی که جستجو با پیچیدگی \(\mathcal{O}(1)\) انجام شود.
				\item \textbf{معایب}: مقیاس‌پذیری کمتر نسبت به روش‌های دیگر و محدودیت در پردازش طولانی‌ترین پیشوندها به دلیل نبود ساختار سلسله‌مراتبی.
			\end{itemize}
			
			\item \textbf{روش‌های مبتنی بر جستجوی باینری (\lr{Binary Search-Based Methods})}:
			\begin{itemize}
				\item \textbf{ویژگی‌ها}: از جستجوی باینری برای یافتن طولانی‌ترین پیشوند استفاده می‌کنند. این روش به‌خصوص برای شبکه‌هایی با پیشوندهای کوتاه‌تر مناسب است.
				\item \textbf{مزایا}: زمان جستجوی ثابت و مناسب در اندازه‌های کوچک و متوسط.
				\item \textbf{معایب}: کارایی حافظه کمتر نسبت به سایر روش‌ها در شبکه‌های بزرگ.
			\end{itemize}
		\end{itemize}
	\end{qsolve}
	
	\begin{qsolve}[ادامه پاسخ]
		\item [ ]
		\begin{itemize}
			\item \textbf{روش‌های مبتنی بر جداول \lr{TCAM}}:
			\begin{itemize}
				\item \textbf{ویژگی‌ها}: در این روش‌ها از حافظه \lr{TCAM} استفاده می‌شود که امکان جستجوی همزمان چندین ورودی را فراهم می‌آورد.
				\item \textbf{مزایا}: سرعت بسیار بالای جستجو و پشتیبانی از طولانی‌ترین پیشوند.
				\item \textbf{معایب}: قیمت بالا و مصرف انرژی زیاد، همچنین نیاز به تنظیم دقیق برای بهینه‌سازی عملکرد.
			\end{itemize} 
		\end{itemize}
		
	\end{qsolve}

\end{itemize}