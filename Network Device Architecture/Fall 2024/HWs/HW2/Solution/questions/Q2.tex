\section{سوال دوم}
شرکت "توسعه‌دهندگان نوآور" تصمیم دارد یک کنفرانس آنلاین برای معرفی محصول جدید خود برگزار کند. در این کنفرانس، تیم‌های مختلف از نقاط مختلف کشور شرکت خواهند کرد. برای برقراری ارتباط بین شرکت‌کنندگان، دو شبکه یکی مبتنی بر تکنیک سوئیچینگ مداری و دیگری مبتنی بر سوئیچینگ بسته‌ای در دسترس است. با توجه به نیازمندی شرکت‌کنندگان، توضیح دهید برای برقراری ارتباط با هریک از این شرکت‌کنندگان استفاده از کدام تکنیک سوئیچینگ مناسب‌تر است.

\begin{enumerate}
	\item 
	تیم فنی (شامل ۵ نفر) - نیاز به ارتباط صوتی و تصویری با کیفیت بالا
	\begin{qsolve}
		با تووجه به توضیحاتی که در سوال اول در مورد این دو تکنیک داده شد، به نظر برای تیم فنی که نیاز به ارتباط صوتی و تصویری با کیفیت بالا دارند، سوئیچینگ مداری گزینه مناسبی است. دلیل این انتخاب این است که در ارتباطات صوتی و تصویری زنده، تأخیر بسیار کم و کیفیت ثابت مورد نیاز است. در سوئیچینگ مداری، یک مسیر انحصاری برای کل زمان ارتباط برقرار می‌شود که می‌تواند نیاز به پهنای باند پایدار و کیفیت ثابت برای تماس‌های صوتی و تصویری را فراهم کند.
	\end{qsolve}
	
	\item 
	تیم بازاریابی (شامل ۱۰ نفر) - نیاز به ارسال و دریافت اطلاعات و اسناد
	\begin{qsolve}
		برای تیم بازاریابی که بیشتر نیاز به ارسال و دریافت اطلاعات و اسناد دارند، سوئیچینگ بسته‌ای گزینه بهتری است. در این نوع ارتباطات، ارسال داده‌ها می‌تواند به صورت غیر همزمان و بهینه انجام شود. همچنین این تکنیک برای ارسال فایل‌های حجیم و مدیریت پهنای باند بسیار مناسب است، چرا که بسته‌ها به صورت مجزا و از مسیرهای مختلف ارسال می‌شوند و نیازی به مسیر انحصاری نیست.
	\end{qsolve}
	
	\item 
	تیم مدیریت (شامل ۳ نفر) - نیاز به ارتباط سریع و مؤثر
	\begin{qsolve}
		برای تیم مدیریت سوئیچینگ بسته‌ای مناسب است. زیرا این تکنیک می‌تواند به صورت کارآمد داده‌های کم‌حجم و سریع را در شبکه انتقال دهد و منابع شبکه را بین چندین کاربر به اشتراک بگذارد. همچنین از آنجا که تیم مدیریت به ارسال پیام‌های فوری و دستورات نیاز دارند، سوئیچینگ بسته‌ای با توانایی ارسال سریع بسته‌ها و توزیع کارآمد داده‌ها، نیازهای آن‌ها را برآورده می‌کند.
	\end{qsolve}
\end{enumerate}