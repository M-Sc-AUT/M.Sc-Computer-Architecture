\section{سوال پنجم}

 \lr{Path-compressed trie}
 جدول جلورانی زیر را رسم کنید.

\begin{latin}
		\begin{tabular}{l l}
			P1 & * \\
			P2 & 10* \\
			P3 & 1001* \\
			P4 & 1011* \\
			P5 & 11101* \\
			P6 & 010011* \\
			P7 & 010101* \\
			P8 & 0100110* \\
		\end{tabular}
\end{latin}

\begin{qsolve}
	ابتدا \lr{Binary Trie} جدول را به‌صورت زیر رسم می‌کنیم:
	
	\begin{center}
		\includegraphics*[width=0.5\linewidth]{pics/q5.pdf}
		\captionof{figure}{درخت \lr{Binary}}
	\end{center}
	
	
	سپس درخت \lr{Path Compress} را به‌صورت زیر رسم می‌کنیم:
	\begin{center}
		\includegraphics*[width=0.5\linewidth]{pics/q7_b.pdf}
		\captionof{figure}{درخت \lr{Path Compress}}
	\end{center}
\end{qsolve}