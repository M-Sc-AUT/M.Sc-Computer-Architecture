\section{سوال دهم}

یکی از روش‌های کاهش هزینه توان مصرفی در \lr{TCAM} روش \lr{Trie-Based Table Partitioning} که خود شامل \lr{Subtree Splitting} و \lr{Post-Order Splitting} است. با اعمال این دو روش بر روی درخت زیر، جدولی مشابه جدول صفحه ۶۴ کتاب بدست آورید. گره‌های آبی رنگ شامل آدرس‌های \lr{Prefix} هستند. (مقدار اندازه بلوک‌های حافظه را ۴ در نظر بگیرید؛ $b=4 $)


\begin{center}
	\includegraphics*[width=0.5\linewidth]{pics/img2.png}
	\captionof{figure}{درخت باینری}
\end{center}


\begin{qsolve}
	مقدار \lr{Value} را برای هر گره می‌نویسیم و در هر مرحله که تکه‌ای از درخت را جدا می‌کنیم، این مقدار را آپدیت می‌کنیم و مقدار و مقدار آن را با رنگ قرمز درکنار هر \lr{Node} مشخص می‌کنیم.
	
	\textbf{باید $2\le Value \le 4 $ را جدا کنیم.}

	\begin{enumerate}
		\item 
		\textbf{:Preorder}
		
		\begin{center}
			\includegraphics*[width=0.7\linewidth]{pics/q10a.pdf}
			\captionof{figure}{Preorder}
		\end{center}
		
	\end{enumerate}
	
\end{qsolve}




\begin{qsolve}
	
	\begin{latin}
		\begin{center}
			\begin{tabular}{|c|c|c|c|}
				\hline
				Prefix & Bucket Prefixes & Bucket Size & Covering Prefix  \\
				\hline\hline
				\texttt{000*} & \texttt{000*},\texttt{00010*},\texttt{00011*} & 3 & \texttt{000*}  \\
				\hline
				\texttt{00*} & \texttt{0011*},\texttt{001110*} & 2 & \texttt{0011*}  \\
				\hline
				\texttt{0*} & \texttt{01*},\texttt{01101*},\texttt{011010*},\texttt{011011*} & 4 & \texttt{01*}  \\
				\hline
				\texttt{10*} & \texttt{100*},\texttt{10010*},\texttt{100110*} & 3 & \texttt{100*}  \\
				\hline
				\texttt{*} & \texttt{1*},\texttt{111*} & 2 & \texttt{1*}  \\
				\hline
			\end{tabular}
		\end{center}
	\end{latin}
	
	
	\begin{enumerate}
		\item [2.] \textbf{:Postorder}
		
		
		\begin{center}
			\includegraphics*[width=0.7\linewidth]{pics/110b.pdf}
			\captionof{figure}{Postorder}
		\end{center}
		
		
		\begin{latin}
			\begin{center}
				\begin{tabular}{|c|c|c|c|}
					\hline
					Prefix & Bucket Prefixes & Bucket Size & Covering Prefix  \\
					\hline\hline
					\texttt{0001*},\texttt{001*} & \texttt{00010*},\texttt{00011*},\texttt{0011*},\texttt{001110*} & 4 & \texttt{0001*},\texttt{0011*}  \\
					\hline
					\texttt{01*} & \texttt{01*},\texttt{01101*},\texttt{011010*},\texttt{011011*} & 4 & \texttt{01*},\texttt{01101*}  \\
					\hline
					\texttt{10*},\texttt{11*} & \texttt{100*},\texttt{10010*},\texttt{100110*},\texttt{111*} & 4 & \texttt{100*},\texttt{111*}  \\
					\hline
					\texttt{*} & \texttt{000*},\texttt{1*} & 2 & \texttt{1*},\texttt{000*}  \\
					\hline
				\end{tabular}
			\end{center}
		\end{latin}
		
	\end{enumerate}
\end{qsolve}