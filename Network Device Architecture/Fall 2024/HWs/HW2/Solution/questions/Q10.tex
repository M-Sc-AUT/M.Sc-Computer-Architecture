\section{سوال دهم}

یکی از روش‌های کاهش هزینه توان مصرفی در \lr{TCAM} روش \lr{Trie-Based Table Partitioning} که خود شامل \lr{Subtree Splitting} و \lr{Post-Order Splitting} است. با اعمال این دو روش بر روی درخت زیر، جدولی مشابه جدول صفحه ۶۴ کتاب بدست آورید. گره‌های آبی رنگ شامل آدرس‌های \lr{Prefix} هستند. (مقدار اندازه بلوک‌های حافظه را ۴ در نظر بگیرید؛ $b=4 $)


\begin{center}
	\includegraphics*[width=0.5\linewidth]{pics/img2.png}
	\captionof{figure}{درخت باینری}
\end{center}


\begin{qsolve}
	
\end{qsolve}