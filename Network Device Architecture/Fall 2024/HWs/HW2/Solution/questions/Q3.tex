\section{سوال سوم}


درخت باینری نشان‌داده شده در زیر، از روی جدول جلورانی در یک مسیریاب ایجاد شده است و جست و جو در جدول جلورانی برای پیدا کردن شماره پورت خروجی با پیمایش این درخت انجام می‌گردد. (در این درخت باینری، گره‌های مشکی، پیشوندهای آدرس جدول جلورانی هستند و مقادیر آن‌ها بیانگر شماره پورت خروجی است.)


\begin{center}
	\includegraphics*[width=0.5\linewidth]{pics/img1.png}
	\captionof{figure}{درخت باینری}
\end{center}



\begin{enumerate}
	\item بسته‌ای با آدرس مقصد \lr{\texttt{90B28FF1}} از کدام پورت خارج می‌گردد؟ (آدرس‌های مقصد در مبنای ۱۶ نمایش داده شده‌اند.)
	
	\begin{qsolve}
		بر اساس درخت داده شده، جدول مسیریابی به‌صورت زیر به‌دست می‌آید:
		
		\begin{latin}
			\begin{center}
				\begin{tabular}{|c|c|}
					\hline
					\textbf{Prefix} & \textbf{Output Port} \\
					\hline\hline
					\texttt{*} & \texttt{10} \\
					\hline\hline
					\texttt{111*} & \texttt{8} \\
					\hline
					\texttt{110*} & \texttt{4} \\
					\hline
					\texttt{1000*} & \texttt{56} \\
					\hline
					\texttt{10010*} & \texttt{22} \\
					\hline
					\texttt{100100*} & \texttt{60} \\
					\hline
					\texttt{100111*} & \texttt{20} \\
					\hline
					\texttt{00*} & \texttt{12} \\
					\hline
					\texttt{0000*} & \texttt{5} \\
					\hline
				\end{tabular}
			\end{center}
		\end{latin}
		
		اگر معادل باینری آدرس داده شده را بنویسیم:
		
		\begin{latin}
			\texttt{(0X90B28FF1) = 1001 0000 1011 0010 1000 1111 1111 0001}
		\end{latin}
		
		می‌بینیم که براساس الگوریتم \lr{Longest Prefix Matching} بیشترین \lr{Matching} را با سطر ۶ جدول دارد. پس درنتیجه، این بسته از پورت شماره ۶۰ خارج می‌شود.
	\end{qsolve}
	
	
	
	
	
	\item بسته‌ای با آدرس مقصد \lr{\texttt{A2AB11C3}} از کدام پورت خارج می‌گردد؟
	\begin{qsolve}
		معادل باینری آدرس داده شده را می‌نویسیم:
		
		\begin{latin}
			\texttt{(0XA2AB11C3) = 1010 0010 1010 1011 0001 0001 1100 0011}
		\end{latin}
		
		بیشترین \lr{Matching} را با سطر ۱ جدول است. پس، این بسته از پورت شماره ۱۰ خارج می‌شود.
	\end{qsolve}
	
	
	
	
	
	
	
	\item آدرس پیشوندی \texttt{10101*} با پورت خروجی ۵۰ را به درخت اضافه نمایید.
	\begin{qsolve}
		\begin{center}
			\includegraphics*[width=0.5\linewidth]{pics/q3_d.pdf}
			\captionof{figure}{درخت جدید}
		\end{center}
	\end{qsolve}
	
	
	
	
	\item آدرس پیشوندی \texttt{10010*} با پورت خروجی ۲۲ را از درخت حذف نمایید.

	\begin{qsolve}
		\begin{center}
			\includegraphics*[width=0.5\linewidth]{pics/q3_c.pdf}
			\captionof{figure}{درخت جدید}
		\end{center}
	\end{qsolve}

\end{enumerate}





