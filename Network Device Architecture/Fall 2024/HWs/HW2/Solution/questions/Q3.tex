\section{سوال سوم}
رده‌بندی ارائه‌دهندگان خدمات اینترنت به سه سطح \lr{Tier1، Tier2, Tier3} صورت می‌گیرد.

\begin{enumerate}
	\item 
	نقش هر یک از این سطوح در معماری شبکه اینترنت را توضیح دهید.
	\begin{qsolve}
		\begin{enumerate}
			\item 
			\textbf{ارائه‌دهندگان سطح 1 (\lr{Tier 1 ISPs}):}\\
			این ارائه‌دهندگان خدمات اینترنت (\lr{ISP}ها) بزرگ‌ترین شرکت‌های ارائه‌دهنده اینترنت هستند که به هیچ ارائه‌دهنده اینترنت دیگری بابت ترانزیت اینترنت هزینه‌ای نمی‌پردازند. آن‌ها مستقیماً به سایر شبکه‌های \lr{Tier 1} متصل می‌شوند و به تبادل داده‌ها می‌پردازند. \lr{Tier 1 ISPs} مسئولیت مدیریت شبکه‌های عظیم بین‌المللی را دارند و ستون فقرات اصلی اینترنت را تشکیل می‌دهند. این شبکه‌ها معمولاً ارتباطات مستقیم با \lr{PoP}های بین‌المللی دارند و داده‌ها را در سطح جهانی انتقال می‌دهند.
			
			
			\item 
			\textbf{ارائه‌دهندگان سطح 2 (\lr{Tier 2 ISPs}):}\\
			این ارائه‌دهندگان اینترنت به صورت محلی یا منطقه‌ای فعالیت می‌کنند و به ارائه‌دهندگان سطح 1 متصل می‌شوند تا به اینترنت جهانی دسترسی پیدا کنند. همچنین ممکن است برای کاهش هزینه‌های ترانزیت به سایر ارائه‌دهندگان سطح 2 نیز ارتباط برقرار کنند. \lr{Tier 2 ISPs} از طریق خرید خدمات از \lr{Tier 1 ISPs} به اینترنت دسترسی دارند و بخشی از ترافیک داده را از طریق تبادل‌های داخلی (پیرینگ) مدیریت می‌کنند.
			
			\item 
			\textbf{ارائه‌دهندگان سطح 3 (\lr{Tier 3 ISPs}):}\\
			این ارائه‌دهندگان، کوچک‌تر و محلی‌تر هستند و عموماً برای ارائه خدمات اینترنت به کاربران نهایی (مصرف‌کنندگان خانگی یا تجاری) فعالیت می‌کنند. \lr{Tier 3 ISPs} برای دسترسی به اینترنت جهانی نیاز به خرید خدمات از \lr{Tier 2 ISPs} یا \lr{Tier 1 ISPs} دارند. آن‌ها اغلب مسئول ارائه خدمات نهایی به کاربران هستند و به شبکه‌های محلی یا شهری متصل می‌شوند.
		\end{enumerate}
	\end{qsolve}
	
	
	\item 
	نقاط حضور (\lr{PoP}) را تعریف کنید و در ادامه ارتباط این نقاط با هزینه‌های سرمایه‌گذاری (\lr{CAPEX}) و نگهداری (\lr{OPEX}) را شرح دهید.
	\begin{qsolve}
		\begin{enumerate}
			\item 
			\textbf{تعریف نقاط حضور (\lr{PoP}):}
			نقاط حضور یا \lr{Point of Presence (PoP)} مکان‌های فیزیکی هستند که در آن‌ها تجهیزات شبکه مانند روترها، سوئیچ‌ها و سرورها قرار دارند و اتصال اینترنت از طریق آن‌ها انجام می‌شود. این نقاط معمولاً در مراکز داده یا ساختمان‌های خاصی قرار دارند که امکان اتصال مستقیم بین ارائه‌دهندگان خدمات اینترنت (\lr{ISPs}) و سایر شبکه‌ها فراهم می‌کنند. نقاط حضور به عنوان مرکز ارتباطات شبکه‌ای برای تبادل ترافیک داده عمل می‌کنند.
			
			\item 
			\textbf{ارتباط \lr{PoP} با هزینه‌های سرمایه‌گذاری (\lr{CAPEX}) و نگهداری (\lr{OPEX}):}
			\begin{itemize}
				\item 
				\lr{CAPEX (Capital Expenditure)}: هزینه‌های سرمایه‌گذاری اولیه برای ایجاد نقاط حضور شامل خرید تجهیزات شبکه (روترها، سرورها، سوئیچ‌ها)، ساخت یا اجاره ساختمان، و زیرساخت‌های لازم (نظیر سیستم‌های خنک‌کننده، تامین برق اضطراری) است. این هزینه‌ها اغلب بسیار زیاد هستند، زیرا ایجاد \lr{PoP}های جدید نیازمند سرمایه‌گذاری قابل توجهی برای تجهیزات و زیرساخت‌های فیزیکی است.
			\end{itemize}

		\end{enumerate}
	\end{qsolve}
	
	
	
	\begin{qsolve}
		\begin{enumerate}
			\item [ ]
			\begin{itemize}
				\item 
				\lr{OPEX (Operating Expenditure)}: هزینه‌های عملیاتی و نگهداری نقاط حضور شامل هزینه‌های برق، نگهداری و تعمیر تجهیزات، نیروی انسانی برای مدیریت و پشتیبانی فنی، و هزینه‌های اجاره مکان است. این هزینه‌ها باید به طور مداوم پرداخت شوند تا نقاط حضور به‌صورت مداوم و بدون وقفه کار کنند. همچنین هزینه‌های عملیاتی شامل هزینه‌های پهنای باند و ترافیک داده نیز می‌شود که ممکن است بر اساس مقدار مصرف متغیر باشد.
			\end{itemize}
			
		\end{enumerate}
		
		در نهایت، میزان \lr{CAPEX} و \lr{OPEX} بسته به مقیاس و محل قرارگیری \lr{PoP}ها متفاوت است. یک \lr{PoP} بزرگ و مرکزی در یک منطقه شهری ممکن است هزینه‌های بالاتری نسبت به \lr{PoP}های کوچک‌تر در مناطق روستایی داشته باشد.
	\end{qsolve}
	
\end{enumerate}