\section{سوال اول}

فرض کنید یک مسیریاب با استفاده از الگوریتم تطابق بیشترین طول پیشوند (\lr{Longest Prefix Matching})، جلو‌رانی (\lr{Forwarding}) بسته‌ها را انجام می‌دهد. در صورتی که جدول جلورانی مسیریاب به صورت زیر باشد:


\begin{latin}
	\begin{center}
		\begin{tabular}{|c|c|}
			\hline
			\textbf{Prefix} & \textbf{Next Hop} \\
			\hline\hline
			\texttt{1010*} & A \\
			\hline
			\texttt{101*} & B \\
			\hline
			\texttt{101011*} & C \\
			\hline
			\texttt{100*} & D \\
			\hline
		\end{tabular}
	\end{center}
\end{latin}



گام بعدی بسته‌های دریافتی با آدرس‌های مقصد زیر را بدست آورید:

\begin{itemize}
	\item \texttt{10101101}
	\begin{qsolve}
		طولانی ترین تطابق با \texttt{101011*} است بنابراین مقصد C است.
	\end{qsolve}
	
	\item \texttt{10111101}
	\begin{qsolve}
		طولانی ترین تطابق با \texttt{101*} است بنابراین مقصد B است.
	\end{qsolve}
	
	\item \texttt{10001101}
	\begin{qsolve}
		طولانی ترین تطابق با \texttt{100*} است بنابراین مقصد D است.
	\end{qsolve}
\end{itemize}


%\begin{center}
%	\includegraphics*[width=0.7\linewidth]{pics/img4.png}
%	\captionof{figure}{مدل‌های برقراری ارتباط میان فرآیند‌ها}
%\end{center}