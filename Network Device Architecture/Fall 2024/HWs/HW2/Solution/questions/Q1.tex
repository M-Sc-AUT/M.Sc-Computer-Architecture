\section{سوال اول}
تفاوت‌های اصلی تکنیک‌های سوئیچینگ مداری و سوئیچینگ بسته‌ای را با استفاده از یک مثال کاربردی توضیح دهید. تأثیر هر یک از این تکنیک‌ها بر کیفیت و سرعت انتقال داده‌ها چیست؟



\begin{qsolve}
	\begin{enumerate}
		\item 
		\textbf{سوئیچینگ مداری (\lr{Circuit Switching}):}\\
		این تکنیک معمولاً در شبکه‌های تلفن ثابت و موبایل مورد استفاده قرار می‌گیرد. در این تکنیک، قبل از شروع انتقال داده، یک مسیر فیزیکی ثابت بین فرستنده و گیرنده برقرار می‌شود و این مسیر تا پایان ارتباط به طور انحصاری مورد استفاده قرار می‌گیرد.
		
		\textbf{مثال:}
		تصور کنید دو نفر در حال مکالمه تلفنی هستند. زمانی که شما شماره‌گیری می‌کنید، یک مسیر ارتباطی مستقیم بین شما و مخاطبتان در شبکه تلفن برقرار می‌شود. در طول مکالمه، این مسیر به طور اختصاصی در اختیار شما و مخاطب قرار دارد و هیچ کاربر دیگری نمی‌تواند از آن استفاده کند.
		
		\textbf{تأثیر بر کیفیت و سرعت:}
		\begin{itemize}
			\item 
			کیفیت: کیفیت در سوئیچینگ مداری بالاست، زیرا مسیر ثابتی اختصاص داده می‌شود و هیچ وقفه‌ای در جریان داده‌ها وجود ندارد.
			
			\item 
			سرعت: سرعت ثابت و قابل پیش‌بینی است، اما ممکن است شبکه کارآمدی لازم را نداشته باشد زیرا مسیر انحصاری حتی در زمان‌های عدم استفاده نیز اشغال می‌ماند.
		\end{itemize}
		
		
		
		
		\item 
		\textbf{سوئیچینگ بسته‌ای (\lr{Packet Switching}):}
		در این روش، داده‌ها به بسته‌های کوچک‌تری تقسیم می‌شوند و هر بسته به صورت مستقل از طریق مسیرهای مختلف به مقصد ارسال می‌شود. پروتکل‌هایی مثل \lr{TCP/IP} از این تکنیک استفاده می‌کنند.
		
		\textbf{مثال:}
		یک ایمیل را تصور کنید. هنگامی که ایمیل ارسال می‌شود، پیام شما به چندین بسته داده کوچک تقسیم می‌شود و این بسته‌ها به صورت جداگانه از طریق شبکه به گیرنده می‌رسند. هر بسته ممکن است از مسیری متفاوت عبور کند و در مقصد دوباره جمع‌آوری و به شکل اصلی بازگردانده شود.
		
		\textbf{تأثیر بر کیفیت و سرعت:}
		\begin{itemize}
			\item 
			کیفیت: کیفیت ارتباط در سوئیچینگ بسته‌ای ممکن است متغیر باشد، زیرا بسته‌ها ممکن است با تأخیر مواجه شوند یا حتی از بین بروند. با این حال، مکانیزم‌های کنترلی برای باز ارسال و تصحیح خطا وجود دارد.
			
			\item 
			سرعت: سرعت به دلیل تقسیم‌بندی داده‌ها و ارسال آنها از مسیرهای مختلف به طور کلی بالاتر است و شبکه از پهنای باند بهتری استفاده می‌کند، اما بسته‌ها ممکن است با تأخیرهای کوچک مواجه شوند.
		\end{itemize}
	\end{enumerate}
\end{qsolve}


\begin{qsolve}
	در نتیجه می‌توان گفت:
	\begin{itemize}
		\item 
		سوئیچینگ بسته‌ای برای انتقال داده‌های حجیم مثل ارسال ایمیل، فایل‌های بزرگ یا وب‌گردی بهتر است زیرا از پهنای باند کارآمدتر استفاده می‌کند و امکان اشتراک‌گذاری منابع بین چندین کاربر را فراهم می‌کند.
		
		\item 
		سوئیچینگ مداری برای ارتباطات حساس به تأخیر مثل مکالمات صوتی یا تصویری زنده مناسب‌تر است زیرا مسیر ثابتی برای جریان داده‌ها وجود دارد و کیفیت ثابتی ارائه می‌دهد.
	\end{itemize}
\end{qsolve}



%\begin{center}
%	\includegraphics*[width=0.7\linewidth]{pics/img4.png}
%	\captionof{figure}{مدل‌های برقراری ارتباط میان فرآیند‌ها}
%\end{center}