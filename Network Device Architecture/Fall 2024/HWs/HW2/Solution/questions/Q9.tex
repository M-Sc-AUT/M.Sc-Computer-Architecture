\section{سوال نهم}

در الگوریتم \lr{Binary search on prefix range} در صورتی که $n$ داده، $m$ بیتی داشته باشیم و از \lr{k-way search} استفاده کنیم.

\begin{itemize}
	\item در این الگوریتم چه ارتباطی بین $n$، $m$ و $k$ وجود دارد؟
	\begin{qsolve}
		در هر جست‌و جو، $\frac{1}{k}$ از جدول مرحله بعدی جست‌و‌جو باقی می‌ماند زیرا باید، $log(k)$ بیت را در جدول جست‌و‌جو کنیم. جدولی $2n$ ردیفی داریم که دارای $m$ بیت در هر ردیف است.
		
	\end{qsolve}
	
	\item در بدترین حالت مرتبه زمانی جستجو و حافظه مورد نیاز را محاسبه نمایید.
	\begin{qsolve}
		در بدترین حالت، پیچیدگی حافظه و زمان به‌ترتیب برابر هستند با $O(N)$ و $O(log(n))$
	\end{qsolve}
	
	\item در صورتی که عمل جستجو توسط پردازنده‌ای با فرکانس ساعت \lr{3.2 MHz} و بر روی یک جدول جلورانی با \(90000\) سطر (\lr{entries}) انجام شود، بیشترین زمان مورد نیاز و متوسط زمان مورد نیاز و میزان حافظه مورد نیاز را محاسبه نمایید.
	\begin{qsolve}
		با فرض اینکه هربار جدول به $k$ قسمت تقسیم شود و سپس مقایسه انجام شود، هر دفعه $k$ دسترسی به حافظه خواهیم داشت.
	\end{qsolve}

\end{itemize}


