\section{سوال چهارم}

درخت سوال ۳ را به فرم \lr{disjoint-prefix binary trie} تبدیل نمایید. حافظه مورد نیاز جهت نگهداری درخت سوال ۳ و سوال ۴ را مقایسه کنید.

\begin{qsolve}
	
	درخت \lr{Disjoint} حاصل به‌صورت زیر است:
	
	\begin{center}
		\includegraphics*[width=0.5\linewidth]{pics/q4.pdf}
		\captionof{figure}{درخت \lr{Disjoint}}
	\end{center}
	
	به این درخت باید ۴ گره جدید دیگر اضاف شود تا درخت کامل شود. در اینجا مجموعا ۱۳ برگ داریم و در این روش فقط برگ‌ها را در حافظه ذخیره می‌کنیم و درنتیجه کلا اطلاعات ۱۳ برگ را درحافظه داریم. اما در سوال ۴ کل درخت که شامل ۱۹ گره است را باید در حافظه نگهداری کنیم. \textbf{درنتیجه، در این روش، حافظه مصرفی کاهش می‌یابد.}
\end{qsolve}