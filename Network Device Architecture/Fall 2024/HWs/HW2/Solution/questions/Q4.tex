\section{سوال چهارم}
با جستجو در اینترنت یک نمونه مسیریاب \lr{IP} با کارآیی بالا مقیاس پذیر (قابل توسعه) را پیدا نموده و معماری آن را شرح دهید.


\begin{qsolve}
	در این سوال به بررسی روتر سیسکو \lr{CRS-1} می‌پردازیم. معماری این روتر را در شکل زیر آورده‌ شده است:
	
	\begin{center}
		\includegraphics*[width=0.7\linewidth]{pics/img2.png}
		\captionof{figure}{روتر \lr{CRS-1}}
	\end{center}
	
	در ادامه به توضیح هریک از بخش‌های این روتر می‌پردازیم.
	
	\begin{enumerate}
		\item 
		\lr{\textbf{:Line Card}} هر \lr{Line card} توسط یک \lr{Midplane} به دو جزء اصلی جدا می شود. ماژول رابط و \lr{MSC}. هر لاین کارت سیسکو \lr{CRS-1} یک نسخه مجزا از جدول مجاورت و پایگاه‌های اطلاعاتی فورواردینگ را حفظ می‌کند و حداکثر مقیاس‌پذیری و کارایی را ممکن می‌سازد.
	
	
	
	
		\item 
		\lr{\textbf{:Interface Module}} ماژول رابط، اتصالات فیزیکی به شبکه، از جمله عملکرد‌های لایه ۱ و ۲ را فراهم می‌کند. ماژول‌های رابط برای این روتر عبارت اند از:
		
		\begin{latin}
			\begin{itemize}
				\item \texttt{1-port OC-768c/STM- 256c PoS}
				\item \texttt{4-port OC- 192c/STM-64c PoS}
				\item \texttt{16-port OC-48c/STM-16c PoS}
				\item \texttt{8-port 10 Gigabit Ethernet}
				\item \texttt{1-port OC-768c/STM- 256c tunable WDMPOS}
				\item \texttt{4-port 10 Gigabit Ethernet tunable WDMPHY}
			\end{itemize} 
		\end{latin}
	\end{enumerate}
\end{qsolve}


\begin{qsolve}
	\begin{enumerate}
		\item [3.]
		\lr{\textbf{:Service Card Module}} 
		یک \lr{Forwarding engine} لایه ۳ با کارآیی بالا است. هر سیسکو \lr{CRS-1 MSC} مجهز به دو \lr{SPP} با کارآیی بالا و انعطاف پذیر است، یکی برای ورودی و دیگری برای پردازش بسته‌های خروجی.
		
		این کارت مسئولیت تمامی پردازش‌های بسته شامل کیفیت خدمات (\lr{QoS}) طبقه‌بندی و شکل‌دهی را بر عهده دارد و مجهز به صف‌های سلسله مراتبی سه سطحی با مجموع ۱۶۰۰۰ صف است.
		
		\item [4.]
		\lr{\textbf{:Cisco Silicon Packet Processor}}
		پیچیده‌ترین \lr{ASIC} موجود امروزی، از ۱۸۸ پردازنده \lr{RISC} سی‌و‌دو بیتی (که هر کدام می‌توانند به طور مستقل از یک کار مجزا کار کنند) در هر تراشه تشکیل شده است که به توان پردازش کاملاً انعطاف پذیر و ۴۰ گیگابیت بر ثانیه کمک می‌کند.
		
		انعطاف‌پذیری \lr{SPP} سیسکو با بکارگیری ویژگی‌های مختلف برای مسیریابی هسته، لبه و همتا بر اساس کد نرم‌افزار، بر روی یک سخت‌افزار را تسهیل می‌کند و نیاز به موتورهای خاص برای مسیریابی هسته در مقابل لبه را از بین می‌برد. سهولت معرفی کد جدید به طور قابل توجهی باعث تسریع زمان عرضه ویژگی‌ها، خدمات و برنامه‌های جدید به بازار می‌شود.
		
		
		\item [5.]
		\lr{\textbf{:Route Processors}}
		مدیریت و حسابداری هر پردازنده سیستم \lr{CRS Route} عملکردهای کنترلر \lr{Rack} را مدیریت می‌کند و با ۴ گیگابایت حافظه با دسترسی تصادفی پویا (\lr{DRAM}) در \lr{RP-B} و ۱۲٫۶ گیگابایت در \lr{PRP} به اضافه یک هارد دیسک ۴۰ گیگابایتی در \lr{RP-B} یا \lr{x32-GB2} پشتیبانی می‌کند.
		
		
		\item [6.]
		یک سیسکو \lr{CRS-1 Distributed Route Processor (DRP)} را می‌توان در هر شکاف لاین کارت موجود قرار داد و با افزایش مقیاس صفحه کنترل یا افزودن سرویس‌های جدید در صورت نیاز به جلوگیری از تنگناهای حافظه یا پردازش کمک می‌کند.
		
		\item [7.]
		\lr{\textbf{:Service-Intelligent Switch Fabric}}
		سوئیچ فابریکی که مسیر ارتباطی بین لاین کارت‌ها را فراهم می‌کند، معماری \lr{Benes} سه مرحله‌ای و خود مسیریابی (برای اولین بار برای مسیریاب‌های \lr{IP}) با بافر ۱۲۹۶$\times$۱۲۹۶ سوئیچینگ غیر مسدود کننده است. از نظر فیزیکی، سیسکو \lr{CRS-1} به هشت صفحه تقسیم می‌شود که بسته‌ها به سلول‌های شکسته شده به طور مساوی توزیع می‌شوند.
		\item [8.]
		\lr{\textbf{:Cisco IOS XR Software}}
		از آنجایی که نرم‌افزار سیستم \lr{CRS-1} بر روی معماری نرم‌افزار مبتنی بر میکروکرنل حفاظت شده از حافظه ساخته شده است، تنها عناصر پردازش ضروری مانند ارسال پیام، مدیریت حافظه، زمان‌بندی فرآیند و توزیع رشته در سطح هسته انجام می‌شوند. این معماری تأثیر هرگونه خرابی نرم‌افزار را در درایورهای دستگاه به حداقل می‌رساند و راه‌اندازی مجدد یا ارتقا این فرآیند‌ها را بدون نیاز به راه‌اندازی مجدد در سطح سیستم تسهیل می‌کند. این معماری مبتنی بر میکروکرنل امکان توزیع فرآیندهای صفحه کنترل، ارسال و مدیریت را برای استفاده کارآمد از منابع و حداکثر عملکرد صفحه کنترل فراهم می‌کند. مجموعه‌ای بسیار ساختاریافته از رابط‌های برنامه‌نویسی کاربردی (\lr{API}) و مکانیسم‌های ارسال پیام تضمین می‌کند که ارتباطات بین فرآیندی به‌طور یکسان و با کارایی یکسان در هر دو سیستم تک پردازنده و چند پردازنده عمل می‌کنند.
		
	\end{enumerate}
\end{qsolve}


