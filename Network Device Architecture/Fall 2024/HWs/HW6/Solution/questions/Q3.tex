\section{سوال سوم}

در شکل زیر یک سوئیچ \lr{8×8} را نشان می‌دهد. همان طور که مشخص است این سوئیچ دارای ساختاری درختی است. تمام لینک‌ها در هر شکل \lr{a} ظرفیت عبور تنها یک بسته در هر برش زمانی را دارند.


\begin{center}
	\includegraphics*[width=1\linewidth]{pics/img1.png}
	\captionof{figure}{ساختار سوئیچ سوال سوم}
	\label{سوئیچ سوال سوم}
\end{center}


\begin{enumerate}
	\item 
	
	الگوی ترافیکی را مثال بزنید که تمام پورت‌های ورودی و خروجی اشغال باشند اما سوئیچ دچار \lr{Blocking} نمی‌شود
	(فرض کنید الگویی که هر پورت ورودی به پورت خروجی هم‌نام خودش \lr{p(in)} به \lr{p(out)} وصل شده باشد امکان‌پذیر نباشد).
	\begin{qsolve}
		\begin{latin}
			\begin{enumerate}
				\item 
				$P_1(out) \ and \ P_0(in)$
				
				\item 
				$P_3(out) \ and \ P_2(in)$
				
				\item 
				$P_5(out) \ and \ P_4(in)$
				
				\item 
				$P_7(out) \ and \ P_6(in)$
			\end{enumerate}
		\end{latin}
	\end{qsolve}
	
	
	\item 
	الگوی ترافیکی را مثال بزنید که نشان دهد در شکل \lr{a} سوئیچ دچار \lr{Internal Blocking} می‌شود.
	
	\begin{qsolve}
		برای مثال اگر $P_0$ پورت ورودی و $P_4$ پورت خروجی باشد، با وجود اینکه پورت ورودی $P_1$ آزاد است و به غیر از $P_4$ تمام پورت‌های خروجی نیز آزاد هستند، به علت \lr{internal blocking} از $P_1$ به هیچ‌یک از پورت‌های $P_2$، $P_3$، $P_5$، $P_6$ و $P_7$ نمی‌توان بسته فرستاد.
	\end{qsolve}
	
	
	
	\item 
	اگر در شکل \lr{b} فرض کنیم خطوط پررنگ‌تر ظرفیت ارسال ۲ بسته در یک برش زمانی را دارند. آیا این تغییر سوئیچ شکل \lr{b} دچار \lr{Internal Blocking} نمی‌شود؟ 
	
		
\end{enumerate}
\newpage


\begin{enumerate}
	\item [ ]
	\begin{qsolve}
		
		بله، برای مثال دو انتقال زیر را درنظر بگیرید.
		\begin{itemize}
			\item $P_0$ ورودی و $P_4$ خروجی
			\item $P_1$ ورودی و $P_5$ خروجی
		\end{itemize}
		
		در این صورت، به دلیل \lr{internal blocking} با وجود آزاد بودن پورت ورودی $P_2$ و پورت خروجی $P_6$ امکان انتقال بسته از $P_2$ به $P_6$ وجود ندارد، زیرا نیاز دارد از بالاترین خط هم‌زمان ۳ بسته در یک برش زمانی ارسال شود که بیش از ظرفیت لینک (دو بسته) است.
		
	\end{qsolve}
	
	
	\item [4.]
	کمترین ظرفیتی که می‌توان به سوئیچ قسمت \lr{a} اضافه کرد که سوئیچ دچار \lr{Internal Blocking} نشود چیست؟
	\begin{qsolve}
		اگر ظرفیت لینک‌های قرمز ۲ بسته در یک برش زمانی و ظرفیت لینک‌های بنفش ۴ بسته در یک برش زمانی باشد، \lr{internal blocking} رخ نخواهد داد.
		
		\begin{center}
			\includegraphics*[width=0.5\linewidth]{pics/img5.png}
		\end{center}
	\end{qsolve}
	
	
	
\end{enumerate}