\section{سوال سوم}

در شکل زیر یک سوئیچ \lr{8×8} را نشان می‌دهد. همان طور که مشخص است این سوئیچ دارای ساختاری درختی است. تمام لینک‌ها در هر شکل \lr{a} ظرفیت عبور تنها یک بسته در هر برش زمانی را دارند.


\begin{center}
	\includegraphics*[width=1\linewidth]{pics/img1.png}
	\captionof{figure}{ساختار سوئیچ سوال سوم}
	\label{سوئیچ سوال سوم}
\end{center}


\begin{enumerate}
	\item 
	
	الگوی ترافیکی را مثال بزنید که تمام پورت‌های ورودی و خروجی اشغال باشند اما سوئیچ دچار \lr{Blocking} نمی‌شود
	(فرض کنید الگویی که هر پورت ورودی به پورت خروجی هم‌نام خودش \lr{p(in)} به \lr{p(out)} وصل شده باشد امکان‌پذیر نباشد).
	
	
	\item 
	الگوی ترافیکی را مثال بزنید که نشان دهد در شکل \lr{a} سوئیچ دچار \lr{Internal Blocking} می‌شود.
	
	
	\item 
	اگر در شکل \lr{b} فرض کنیم خطوط پررنگ‌تر ظرفیت ارسال ۲ بسته در یک برش زمانی را دارند. آیا این تغییر سوئیچ شکل \lr{b} دچار \lr{Internal Blocking} نمی‌شود؟ 
	
	
	\item 
	کمترین ظرفیتی که می‌توان به سوئیچ قسمت \lr{a} اضافه کرد که سوئیچ دچار \lr{Internal Blocking} نشود چیست؟
	
	
\end{enumerate}