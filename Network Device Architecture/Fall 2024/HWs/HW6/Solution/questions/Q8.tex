\section{سوال هشتم}

اجزای یک سوئیچ \lr{OpenFlow} نسخه \lr{5.1} را نشان دهید و هر کدام را شرح دهید.


\begin{qsolve}
	 \begin{enumerate}
	 	\item \textbf{\lr{Flow Table} (جدول جریان):} 
	 	این جدول، قوانین مربوط به جریان‌ها را ذخیره کرده و تصمیم‌گیری‌های لازم برای بسته‌های ورودی را انجام می‌دهد.
	 	\begin{itemize}
	 		\item \lr{Match Fields:} فیلدهای تطبیق مانند آدرس \lr{IP} و شماره پورت.
	 		\item \lr{Actions:} اقداماتی مانند ارسال به پورت مشخص یا حذف بسته.
	 		\item \lr{Counters:} شمارنده‌هایی برای ثبت تعداد و حجم بسته‌های پردازش‌شده.
	 	\end{itemize}
	 	
	 	\item \textbf{\lr{Group Table} (جدول گروه):} 
	 	برای انجام عملیات پیشرفته‌تر مانند \lr{Multicast} یا \lr{Load Balancing}.
	 	\begin{itemize}
	 		\item تعریف گروه‌هایی از اقدامات.
	 		\item ارسال بسته به چندین مقصد به صورت همزمان.
	 	\end{itemize}
	 	
	 	\item \textbf{\lr{Meter Table} (جدول اندازه‌گیری):} 
	 	مدیریت پهنای باند و اعمال سیاست‌های \lr{QoS}.
	 	\begin{itemize}
	 		\item اندازه‌گیری نرخ جریان داده.
	 		\item اولویت‌بندی جریان‌ها.
	 	\end{itemize}
	 	
	 	\item \textbf{\lr{Packet Buffer} (بافر بسته):} 
	 	ذخیره موقت بسته‌هایی که در انتظار پردازش یا ارسال به کنترل‌کننده هستند.
	 	
	 	\item \textbf{\lr{OpenFlow Channel} (کانال ارتباطی):} 
	 	ارتباط بین سوئیچ و کنترل‌کننده \lr{SDN}.
	 	\begin{itemize}
	 		\item ارسال و دریافت پیام‌های کنترل.
	 		\item تضمین ارتباط امن.
	 	\end{itemize}
	 	
	 	\item \textbf{\lr{Pipeline} (پایپ‌لاین):} 
	 	مجموعه‌ای از جدول‌های جریان که به صورت متوالی پردازش می‌شوند.
	 	
	 	\item \textbf{\lr{Statistics Collection} (جمع‌آوری آمار):} 
	 	جمع‌آوری آمار مربوط به جریان‌ها، پورت‌ها و پهنای باند.
	 	
	 	\item \textbf{\lr{Secure Channel} (کانال امن):} 
	 	ارتباط امن بین کنترل‌کننده و سوئیچ با استفاده از رمزنگاری.
	 \end{enumerate}
\end{qsolve}