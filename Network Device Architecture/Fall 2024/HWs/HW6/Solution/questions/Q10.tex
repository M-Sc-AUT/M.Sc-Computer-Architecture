\section{سوال دهم}

معماری سوئیچ‌های نسل جدید \lr{Huawei} و \lr{Intel} را بررسی کرده و نوع پیاده‌سازی و ویژگی‌های سوئیچ‌های \lr{OpenFlow} مانند عملیات \lr{Pipelining} را شرح دهید.

\begin{qsolve}
	\begin{enumerate}
		\item \textbf{معماری سوئیچ‌های نسل جدید \lr{Huawei}:}
		\begin{itemize}
			\item \textbf{معماری سخت‌افزار-محور و نرم‌افزار-محور:} سوئیچ‌های نسل جدید \lr{Huawei} ترکیبی از معماری سخت‌افزار-محور و نرم‌افزار-محور را ارائه می‌دهند. از تراشه‌های پردازش سریع برای عملیات \lr{Data Plane} و از کنترل‌کننده‌های نرم‌افزاری برای مدیریت و کنترل استفاده می‌شود.
			\item \textbf{پشتیبانی از \lr{OpenFlow}:} سوئیچ‌های \lr{Huawei} از پروتکل \lr{OpenFlow} پشتیبانی کرده و قابلیت پیاده‌سازی جریان‌های پیچیده را دارند.
			\item \textbf{ویژگی‌های کلیدی:}
			\begin{itemize}
				\item پشتیبانی از \lr{SDN}: قابلیت برنامه‌ریزی شبکه و مدیریت مرکزی.
				\item عملیات موازی: استفاده از معماری پیشرفته برای پردازش چندین جریان به‌صورت موازی.
				\item کیفیت خدمات (\lr{QoS}): پشتیبانی از سیاست‌های پیشرفته \lr{QoS} برای کنترل ترافیک شبکه.
				\item امنیت: استفاده از ماژول‌های سخت‌افزاری و نرم‌افزاری برای جلوگیری از حملات سایبری.
			\end{itemize}
		\end{itemize}
		
		\item \textbf{معماری سوئیچ‌های نسل جدید \lr{Intel}:}
		\begin{itemize}
			\item \textbf{معماری سخت‌افزار عمومی (\lr{Commodity Hardware}):} سوئیچ‌های \lr{Intel} اغلب از پردازنده‌های \lr{FPGA} و \lr{ASIC} برای عملیات سریع شبکه استفاده می‌کنند. این معماری انعطاف‌پذیر بوده و برای اجرای برنامه‌های \lr{SDN} بهینه‌سازی شده است.
			\item \textbf{پشتیبانی از \lr{P4}:} سوئیچ‌های \lr{Intel} از زبان برنامه‌نویسی \lr{P4} برای تعریف رفتار سوئیچ استفاده می‌کنند. این قابلیت به توسعه‌دهندگان امکان می‌دهد تا عملیات سفارشی در \lr{Data Plane} را پیاده‌سازی کنند.
			\item \textbf{ویژگی‌های کلیدی:}
			\begin{itemize}
				\item پشتیبانی از \lr{OpenFlow} و \lr{P4}: امکان استفاده از هر دو برای توسعه شبکه‌های انعطاف‌پذیر.
				\item عملیات کم‌تأخیر: طراحی‌شده برای تأخیر بسیار پایین.
				\item پشتیبانی از آنالیز داده: جمع‌آوری و پردازش داده‌ها به‌صورت بلادرنگ برای بهینه‌سازی شبکه.
			\end{itemize}
		\end{itemize}
		
		\item \textbf{ویژگی‌های \lr{OpenFlow} مانند عملیات \lr{Pipelining}:}
		\begin{itemize}
			\item \textbf{عملیات \lr{Pipelining}:}
			\begin{itemize}
				\item \textbf{تعریف:} عملیات \lr{Pipelining} در \lr{OpenFlow} به معنای پردازش بسته‌ها به‌صورت مرحله‌ای از طریق مجموعه‌ای از جدول‌های جریان (\lr{Flow Tables}) است. بسته‌ها در هر مرحله پردازش شده و تصمیم‌گیری در مورد آن‌ها انجام می‌شود.
				\item \textbf{نحوه عملکرد:}
				\begin{itemize}
					\item بسته وارد اولین جدول جریان (\lr{Flow Table}) می‌شود.
					\item اگر قانون تطابق پیدا کند، عملیات مشخص‌شده روی بسته اجرا می‌شود.
					\item بسته به جدول بعدی منتقل می‌شود (در صورت نیاز) تا پردازش بیشتری انجام شود.
					\item در نهایت، بسته به مقصد نهایی ارسال می‌شود یا حذف می‌گردد.
				\end{itemize}
			\end{itemize}
		\end{itemize}
	\end{enumerate}
\end{qsolve}
\newpage


\begin{qsolve}
	\begin{enumerate}
		\item [ ]
		\begin{itemize}
			\begin{itemize}
				\item \textbf{مزایا:} پردازش مرحله‌ای و ماژولار. امکان تعریف قوانین پیچیده و چندلایه. بهینه‌سازی پردازش بسته‌ها در شبکه.
			\end{itemize}
			
			
			\item \textbf{پشتیبانی از \lr{Group Table}:}
			\begin{itemize}
				\item \textbf{تعریف:} جدول گروه برای تعریف اقدامات پیچیده‌تر مانند مسیریابی چندبخشی (\lr{Multicast}) یا توزیع بار (\lr{Load Balancing}) استفاده می‌شود.
				\item \textbf{ویژگی‌ها:} امکان ارسال بسته به چندین پورت. مدیریت بهتر منابع شبکه.
			\end{itemize}
			\item \textbf{پشتیبانی از \lr{Meter Table}:}
			\begin{itemize}
				\item \textbf{تعریف:} این جدول برای مدیریت پهنای باند و سیاست‌های \lr{QoS} استفاده می‌شود.
				\item \textbf{عملکرد:} اندازه‌گیری نرخ ارسال بسته‌ها. اعمال محدودیت یا اولویت‌بندی بسته‌ها بر اساس سیاست‌های تعریف‌شده.
			\end{itemize}
		\end{itemize}
	\end{enumerate}
\end{qsolve}





