\section{سوال یازدهم}

معماری سوئیچ‌های مبتنی بر چارچوب \lr{ForCES} را بررسی کنید و ویژگی‌های این چارچوب را شرح دهید.

\begin{qsolve}
	\begin{enumerate}
		\item \textbf{عناصر کنترلی (\lr{Control Elements - CEs}):}
		\begin{itemize}
			\item مسئول تصمیم‌گیری در مورد مسیردهی، سیاست‌ها، و مدیریت جریان‌ها هستند.
			\item کنترل‌کننده‌ها با ارسال دستورات به عناصر فورواردینگ عملیات شبکه را تنظیم می‌کنند.
		\end{itemize}
		
		\item \textbf{عناصر فورواردینگ (\lr{Forwarding Elements - FEs}):}
		\begin{itemize}
			\item مسئول ارسال بسته‌ها طبق قوانینی هستند که از عناصر کنترلی دریافت می‌کنند.
			\item از عناصر سخت‌افزاری یا نرم‌افزاری تشکیل شده و بسته‌ها را با توجه به جداول فورواردینگ پردازش می‌کنند.
		\end{itemize}
		
		\item \textbf{پروتکل ارتباطی \lr{ForCES}:}
		\begin{itemize}
			\item پروتکل استانداردی برای ارتباط بین \lr{CEs} و \lr{FEs}.
			\item این پروتکل وظیفه انتقال اطلاعات کنترل، تنظیمات، و وضعیت شبکه را برعهده دارد.
		\end{itemize}
		
		\item \textbf{مدل داده \lr{ForCES}:}
		\begin{itemize}
			\item تعریف ساختار و فرمت داده‌هایی که بین عناصر کنترلی و فورواردینگ تبادل می‌شوند.
			\item مدل داده قابل برنامه‌ریزی است و امکان توسعه آسان را فراهم می‌کند.
		\end{itemize}
	\end{enumerate}
	
	\textbf{ویژگی‌های چارچوب \lr{ForCES}:}
	
	\begin{enumerate}
		\item \textbf{جداسازی کنترلی و فورواردینگ:}
		\begin{itemize}
			\item چارچوب \lr{ForCES} ارتباط مستقلی بین عناصر کنترلی و فورواردینگ ایجاد می‌کند.
			\item این جداسازی باعث افزایش مقیاس‌پذیری و انعطاف‌پذیری در شبکه می‌شود.
		\end{itemize}
		
		\item \textbf{پشتیبانی از معماری‌های متنوع:}
		\begin{itemize}
			\item این چارچوب برای انواع معماری‌های شبکه از جمله \lr{SDN} و شبکه‌های سنتی مناسب است.
			\item می‌تواند در محیط‌های سخت‌افزاری یا مجازی استفاده شود.
		\end{itemize}
		
		\item \textbf{انعطاف‌پذیری بالا:}
		\begin{itemize}
			\item به توسعه‌دهندگان اجازه می‌دهد قوانین فورواردینگ را به‌صورت داینامیک تغییر دهند.
			\item از عملیات پیچیده مانند تغییر مسیر، سیاست‌های امنیتی، و کیفیت خدمات (\lr{QoS}) پشتیبانی می‌کند.
		\end{itemize}
		
		\item \textbf{استاندارد باز:}
		\begin{itemize}
			\item چارچوب \lr{ForCES} توسط \lr{IETF} توسعه داده شده و به عنوان یک استاندارد باز عمل می‌کند.
			\item این ویژگی امکان سازگاری و همکاری بین تولیدکنندگان مختلف تجهیزات شبکه را فراهم می‌کند.
		\end{itemize}
	\end{enumerate}
\end{qsolve}
\newpage

\begin{qsolve}
	\begin{enumerate}
		\item [5.] \textbf{امنیت بالا:}
		\begin{itemize}
			\item پشتیبانی از ارتباطات امن بین \lr{CEs} و \lr{FEs}.
			\item استفاده از روش‌های رمزنگاری و مکانیزم‌های احراز هویت برای محافظت از داده‌ها.
		\end{itemize}
		
		\item [6.] \textbf{مدیریت ساده‌تر:}
		\begin{itemize}
			\item جداسازی وظایف کنترل و فورواردینگ باعث ساده‌تر شدن مدیریت شبکه می‌شود.
			\item امکان نظارت و تحلیل بهتر عملکرد شبکه را فراهم می‌کند.
		\end{itemize}
	\end{enumerate}
\end{qsolve}