\section{سوال ششم}

\begin{itemize}
	\item (الف) مزایا و معایب سوئیچ‌های \lr{Banyan} را شرح دهید.
	\begin{qsolve}
		\begin{enumerate}
			\item 
			\textbf{مزایا:}
			
			\begin{itemize}
				\item \textbf{کارایی بالا:}
				\lr{Banyan}
				به دلیل طراحی چندمرحله‌ای، تأخیر کمتری نسبت به سوئیچ‌های تک‌مرحله‌ای دارد.
				\item \textbf{پیاده‌سازی ساده:}
				معماری ساده‌ای داشته و نیاز به اجزای پیچیده ندارد.
				\item \textbf{قابلیت مقیاس‌پذیری:}
				امکان گسترش اندازه سوئیچ با اضافه کردن مراحل یا گره‌ها وجود دارد.
				\item \textbf{حداقل مسیریابی:}
				ساختار مرتب‌شده‌ای دارد که مسیریابی را آسان و با حداقل تأخیر ممکن می‌سازد.
			\end{itemize}
			
			
			\item 
			\textbf{معایب:}
			
			\begin{itemize}
				\item \textbf{بلاک شدن داخلی (\lr{Internal Blocking}):}
				اگر چند بسته بخواهند از یک لینک مشترک استفاده کنند، ممکن است بلاک شدن رخ دهد.
				\item \textbf{عدم تحمل خطا:}
				خرابی یک گره یا لینک می‌تواند کل سیستم را مختل کند.
				\item \textbf{الگوهای ترافیکی محدود:}
				الگوهای خاص ترافیک ممکن است بهره‌وری و عملکرد را کاهش دهند.
				\item \textbf{پیچیدگی در کنترل ترافیک:}
				برای جلوگیری از بلاک شدن داخلی، به کنترل‌کننده‌های پیچیده نیاز است.
			\end{itemize}
		\end{enumerate}
	\end{qsolve}
	
	
	\item (ب) یک سوئیچ \lr{16×16 Banyan} رسم کنید که شامل \lr{Shuffle} و \lr{Unshuffled} باشد.
	\begin{qsolve}
		سوئیچ \lr{Banyan} با اندازه \lr{16×16} شامل چهار مرحله است زیرا $4^2 = 16$ و هر مرحله از سوئیچ‌های \lr{2×2} تشکیل شده است.
		
		\begin{enumerate}
			\item تعداد مراحل برابر با \(\log_2(16) = 4\) است.
			\item در هر مرحله، \(8\) سوئیچ \lr{2×2} مورد نیاز است.
			\item اتصالات \lr{Shuffle} و \lr{Unshuffle} به این صورت انجام می‌شود:
			\begin{itemize}
				\item \lr{Shuffle:} خروجی \(i\) به ورودی \((i \times 2) \mod 16\) وصل می‌شود.
				\item \lr{Unshuffle:} خروجی \(i\) به ورودی \((i / 2)\) یا \(i / 2 + 8\) (برای اندیس‌های فرد) متصل می‌شود.
			\end{itemize}
		\end{enumerate}
	\end{qsolve}
\end{itemize}
\newpage


\begin{center}
	\includegraphics*[width=1\linewidth]{pics/img24.pdf}
\end{center}
