\section{سوال هفتم}

فرض کنید در یک شبکه، مسیریابی به‌صورت خودکار از کوتاه‌ترین مسیر برای هر جریان استفاده می‌کند. این امر باعث شده است که یک لینک مشخص به ظرفیت حداکثری خود برسد و دچار ازدحام شود، در حالی که سایر لینک‌ها کمتر از ظرفیت خود استفاده می‌شوند.

\begin{enumerate}
	\item 
	با در نظر گرفتن ظرفیت هر لینک و نیازمندی‌های پهنای باند برای هر جریان، یک طرح توزیع بهینه برای جریان‌ها ارائه دهید که بار را در شبکه به طور یکنواخت توزیع کند.
	
	\begin{qsolve}
		برای توزیع یکنواخت بار در شبکه، می‌توان از رویکردهای زیر استفاده کرد:
		
		\begin{enumerate}
			\item \textbf{استفاده از الگوریتم‌های مسیریابی چندمسیره (\lr{Multipath Routing}):} 
			به‌جای ارسال تمام جریان‌ها از یک مسیر، می‌توان ترافیک را به چند مسیر تقسیم کرد. به عنوان مثال، استفاده از پروتکل‌هایی مانند \lr{ECMP (Equal-Cost Multi-Path)} پیشنهاد می‌شود.
			
			\item \textbf{مدل‌سازی ریاضی:} 
			با تعریف مدل بهینه‌سازی ریاضی، توزیع بار شبکه را می‌توان به شکل زیر بهینه کرد:
			\begin{itemize}
				\item \textbf{متغیر تصمیم:} \( x_{ij} \) مقدار ترافیک ارسال شده از جریان \( j \) روی لینک \( i \).
				\item \textbf{محدودیت:} 
				\[
				\sum_j x_{ij} \leq C_i \quad \forall i
				\]
				که \( C_i \) ظرفیت لینک \( i \) است.
				\item \textbf{هدف:} 
				کمینه‌سازی ازدحام یا توزیع یکنواخت بار:
				\[
				\max_i \frac{\sum_j x_{ij}}{C_i}.
				\]
			\end{itemize}
			
			\item \textbf{مراحل طراحی:}
			\begin{itemize}
				\item \textbf{تحلیل شبکه:} شناسایی لینک‌های پرترافیک و مسیرهای جایگزین.
				\item \textbf{تقسیم ترافیک:} انتقال بخشی از جریان‌ها به مسیرهای کم‌مصرف‌تر.
				\item \textbf{بازتخصیص منابع:} بر اساس ظرفیت لینک‌ها و نیازمندی‌های جریان‌ها، توزیع بهینه انجام می‌شود.
			\end{itemize}
		\end{enumerate}
	\end{qsolve}
	
	
	\item 
	نشان دهید که این بازطراحی چگونه می‌تواند تأخیر ناشی از ازدحام را کاهش دهد. برای این منظور، فرض کنید ظرفیت لینک ازدحام‌کرده ۱۰۰ مگابیت در ثانیه است و میزان ترافیک جاری روی آن به ۱۲۰ مگابیت در ثانیه رسیده است. توزیع جدید را طوری طراحی کنید که استفاده از لینک به کمتر از ۸۰ درصد ظرفیت برسد و میزان تأخیر را محاسبه و با وضعیت اولیه مقایسه کنید.
\end{enumerate}

\begin{enumerate}
	\item [ ]
	
	\begin{qsolve}
		\paragraph{وضعیت اولیه:}
		\begin{itemize}
			\item \textbf{ظرفیت لینک:} \( C = 100 \, \text{\lr{Mbps}} \).
			\item \textbf{ترافیک جاری:} \( T = 120 \, \text{\lr{Mbps}} \).
			\item \textbf{درصد استفاده:} 
			\[
			\frac{T}{C} \times 100 = \frac{120}{100} \times 100 = 120\%.
			\]
			\item \textbf{مدل تأخیر:} فرض کنیم تأخیر بر اساس رابطه \( D = \frac{1}{C - T} \) محاسبه شود. در این حالت:
			\[
			D_{\text{initial}} = \frac{1}{100 - 120} = \infty.
			\]
			ازدحام شدید باعث تأخیر بی‌نهایت می‌شود.
		\end{itemize}
		
		\paragraph{وضعیت بازطراحی‌شده:}
		\begin{itemize}
			\item \textbf{هدف:} کاهش استفاده لینک به \( 80\% \) ظرفیت:
			\[
			\text{Target Traffic} = 0.8 \times 100 = 80 \, \text{\lr{Mbps}}.
			\]
			\item \textbf{بازتوزیع جریان‌ها:} 
			با انتقال \( 40 \, \text{\lr{Mbps}} \) ترافیک به مسیرهای دیگر، ترافیک لینک به \( T_{\text{new}} = 80 \, \text{\lr{Mbps}} \) کاهش می‌یابد.
		\end{itemize}
		
		\paragraph{محاسبه تأخیر در وضعیت جدید:}
		با استفاده از رابطه \( D = \frac{1}{C - T} \) در وضعیت جدید:
		\[
		D_{\text{new}} = \frac{1}{100 - 80} = \frac{1}{20} = 0.05 \, \text{\lr{ms}}.
		\]
		
		\paragraph{مقایسه تأخیر:}
		\begin{itemize}
			\item وضعیت اولیه: تأخیر بی‌نهایت به دلیل ازدحام شدید.
			\item وضعیت بازطراحی‌شده: کاهش تأخیر به \( 0.05 \, \text{\lr{ms}} \).
		\end{itemize}
	\end{qsolve}
	
\end{enumerate}
