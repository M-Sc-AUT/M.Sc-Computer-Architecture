\section{سوال پنجم}

فرض کنید یک منبع با حجم نامحدودی از اطلاعات برای ارسال، از یک کنترل حلقه بسته (\lr{closed-loop control}) استفاده می‌کند تا نرخ ارسال خود را براساس اطلاعات بازخورد (\lr{feedback}) تنظیم کند. در صورتی که اطلاعات بازخورد نشان دهد هیچ ترافیکی (\lr{traffic}) در مسیر وجود ندارد، منبع به‌صورت پیوسته نرخ ارسال خود را به‌صورت خطی (\lr{linear}) افزایش می‌دهد. اما اگر اطلاعات بازخورد حاکی از وجود ترافیک در مسیر باشد، منبع نرخ ارسال را به صفر کاهش می‌دهد و سپس این چرخه را با افزایش تدریجی نرخ ارسال ادامه می‌دهد تا بار دیگر ترافیک شناسایی شود. حال فرض کنید که مدت زمانی معادل $T$ ثانیه طول می‌کشد تا اطلاعات بازخورد پس از وقوع ترافیک به منبع برسد. نمودار نرخ ارسال منبع را نسبت به زمان برای مقادیر کوچک و بزرگ $T$ ترسیم کنید و توضیح دهید که تأخیر انتشار $T$ \lr{(Propagation Delay)} چه نقشی در این کنترل حلقه بسته ایفا می‌کند.