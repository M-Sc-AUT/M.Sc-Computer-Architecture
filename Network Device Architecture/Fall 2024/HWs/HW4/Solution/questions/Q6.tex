\section{سوال ششم}

شبکه زیر را درنظر بگیرید:

\begin{center}
	\includegraphics*[width=0.5\linewidth]{pics/img1.png}
	\captionof{figure}{شبکه سوال ۶}
\end{center}


فرض کنید که ارتباط‌های زیر به ترتیب (چپ به راست) باید ایجاد شوند:

\begin{latin}
	$$ 
	5 \rightarrow 8, \quad 1 \rightarrow 8, \quad 2 \rightarrow 4, \quad 3 \rightarrow 8, \quad 3 \rightarrow 5, \quad 2 \rightarrow 1, \quad 1 \rightarrow 3, \quad 3 \rightarrow 6, \quad 6 \rightarrow 7, \quad 7 \rightarrow 8
	$$
\end{latin}

\begin{enumerate}
	\item 
	با استفاده از الگوریتم مسیریابی کوتاه‌ترین مسیر (\lr{Shortest-Path}) بیشترین ارتباطی که می‌توانید را برقرار کنید. ارتباط‌های قطع شده را نیز مشخص کنید.
	\begin{qsolve}
		با فرض اینکه ظرفیت هر لینک ۱ واحد و ظرفیتی که هر ارتباط به خود اختصاص می‌دهد نیز ۱ واحد باشد، می‌توان گفت که هر ارتباط فقط یک لینک را اشغال می‌کند و هر لینک صرفا ببرای یک ارتباط استفاد می‌شود.
		
		ابتدا ارتباط تک گام را رسم می‌کنیم سپس بررسی می‌کنیم که کدام ارتباط دو گام را اگر بررسی کنیم بهتر است و ارتباطات به بیشترین تعداد خود می‌رسد.
		
		\begin{center}
			\includegraphics*[width=0.4\linewidth]{pics/Q6_a.pdf}
			\captionof{figure}{بیشترین ارتباطات قابل رسم}
		\end{center}
	\end{qsolve}
	\newpage
\end{enumerate}


\begin{enumerate}
	\item [ ]
	\begin{qsolve}
		همه ارتباطات به جز ارتباط $3 \rightarrow 8 $ برقرار هستند.
	\end{qsolve}
	
	
	\item [2. ]
	با استفاده از الگوریتم مسیریابی \lr{Constraint Shortest-Path} بیشترین ارتباطی که می‌توانید را برقرار کنید. ارتباط‌های قطع شده را نیز مشخص کنید.
	
	\begin{qsolve}
		اگر در این قسمت نیز همان فرض قسمت قبل را درنظر بگیریم، الگوریتم \lr{CSPF} همانند الگوریتم \lr{Shortest Path} عمل می‌کند و نتیجه با قسمت قبل یکسان می‌شود.
	\end{qsolve}
	
	
	
	
	\item [3. ]
	آیا می‌توانید الگوریتم \lr{Constraint Shortest-Path} را بهبود دهید؟
	\begin{qsolve}
		اگر ظرفیت لینک‌ها را تقسیم کنیم، الگوریتم \lr{CSPF} می‌تواند لینک $3 \rightarrow 8 $ را نیز برقرار کند، برای افزایش سرعت الگوریتم نیز می‌توان آن را \lr{Pipeline} کرد.
	\end{qsolve}
\end{enumerate}