\section{سوال سوم}

با توجه به مدیریت صف با استفاده از تشخیص زودهنگام تصادفی (\lr{RED - Random Early Detection}):

\begin{enumerate}
	\item 
	توضیح دهید که چرا \lr{RED} به جلوگیری از شناسایی ترافیک \lr{TCP} از طریق فرستنده‌ها و کاهش هم‌زمان نرخ انتقال آن‌ها کمک می‌کند.
	
	
	\item 
	تأثیر \lr{RED} بر روی توان شبکه (\lr{Throughput}) را بررسی کنید.
	
	
	\item 
	پیچیدگی پیاده‌سازی الگوریتم \lr{RED} را بررسی کنید.
	
	
	\item 
	توضیح دهید که اگر به جای استفاده از طول متوسط صف (\lr{average queue length}) از طول لحظه‌ای صف (\lr{instantaneous queue length}) استفاده شود، چه پیامد‌هایی خواهد داشت.
	
	
	\item 
	راه‌هایی برای پیدا کردن مقادیر معقول برای پارامترهای \lr{RED} (یعنی $Min_{th}$ و $Max_{th}$ و احتمال افت بسته زمانی که طول متوسط صف به $Max_{th}$​ می‌رسد) را بررسی کنید.
\end{enumerate}



