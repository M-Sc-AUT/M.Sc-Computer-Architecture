\section{سوال دوم}

فرض کنید که در یک سیستم صف عادلانه وزن‌دار (\lr{Weighted Fair-Queuing system})، یک بسته با برچسب اتمام $F$ \lr{(finish tag)} در زمان $t$ وارد خدمت می‌شود. آیا ممکن است بسته‌ای بعد از زمان $t$ به سیستم برسد و برچسب اتمام آن کمتر از $F$ باشد؟ اگر بله، مثالی بزنید و اگر خیر، توضیح دهید.

\begin{qsolve}
	\textbf{بله، ممکن است.}
	
	
	در \lr{WFQ}، مقدار (\( F \)) به عوامل زیر بستگی دارد:
	\begin{itemize}
		\item \textbf{وزن جریان (\( W \))}: جریان‌هایی که وزن بیشتری دارند، ممکن است برچسب اتمام کمتری دریافت کنند.
		\item \textbf{اندازه بسته (\( L \))}: بسته‌های کوچکتر معمولاً زودتر تمام می‌شوند و برچسب اتمام کمتری دارند.
		\item \textbf{زمان ورود (\( t \))}: بسته‌ای که بعد از زمان \( t \) وارد شود، ممکن است به دلیل وزن بیشتر یا اندازه کوچکتر، برچسب اتمام کمتری دریافت کند.
	\end{itemize}
	
	
	برای مثال فرض شود دو جریان به‌صورت زیر داریم:
	
	\begin{itemize}
		\item جریان \( A \) با وزن \( W_A = 1 \)
		\item جریان \( B \) با وزن \( W_B = 2 \)
	\end{itemize}
	
	\textbf{سناریو:}
	\begin{enumerate}
		\item در زمان \( t = 10 \)، بسته \( P_1 \) از جریان \( A \) وارد سیستم می‌شود:
		\begin{itemize}
			\item اندازه بسته \( L_1 = 100 \)
			\item برچسب شروع (\( S_1 \)) برابر با \( t \) است: \( S_1 = 10 \)
			\item برچسب اتمام (\( F_1 \)): \( F_1 = S_1 + \frac{L_1}{W_A} = 10 + \frac{100}{1} = 110 \)
		\end{itemize}
		
		\item در زمان \( t = 15 \)، بسته \( P_2 \) از جریان \( B \) وارد سیستم می‌شود:
		\begin{itemize}
			\item اندازه بسته \( L_2 = 50 \)
			\item برچسب شروع (\( S_2 \)) برابر با حداکثر زمان جاری (\( 15 \)) یا برچسب اتمام جریان قبلی است: \( S_2 = 15 \)
			\item برچسب اتمام (\( F_2 \)): \( F_2 = S_2 + \frac{L_2}{W_B} = 15 + \frac{50}{2} = 40 \)
		\end{itemize}
	\end{enumerate}
	
	در نتیجه، بسته \( P_2 \) که در زمان \( t = 15 \) وارد سیستم شده است، برچسب اتمام \( F_2 = 40 \) دارد که \textbf{کمتر از \( F_1 = 110 \)} است، با اینکه \( P_2 \) بعد از \( P_1 \) وارد سیستم شده است.
	
\end{qsolve}