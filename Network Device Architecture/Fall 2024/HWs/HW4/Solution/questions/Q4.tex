\section{سوال چهارم}


یک شبکه از الگوریتم \lr{RED (Random Early Detection)} برای مدیریت ازدحام استفاده می‌کند. ظرفیت صف بین آستانه‌های حداقل (\lr{min-threshold}) و حداکثر (\lr{max-threshold}) تنظیم شده است. در این شبکه:


آستانه حداقل برابر ۲۰ بسته و آستانه حداکثر برابر ۵۰ بسته است. اگر طول صف از آستانه حداقل عبور کند، احتمال حذف بسته‌ها به تدریج افزایش می‌یابد و با رسیدن به آستانه حداکثر، این احتمال به ۱۰۰٪ می‌رسد.

\begin{enumerate}
	\item 
	اگر طول صف در لحظه‌ای به ۴۰ بسته برسد، با توجه به مقادیر حداقل و حداکثر، نرخ حذف بسته‌ها را محاسبه کنید.
	
	
	\item 
	یک سناریو شبیه‌سازی کنید که در آن طول صف به طور پیوسته افزایش می‌یابد و تأثیر الگوریتم \lr{RED} بر ترافیک شبکه را تحلیل کنید. مشخص کنید که چگونه \lr{RED} می‌تواند به کاهش ازدحام و جلوگیری از پر شدن کامل صف کمک کند و چه تاثیری بر تأخیر و نرخ ازدحام در شبکه دارد.
\end{enumerate}