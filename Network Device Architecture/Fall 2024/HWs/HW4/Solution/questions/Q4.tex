\section{سوال چهارم}


یک شبکه از الگوریتم \lr{RED (Random Early Detection)} برای مدیریت ازدحام استفاده می‌کند. ظرفیت صف بین آستانه‌های حداقل (\lr{min-threshold}) و حداکثر (\lr{max-threshold}) تنظیم شده است. در این شبکه:


آستانه حداقل برابر ۲۰ بسته و آستانه حداکثر برابر ۵۰ بسته است. اگر طول صف از آستانه حداقل عبور کند، احتمال حذف بسته‌ها به تدریج افزایش می‌یابد و با رسیدن به آستانه حداکثر، این احتمال به ۱۰۰٪ می‌رسد.

\begin{enumerate}
	\item 
	اگر طول صف در لحظه‌ای به ۴۰ بسته برسد، با توجه به مقادیر حداقل و حداکثر، نرخ حذف بسته‌ها را محاسبه کنید.
	
	\begin{qsolve}
		الگوریتم \lr{RED} از یک سیاست احتمالی برای حذف بسته‌ها استفاده می‌کند. احتمال حذف بسته (\( P_{drop} \)) با توجه به طول صف (\( q \)) و مقادیر آستانه حداقل (\( Min_{th} \)) و حداکثر (\( Max_{th} \)) به صورت زیر محاسبه می‌شود:
		\[
		P_{drop} = \frac{q - Min_{th}}{Max_{th} - Min_{th}}
		\]
		
		در اینجا داریم:
		\begin{itemize}
			\item \( Min_{th} = 20 \)
			\item \( Max_{th} = 50 \)
			\item \( q = 40 \)
		\end{itemize}
		
		بنابراین می توان نوشت:
		\[
		P_{drop} = \frac{40 - 20}{50 - 20} = \frac{20}{30} = 0.6667
		\]
		
		بنابراین، احتمال حذف بسته در طول صف \( q = 40 \) برابر با \( 66.67\% \) است.
		
	\end{qsolve}
	
	
	\item 
	یک سناریو شبیه‌سازی کنید که در آن طول صف به طور پیوسته افزایش می‌یابد و تأثیر الگوریتم \lr{RED} بر ترافیک شبکه را تحلیل کنید. مشخص کنید که چگونه \lr{RED} می‌تواند به کاهش ازدحام و جلوگیری از پر شدن کامل صف کمک کند و چه تاثیری بر تأخیر و نرخ ازدحام در شبکه دارد.
	\begin{qsolve}
		فرض کنید طول صف به تدریج افزایش می‌یابد. رفتار \lr{RED} را در سه مرحله زیر بررسی می‌کنیم:
		
		
		\begin{enumerate}
			\item \textbf{{مرحله اول: طول صف کمتر از \( Min_{th} \)}}
			
			
			\begin{itemize}
				\item در این مرحله، هیچ بسته‌ای حذف نمی‌شود زیرا طول صف کمتر از \( Min_{th} \) است.
				\item \lr{RED} به بسته‌ها اجازه می‌دهد بدون هیچ مداخله‌ای وارد صف شوند.
			\end{itemize}
			\textbf{تأثیر بر شبکه:}
			\begin{itemize}
				\item تأخیر کم
				\item نرخ ارسال ترافیک بالا
				\item بدون افت بسته
			\end{itemize}
		\end{enumerate}
	\end{qsolve}
\end{enumerate}

\begin{enumerate}
	\item [ ]
	\begin{qsolve}
			
		\begin{enumerate}
			\item [(ب)] \textbf{{مرحله دوم: طول صف بین \( Min_{th} \) و \( Max_{th} \)}}
			\begin{itemize}
				\item در این مرحله، احتمال حذف بسته‌ها به صورت خطی با افزایش طول صف، افزایش می‌یابد.
				\item هدف اصلی در این مرحله جلوگیری از پر شدن کامل صف با حذف تدریجی بسته‌ها است.
			\end{itemize}
			\textbf{تأثیر بر شبکه:}
			\begin{itemize}
				\item احتمال حذف بسته‌ها افزایش می‌یابد، اما به صورت کنترل‌شده.
				\item تأخیر اندکی افزایش پیدا می‌کند.
				\item نرخ افت بسته‌ها متوسط است.
			\end{itemize}
			
			
			
			\item [(ج)] \textbf{{مرحله سوم: طول صف برابر یا بیشتر از \( Max_{th} \)}}
			\begin{itemize}
				\item در این مرحله، احتمال حذف بسته‌ها به \( 100\% \) می‌رسد.
				\item بسته‌های ورودی جدید به صورت کامل حذف می‌شوند تا فضای بیشتری برای بسته‌های موجود فراهم شود.
			\end{itemize}
			\textbf{تأثیر بر شبکه:}
			\begin{itemize}
				\item جلوگیری از پر شدن کامل صف و کاهش پدیده \lr{Global Synchronization}.
				\item تأخیر در این مرحله بسیار زیاد است.
				\item نرخ افت بسته‌ها بسیار بالا است.
			\end{itemize}
		\end{enumerate}
		
		
		
		\textbf{{نقش \lr{RED} در کاهش ازدحام و تأخیر}}
		\begin{itemize}
			\item \textbf{کاهش ازدحام:} \lr{RED} با حذف تدریجی بسته‌ها، از پر شدن کامل صف جلوگیری می‌کند.
			\item \textbf{کاهش تأخیر:} با مدیریت طول صف و جلوگیری از پر شدن کامل آن، تأخیر صف‌بندی (\lr{Queuing Delay}) کاهش می‌یابد.
			\item \textbf{بهبود توزیع منابع:} \lr{RED} به بسته‌ها اجازه می‌دهد به صورت عادلانه وارد صف شوند و جریان‌های مختلف شبکه فرصت‌های برابری برای استفاده از منابع داشته باشند.
			\item \textbf{جلوگیری از پدیده \lr{Global Synchronization}:} حذف تصادفی بسته‌ها مانع از کاهش هم‌زمان نرخ ارسال جریان‌های \lr{TCP} می‌شود.
		\end{itemize}
	\end{qsolve}
\end{enumerate}