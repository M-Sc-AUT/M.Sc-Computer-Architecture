\section{سوال اول}

فرض کنید یک شبکه دارای دو نوع ترافیک با اولویت بالا و پایین است. نرخ ورود بسته‌های اولویت بالا ۶ بسته در ثانیه و نرخ ورود بسته‌های اولویت پایین ۴ بسته در ثانیه است. حداکثر ظرفیت صف برابر با ۱۰ بسته بوده و از روش \lr{FIFO} برای مدیریت صف‌ها استفاده می‌شود.

\begin{enumerate}
	\item 
	اگر صف به حداکثر ظرفیت خود برسد تعداد بسته‌های هر دسته که در صف باقی می‌مانند را محاسبه کنید. همچنین فرض کنید ورود بسته‌ها به مدت ۵ ثانیه ادامه داشته باشد. نشان دهید که در این سناریو استفاده از \lr{FIFO} ممکن است باعث افزایش زمان	 انتظار برای بسته‌های اولویت بالا شود.
	
	\begin{qsolve}
		
	\end{qsolve}
	
	
	\item 
	روش \lr{HOL Priority Queueing} را به عنوان جایگزین پیشنهاد دهید و تحلیل کنید که چگونه استفاده از این روش می‌تواند زمان انتظار برای بسته‌های اولویت بالا را کاهش دهد و تأثیر آن بر بسته‌های اولویت پایین را ارزیابی کنید.
	
	\begin{qsolve}
		
	\end{qsolve}
	
	
\end{enumerate}