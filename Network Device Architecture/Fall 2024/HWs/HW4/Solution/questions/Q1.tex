\section{سوال اول}

فرض کنید یک شبکه دارای دو نوع ترافیک با اولویت بالا و پایین است. نرخ ورود بسته‌های اولویت بالا ۶ بسته در ثانیه و نرخ ورود بسته‌های اولویت پایین ۴ بسته در ثانیه است. حداکثر ظرفیت صف برابر با ۱۰ بسته بوده و از روش \lr{FIFO} برای مدیریت صف‌ها استفاده می‌شود.

\begin{enumerate}
	\item 
	اگر صف به حداکثر ظرفیت خود برسد تعداد بسته‌های هر دسته که در صف باقی می‌مانند را محاسبه کنید. همچنین فرض کنید ورود بسته‌ها به مدت ۵ ثانیه ادامه داشته باشد. نشان دهید که در این سناریو استفاده از \lr{FIFO} ممکن است باعث افزایش زمان	 انتظار برای بسته‌های اولویت بالا شود.
	
	\begin{qsolve}
		\begin{itemize}
			\item \textbf{ورود بسته‌ها به شبکه:}
			\begin{itemize}
				\item نرخ ورود بسته‌های \lr{High Priority}: ۶ بسته در ثانیه
				\item نرخ ورود بسته‌های \lr{Low Priority}: ۴ بسته در ثانیه
				\item کل نرخ ورود: \( 6 + 4 = 10 \) بسته در ثانیه
			\end{itemize}
			با توجه به نرخ ورود برابر با ظرفیت صف، صف به حداکثر ظرفیت خود می‌رسد.
			
			\item \textbf{وضعیت صف در پایان ۵ ثانیه:}
			\begin{itemize}
				\item تعداد بسته‌های \lr{High Priority}: \( 6 \times 5 = 30 \) بسته
				\item تعداد بسته‌های \lr{Low Priority}: \( 4 \times 5 = 20 \) بسته
				\item مجموع: \( 30 + 20 = 50 \) بسته
			\end{itemize}
			با توجه به ظرفیت صف که برابر ۱۰ بسته است، بسته‌های زیادی کنار گذاشته می‌شوند.
			
			\item \textbf{اثر روش \lr{FIFO}:}
			\begin{itemize}
				\item بسته‌های \lr{High Priority} که پس از پر شدن صف می‌رسند، حذف می‌شوند.
				\item بسته‌های \lr{Low Priority} که زودتر وارد شده‌اند، ممکن است پردازش شوند.
				\item این امر باعث افزایش زمان انتظار و احتمال حذف بسته‌های \lr{High Priority} می‌شود.
			\end{itemize}
		\end{itemize}
	\end{qsolve}
	
	
	\item 
	روش \lr{HOL Priority Queueing} را به عنوان جایگزین پیشنهاد دهید و تحلیل کنید که چگونه استفاده از این روش می‌تواند زمان انتظار برای بسته‌های اولویت بالا را کاهش دهد و تأثیر آن بر بسته‌های اولویت پایین را ارزیابی کنید.	
\end{enumerate}
\newpage


\begin{enumerate}
	\item [ ]
	\begin{qsolve}
		در روش \lr{HOL Priority Queueing}:
		
		\begin{itemize}
			\item بسته‌های \lr{High Priority} همیشه اولویت پردازش دارند.
			\item بسته‌های \lr{Low Priority} تنها زمانی پردازش می‌شوند که صف \lr{High Priority} خالی باشد.
		\end{itemize}
		
		\textbf{مزایای بسته‌های \lr{High Priority}:}
		\begin{itemize}
			\item کاهش زمان انتظار: بسته‌های \lr{High Priority} به محض ورود پردازش می‌شوند.
			\item کاهش احتمال حذف: این بسته‌ها کمتر احتمال دارد که به دلیل پر شدن صف حذف شوند.
		\end{itemize}
		
		\textbf{تأثیر بر بسته‌های \lr{Low Priority}:}
		\begin{itemize}
			\item افزایش زمان انتظار: بسته‌های \lr{Low Priority} تنها زمانی پردازش می‌شوند که هیچ بسته \lr{High Priority} در صف نباشد.
			\item احتمال حذف: به دلیل حجم بالای ترافیک \lr{High Priority}، احتمال حذف بسته‌های \lr{Low Priority} افزایش می‌یابد.
		\end{itemize}
		
		
		می‌توان جدولی به‌صورت زیر از مقایسه این دو روش ارائه نمود:
		
		
	\end{qsolve}
	
	\begin{table}[h!]
		\centering
		\resizebox{\textwidth}{!}{%
			\begin{tabular}{|l|l|l|}
				\hline
				\textbf{\textbf{ویژگی}} & \lr{\textbf{FIFO}} & \lr{\textbf{HOL Priority Queueing}} \\
				\hline
				\textbf{زمان انتظار بسته‌های \lr{High Priority}} & ممکن است افزایش یابد & به حداقل می‌رسد \\
				\hline
			\textbf{	زمان انتظار بسته‌های \lr{Low Priority}} & معمولاً کمتر است & افزایش می‌یابد \\
				\hline
				\textbf{احتمال حذف بسته‌ها} & هر دو نوع بسته ممکن است حذف شوند & بیشتر بسته‌های \lr{Low Priority} حذف می‌شوند \\
				\hline
				\textbf{پیچیدگی اجرایی} & ساده‌تر & پیچیده‌تر به دلیل مدیریت اولویت \\
				\hline
			\end{tabular}%
		}
	\end{table}
	
	\begin{qsolve}
		بنابر این:
		
		\begin{itemize}
			\item اگر هدف کاهش تأخیر برای بسته‌های \lr{High Priority} باشد، روش \lr{HOL Priority Queueing} پیشنهاد می‌شود.
			\item اگر هدف پردازش عادلانه باشد، روش \lr{FIFO} مناسب‌تر است.
		\end{itemize}
	\end{qsolve}
	
	
\end{enumerate}