\section{سوال اول}

در فرآیند طبقه‌بندی (\lr{classification}) بسته‌ها:  
\begin{enumerate}
	\item تشخیص جریان‌های ترافیکی چگونه انجام می‌شود؟
	\begin{qsolve}
		
		تشخیص جریان‌های ترافیکی (\lr{Traffic Flows Identification}) در فرآیند طبقه‌بندی بسته‌ها به شناسایی و گروه‌بندی بسته‌هایی که به یک جریان خاص تعلق دارند، اشاره دارد. این جریان‌ها معمولاً بر اساس اطلاعات موجود در سربرگ بسته‌ها (\lr{Packet Headers}) تعریف می‌شوند. به عنوان مثال، یک جریان می‌تواند شامل تمامی بسته‌هایی باشد که:
		\begin{itemize}
			\item دارای آدرس مبدأ (\lr{Source Address}) و مقصد (\lr{Destination Address}) یکسان هستند.
			\item از یک پروتکل مشخص مانند \lr{TCP} یا \lr{UDP} استفاده می‌کنند.
			\item دارای شماره پورت‌های مشخص مبدأ و مقصد هستند.
		\end{itemize}
		برای تشخیص این جریان‌ها، معمولاً از روش‌هایی مانند:
		\begin{itemize}
			\item تطبیق فیلدهای سربرگ (\lr{Header Field Matching})
			\item جداول حالت جریان (\lr{Flow State Tables})
		\end{itemize}
		استفاده می‌شود که امکان نگهداری اطلاعات مرتبط با هر جریان را فراهم می‌کند.
		
	\end{qsolve}
	
	
	\newpage
	\item انواع روش‌های طبقه‌بندی بسته‌ها را با ذکر ویژگی‌های کلی بیان کنید.
	\begin{qsolve}
		
		روش‌های طبقه‌بندی بسته‌ها را می‌توان به دو دسته کلی تقسیم کرد:
		\begin{enumerate}
			\item \textbf{روش‌های مبتنی بر \lr{Header}:}
			\begin{itemize}
				\item این روش‌ها بسته‌ها را بر اساس اطلاعات موجود در \lr{Header} مانند آدرس‌های \lr{IP}، شماره پورت‌ها و پروتکل‌ها طبقه‌بندی می‌کنند.
				\item سریع هستند و به منابع پردازشی کمتری نیاز دارند.
				\item محدود به اطلاعات قابل مشاهده در سربرگ بوده و برای تحلیل داده‌های رمزنگاری‌شده مناسب نیستند.
			\end{itemize}
			\item \textbf{روش‌های مبتنی بر \lr{Content}:}
			\begin{itemize}
				\item در این روش‌ها محتوای بسته‌ها (مانند داده‌های درون \lr{payload}) مورد تحلیل قرار می‌گیرد.
				\item این روش‌ها دقت بالاتری در شناسایی نوع داده یا کاربرد خاص دارند.
				\item برای بسته‌های رمزنگاری‌شده یا حجم بالای داده‌ها، نیاز به پردازش سنگین و منابع بیشتر دارند.
			\end{itemize}
			\item \textbf{روش‌های هندسی (\lr{Geometric Methods}):}
			\begin{itemize}
				\item این روش‌ها با استفاده از تکنیک‌های هندسی و تجزیه و تحلیل فضایی برای طبقه‌بندی بسته‌ها و جریان‌ها عمل می‌کنند.
				\item به عنوان مثال، از الگوریتم‌های مانند \lr{k-means clustering} و \lr{SVM} استفاده می‌شود که می‌توانند داده‌ها را به‌طور مؤثری در فضاهای چندبعدی دسته‌بندی کنند.
				\item این روش‌ها معمولاً برای داده‌های با ویژگی‌های پیچیده و حجم بالای اطلاعات مناسب هستند.
				\item یکی از مزایای اصلی این روش‌ها، توانایی آن‌ها در شناسایی مرزهای غیرخطی بین کلاس‌هاست.
				\item این روش‌ها به محاسبات پیچیده‌تری نیاز دارند و در مقیاس‌های بزرگ می‌توانند منابع زیادی مصرف کنند.
			\end{itemize}
		\end{enumerate}
	\end{qsolve}
	\newpage
	
	
	
	\item معیارهای کارآیی روش‌های طبقه‌بندی بسته‌ها به بیان کرده و به اختصار شرح دهید.
	\begin{qsolve}
		
		کارآیی روش‌های طبقه‌بندی بسته‌ها بر اساس معیارهای زیر سنجیده می‌شود:
		\begin{itemize}
			\item \textbf{دقت:} میزان بسته‌هایی که به درستی طبقه‌بندی شده‌اند. دقت بالا برای جلوگیری از تداخل جریان‌ها و کاهش خطا اهمیت دارد.
			\item \textbf{سرعت:} سرعت طبقه‌بندی بسته‌ها، به خصوص در شبکه‌های پرظرفیت، باید بالا باشد. روش‌های سریع مانند \lr{Header-Based} برای کاربردهای بلادرنگ مناسب‌تر هستند.
			\item \textbf{مصرف حافظه:} روشی که از حافظه کمتری استفاده کند، به خصوص در مقیاس‌های بزرگ، کارآمدتر است. بهینه‌سازی ساختار داده‌ها مانند استفاده از \lr{Trie} یا \lr{Bitmap} می‌تواند مفید باشد.
			\item \textbf{قابلیت مقیاس‌پذیری:} توانایی روش برای مدیریت حجم زیاد قوانین و جریان‌ها. روش‌هایی که از ساختارهای موازی و الگوریتم‌های موثر استفاده می‌کنند، مقیاس‌پذیرتر هستند.
			\item \textbf{انعطاف‌پذیری:} توانایی سازگاری روش با تغییرات در قوانین یا شرایط شبکه. روش‌های مبتنی بر یادگیری ماشین اغلب انعطاف‌پذیری بیشتری دارند.
			\item \textbf{پیچیدگی محاسباتی:} میزان منابع محاسباتی مورد نیاز، که در روش‌های پیچیده‌تر مانند تحلیل محتوا، بیشتر است.
		\end{itemize}
		
		این معیارها بسته به نوع کاربرد و نیاز شبکه ممکن است اولویت‌بندی متفاوتی داشته باشند.
		
	\end{qsolve}
\end{enumerate}





