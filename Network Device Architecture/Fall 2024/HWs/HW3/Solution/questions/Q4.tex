\section{سوال چهارم}

جدول \lr{Classifier} زیر را در نظر بگیرید که در آن هر قانون با مجموعه‌ای از \lr{Fields} همراه است.

\begin{latin}
	\begin{center}
		\begin{tabular}{|c|c|c|c|c|c|}
			\hline
			\textbf{Rule} & \textbf{B1} & \textbf{B2} & \textbf{B3} & \textbf{B4} & \textbf{Action} \\
			\hline\hline
			R1 & \texttt{110*} & \texttt{01*} & \texttt{1*} & \texttt{101*} & Act1 \\
			\hline\hline
			R2 & \texttt{10*} & \texttt{0*} & \texttt{11*} & \texttt{10*} & Act2 \\
			\hline
			R3 & \texttt{1*} & \texttt{11*} & \texttt{10*} & \texttt{1*} & Act3 \\
			\hline
			R4 & \texttt{11*} & \texttt{01*} & \texttt{1*} & \texttt{10*} & Act4 \\
			\hline
		\end{tabular}
	\end{center}
\end{latin}


\begin{enumerate}
	\item 
	روش \lr{‫‪Cross-Producting‬‬} را برای قوانین داده‌شده اجرا کنید و تمامی مقادیر منحصر به فرد برای هر بعد فیلد را شناسایی کنید.
	\begin{qsolve}
		\begin{enumerate}
			\item 
			اگر ایندکس های ما $ arr[2][1][2],[1,2][2] $،
			\lr{Rule=R1} می‌شود
		
		
			\item 
			اگر ایندکس های $ arr[1][0,1][2],[1,2] $ باشد، \lr{Rule=R2} می‌شود
			
			
			\item 
			اگر ایندکس های $ arr[1,2,3][3][1][1,2,3] $ باشد، \lr{Rule=R3} می‌شود
			
			
			\item 
			اگر ایندکس های $ arr[2,3][1][1,2][1,2] $ باشد، \lr{Rule=R4} می‌شود
			
			
		
		\end{enumerate}
		
		همچنین بازه‌های فیلد‌های ما به‌صورت زیر به‌دست می‌آید:
		
		\begin{latin}
			\begin{center}
				\begin{tabular}{|c|c|}
					\hline
					\multirow{4}{*}{F1} & \texttt{0000*} - \texttt{0111*} \\ \cline{2-2}
					& \texttt{1000*} - \texttt{1011*} \\ \cline{2-2}
					& \texttt{1100*} - \texttt{1101*} \\ \cline{2-2}
					& \texttt{1110*} - \texttt{1111*} \\ \hline
				\end{tabular}
			\end{center}
			
			
			\begin{center}
				\begin{tabular}{|c|c|}
					\hline
					\multirow{4}{*}{F2} & \texttt{0000*} - \texttt{0011*} \\ \cline{2-2}
					& \texttt{0100*} - \texttt{0111*} \\ \cline{2-2}
					& \texttt{1000*} - \texttt{1011*} \\ \cline{2-2}
					& \texttt{1100*} - \texttt{1111*} \\ \hline
				\end{tabular}
			\end{center}
			
			
			
			\begin{center}
				\begin{tabular}{|c|c|}
					\hline
					\multirow{3}{*}{F3} & \texttt{0000*} - \texttt{0111*} \\ \cline{2-2}
					& \texttt{1000*} - \texttt{1011*} \\ \cline{2-2}
					& \texttt{1100*} - \texttt{1111*} \\ \hline
				\end{tabular}
			\end{center}
			
			
			
			\begin{center}
				\begin{tabular}{|c|c|}
					\hline
					\multirow{4}{*}{F4} & \texttt{0000*} - \texttt{0111*} \\ \cline{2-2}
					& \texttt{1010*} - \texttt{1011*} \\ \cline{2-2}
					& \texttt{1000*} - \texttt{1001*} \\ \cline{2-2}
					& \texttt{1100*} - \texttt{1111*} \\ \hline
				\end{tabular}
			\end{center}
		\end{latin}
		
	\end{qsolve}
\end{enumerate}
\newpage


\begin{enumerate}
	\item [2. ]
	به اختصار چگونگی \lr{Cross-Producing} در تطبیق قانون برای طبقه‌بندی بسته در مقیاس بزرگ را شرح دهید، به‌ویژه با تمرکز بر کارایی حافظه و سرعت جست و جو.
	\begin{qsolve}
		روش \lr{Cross-Producing} در تطبیق قوانین برای طبقه‌بندی بسته به‌طور خاص در مقیاس بزرگ، با هدف بهبود کارایی حافظه و سرعت جست‌وجو طراحی شده است. این روش به جای بررسی تک‌تک قوانین به صورت جداگانه، از ترکیب مقادیر در فیلدهای مختلف برای تولید یک فضای اشتراکی از مقادیر استفاده می‌کند. در ادامه، به اختصار ویژگی‌ها و مزایای این روش بررسی می‌شود:
		
		\begin{enumerate}
			\item \textbf{کاهش فضای جست‌وجو:}\\
			با استفاده از ترکیب مقادیر از فیلدهای مختلف (به‌جای پردازش تک‌تک قوانین)، فضای جست‌وجوی قوانین به شدت کاهش می‌یابد. این موضوع منجر به کاهش تعداد مقایسه‌های لازم برای یافتن یک تطبیق می‌شود.
			
			
			\item \textbf{استفاده بهینه از حافظه:}\\
			روش \lr{Cross-Producing} به گونه‌ای طراحی شده است که از ساختارهای داده‌ای مانند \lr{bitmaps} یا جداول فشرده برای ذخیره مقادیر استفاده می‌کند. این ساختارها فضای حافظه مورد نیاز را بهینه می‌کنند، زیرا فقط ترکیبات مرتبط ذخیره می‌شوند و از ذخیره داده‌های غیرضروری جلوگیری می‌شود.
			
			
			
			\item \textbf{سرعت بالا در جست‌وجو:}\\
			با ترکیب مقادیر و تولید کلیدهای یکتا برای هر ترکیب، جست‌وجو در یک مرحله انجام می‌شود. این موضوع به دلیل کاهش مقایسه‌های مکرر در میان قوانین بهبود محسوسی در زمان جست‌وجو ایجاد می‌کند.
			
			
			
			\item \textbf{مقیاس‌پذیری بالا:}\\
			در شبکه‌های بزرگ با تعداد زیاد قوانین، این روش به دلیل کاهش پیچیدگی زمانی و استفاده بهینه از منابع، مقیاس‌پذیری بسیار خوبی دارد. با افزایش تعداد قوانین، رشد مصرف منابع به صورت خطی یا کمتر از آن است.
			
			
			
			\item \textbf{کاربرد در سخت‌افزار:}\\
			این روش برای پیاده‌سازی در سخت‌افزارهایی مانند \lr{TCAM} (تطبیق سریع محتوا) نیز مناسب است، زیرا ساختار ساده و قابل پیش‌بینی آن، امکان استفاده از منابع سخت‌افزاری برای پردازش موازی را فراهم می‌کند.
		\end{enumerate}
	\end{qsolve}
	
	\item [3. ]
	مزایا و معایب استفاده از \lr{Cross-Producing} را برای این جدول طبقه‌بندی شرح دهید.
	\begin{qsolve}
		\begin{enumerate}
			\item \textbf{مزایا:}
			\begin{itemize}
				\item \textbf{بهینه‌سازی فضای جست‌وجو:}
				روش \lr{Cross-Producing} به کمک ترکیب مقادیر فیلدها (مانند \lr{B1} تا \lr{B4}) می‌تواند فضای جست‌وجو را کاهش دهد. در این جدول، با ترکیب مقادیر ممکن از هر فیلد، مجموعه‌ای یکتا از کلیدهای ترکیبی ایجاد می‌شود که جست‌وجو را سریع‌تر می‌کند.
				
				\item \textbf{افزایش کارایی حافظه:}
				برای جدول‌های کوچک مانند جدول فوق، ذخیره ترکیبات در قالب \lr{bitmaps} باعث کاهش مصرف حافظه می‌شود، زیرا فقط ترکیبات ضروری ذخیره می‌گردند.
			\end{itemize}
			
		\end{enumerate}
	\end{qsolve}
	\newpage
	\begin{qsolve}
		\begin{enumerate}
			\item [ ]
			\begin{itemize}
				\item \textbf{سرعت جست‌وجوی بالا:}
				در روش \lr{Cross-Producing}، پس از تولید ترکیبات از فیلدها (مانند ترکیب \texttt{B1} با \texttt{B2} و ...)، جست‌وجو برای تطبیق قانون به صورت مستقیم روی کلیدهای تولید‌شده انجام می‌شود، که نسبت به روش‌های سنتی مقایسه خط به خط، سرعت بیشتری دارد.
				
				
				\item \textbf{مقیاس‌پذیری:}
				در صورتی که تعداد قوانین (مانند \lr{R1} تا \lr{R4}) یا تعداد فیلدها افزایش یابد، روش \lr{Cross-Producing} به دلیل ساختار منظم، می‌تواند عملکرد مطلوبی ارائه دهد.
				
				\item \textbf{کاهش سربار محاسباتی:}
				ترکیب‌های از پیش محاسبه‌شده، نیاز به مقایسه تک‌تک فیلدها را در زمان جست‌وجو از بین می‌برد.
			\end{itemize}
			
			
			\item [(ب)]
			\textbf{معایب:}
			\begin{itemize}
				\item \textbf{پیچیدگی پیش‌پردازش:}
				تولید ترکیبات تمام فیلدها (به‌ویژه در جدول‌هایی با فیلدها یا قوانین زیاد) می‌تواند زمان‌بر و پرهزینه باشد. مثلاً برای جدول فوق، باید تمام مقادیر ممکن برای \lr{B1}، \lr{B2}، \lr{B3}، و \lr{B4} تولید شوند که می‌تواند در جداول پیچیده‌تر مشکل‌ساز شود.
				
				\item \textbf{افزایش نیاز به حافظه در مقیاس بزرگ:}
				اگرچه روش \lr{Cross-Producing} حافظه را برای جداول کوچک بهینه می‌کند، در جداول بزرگ با فیلدهای بیشتر و ترکیبات پیچیده‌تر، تعداد کلیدهای تولیدشده ممکن است بسیار زیاد شود و به مصرف بیش از حد حافظه منجر شود.
				
				\item \textbf{عدم انعطاف‌پذیری در تغییر قوانین:}
				اگر قوانین موجود (مانند تغییر در \lr{R1} یا اضافه‌شدن قوانین جدید) تغییر کنند، کل ترکیبات باید مجدداً محاسبه شوند، که باعث کاهش کارایی در جداول دینامیک می‌شود.
				
				\item \textbf{احتمال برخورد ترکیبات (\lr{Collision}):}
				در موارد خاص که فیلدها الگوهای مشترک داشته باشند (مانند \texttt{*} در قوانین جدول فوق)، ترکیبات ممکن است به نتایج تکراری یا اشتباه منجر شوند که نیاز به الگوریتم‌های رفع برخورد دارد.
				
				\item \textbf{عدم تطابق با جداول بسیار کوچک:}
				برای جداول با تعداد قوانین و فیلدهای کم (مانند جدول بالا)، روش \lr{Cross-Producing} ممکن است از نظر پیچیدگی پردازش اولیه نسبت به جست‌وجوی خط به خط معمولی کارایی کمتری داشته باشد.
			\end{itemize}
		\end{enumerate}
	\end{qsolve}
\end{enumerate}