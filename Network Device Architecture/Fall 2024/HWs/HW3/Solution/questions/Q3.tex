\section{سوال سوم}

جدول \lr{Classifier} زیر را نظر بگیرید.

\begin{latin}
	\begin{center}
		\begin{tabular}{|c|c|c|c|c|c|}
			\hline
			\textbf{Rule} & \textbf{F1} & \textbf{F2} & \textbf{F3} & \textbf{F4} & \textbf{Action} \\
			\hline\hline
			R1 & \texttt{01*} & \texttt{10*} & \texttt{5} & \texttt{(7,12)} & Act0 \\
			\hline\hline
			R2 & \texttt{00*} & \texttt{11*} & \texttt{8} & \texttt{(6,9)} & Act1 \\
			\hline
			R3 & \texttt{10*} & \texttt{1*} & \texttt{9} & \texttt{(4,6)} & Act2 \\
			\hline
			R4 & \texttt{0*} & \texttt{01*} & \texttt{3} & \texttt{(10,14)} & Act1 \\
			\hline
			R5 & \texttt{11*} & \texttt{10*} & \texttt{7} & \texttt{(6,8)} & Act0 \\
			\hline
			R6 & \texttt{0*} & \texttt{11*} & \texttt{6} & \texttt{(11,13)} & Act3 \\
			\hline
			R7 & \texttt{*} & \texttt{00*} & \texttt{7} & \texttt{(8,12)} & Act1 \\
			\hline
		\end{tabular}
	\end{center}
\end{latin}




\begin{enumerate}
	\item 
	بر اساس فیلدهای \lr{F1} و \lr{F2} فضای دو بعدی هندسی را رسم کنید و هر قانون (\lr{R1} تا \lr{R7}) را به ناحیه‌های مربوطه بر اساس مقادیر \lr{F1} و \lr{F2} نگاشت کنید و هر ناحیه را در نمودار با برچسب مربوط به قانون مشخص نمایید.
	
	\item 
	با استفاده از الگوریتم \lr{Cross-Producing} ماتریس تصمیم‌گیری را ایجاد کنید.
	
	\item 
	توضیح دهید که ستون‌های \lr{F3} و \lr{F4} چگونه بر روی \lr{Actions} در هر ناحیه تأثیر می‌گذارند.
\end{enumerate}