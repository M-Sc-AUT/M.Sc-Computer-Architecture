\section{سوال پنجم - عملی}
شبکه‌های کانولوشنی با توجه به توانایی آن‌ها در استخراج و یادگیری خودکار ویژگی‌ها، مقاومت نسبت به تغییرات و کارایی آن‌ها در مقابل پیچیدگی‌های وظیفه‌ی بازشناسی چهره، یک عنصر اساسی در اکثر اسن سیستم‌ها هستند. در این تمرین قصد داریم که با استفاده از شبکه‌های عصبی کانولوشنی به تحلیل احساسات چهره\footnote{\lr{Facial expression recognition}} و طبقه‌بندی آن‌ها از روی تصویر بپردازیم. مجموعه داده‌ی این تمرین شامل ۱۲۰۰ تصویر نمونه‌گیری شده از هر کلاس مجموعه \lr{\href{https://paperswithcode.com/dataset/affectnet}{AffectNet}} می‌باشد. مجموعه داده \lr{AffectNet} شامل ۴۵۰ هزار تصویر چهره با ۸ حالت مختلف می‌باشد که شکل \ref{نمونه‌هایی از مجموعه داده AffectNet} نمونه‌هایی از آن را نشان می‌دهد.



\begin{center}
	\includegraphics*[width=0.6\linewidth]{pics/img2.png}
	\captionof{figure}{نمونه‌هایی از مجموعه داده \lr{AffectNet}}
	\label{نمونه‌هایی از مجموعه داده AffectNet}
\end{center}



\begin{enumerate}
	\item \textbf{پیش‌پردازش و داده افزایی: }
	مجموعه داد را از این \href{https://drive.google.com/file/d/1jTou5SjDGMgHkcJ38DrlwvNvZ8YCe2IR/view?usp=sharing}{لینک} دانلود کنید و از هر کلاس سه نمونه را نمایش دهید. برای افزایش سرعت آموزش، تمامی تصاویر را به بازه $[0,1]$ نرمال‌سازی کنید. همچنین داده‌ها را با پردازش مناسب افزونه کنید. توضیح دهید که به‌نظر شما استفاده از چه پردازش‌هایی در این حالت مناسب است و چرا در این مسئله نیاز به داده‌افزایی وجود دارد؟ از هر کلاس سه نمونه‌ی افزونه شده را نمایش دهید و همچنین تعداد کل نمونه‌ها پیش و پس از داده‌افزایی را در گزارش خود بیاورید.
	
	
	\item \textbf{یادگیری انتقالی }
	یک رویکرد رایج در هوش مصنوعی است که از یک مدل از قبل آموزش دیده برای یک وظیفه متفاوت اما مرتبط استفاده می‌کند و آن را با وظایف جدید تطبیق می‌دهد. با استفاده از شبکه پیش آموزش دیده \lr{VGG16} وظیفه بازشناسی حالت چهره را بر روی مجموعه داده ارائه شده انجام دهید. برای فرآیند آموزش، از داده‌های موجود در پوشه \lr{Train} استفاده کنید. نمودار خطا و دقت در فرآیند آموزش و نمودار \lr{ROC} و ماتریس درهم‌ریختگی را برای داده‌های موجود در پوشه \lr{Validation} گزارش کنید.
	
	به‌کارگیری شبکه‌های ازپیش آموزش دیده به‌طور خاص در زمانی که داده‌ی کمی وجود دارد مزایای زیادی دارد اما این شبکه‌ها با توجه به معماری از‌پیش تعریف شده و نسبتا سنگین آنها برای استفاده در ابزار‌های کاربردی مانند تلفن‌همراه مناسب نیستند. مدل‌های موجود در تلفن های همراه باید نیاز‌های ذخیره‌سازی را به حداقل برسانند و درعین حال افت عملکرد قابل توجهی نداشته باشند. برای دستیابی به این امر، در \href{https://arxiv.org/pdf/1807.08775.pdf}{این مقاله} سه معماری سبک از سه شبکه کانولوشنی مطرح یعنی \lr{VGG }، \lr{AlexNet} و \lr{MobileNet} مطرح شده است. نتایج به‌دست آمده نشان می‌دهد که این سه معماری عملکرد مشابهی نسبت به آخرین مدل‌های پیشرو در این زمینه دارند.
	
	
	\item معماری مطرح شده برای شبکه \lr{VGG} که جزئیات آن در شکل \ref{معماری شبکه VGG ارائه شده در مقاله} آمده است را پیاده‌سازی کنید. این مدل را بر روی مجموعه داده ارائه شده آموزش دهید و نمودار خطا و دقت آن را رسم کنید. همچنین با استفاده از داده موجود در پوشه \lr{Validation} مدل را تست کنید و نمودار \lr{ROC} و ماتریس درهم‌ریختگی آن را گزارش کنید. تعداد پارامتر‌های این مدل و عملکرد آن را با مدل قسما قبل مقایسه و تحلیل کنید.

	
	\begin{center}
		\includegraphics*[width=0.4\linewidth]{pics/img3.png}
		\captionof{figure}{معماری شبکه \lr{VGG} ارائه شده در مقاله}
		\label{معماری شبکه VGG ارائه شده در مقاله}
	\end{center}
	
	
	\item برای درک هرچه بهتر عملکرد شبکه‌های کانولوشنی ابزار‌های متنوعی وجود دارد. یکی از این ابزار‌ها نقشه‌ی فعال‌سازی کلاس\footnote{\lr{Class activation map}} یا به اختصار \lr{CAM} است
	که یک نمونه از آن در شکل \ref{label} آمده است. بررسی کنید که استفاده از این ابزار چه پیش‌بینی برای بهبود شبکه‌های کانولوشنی فراهم می‌آورد. برای دو نمونه با اشتباه دسته بندی شده و دو نمونه به درستی دسته بندی شده‌ی به ازای هر کلاس در مدل سوال ۳ نقشه‌ی قعالسازی کلاس را به‌دست آورید و با تحلیل نتایج به‌دست آمده، رویکردی برای بهبود شبکه پیشنهادی سوال ۳ ارائه دهید.

\end{enumerate}


