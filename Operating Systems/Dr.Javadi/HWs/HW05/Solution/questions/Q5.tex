\section{سوال پنجم}

به سوالات زیر پاسخ کامل دهید.

\begin{enumerate}
	\item زمان دسترسی موثر را برای رشته ارجاعات زیر برای هر یک از الگوریتم های بهینه و FIFO و LRU را بدست آورید. (تعداد فریم‌ها ۴، سربار خطای صفحه ۵میلی‌ثانیه و زمان دسترسی به RAM ۵۰۰ نانو‌ثانیه است)
	
	\begin{latin}
	0, 3, 1, 4, 0, 5, 2, 1, 4, 5, 4, 5, 0
	\end{latin}
	
	\item با فرض موجود بودن r فریم در حافظه اصلی و n صفحه r<n برای رشته مراجعات زیر، تعداد خطا‌های صفحه را برای الگوریتم LRU مشخص کنید.
	
	\begin{latin}
		1, 1, 2, 1, 2, 3, 1, 2, 3, 4, ..., 1, 2, 3, 4, 5, ..., n 
	\end{latin}
	
\end{enumerate}


\begin{qsolve}
	\begin{enumerate}
		\item  برای محاسبه زمان دسترسی موثر به حافظه برای هر الگوریتم، از فرمول زیر استفاده می‌شود:
		
		زمان دسترسی موثر = تعداد صفحات ارجاعی به حافظه × زمان دسترسی به رم + تعداد خطاهای صفحه × سربار خطای صفحه
		
		\begin{enumerate}
			\item الگوریتم :Optimal
			در الگوریتم Optimal، تعداد خطاهای صفحه مساوی با تعداد ارجاعات به صفحات منهای تعداد صفحات موجود در حافظه است. برای رشته داده شده:
			
			تعداد خطاهای صفحه = تعداد ارجاعات − تعداد صفحات مختلف
			
			$13 - 6 = 7$
			
			
			\item الگوریتم :FIFO
			در الگوریتم FIFO، هر زمان که یک صفحه جدید وارد حافظه می‌شود، صفحه‌ای که اولین بار وارد حافظه شده بود از حافظه حذف می‌شود.
			
			تعداد خطاهای صفحه = تعداد ارجاعات + تعداد صفحات موجود در حافظه
			
			$13 + 4 = 17$
			
			\item الگوریتم :LRU
			در الگوریتم LRU، صفحه‌ای که مدت زمان استفاده آن کمترین بوده، از حافظه حذف می‌شود.
			
			تعداد خطاهای صفحه = تعداد ارجاعات + تعداد صفحات موجود در حافظه × (سربار خطای صفحه − ۱)
			
			$13 + 4 \times (5 - 1) = 33$
		\end{enumerate}
		
		\item برای محاسبه تعداد خطا‌های صفحه در الگوریتم می‌توان از فرمول زیر استفاده کرد.
		
		$ n - r + 1 $
		
		در اینجا داریم: 
		$n - r + 1 = n - 1$
	\end{enumerate}

\end{qsolve}







