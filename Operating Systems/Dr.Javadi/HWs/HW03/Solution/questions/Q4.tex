\section{سوال چارم}

‫دو‬ ‫پردازه‬ ‫برای ‫حل‬ ‫مسئله‬‫ی‬ ‫ناحیه‬ ‫بحرانی‬ ‫از‬ ‫روش‬ ‫زیر‬ ‫استفاده‬ ‫کردند‬. متغیر های s1 و s2 بین دو پردازه مشترک هستند و یک مقدار Boolean دارند که ‫در‬ ‫ابتدای‬ ‫اجرای‬ ‫برنامه‬ ‫به‬ ‫صورت‬ ‫تصادفی‬ ‫مقدار‬ ‫دهی‬ ‫شده‬ ‫اند‬.
\begin{latin}
\begin{lstlisting}
while(s1 != s2);			while(s1 == s2);
  // critical section			  // critical section
s2 =! s1; 				s2 == s1;			
\end{lstlisting}
\end{latin}

\begin{enumerate}
	\item بررسی کنید و توضیح دهید که هرکدام از سه شرط Progress و exclution Manual و waiting Nounded برآورده می‌شود یا خیر؟
	
	\item راه‌حلی برای عدم نقض هرکدام از شرط های بالا ارائه دهید.
\end{enumerate}


\begin{qsolve}
	\begin{enumerate}
		\item \textbf{بررسی شرایط:}
		\begin{itemize}
			\item \textbf{شرط :Progress\\}
در هر دو حلقه بالا، شرط Progress برآورده می‌شود. در هر حلقه، پردازه‌ها قبل از ورود به بخش بحرانی مقادیر متغیرهای s1 و s2 را بررسی می‌کنند و در صورتی که مقادیر آنها با هم متفاوت باشند، وارد بخش بحرانی می‌شوند. این شرط تضمین می‌کند که حداقل یک پردازه در هر حلقه در بخش بحرانی حضور داشته باشد.
			
			\item \textbf{شرط manual :exclusion\\}
در حلقه اول و دوم، شرط manual Exclusion برآورده می‌شود. زیرا پردازه‌ها قبل از ورود به بخش بحرانی، مقادیر s1 و s2 را بررسی می‌کنند و در صورتی که مقادیر آنها با هم برابر نباشند، وارد بخش بحرانی نمی‌شوند. این شرط تضمین می‌کند که همزمان حداکثر یک پردازه در بخش بحرانی حضور داشته باشد.
			
			
			\item \textbf{شرط bounded :waiting}
در هیچ یک از حلقه‌ها، شرط bounded Waiting برآورده نمی‌شود. زیرا در هر حلقه، پردازه‌ها به صورت بی‌پایان منتظر می‌مانند تا مقادیر s1 و s2 با هم متفاوت یا برابر شوند. این باعث می‌شود که پردازه‌ها در صورتی که مقادیر s1 و s2 با هم برابر نباشند، به صورت بی‌نهایت در حالت انتظار قرار بگیرند.
		\end{itemize}
		
		
		\item \textbf{راه‌حل برای عدم نقض شرایط:}
		\begin{itemize}
			\item برای رعایت شرط bounded Waiting می‌توان از یک تاخیر استفاده کرد. به این صورت که پردازه‌ها پس از بررسی مقادیر s1 و s2، یک تاخیر کوتاه داشته باشند و سپس دوباره مقادیر را بررسی کنند. این کار باعث می‌شود که پردازه‌ها در صورتی که مقادیر s1 و s2 با هم متفاوت یا برابر باشند،در حلقه انتظار کوتاهی داشته باشند و منتظر تغییر مقادیر باشند. این روش می‌تواند به صورت زیر پیاده‌سازی شود:
			
\begin{latin}
\begin{lstlisting}
while(s1 != s2);
  // critical section
s2 =! s1; 
delay();
\end{lstlisting}
\end{latin}
			
		\end{itemize}
	\end{enumerate}
\end{qsolve}
