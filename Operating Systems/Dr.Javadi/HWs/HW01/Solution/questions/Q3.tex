\section{در هر یک از موارد زیر، پردازنده از چه حالتی به چه حالت دیگری تغییر وضعیت می‌دهد؟ (منظور از حالت، وضعیت های مختلف پردازنده شامل ,Waiting ,Ready ,Terminated ,New Running است)}


الف) در حین اجرای پردازنده، کاربر کلیدی را فشار داده و وقفه ای با اولویت بالا تر در سیستم اتفاق می‌افتد.

ب) پردازه در حین اجرای کد خود، به جایی می‌رسد که نیاز به دریافت داده‌ها از طریق شبکه دارد

پ) رویداد مربوط به یکی از دستگاه های ورودی/خروجی به پایان رسیده و داده‌های مورد نیاز پردازنده آماده می‌شود.

ت) برنامه ای به زبان \texttt{C} تابع \texttt{exit()} را فراخوانی می‌کند.

ث) زمان‌بندی سیستم‌عامل، پردازنده ای را از صف انتظار خارج کرده و به آن اجازه اجرا می‌دهد.


\begin{qsolve}
در هر یک از موارد زیر، پردازه از یک حالت به حالت دیگر تغییر وضعیت می‌دهد:
\begin{enumerate}
	\item \textbf{وقفه با اولویت بالا:}
	\begin{enumerate}
		\item وضعیت قبلی: Running
		\item وضعیت جدید: Interrupted/Waiting
		
		 وقفه با اولویت بالا می‌تواند پردازه‌ای که در حال اجرا است را متوقف کند و به حالت وقفه (Interrupted) یا انتظار (Waiting) برود.
	\end{enumerate}
	\item \textbf{درخواست دریافت داده‌ها از شبکه:}
	\begin{enumerate}
		\item وضعیت قبلی: Running
		\item وضعیت جدید: Waiting
		
		 پردازه در حال اجرا برای دریافت داده‌ها از شبکه نیاز به انتظار (Waiting) دارد تا داده‌های مورد نیاز آماده شوند.
	\end{enumerate}
	\item \textbf{پایان رسیدن رویداد دستگاه ورودی/خروجی:}
	\begin{enumerate}
		\item وضعیت قبلی: Waiting
		\item وضعیت جدید: Ready
		
		 رویداد مربوط به دستگاه ورودی/خروجی به پایان رسیده و داده‌های مورد نیاز پردازنده آماده می‌شود، در نتیجه پردازه از حالت انتظار (Waiting) به حالت آماده (Ready) تغییر وضعیت می‌دهد.
	\end{enumerate}
	\item \textbf{فراخوانی تابع \texttt{exit()}:}
	\begin{enumerate}
		\item وضعیت قبلی: Running
		\item وضعیت جدید: Terminated
		
		 هنگامی که برنامه C تابع exit() را فراخوانی می‌کند، پردازه به حالت پایانی (Terminated) می‌رود و اجرای آن به پایان می‌رسد.
	\end{enumerate}
	\item پردازنده در هنگام اجرای کد خود، به جایی می‌رسد که نیاز به دریافت داده‌ها از شبکه دارد.
	\begin{enumerate}
		\item وضعیت قبلی: Ready
		\item وضعیت جدید: Waiting
	\end{enumerate}
\end{enumerate}

\end{qsolve}