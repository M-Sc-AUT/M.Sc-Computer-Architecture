\section{به سوالات زیر پاسخ دهید}




الف)‌ هدف از DMA چیست؟
\begin{qsolve}
	هدف اصلی استفاده از DMA افزایش سرعت انتقال داده ها بین دستگاه‌های I/O و حافظه سیستم است. با استفاده از DMA می‌توان مستقیما و بدون نیاز به مداخله CPU به I/O و حافظه‌ها دسترسی پیدا کرد و دیتا را به‌صورت مستقیم انتقال داد.
\end{qsolve}







ب) ‫چگونه‬ ‫می‬ ‫توان‬ ‫سیستمی ‬‫طراحی‬ ‫کرد‬ ‫که‬ ‫اجازه‬ ‫ی‬ ‫انتخاب‬ ‫یک‬ ‫سیستم‬ ‫عامل‬ ‫از‬ ‫چند‬ ‫سیستم‬ ‫عامل‬ ‫را‬ ‫هنگام‬ بوت شدن به کاربر بدهد؟ برنامه‌ی Bootstrap ‫برای‬ ‫این‬ ‫منظور‬ ‫چه‬ ‫کاری‬ ‫باید‬ ‫انجام‬ ‫دهد؟‬
\begin{qsolve}
	برای این کار باید برنامه ای نوشته شود که در زمان Boot سیستم اجرا شود. این برنامه باید توانایی نمایش رابط کاربری مناسب (مانند منو انتخاب سیستم‌عامل) را داشته باشد و بتواند با ارتباط با I/O های سیستم ورودی کاربر که OS انتخابی آن است را دریافت کند.
	
	همچنین این برنامه باید قابلیت این را داشته باشد که پس از دریافت سیستم‌عامل انتخابی توسط کاربر، بتواند فایل ها و داده های مورد نیاز برای اجرای OS را بارگذاری کند و پس از اجرای OS باید برنامه Bootstrap کنترل سیستم را به سیستم‌عامل بدهد.
\end{qsolve}








پ) ‫توضیح‬ ‫دهید‬ ‫که‬ ‫تفاوت‬ ‫بین‬ ‫حالت‬ ‫کرنل‬ ‫و‬ ‫حالت‬ ‫کاربر‬ ‫چگونه‬ ‫به‬ ‫حفاظت‬ ‫و‬ ‫امنیت‬ سیستم ‬‫کمک‬ ‫میکند‪.‬‬ ‫کدام یک ‬‫از‬ ‫دستورات‬ ‫زیر‬ ‫باید‬ ‫در‬ ‫حالت‬ ‫کرنل‬ ‫اجرا‬ ‫شوند؟‬
\begin{latin}
	\begin{enumerate}[label=\Alph*)]
		\item ‫‪Set value of timer
		\item Read the clock
		\item Clear memory
		\item ‫‪Issue a trap instruction
		\item ‫‪Turn off interrupts
		\item ‫‪Modify ‫‪entries‬‬ in device-status table
		\item ‫‪Switch from user to kernel mode
		\item ‫‪Access I/O device
	\end{enumerate}
\end{latin}

\begin{qsolve}
حالت کرنل و حالت کاربر دو حالت اجرایی در OS هستند که تفاوت های مهمی در امنیت و حفاظت سیستم دارند. در حالت کرنل، برنامه‌ها و سرویس‌های سیستم با دسترسی کامل به منابع سخت‌افزاری و سیستم عامل اجرا می‌شوند و در حالت کاربر، برنامه‌ها تنها با دسترسی محدود به منابع سیستم عامل اجرا می‌شوند.

حالت کرنل به دلایل زیر به حافظت و امنیت سیستم کمک می‌کند:
\begin{enumerate}
	\item \textbf{محدودیت دسترسی:‌ }در حالت کاربر، برنامه‌ها دسترسی محدودتری به منابع سیستم عامل دارند و نمی‌توانند به منابع حساس مثل حافظه سیستم یا دستگاه‌های I/O مستقیماً دسترسی داشته باشند. در حالت کرنل، سرویس‌ها و برنامه‌های سیستم عامل با دسترسی کامل به منابع سیستم عامل اجرا می‌شوند، اما این دسترسی برای برنامه‌های کاربردی محدود می‌شود.
	
	\item \textbf{جدا بودن فضای آدرسی: }در حالت کرنل و کاربر، فضای آدرسی برای برنامه‌ها جداگانه تعیین می‌شود. در حالت کاربر، برنامه‌ها تنها به فضای آدرسی خودشان دسترسی دارند و نمی‌توانند به فضای آدرسی برنامه‌های دیگر یا سیستم عامل دسترسی داشته باشند. این از خطرات نفوذ و دسترسی غیرمجاز جلوگیری می‌کند.
	
	
	\item \textbf{محدودیت دسترسی به سخت‌افزار: }در حالت کاربر، برنامه‌ها نمی‌توانند به دستگاه‌های سخت‌افزاری مستقیماً دسترسی داشته باشند و باید از طریق واسط‌های سیستم عامل از آن‌ها استفاده کنند. در حالت کرنل، سرویس‌ها و برنامه‌های سیستم عامل می‌توانند به طور مستقیم با دستگاه‌های سخت‌افزاری ارتباط برقرار کنند
	
	دستوراتی که مستقیماً با منابع سیستم عامل یا سخت‌افزار ارتباط برقرار می‌کنند، مانند تنظیم تایمر، پاکسازی حافظه، صدا زدن دستور توقف (trap) و دسترسی به دستگاه‌های I/O، در حالت کرنل باید اجرا شوند. این دستورات نیاز به دسترسی به منابع حساس سیستم دارند که در حالت کاربر محدود می‌شود. دستوراتی که مستقیماً با منابع کاربردی برنامه‌ها ارتباط برقرار می‌کنند، مانند خواندن ساعت، تغییر I/O و مدیریت حافظه، در حالت کاربر باید اجرا شوند. این دستورات معمولاً نیاز به دسترسی مستقیم به منابع کاربران دارند و در حالت کرنل اجرا نمی‌شوند.
	
	بنابراین فقط مورد B است که باید در حالت کاربر اجرا شود و بقیه موارد همگی در حالت کرنل اجرا می‌شوند.
\end{enumerate}
\end{qsolve}







د) در یک محیط programming Multi و sharing Time ‫چند‬ ‫کا‬ربر به‬ ‫صورت‬ ‫همزمان‬ ‫سیستم‬ ‫را‬ ‫به‬ ‫اشتراک‬ می‌گذارند ‫و‬ این‬ ‫وضعیت‬ ‫می‬‫تواند‬ ‫منجر‬ به مشکلات امنیتی‬ ‫مختلف‬ ‫شود. ۲ مورد از این مشکلات را نام ببرید.
\begin{qsolve}
\begin{enumerate}
	\item \textbf{کلاهبرداری از داده‌ها: }وجود چند کاربر در یک سیستم به اشتراک گذاشته شده ممکن است باعث افزایش ریسک کلاهبرداری از داده‌ها شود. اگر یک کاربر از طریق آسیب‌پذیری‌های امنیتی در سیستم، به داده‌های دیگری که توسط کاربران دیگر در حافظه سیستم قرار دارد، دسترسی پیدا کند، می‌تواند اطلاعات حساس را بدون اجازه و به طور غیرمجاز به دست آورد. این مشکل می‌تواند منجر به فاش شدن اطلاعات شخصی، رمزهای عبور، داده‌های حساس کسب و کار و سایر اطلاعات محرمانه شود.
	
	\item \textbf{تداخل در حافظه و منابع سیستم: } وجود چند کاربر در یک سیستم به اشتراکزمان، ممکن است منجر به تداخل‌های حافظه و منابع سیستم شود. زمانی که چند کاربر به صورت همزمان در حال اجرای برنامه‌ها و پردازش‌های مختلف هستند، ممکن است منابع سیستم مانند حافظه، پردازنده و I/O به طور ناهمزمان و نامتعادل مورد استفاده قرار گیرند. این موضوع می‌تواند منجر به افزایش زمان پاسخ و کندی عملکرد برنامه‌ها شود. همچنین، در مواردی که هر کاربر به منابع سیستم با دسترسی محدود دسترسی دارد، تداخل‌ها می‌توانند باعث کاهش کارایی و عملکرد کاربران شود.
\end{enumerate}
\end{qsolve}