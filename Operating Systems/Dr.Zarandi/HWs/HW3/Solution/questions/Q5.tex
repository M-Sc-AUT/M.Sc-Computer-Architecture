\section{سوال پنجم}

به سوالات زیر در رابطه با مدل‌های سیستم‌های عامل پاسخ دهید.

\begin{enumerate}
	\item 
	سیستم‌های عامل اولیه از چه مدلی پیروی می‌کردند و دو مورد از معایب این مدل را توضیح دهید.
	\begin{qsolve}
		سیستم‌های عامل اولیه عمدتاً از مدل \lr{Monolithic} پیروی می‌کردند. در این مدل، تمام بخش‌های سیستم عامل به صورت یکپارچه درون یک هسته بزرگ قرار داشتند.
		
		\begin{itemize}
			\item 
			مشکل در نگهداری و دیباگ کردن: چون همه بخش‌های سیستم عامل در یک هسته واحد قرار داشتند، خطا در یکی از بخش‌ها می‌توانست کل سیستم را تحت تأثیر قرار دهد. این موضوع باعث می‌شد دیباگ کردن و نگهداری سیستم پیچیده و زمان‌بر باشد.
			
			\item 
			عدم انعطاف‌پذیری: در این مدل، اضافه کردن یا حذف یک سرویس یا قابلیت جدید به سیستم عامل بسیار دشوار بود، چرا که تمام بخش‌های سیستم به یکدیگر وابسته بودند. هر تغییری نیازمند بازبینی و بازطراحی بخش‌های دیگر بود.
		\end{itemize}
		
	\end{qsolve}
	
	
	\item 
	فواید کاهش سایز \lr{kernel} (در مدل \lr{Microkernel}) چیست؟ سه نمونه از عملیات‌هایی که در \lr{Kernel} نگه داشته می‌شوند را نام ببرید.
	
	\begin{qsolve}
		\begin{itemize}
			\item 
			افزایش پایداری و امنیت: با کوچک کردن هسته، بسیاری از سرویس‌ها از هسته خارج می‌شوند و به‌عنوان سرویس‌های کاربری پیاده‌سازی می‌شوند. این باعث می‌شود که اگر یکی از این سرویس‌ها دچار مشکل شود، بر عملکرد هسته تأثیر نگذارد و سیستم به کار خود ادامه دهد.
			
			\item 
			قابلیت گسترش و انعطاف‌پذیری: در مدل ریزهسته‌ای، سرویس‌های اضافی مانند مدیریت فایل‌ها یا درایورهای دستگاه‌ها می‌توانند به‌طور جداگانه بارگذاری یا حذف شوند. این امر امکان گسترش سیستم بدون نیاز به تغییر در هسته اصلی را فراهم می‌کند.
			
			\item 
			افزایش امنیت: به دلیل کوچک بودن هسته، تعداد عملیات‌هایی که در سطح هسته انجام می‌شوند کاهش می‌یابد، و همین باعث می‌شود که نقاط آسیب‌پذیر کمتری برای حملات وجود داشته باشد.
		\end{itemize}
	\end{qsolve}
	
	
	\item 
	تفاوت میان مدل لایه ای و مدل ماژولار چیست و چه عاملی باعث برتری مدل ماژولار می‌شود؟
	\begin{qsolve}
		\begin{itemize}
			\item 
			ساختار: در مدل لایه‌ای، سیستم عامل به لایه‌های مجزا تقسیم می‌شود که هر لایه فقط می‌تواند با لایه‌های مجاور خود ارتباط برقرار کند. در حالی که در مدل ماژولار، سیستم عامل به ماژول‌های جداگانه‌ای تقسیم می‌شود که هر ماژول به صورت مستقل از ماژول‌های دیگر قابل بارگذاری و حذف است و نیازی به وابستگی به لایه‌های مجاور ندارد.
			
			\item 
			ارتباط بین بخش‌ها: در مدل لایه‌ای، ارتباط بین لایه‌ها باید به ترتیب و از پایین به بالا یا بالعکس انجام شود. اما در مدل ماژولار، هر ماژول می‌تواند به طور مستقیم با هر بخش دیگری از سیستم ارتباط برقرار کند.
		\end{itemize}
		
	\end{qsolve}
	
	
	\begin{qsolve}
		\begin{itemize}
			\item 
			انعطاف‌پذیری مدل ماژولار بر مدل لایه‌ای برتری دارد. در مدل ماژولار، هر ماژول به صورت مستقل قابل توسعه، بارگذاری یا حذف است، بدون اینکه نیاز باشد کل سیستم عامل را تغییر داد یا از نو طراحی کرد. این امکان باعث می‌شود سیستم عامل در مواجهه با نیازهای جدید یا تغییرات سخت‌افزاری به‌سادگی گسترش یابد.
		\end{itemize}
	\end{qsolve}
	
\end{enumerate}