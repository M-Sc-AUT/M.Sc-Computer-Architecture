\section{سوال سوم}

از میان عملیات‌هایی که نیاز به \lr{System Call} دارند ۳ مثال نام ببرید و توضیح دهید که اگر هر عملیات در لایه \lr{User} انجام نشود چه مشکل‌هایی را می‌تواند به وجود بیاورد.


\begin{qsolve}
	\begin{enumerate}
		\item 
	\textbf{	عملیات خواندن و نوشتن فایل‌ها (\lr{File Read/Write})}\\
		 برای دسترسی به سیستم فایل و خواندن یا نوشتن داده‌ها در فایل‌ها، نیاز به \lr{System Call} داریم. دسترسی مستقیم کاربران به هارد دیسک یا سیستم فایل از طریق \lr{User mode} بدون استفاده از \lr{System Call} بسیار خطرناک است. زیر می‌تواند مشکلات زیر را ایجاد کند:
		 
		 \begin{itemize}
		 	\item 
		 	عدم کنترل دسترسی: اگر \lr{User} ها به صورت مستقیم و بدون استفاده از سیستم‌عامل به فایل‌ها دسترسی پیدا کنند، امنیت فایل‌ها به خطر می‌افتد و احتمال خرابی داده‌ها وجود دارد.
		 	
		 	\item 
		 	مدیریت ضعیف منابع: بدون واسطه‌ای مانند \lr{System Call} امکان مدیریت مناسب منابع و جلوگیری از استفاده نادرست یا بیش‌ازحد منابع وجود نخواهد داشت.
		 \end{itemize}
		
		
		
		\item 
	\textbf{	تخصیص و آزادسازی حافظه (\lr{Memory Allocation/Deallocation})}\\
		فرآیند تخصیص حافظه به برنامه‌ها توسط \lr{System Call‌}هایی مانند \lr{malloc} یا \lr{new} انجام می‌شود. سیستم‌عامل کنترل می‌کند که چه بخشی از حافظه به برنامه اختصاص داده شود و چه زمانی حافظه باید آزاد شود. مشکلات این دسته به‌صورت زیر عنوان می‌شود:
		
		\begin{itemize}
			\item 
			خرابی یا دسترسی غیرمجاز به حافظه: اگر کاربر مستقیماً به حافظه دسترسی پیدا کند، ممکن است به بخش‌هایی از حافظه دسترسی داشته باشد که به برنامه‌های دیگر اختصاص داده شده‌اند، که این می‌تواند منجر به خرابی سیستم یا دسترسی غیرمجاز به اطلاعات شود.
			
			\item 
			مدیریت نادرست حافظه: بدون \lr{System Call}، امکان نشت حافظه (\lr{memory leak}) و استفاده بی‌رویه از منابع حافظه وجود دارد، چرا که سیستم‌عامل نمی‌تواند حافظه‌ای که دیگر مورد استفاده نیست را آزاد کند.
		\end{itemize}
		
		
		\item 
		\textbf{اجرای فرآیند جدید (\lr{Process Creation})}\\
		عملیات ایجاد یک فرآیند جدید توسط سیستم‌عامل انجام می‌شود که از طریق \lr{System Call‌}هایی مانند \lr{fork} یا \lr{exec} در سیستم‌عامل‌های یونیکسی انجام می‌شود. این \lr{System Call}ها امکان اجرای برنامه‌های جدید در محیط سیستم‌عامل را فراهم می‌کنند. مشکلات این مثال به صورت زیر معرفی می‌شود:
		
		\begin{itemize}
			\item 
			تداخل در مدیریت فرآیندها: اگر کاربران بتوانند به‌طور مستقیم فرآیند جدید ایجاد کنند، سیستم‌عامل نمی‌تواند به درستی فرآیندها را زمان‌بندی کند و منابع را به طور عادلانه بین آن‌ها توزیع نماید.
			
			\item 
			نقض امنیت: ایجاد فرآیندهای جدید بدون کنترل سیستم‌عامل می‌تواند منجر به اجرای کدهای مخرب یا دسترسی‌های غیرمجاز به منابع سیستم شود، که امنیت کل سیستم را به خطر می‌اندازد.
		\end{itemize}
	\end{enumerate}
\end{qsolve}
