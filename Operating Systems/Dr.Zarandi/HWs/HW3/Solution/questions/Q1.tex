\section{سوال اول}

به سوالات زیر در رابطه با محیط‌های محاسباتی (\lr{Computing environment}) پاسخ دهید.

\begin{enumerate}
	\item 
	مدل‌های \lr{Client-server} و \lr{Peer to peer} را تعریف و با یکدیگر مقایسه کنید.
	
	\begin{qsolve}
		\begin{enumerate}
			\item 
			\textbf{معماری \lr{Client-server}: }\\
مطابق با توضیحات صفحه ۴۳ کتاب \lr{Silberschatz} می‌توان گفت معماری شبکه‌های امروزی معمولاً به این شکل است که سرورها به درخواست‌هایی که از طرف کلاینت‌ها (کاربران) می‌آیند پاسخ می‌دهند. به این مدل از سیستم‌های توزیع‌شده، سیستم کلاینت-سرور می‌گویند. به طور کلی، سرورها به دو دسته تقسیم می‌شوند: سرورهای محاسباتی و سرورهای فایل.

		
		\end{enumerate}
	\end{qsolve}
	
	\item 
	 \lr{Virtualization} 
	و \lr{Emulation} را تعریف کنید و تفاوت‌های آن‌ها را ذکر کنید.
	
	\begin{qsolve}
		
	\end{qsolve}
	
	\item 
	سه نمونه از دسته سرویس‌های ابری را نام ببرید و به صورت مختصر توضیح دهید.
	
	\begin{qsolve}
		
	\end{qsolve}
\end{enumerate}