\section{سوال دوم}

دو روش برای ارتباط میان فرآیندها \lr{Message passing} و \lr{Shared memory} است. آن‌ها را تعریف و با یکدیگر مقایسه کنید.

\begin{qsolve}
	در مدل \lr{Message passing} یک بخش از حافظه که بین فرآیندهای همکار به اشتراک گذاشته می‌شود، ایجاد می‌گردد. سپس فرآیندها می‌توانند با خواندن و نوشتن داده‌ها در این بخش مشترک، اطلاعات را مبادله کنند. در مدل \lr{Message passing}، ارتباط از طریق پیام‌هایی که بین فرآیندهای همکار رد و بدل می‌شود، انجام می‌گیرد. تفاوت این دو مدل ارتباطی در شکل زیر نشان داده شده است.
	
	\begin{center}
		\includegraphics*[width=0.7\linewidth]{pics/img4.png}
		\captionof{figure}{مدل‌های برقراری ارتباط میان فرآیند‌ها}
	\end{center}
	
	هر دو مدل ذکر شده در سیستم‌عامل‌ها رایج هستند و بسیاری از سیستم‌ها هر دو را پیاده‌سازی می‌کنند. \lr{Message passing} برای تبادل مقادیر کوچکتر داده‌ها مفید است، زیرا نیازی به جلوگیری از تداخلات نیست. \lr{Message passing} همچنین در سیستم‌های توزیع‌شده نسبت به حافظه مشترک آسان‌تر پیاده‌سازی می‌شود. \lr{Shared memory} می‌تواند سریع‌تر از پیام‌رسانی باشد، زیرا سیستم‌های \lr{Message passing} معمولاً از طریق فراخوانی‌های سیستمی پیاده‌سازی می‌شوند و بنابراین نیازمند مداخله هسته هستند که زمان‌بر است. در سیستم‌های \lr{Shared memory} فراخوانی سیستمی تنها برای ایجاد بخش‌های حافظه مشترک لازم است. پس از ایجاد حافظه مشترک، تمام دسترسی‌ها مانند دسترسی‌های معمولی به حافظه انجام می‌شود و نیازی به کمک هسته نیست.
\end{qsolve}