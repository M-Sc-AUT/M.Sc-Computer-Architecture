\section{سوال دوم}

تصور کنید در یک سیستم ۵ فرآیند وجود دارد، که زمان ورود \lr{Arrival Time} و زمان پردازش \lr{CPU Burst Time} آن به‌صورت زیر می‌باشد.


\begin{center}
	\begin{tabular}{||c|c|c||}
		\hline 
		مدت زمان پردازش & زمان ورود & فرآیند \\
		\hline \hline
		10 & 0 & $P_1$ \\
		5 & 1 & $P_2$ \\
		8 & 2 & $P_3$ \\
		6 & 3 & $P_4$ \\
		4 & 4 & $P_5$ \\
		\hline
	\end{tabular}
\end{center}



فرض کنید کوانتوم زمانی برابر با ۳ واحد زمانی است.

\begin{enumerate}
	\item 
	نمودار گانت مربوط به این فرآیند ها را رسم کنید
	
	\begin{qsolve}
		با فرض \lr{Preemptive} بودن، نمودار گانت به‌صورت زیر به‌دست می‌آید:
		
		
		\begin{center}
			\includegraphics*[width=1\linewidth]{pics/Q2.pdf}
			\captionof{figure}{گانت چارت}
		\end{center}
	\end{qsolve}
	
	
	\item 
	زمان تکمیل \lr{(‫‪Time‬‬‫‪Completion)‬‬} و زمان بازگشت \lr{(Turnaround Time)} و زمان انتظار \lr{(Waiting Time)} هر فرآیند را محاسبه کنید.
	
	\begin{qsolve}
		می‌دانیم:
		$$ TAT = CT - AT $$
		
		$$ WT = TAT - BT $$
		
		بنابراین محاسبه می‌شود:
		
		\begin{center}
			\begin{tabular}{||c|c|c|c||}
				\hline 
				پردازه & زمان تکمیل \lr{(CT)} & زمان بازگشت \lr{(TAT)} & زمان انتظار \lr{(WT)} \\
				\hline \hline
				$P_1$ & 33 & 33 & 23 \\
				$P_2$ & 20 & 19 & 14 \\
				$P_3$ & 32 & 30 & 22 \\
				$P_4$ & 26 & 23 & 17 \\
				$P_5$ & 30 & 26 & 22 \\
				\hline
			\end{tabular}
		\end{center}
	\end{qsolve}


	\item 
	میانگین زمان انتظار \lr{(Average Waiting Time)} و میانگین زمان بازگشت \lr{Turnaround Average Time} را محاسبه کنید.
\end{enumerate}


\begin{enumerate}
	\item [ ]
	\begin{qsolve}
		$$ \text{\lr{Average Waiting Time}} = \frac{(23+14+22+17+22)}{5} = \frac{98}{5}=19.6 
		$$
		
		$$ \text{\lr{Average Turnaround Time}} = \frac{(33+19+30+23+26)}{5} = \frac{131}{5}=26.2
		$$
	\end{qsolve}
\end{enumerate}

