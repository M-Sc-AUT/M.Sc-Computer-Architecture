\section{سوال دوم}

تصور کنید در یک سیستم ۵ فرآیند وجود دارد، که زمان ورود \lr{Arrival Time} و زمان پردازش \lr{CPU Burst Time} آن به‌صورت زیر می‌باشد.


\begin{center}
	\begin{tabular}{||c|c|c||}
		\hline 
		مدت زمان پردازش & زمان ورود & فرآیند \\
		\hline \hline
		10 & 0 & $P_1$ \\
		5 & 1 & $P_2$ \\
		8 & 2 & $P_3$ \\
		6 & 3 & $P_4$ \\
		4 & 4 & $P_5$ \\
		\hline
	\end{tabular}
\end{center}



فرض کنید کوانتوم زمانی برابر با ۳ واحد زمانی است.

\begin{enumerate}
	\item 
	نمودار گانت مربوط به این فرآیند ها را رسم کنید
	
	
	\item 
	زمان تکمیل \lr{(‫‪Time‬‬‫‪Completion)‬‬} و زمان بازگشت \lr{(Turnaround Time)} و زمان انتظار \lr{(Waiting Time)} هر فرآیند را محاسبه کنید.


	\item 
	میانگین زمان انتظار \lr{(Average Waiting Time)} و میانگین زمان بازگشت \lr{Turnaround Average Time} را محاسبه کنید.


\end{enumerate}

