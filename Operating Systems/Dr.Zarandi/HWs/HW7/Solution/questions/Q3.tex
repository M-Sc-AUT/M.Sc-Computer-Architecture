\section{سوال سوم}


فرض کنید یک سیستم دارای ۶ فرایند $ (P_0, P_1, P_2, P_3, P_4, P_5, P_6) $ و چهر نوع منبع  $(A, B, C, D)$ است که از هرکدام به ترتیب و درمجموع $ (14, 10, 9, 12) $ موجود است. جدول زیر وضعیت فعلی تخصیص منابع را نشان می‌دهد.



\begin{center}
	\begin{tabular}{||c|c|c|c|c||}
		\hline 
		Process & A & B & C & D\\
		\hline \hline
		$P_0$ & 1 & 1 & 0 & 2 \\
		$P_1$ & 2 & 1 & 1 & 0 \\
		$P_2$ & 0 & 1 & 2 & 3 \\
		$P_3$ & 2 & 0 & 1 & 2 \\
		$P_4$ & 1 & 2 & 1 & 1 \\
		$P_5$ & 1 & 2 & 0 & 0 \\
		\hline
	\end{tabular}
\end{center}


و جدول زیر بیشترین مقدار منابع مورد نیاز هر فرآیند را نشان می‌دهد:

\begin{center}
	\begin{tabular}{||c|c|c|c|c||}
		\hline 
		Process & A & B & C & D\\
		\hline \hline
		$P_0$ & 4 & 1 & 2 & 3 \\
		$P_1$ & 6 & 3 & 5 & 7 \\
		$P_2$ & 2 & 5 & 3 & 9 \\
		$P_3$ & 5 & 2 & 2 & 4 \\
		$P_4$ & 4 & 3 & 3 & 5 \\
		$P_5$ & 4 & 5 & 2 & 6 \\
		\hline
	\end{tabular}
\end{center}


\begin{enumerate}
	\item 
	آیا سیستم در حالت امن است؟
	
	\begin{qsolve}
		ابتدا ماتریس \lr{Need} را محاسبه می‌کنیم. می‌دانیم که:
		
		$$ Need = Max - Alloc $$
		
		پس ماتریس \lr{Need} به‌صورت زیر محاسبه می‌شود:
		
		\begin{center}
			\begin{tabular}{||c|c|c|c|c||}
				\hline 
				Process & A & B & C & D\\
				\hline \hline
				$P_0$ & 3 & 0 & 2 & 1 \\
				$P_1$ & 4 & 2 & 4 & 7 \\
				$P_2$ & 2 & 4 & 1 & 6 \\
				$P_3$ & 3 & 2 & 1 & 2 \\
				$P_4$ & 3 & 1 & 2 & 4 \\
				$P_5$ & 3 & 3 & 2 & 6 \\
				\hline
			\end{tabular}
		\end{center}
		
		
		با به‌دست آوردن ماتریس \lr{Need} و مقادیر \lr{Available} که در صورت سوال داده‌شده است (۱۲، ۹، ۱۰، ۱۴) می‌توان گفت که سیستم در‌حالت امن است و همه فرآیند ها می‌توانند بدون مشکل اجرا شوند.
		
	\end{qsolve}
	
	
	
	\item  
	اگر فرآیند $P_1 $ درخواست $ [1, 1, 2, 2] $ از منابع را ارسال کند، آیا این درخواست قابل قبول است؟
	
\end{enumerate}
\newpage


\begin{enumerate}
	\item [ ]
	\begin{qsolve}
		بله قابل قبول است. پس از این درخواست مقدار \lr{Available} به‌صورت $(13,9,7,10)$ می‌شود.
	\end{qsolve}
\end{enumerate}