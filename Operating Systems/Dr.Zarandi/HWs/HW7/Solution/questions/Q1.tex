\section{سوال اول}

عبارات و اصطلاحات زیر را تعریف کنید:


\begin{enumerate}
	\item 
	\lr{:CPU Burst Time}
	\begin{qsolve}
		\lr{CPU Burst Time}
		 یعنی مدت زمانی که یک فرآیند (یک برنامه یا کار در حال اجرا) به طور پیوسته از \lr{CPU} استفاده می‌کند. به عبارت دیگر، این زمان مشخص می‌کند که یک فرآیند چقدر زمان نیاز دارد تا کارهای خود را با پردازنده انجام دهد.
		 
		 فرض شود یک برنامه در حال اجرا است. این برنامه ممکن است در زمان‌های مختلف به \lr{CPU} نیاز داشته باشد تا محاسبات یا پردازش‌هایی انجام دهد. \lr{CPU Burst Time} همان مدت زمانی است که این فرآیند به طور مداوم در حال استفاده از \lr{CPU} است تا کار خود را انجام دهد. بعد از این مدت، ممکن است برنامه نیاز به انتظار برای \lr{I/O} (مثلاً خواندن داده از دیسک یا شبکه) داشته باشد، و در این زمان دیگر پردازنده در اختیار برنامه نخواهد بود.
	\end{qsolve}
	
	
	
	
	
	\item 
	\lr{‫‪:Turnaround‬‬ Time}
	\begin{qsolve}
		\lr{‫‪:Turnaround‬‬ Time}
		 مجموع زمانی است که از زمان شروع یک فرآیند تا زمان اتمام آن طول می‌کشد. این زمان شامل:
		 
		\begin{enumerate}
			\item 
			زمان اجرای فرآیند \lr{(CPU Burst Time)}
			
			\item 
			زمان انتظار برای منابع دیگر (مانند \lr{I/O} یا دسترسی به پردازنده)
			
			\item 
			زمان ارسال و دریافت ورودی/خروجی
		\end{enumerate}
		
		به طور کلی داریم:
		$$ \text{Turnaround Time} = T_{Begin} - T_{End} $$
	
	
	\end{qsolve}
\end{enumerate}
\newpage


\begin{enumerate}
	\item [3. ]
	بن‌بست:
	\begin{qsolve}
		بن‌بست یا \lr{Deadlock} یک وضعیت در سیستم‌های عامل است که در آن دو یا چند فرآیند یا \lr{thread} به طوری به یکدیگر وابسته می‌شوند که هیچ کدام از آن‌ها قادر به ادامه اجرای خود نیستند. این وضعیت زمانی اتفاق می‌افتد که:
		\begin{enumerate}
			\item 
			هر فرآیند یک یا چند منبع را در اختیار دارد.
			
			\item 
			هر فرآیند منتظر منبع دیگری است که توسط فرآیند دیگر نگهداری می‌شود.
		\end{enumerate}
		
		در نتیجه، هیچ یک از فرآیندها نمی‌توانند ادامه یابند، زیرا هر کدام به منابعی نیاز دارند که در حال حاضر توسط دیگران قفل شده است.
	\end{qsolve}
	
	
	
	
	
	\item [4. ]
	حالت امن:
	\begin{qsolve}
		حالت امن \lr{(Safe State)} به حالتی اطلاق می‌شود که سیستم در آن قادر است به گونه‌ای منابع را تخصیص دهد که هیچ‌گاه به \lr{Deadlock} منتهی نشود. به عبارت دیگر، در حالت امن، سیستم می‌تواند به راحتی منابع را بین فرآیندها تخصیص دهد بدون اینکه در هر مرحله‌ای وارد بن‌بست شود.
		
		برای بررسی اینکه آیا سیستم در حالت امن است یا خیر، از الگوریتم‌های مانند الگوریتم \lr{Banker’s Algorithm} استفاده می‌شود، که بررسی می‌کند آیا می‌توان به نحوی منابع را تخصیص داد که همواره به فرآیندها اجازه داده شود تا به طور کامل به اتمام برسند.
	\end{qsolve}
\end{enumerate}



%\begin{center}
%	\includegraphics*[width=0.3\linewidth]{pics/img1.png}
%	\captionof{figure}{ساختار عمومی مسئله ناحیه بحرانی}
%\end{center}