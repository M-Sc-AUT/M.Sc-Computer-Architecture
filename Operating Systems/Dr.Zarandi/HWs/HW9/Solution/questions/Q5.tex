\section{سوال پنجم}

در چه حالاتی (ترتیبی از درخواست‌ها) استفاده از الگوریتم \lr{C-SCAN} بهتر از \lr{SCAN} می‌باشد؟ با ذکر مثال دلیل آورید. توجه کنید منظور از بهتر بودن لزوماً کمتر بودن \lr{Head movement} نمی‌باشد.


\begin{qsolve}
	
	الگوریتم \lr{C-SCAN} بهتر از \lr{SCAN} در شرایطی است که حرکت سر دیسک (\lr{Head movement}) به دلیل بازگشت به ابتدای دیسک در الگوریتم \lr{SCAN} بیشتر می‌شود. در الگوریتم \lr{C-SCAN} پس از رسیدن به انتهای دیسک، سر به طور مستقیم به ابتدای دیسک بازمی‌گردد و سپس به سمت انتهای دیگر حرکت می‌کند، در حالی که در الگوریتم \lr{SCAN} سر پس از رسیدن به انتهای دیسک جهت خود را تغییر داده و دوباره به سمت دیگر حرکت می‌کند.
	
	به عبارت دیگر، استفاده از \lr{C-SCAN} زمانی مناسب است که تعداد درخواست‌ها به سمت یکی از انتهای دیسک متمرکز باشد و نیاز به بازگشت به ابتدای دیسک در \lr{SCAN} موجب افزایش \lr{Head movement} شود.
	
	\textbf{مثال:}
	
	فرض کنید درخواست‌ها به صورت زیر باشند:
	
	$$ 7, 140, 23, 98, 17, 102, 48, 52, 57, 12 $$
	
	مقدار اولیه سر (\lr{head}) روی \lr{50} است و بازه دیسک از \lr{0} تا \lr{199} است.
	
	در الگوریتم \lr{SCAN}:
	\begin{enumerate}
		\item 
		ابتدا سر در موقعیت \lr{50} است و به سمت \lr{0} حرکت می‌کند.
		
		\item 
		درخواست‌ها به ترتیب بررسی می‌شوند: \lr{48, 23, 17, 12, 7}.
		
		\item 
		پس از رسیدن به \lr{0}، سر جهت خود را تغییر داده و به سمت \lr{199} حرکت می‌کند.
		
		\item 
		درخواست‌های باقی‌مانده بررسی می‌شوند: \lr{52, 57, 98, 102, 140}.
	\end{enumerate}
	
	در الگوریتم \lr{C-SCAN}:
	\begin{enumerate}
		\item 
		ابتدا سر در موقعیت \lr{50} است و به سمت \lr{0} حرکت می‌کند.
		
		\item 
		درخواست‌ها به ترتیب بررسی می‌شوند: \lr{48, 23, 17, 12, 7}.
		
		\item 
		پس از رسیدن به \lr{0}، سر به طور مستقیم به \lr{199} بازمی‌گردد.
		
		\item 
		درخواست‌های باقی‌مانده بررسی می‌شوند: \lr{52, 57, 98, 102, 140}.
	\end{enumerate}
	
	در این مثال، استفاده از \lr{C-SCAN} باعث می‌شود که سر دیسک پس از رسیدن به \lr{0} به طور مستقیم به \lr{199} بازگردد و نیازی به حرکت مجدد به سمت \lr{0} نخواهد بود، در حالی که در \lr{SCAN} سر پس از رسیدن به \lr{0} باید دوباره به سمت \lr{199} حرکت کند.
	
	بنابراین، \lr{C-SCAN} در این حالت می‌تواند سریع‌تر باشد زیرا نیازی به بازگشت به ابتدای دیسک نیست.
	
\end{qsolve}