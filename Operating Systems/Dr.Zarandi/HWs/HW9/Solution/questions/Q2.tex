\section{سوال دوم}

فرض کنید از صفحه‌آوری مبتنی بر درخواست (\lr{demand paging}) استفاده می‌کنیم. جدول صفحات در حافظه اصلی نگهداری می‌شود که زمان دسترسی به آن ۱۱۰ نانوثانیه است. بنابراین ویژگی‌های حافظه ثانویه در این سیستم سرویس‌دهی به نقص صفحه در ۶۵ درصد مواقع ۴ میلی‌ثانیه و باقی مواقع ۲۱۰ میلی‌ثانیه طول می‌کشد. با این مفروضات بیشترین نرخ نقص صفحه چقدر می‌تواند باشد تا زمان مؤثر دسترسی بیشتر از ۲۰۰ نانوثانیه نشود؟


\begin{qsolve}
	\[
	\text{\lr{EAT}} = \text{\lr{(1 - p)}} \times \text{\lr{memory access time}} + \text{\lr{p}} \times \text{\lr{page fault service time}}
	\]
	
	که در آن:
	\begin{itemize}
		\item \(\text{\lr{p}}\): نرخ نقص صفحه (\lr{page fault rate})
		\item \(\text{\lr{memory access time}} = 110\) نانوثانیه
		\item \(\text{\lr{page fault service time}}\): زمان متوسط سرویس‌دهی به نقص صفحه
	\end{itemize}
	
	\textbf{محاسبه زمان متوسط سرویس‌دهی به نقص صفحه}
	
	زمان سرویس‌دهی به نقص صفحه به دو حالت بستگی دارد:
	\begin{itemize}
		\item در \(65\) درصد مواقع: \(4\) میلی‌ثانیه (\(4000000\) نانوثانیه)
		\item در \(35\) درصد مواقع: \(210\) میلی‌ثانیه (\(210000000\) نانوثانیه)
	\end{itemize}
	
	میانگین زمان سرویس‌دهی (\lr{average page fault service time}) برابر است با:
	\[
	\text{\lr{average service time}} = 0.65 \times 4000000 + 0.35 \times 210000000
	\]
	
	\[
	\text{\lr{average service time}} = 2600000 + 73500000 = 76100000 \text{\lr{ nanoseconds}}
	\]
	
	
	\textbf{جایگذاری در فرمول \lr{EAT}}
	
	فرمول \lr{EAT} را داریم:
	\[
	200 = (1 - p) \times 110 + p \times 76100000
	\]
	
	حل برای \(p\):
	\[
	200 = 110 - 110p + 76100000p
	\]
	
	\[
	200 - 110 = 76100000p - 110p
	\]
	
	\[
	90 = 76099990p
	\]
	
\end{qsolve}
\newpage

\begin{qsolve}
	\[
	p = \frac{90}{76099990}
	\]
	
	\[
	p \approx 0.00000118
	\]
	
	\textbf{بنابراین:}
	
	بیشترین نرخ نقص صفحه (\(\text{\lr{p}}\)) که زمان مؤثر دسترسی (\(\text{\lr{EAT}}\)) از \(200\) نانوثانیه بیشتر نشود، برابر است با:
	\[
	p \approx 0.000118 \text{\lr{ یا }} 0.0118\%.
	\]
\end{qsolve}