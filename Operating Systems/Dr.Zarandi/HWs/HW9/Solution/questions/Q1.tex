\section{سوال اول}



با فرض وجود سه قاب (\lr{frame}) از الگوریتم‌های \lr{FIFO}، \lr{LRU} و بهینه (\lr{optimal}) برای رشته‌های رجوع به صفحه (\lr{page}) زیر با ذکر مراحل استفاده کرده (از چپ به راست) و در نهایت تعداد نقص صفحه (\lr{page fault}) را به ازای هر الگوریتم به دست آورید.

\begin{latin}
	\begin{itemize}
		\item \lr{3, 7, 3, 7, 6, 5, 6, 3, 3, 8, 7, 7, 9, 5, 6, 0, 2, 4, 3, 5}
		\item \lr{7, 6, 7, 5, 2, 3, 5, 7, 6, 6, 4, 3, 3, 2, 0, 8, 2, 7, 8, 7}
		\item \lr{5, 4, 6, 8, 3, 5, 2, 7, 1, 7, 8, 1, 7, 1, 2, 3, 6, 2, 8, 5}
	\end{itemize}
	
\end{latin}


\begin{qsolve}
	
	\textbf{رشته اول:} \lr{3, 7, 3, 7, 6, 5, 6, 3, 3, 8, 7, 7, 9, 5, 6, 0, 2, 4, 3, 5}
	\begin{itemize}
		\item \lr{FIFO}: تعداد نقص صفحه: \lr{10}
		\item \lr{LRU}: تعداد نقص صفحه: \lr{9}
		\item \lr{Optimal}: تعداد نقص صفحه: \lr{8}
	\end{itemize}
	
	\textbf{رشته دوم:} \lr{7, 6, 7, 5, 2, 3, 5, 7, 6, 6, 4, 3, 3, 2, 0, 8, 2, 7, 8, 7}
	\begin{itemize}
		\item \lr{FIFO}: تعداد نقص صفحه: \lr{11}
		\item \lr{LRU}: تعداد نقص صفحه: \lr{9}
		\item \lr{Optimal}: تعداد نقص صفحه: \lr{7}
	\end{itemize}
	
	\textbf{رشته سوم:} \lr{5, 4, 6, 8, 3, 5, 2, 7, 1, 7, 8, 1, 7, 1, 2, 3, 6, 2, 8, 5}
	\begin{itemize}
		\item \lr{FIFO}: تعداد نقص صفحه: \lr{12}
		\item \lr{LRU}: تعداد نقص صفحه: \lr{10}
		\item \lr{Optimal}: تعداد نقص صفحه: \lr{8}
	\end{itemize}
	
	 باتوجه به محاسبات انجام شده، الگوریتم \lr{Optimal} کمترین تعداد نقص صفحه را دارد، زیرا آینده را پیش‌بینی می‌کند و صفحه‌ای را که دیرتر مورد نیاز است جایگزین می‌کند. الگوریتم \lr{LRU} نیز عملکرد بهتری نسبت به \lr{FIFO} دارد زیرا صفحات کمتر استفاده شده اخیر را جایگزین می‌کند.
	
\end{qsolve}




%\begin{center}
%	\includegraphics*[width=0.3\linewidth]{pics/img1.png}
%	\captionof{figure}{ساختار عمومی مسئله ناحیه بحرانی}
%\end{center}