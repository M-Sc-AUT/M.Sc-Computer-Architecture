\section{سوال چهارم}


با توجه به لیست درخواست شده (از چپ به راست) ترتیب دسترسی به فضاهای خواسته شده را با استفاده از الگوریتم‌های \lr{SCAN}، \lr{C-SCAN}، \lr{LOOK}، \lr{C-LOOK} و \lr{SSTF (Shortest Seek Time First)} را بنویسید و همچنین مقادیر \lr{Head movement} را به ازای هر الگوریتم نیز به دست آورید.

- مقدار اولیه سر (\lr{head}) بر روی \lr{50} است و بازه دیسک از \lr{0} تا \lr{199} است.


\begin{latin}
	\begin{itemize}
		\item \lr{57, 140, 23, 98, 7, 102, 48, 52, 17, 12}
	\end{itemize}
\end{latin}


\begin{qsolve}
	\textbf{1. \lr{SCAN}:}
	
	الگوریتم \lr{SCAN} به این صورت عمل می‌کند که سر ابتدا در یک جهت حرکت می‌کند تا به انتهای دیسک برسد، سپس جهت خود را تغییر داده و به سمت دیگر حرکت می‌کند.
	
	- ابتدا سر در موقعیت \lr{50} است و به سمت \lr{0} حرکت می‌کند.
	- در مسیر به سمت \lr{0}، درخواست‌ها را به ترتیب بررسی می‌کنیم: \lr{48, 23, 17, 12, 7}.
	- پس از رسیدن به \lr{0}، جهت سر تغییر کرده و به سمت \lr{199} حرکت می‌کند.
	- در مسیر به سمت \lr{199}، درخواست‌های باقی‌مانده را به ترتیب بررسی می‌کنیم: \lr{52, 57, 98, 102, 140}.
	
	ترتیب دسترسی: \lr{48, 23, 17, 12, 7, 0, 52, 57, 98, 102, 140}
	
	
	
	
	مجموع \lr{Head movement}:
	\[
	|50 - 48| + |48 - 23| + |23 - 17| + |17 - 12| + |12 - 7| = 2 + 25 + 6 + 5 + 5 = 43
	\]
	\[
	|7 - 0| + |0 - 52| + |52 - 57| + |57 - 98| + |98 - 102| + |102 - 140| = 7 + 52 + 5 + 41 + 4 + 38 = 147
	\]
	\[
	\text{مجموع} = 43 + 147 = 190
	\]
	
	\textbf{2. \lr{C-SCAN}:}
	
	الگوریتم \lr{C-SCAN} مشابه \lr{SCAN} است، اما پس از رسیدن به انتهای دیسک، سر به طور مستقیم به ابتدای دیسک بازمی‌گردد.
	
	- ابتدا سر در موقعیت \lr{50} است و به سمت \lr{0} حرکت می‌کند.
	- در مسیر به سمت \lr{0}، درخواست‌ها را به ترتیب بررسی می‌کنیم: \lr{48, 23, 17, 12, 7}.
	- پس از رسیدن به \lr{0}، سر به طور مستقیم به \lr{199} بازمی‌گردد و سپس درخواست‌های باقی‌مانده را به ترتیب بررسی می‌کنیم: \lr{52, 57, 98, 102, 140}.
	
	ترتیب دسترسی: \lr{48, 23, 17, 12, 7, 0, 199, 52, 57, 98, 102, 140}
	


مجموع \lr{Head movement}:
	\[
	|50 - 48| + |48 - 23| + |23 - 17| + |17 - 12| + |12 - 7| = 2 + 25 + 6 + 5 + 5 = 43
	\]
	\[
	|7 - 0| + |0 - 199| + |199 - 52| + |52 - 57| + |57 - 98| + |98 - 102| + |102 - 140| 
	\]
	\[
	= 7 + 199 + 147 + 5 + 41 + 4 + 38 = 441
	\]
	\[
	\text{مجموع} = 43 + 441 = 484
	\]
\end{qsolve}
\newpage

\begin{qsolve}
	\textbf{3. \lr{LOOK}:}
	
	الگوریتم \lr{LOOK} مشابه \lr{SCAN} است، با این تفاوت که سر پس از رسیدن به آخرین درخواست، جهت خود را تغییر می‌دهد و نیازی به حرکت به انتهای دیسک ندارد.
	
	- ابتدا سر در موقعیت \lr{50} است و به سمت \lr{0} حرکت می‌کند.
	- در مسیر به سمت \lr{0}، درخواست‌ها را به ترتیب بررسی می‌کنیم: \lr{48, 23, 17, 12, 7}.
	- پس از رسیدن به \lr{7}، جهت سر تغییر کرده و به سمت \lr{140} حرکت می‌کند.
	- در مسیر به سمت \lr{140}، درخواست‌های باقی‌مانده را به ترتیب بررسی می‌کنیم: \lr{52, 57, 98, 102, 140}.
	
	ترتیب دسترسی: \lr{48, 23, 17, 12, 7, 140, 102, 98, 57, 52}
	
	مجموع \lr{Head movement}:
	\[
	|50 - 48| + |48 - 23| + |23 - 17| + |17 - 12| + |12 - 7| = 2 + 25 + 6 + 5 + 5 = 43
	\]
	\[
	|7 - 140| + |140 - 102| + |102 - 98| + |98 - 57| + |57 - 52| = 133 + 38 + 4 + 41 + 5 = 221
	\]
	\[
	\text{مجموع} = 43 + 221 = 264
	\]
	
	
	
	
	\textbf{4. \lr{C-LOOK}:}
	
	الگوریتم \lr{C-LOOK} مشابه \lr{LOOK} است، اما پس از رسیدن به آخرین درخواست، سر به طور مستقیم به کوچکترین درخواست بازمی‌گردد.
	
	- ابتدا سر در موقعیت \lr{50} است و به سمت \lr{0} حرکت می‌کند.
	- در مسیر به سمت \lr{0}، درخواست‌ها را به ترتیب بررسی می‌کنیم: \lr{48, 23, 17, 12, 7}.
	- پس از رسیدن به \lr{7}، سر به طور مستقیم به \lr{140} بازمی‌گردد و سپس درخواست‌های باقی‌مانده را به ترتیب بررسی می‌کنیم: \lr{52, 57, 98, 102, 140}.
	
	ترتیب دسترسی: \lr{48, 23, 17, 12, 7, 140, 102, 98, 57, 52}
	
	مجموع \lr{Head movement}:
	\[
	|50 - 48| + |48 - 23| + |23 - 17| + |17 - 12| + |12 - 7| = 2 + 25 + 6 + 5 + 5 = 43
	\]
	\[
	|7 - 140| + |140 - 102| + |102 - 98| + |98 - 57| + |57 - 52| = 133 + 38 + 4 + 41 + 5 = 221
	\]
	\[
	\text{مجموع} = 43 + 221 = 264
	\]
	
	
	
	\textbf{5. \lr{SSTF (Shortest Seek Time First)}:}
	
	الگوریتم \lr{SSTF} به این صورت عمل می‌کند که سر همیشه به نزدیک‌ترین درخواست حرکت می‌کند.
	
	- ابتدا سر در موقعیت \lr{50} است.
	- نزدیک‌ترین درخواست به \lr{50}، \lr{48} است.
	- سپس نزدیک‌ترین درخواست به \lr{48}، \lr{52} است.
	- بعد از آن نزدیک‌ترین درخواست به \lr{52}، \lr{57} است.
	- سپس نزدیک‌ترین درخواست به \lr{57}، \lr{98} است.
	- و به همین ترتیب ادامه می‌دهیم.
	
	ترتیب دسترسی: \lr{48, 52, 57, 98, 102, 140, 23, 17, 12, 7}
	
	مجموع \lr{Head movement}:
	\[
	|50 - 48| + |48 - 52| + |52 - 57| = 2 + 4 + 5 = 11
	\]
	\[
	|57 - 98| + |98 - 102| + |102 - 140| = 41 + 4 + 38 = 83
	\]
	\[
	|140 - 23| + |23 - 17| + |17 - 12| + |12 - 7| = 117 + 6 + 5 + 5 = 133
	\]
	\[
	\text{مجموع} = 11 + 83 + 133 = 227
	\]
	
\end{qsolve}