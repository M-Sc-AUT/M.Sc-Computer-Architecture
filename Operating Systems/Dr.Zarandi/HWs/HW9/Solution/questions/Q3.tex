\section{سوال سوم}


یک حافظه فیزیکی با ۱۰۲۴ قاب (\lr{frame}) تحت نگاشت یک فضای آدرس‌دهی منطقی شامل ۲۰۴۸ صفحه که اندازه هر صفحه آن ۴ کیلوبایت می‌باشد، قرار گرفته است. برای آدرس‌دهی منطقی و آدرس‌دهی فیزیکی این فضا به چه تعداد بیت نیاز داریم؟

\begin{qsolve}
	
	
	\textbf{آدرس‌دهی منطقی (\lr{Logical Addressing})}
	
	فضای آدرس‌دهی منطقی شامل \(2048\) صفحه است و اندازه هر صفحه \(4 \text{\lr{ KB}}\) (\(2^{12}\) بایت) است. بنابراین فضای آدرس‌دهی منطقی برابر است با:
	\[
	2048 \times 2^{12} \text{\lr{ bytes}} = 2^{11} \times 2^{12} = 2^{23} \text{\lr{ bytes}}
	\]
	
	برای آدرس‌دهی \(2^{23}\) بایت، به \(23\) بیت نیاز داریم.
	
	
	\textbf{آدرس‌دهی فیزیکی (\lr{Physical Addressing})}
	
	حافظه فیزیکی شامل \(1024\) قاب (\lr{frame}) است و اندازه هر قاب برابر با اندازه یک صفحه (\(4 \text{\lr{ KB}}\)) است. بنابراین فضای آدرس‌دهی فیزیکی برابر است با:
	\[
	1024 \times 2^{12} \text{\lr{ bytes}} = 2^{10} \times 2^{12} = 2^{22} \text{\lr{ bytes}}
	\]
	
	برای آدرس‌دهی \(2^{22}\) بایت، به \(22\) بیت نیاز داریم.
	
\end{qsolve}