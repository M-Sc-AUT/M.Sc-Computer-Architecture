\section{سوال سوم}

مسیر اجرای کد زیر را در گراف حالت فرایند (\lr{Process state}) از شروع اجرا تا پایان اجرا مشخص کنید. توجه کنید سیستمی که این قطعه در آن اجرا می‌شود تک پردازنده می‌باشد.


\begin{latin}
\begin{lstlisting}[caption=Code of Q3, label=cpp_code_example]
int main()
{
	int n;
	scanf("%d", &n);
	n *= 10;
	printf("%d", n);
	
	return 0;
}
\end{lstlisting}
\end{latin}


\begin{qsolve}
	مطابق با تعریف آقای \lr{Silberschatz} در صفحه ۱۰۸، گراف حالت فرآیند به صورت زیر است:
	
	
	\begin{center}
		\includegraphics*[width=0.7\linewidth]{pics/img1.png}
		\captionof{figure}{ساختار \lr{Process State}}
	\end{center}
	
	طبق این تعریف می‌توان غرایند اجرای این کد را به‌صورت زیر نوشت:
	
	\begin{enumerate}
		\item 
		\textbf{\lr{:New}}\\
		فرآیند در وضعیت \lr{New} شروع می‌شود، زمانی که برنامه (یعنی این کد) ایجاد می‌شود، اما هنوز برای اجرا پذیرفته نشده است. در این مرحله، سیستم‌عامل فرآیند را در حافظه تنظیم کرده و آن را برای اجرا آماده می‌کند.
		
		
		
		
		\item 
		\textbf{\lr{:Ready}}\\
		پس از ایجاد فرآیند، به وضعیت \lr{Ready} می‌رویم. و در انتظار تخصیص پردازنده می‌مانیم. این وضعیت نشان می‌دهد که فرآیند در حافظه بارگذاری شده و برای اجرا آماده است، اما منتظر است تا پردازنده به آن اختصاص داده شود.
		
		
		
		
		\item 
		\textbf{\lr{:Running}}\\
		پس از اینکه زمان‌بند پردازنده این فرآیند را انتخاب کرد، به وضعیت \lr{Running} وارد می‌شود. در اینجا فرآیند مراحل زیر را طی می‌کند:		

	\end{enumerate}
\end{qsolve}


\begin{qsolve}[ادامه پاسخ ۳]
	\begin{enumerate}
		\item [ ]
		\begin{itemize}
				\item 						تابع \lr{\texttt{scanf()}} منتظر ورودی است، بنابراین در حالی که منتظر ورودی کاربر است، فرآیند ممکن است به طور موقت به وضعیت در انتظار \lr{Waiting} منتقل شود (اگر ورودی زمان‌بر باشد یا به \lr{I/O} نیاز داشته باشد).
		
				\item 						پس از دریافت ورودی، فرآیند مقدار \texttt{n} را در ۱۰ ضرب می‌کند که یک محاسبه است و در این مرحله در وضعیت در حال \lr{Running} باقی می‌ماند.
					
				\item 						سپس تابع \lr{\texttt{printf()}} برای نمایش نتیجه فراخوانی می‌شود که شامل عملیات \lr{I/O} است. اگر در عملیات \lr{I/O} تأخیر باشد، ممکن است به طور موقت دوباره به وضعیت \lr{Waiting} بازگردد.
					
	
			\end{itemize}
			
			\item [4.]
			\textbf{\lr{:Waiting}}\\					این وضعیت زمانی اعمال می‌شود که فرآیند نیاز دارد برای یک عملیات \LR{I/O} (مانند \lr{\texttt{scanf()}} یا \lr{\texttt{printf()}}) منتظر بماند. فرآیند ممکن است به طور موقت در وضعیت \lr{Waiting} باشد اگر عملیات \lr{I/O} با تأخیر انجام شود یا منتظر ورودی کاربر باشد، اما پس از اتمام
			 \lr{I/O} دوباره به آماده یا در حال اجرا بازمی‌گردد.
			
			
			\item [5.]
			\textbf{\lr{:Terminated}}\\				پس از اتمام تمام دستورات (رسیدن به \lr{\texttt{return 0;}} ) فرآیند وارد وضعیت \lr{Terminated} می‌شود، به این معنی که اجرای آن به پایان رسیده و از سیستم خارج می‌شود.
		
		
		
\end{enumerate}
\end{qsolve}