% source: http://mjgeiger.github.io/OS/prev/sp17/homework/OSsp17_hw2_soln.pdf

\section{سوال دوم}

کد زیر را در نظر بگیرید. تابع \texttt{create\_thread()} یک ریسمان جدیدی را در فرایند فراخوانی شروع می‌کند. چند فرایند منحصر به فرد ایجاد می‌شود؟ چه تعداد رشته منحصر به فرد ایجاد می‌شود؟ توضیح دهید.


\begin{latin}
\begin{lstlisting}[caption=Code of Q2, label=cpp_code_example]
pid_t pid;
pid = fork();
if (pid == 0) 
{ /* Child process */
	fork();
	thread_create(...);
}
fork();

\end{lstlisting}
\end{latin}


\begin{qsolve}
	این کد درمجموع، ۶ فرآیند و دو نخ ایجاد می‌کند که توضیحات آن را در ادامه می‌دهیم.
	
	در اولین فراخوانی \lr{\texttt{fork()}} که در خط ۲ انجام می‌شود، یک کپی از \lr{Process} اصلی ایجاد می‌شود. تا اینجا ۲ \lr{Process} داریم.
	
	در فراخوانی دوم که در خط ۵ اتفاق می‌افتد، تنها توسط \lr{Process} فرزند که حاصل از فراخوانی اول است اجرا می‌شود. تا اینجا ۳ \lr{Process} داریم.
	
	اکنون دو \lr{Process} در حال اجرای کد داخل شرط \texttt{if} هستند، به این معنی که هر دو این فرآیندها \texttt{thread\_create()} را فراخوانی می‌کنند (در این نقطه ۳ فرآیند و ۲ نخ داریم).
	
	البته نکته‌ای که باید در صورت این مسئله واضح‌تر بیان می‌شد این است که هر نخ تازه ایجاد شده یک تابع متفاوت از تابعی که هم‌اکنون در حال اجرا است، شروع می‌کند. یکی از آرگومان‌های تابع \texttt{thread\_create()} اشاره‌گری به تابعی است که باید اجرا شود. بنابراین نخ‌ها به فراخوانی \texttt{fork()} آخر نمی‌رسند.
	
	هر سه \lr{Process} فراخوانی نهایی به \texttt{fork()} را در خط ۸ اجرا می‌کنند، بنابراین هر \lr{Process} در این نقطه خود را کپی می‌کند (در مجموع ۶ فرآیند و ۲ نخ داریم).
	
	
	البته با توجه به اینکه هر \lr{Process} به عنوان یک \lr{single-thread} آغاز می‌شود، شاید بهتر باشد بگوییم این قطعه کد در مجموع شش \lr{Process} و هشت نخ ایجاد می‌کند (دو نخ توسط فراخوانی‌های \texttt{thread\_create()} و شش نخ که مربوط به شش فرآیند تک‌نخی هستند).
	
\end{qsolve}