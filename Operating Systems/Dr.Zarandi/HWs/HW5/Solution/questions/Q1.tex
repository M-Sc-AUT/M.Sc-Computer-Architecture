\section{سوال اول}

در خصوص انواع فرایندها به سوالات زیر پاسخ دهید.
\begin{enumerate}
	\item 
	فرایند فرزند چگونه منابع مورد نیاز خود را تامین می‌کند؟ آیا می‌تواند از منابع والد استفاده کند؟
	\begin{qsolve}
		مطابق با توضیحات آقای \lr{Silberschatz} در صفحه ۱۱۷، فرایند فرزند در هنگام ایجاد توسط سیستم‌عامل می‌تواند منابع خود را به چندین روش تأمین کند:
		\begin{enumerate}
			\item 
			\textbf{تخصیص منابع از والد:} در بسیاری از سیستم‌عامل‌ها، فرایند فرزند می‌تواند بخشی از منابع فرایند والد را به ارث ببرد. به عنوان مثال، اگر والد دارای فایل‌های باز، حافظه و یا دسترسی به برخی دستگاه‌ها باشد، این منابع ممکن است به فرایند فرزند به ارث برسند.
			
			
			\item 
			\textbf{تخصیص منابع جدید:} فرایند فرزند می‌تواند از سیستم‌عامل درخواست منابع جدید کند، مانند حافظه اضافی یا دسترسی به فایل‌ها. این منابع از منابع کلی سیستم اختصاص می‌یابند و مستقل از والد هستند.
			
			
			\item 
			\textbf{به ارث بردن حافظه و داده‌ها:} معمولاً در سیستم‌های مبتنی بر \lr{UNIX}، فرایند فرزند یک کپی از فضای حافظه والد خود را به ارث می‌برد. این به معنای آن است که فرایند فرزند در ابتدا از داده‌ها و متغیرهای والد خود کپی‌ای مستقل دارد.
		\end{enumerate}
	
	\end{qsolve}
	
	
	
	\item 
	همانطور که می‌دانید فرایند فرزند ممکن است پیش از اتمام اجرا توسط فرایند والد به پایان برسد. توضیح دهید که فرایند والد به چه دلایلی ممکن است تصمیم بگیرد فرایند فرزند پایان یابد.
	\begin{qsolve}
		فرایند والد ممکن است به دلایل زیر تصمیم بگیرد فرایند فرزند خود را خاتمه دهد:
		
		\begin{enumerate}
			\item 
			\textbf{خطا در اجرای فرزند:} اگر فرایند فرزند دچار خطا یا مشکل شود (مثلاً دسترسی غیرمجاز به حافظه)، والد ممکن است تشخیص دهد که بهتر است فرزند را خاتمه دهد.
			
			\item 
			\textbf{نیاز به بازپس‌گیری منابع:} در صورت نیاز به منابع بیشتر، فرایند والد می‌تواند فرزند را پایان دهد تا منابع مصرفی آن آزاد شوند و به والد یا سایر فرایندها اختصاص یابند.
			
			\item 
			\textbf{پایان زودتر از موعد والد:} در برخی شرایط، اگر والد تصمیم بگیرد زودتر از موعد پایان یابد، ممکن است فرزند نیز خاتمه پیدا کند، چرا که فرایند فرزند دیگر بدون والد قابل اجرا نیست.
		\end{enumerate}
	\end{qsolve}
	
	\begin{qsolve}
		\begin{enumerate}
			\item [(د)]
			\textbf{دستور صریح از والد:} فرایند والد می‌تواند به طور مستقیم و با دستوراتی مانند \texttt{kill} یا \texttt{terminate} در سیستم‌های مبتنی بر \lr{UNIX} فرایند فرزند را پایان دهد.
		\end{enumerate}
	\end{qsolve}
	
	
	
	
	\item 
	هنگامی که فرایند والد به دستور \lr{\texttt{wait()}} می‌رسد، چه اتفاقی رخ می‌دهد؟
	\begin{qsolve}
		هنگامی که فرایند والد به دستور \texttt{wait()} می‌رسد، اتفاقات زیر رخ می‌دهند:
		\begin{enumerate}
			\item 
			\textbf{توقف اجرای والد:} فرایند والد به حالت \lr{Waiting} می‌رود و منتظر می‌ماند تا فرایند فرزند خاتمه یابد.
			
			\item 
			\textbf{انتظار برای پایان فرزند:} سیستم‌عامل اجرای والد را به حالت تعلیق در می‌آورد تا زمانی که فرایند فرزند به پایان برسد.
			
			\item 
			\textbf{دریافت کد بازگشتی فرزند:} پس از اتمام فرایند فرزند، کد بازگشتی آن به والد منتقل می‌شود. این کد می‌تواند وضعیت خروجی فرایند فرزند را نشان دهد، مثلاً موفقیت یا شکست فرایند.
			
			\item 
			\textbf{بازگشت والد به اجرا:} بعد از اینکه فرزند خاتمه یافت و والد اطلاعات لازم را دریافت کرد، والد دوباره به حالت \lr{Ready} بازگشته و اجرای آن ادامه می‌یابد.
		\end{enumerate}
	\end{qsolve}
\end{enumerate}


%\begin{center}
%	\includegraphics*[width=0.7\linewidth]{pics/img3.png}
%	\captionof{figure}{ساختار یک ۳ عدد \lr{VM}}
%\end{center}