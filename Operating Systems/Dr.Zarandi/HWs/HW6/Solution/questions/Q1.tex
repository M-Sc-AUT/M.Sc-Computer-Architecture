\section{سوال اول}

ناحیه بحرانی را تعریف کنید و شروط لازم و کافی را برای آن نام ببرید و به صورت مختصر توضیح دهید.


\begin{qsolve}
	مطابق با تعریف کتاب آقای \lr{silberschatz} در صفحه ۲۶۰، می‌توان گفت: ناحیه بحرانی بخشی از برنامه است که در آن فرآیندها یا رشته‌ها به منابع مشترک دسترسی پیدا می‌کنند که ممکن است باعث تداخل و ناسازگاری در نتایج شود. در یک سیستم \lr{Multi task} برای جلوگیری از مشکلات ناشی از دسترسی هم‌زمان به منابع مشترک، باید دسترسی به ناحیه بحرانی به‌درستی مدیریت شود. در شکل زیر این این مشکل آورده شده است:
	
	\begin{center}
		\includegraphics*[width=0.3\linewidth]{pics/img1.png}
		\captionof{figure}{ساختار عمومی مسئله ناحیه بحرانی}
	\end{center}
	
	در ادامه اگر منظور از شروط لازم و کافی، شروط لازم و کافی برای حل مشکل ناحیه بحرانی باشد می‌توان آن را به ۳ دست زیر تقسیم نمود:
	
	\begin{enumerate}
		\item 
		\textbf{انحصار متقابل یا \lr{Mutual exclusion}:}
		اگر فرآیند $P_i$ در حال اجرا در ناحیه بحرانی خود باشد، هیچ فرآیند دیگری نمی‌تواند در ناحیه بحرانی خود اجرا شود. یا یه عبارتی دیگر، در هر لحظه، فقط یک فرآیند اجازه دارد که وارد ناحیه بحرانی شود. این شرط مانع از دسترسی هم‌زمان چندین فرآیند به منابع مشترک می‌شود.
		
		
		\item 
		\textbf{پیشرفت یا \lr{Progress}:}
		 در صورتی که هیچ فرآیندی در ناحیه بحرانی نباشد، فرآیندهای آماده‌ی ورود به ناحیه بحرانی نباید بدون دلیل منتظر بمانند. این شرط تضمین می‌کند که در صورت امکان، فرآیندهای آماده به ناحیه بحرانی دسترسی پیدا کنند.
		 
		 
		 \item 
		 \textbf{انتظار محدود یا \lr{Bounded Waiting}:}
		 هر فرآیند نمی‌تواند برای همیشه منتظر بماند تا وارد ناحیه بحرانی شود. این شرط تضمین می‌کند که پس از مدتی محدود، هر فرآیند می‌تواند به ناحیه بحرانی دسترسی پیدا کند و به \lr{Starvation} دچار نمی‌شود.
	\end{enumerate}
\end{qsolve}




