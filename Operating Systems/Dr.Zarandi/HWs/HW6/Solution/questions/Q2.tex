\section{سوال دوم}

دو روش برای مدیریت نواحی بحرانی به صورت \lr{Preemptive} و \lr{Non preemptive} می‌باشد. این دو روش را توضیح دهید و برای هرکدام یک مثال بیاورید که در چه نوع سیستم‌هایی بهتر است استفاده شوند.




\begin{qsolve}
	همانطور که در صورت سوال نیز بیان شد، دو روش کلی برای مدیریت نواحی بحرانی در سیستم‌عامل‌ها استفاده می‌شود: روش \lr{Preemptive} و روش \lr{Non-Preemptive}
	
	\begin{enumerate}
		\item 
		در روش \lr{Preemptive} سیستم عامل می‌تواند یک فرآیند را در هنگام اجرای ناحیه بحرانی به‌صورت خودکار متوقف کند و کنترل را به فرآیند دیگری واگذار کند. در این حالت، فرآیند می‌تواند با قطع ناگهانی (که اصطلاحا به این کار \lr{Preempt} کردن گفته می‌شود) از ناحیه بحرانی خارج شود. این روش انعطاف‌پذیر است و امکان اجرای همزمان چند فرآیند را فراهم می‌آورد. این روش در سیستم‌های \lr{Time-Sharing} مانند سیستم‌های عامل دسکتاپ (\lr{Windows، Linux، macOS}) که نیاز به مدیریت هم‌زمان چندین برنامه را دارند، بسیار مناسب است. به دلیل نیاز به پاسخ‌دهی سریع به تعاملات کاربر و اجرای همزمان برنامه‌ها، این سیستم‌ها از روش پیش‌دستانه بهره می‌برند تا اطمینان حاصل شود که هیچ فرآیندی به طور نامحدود در ناحیه بحرانی باقی نمی‌ماند.
		
		
		
		\item 
		روش  \lr{None-Preemptive} فرآیند پس از ورود به ناحیه بحرانی بدون امکان قطع توسط سیستم عامل تا پایان کارش در ناحیه بحرانی باقی می‌ماند. در این روش، کنترل به فرآیند دیگری منتقل نمی‌شود مگر اینکه فرآیند به‌طور کامل کار خود را به پایان رسانده و ناحیه بحرانی را ترک کند. این روش برای سیستم‌هایی که نیاز به کنترل دقیق در دسترسی به منابع مشترک دارند، مناسب است. این روش در سیستم‌های \lr{Real-Time} که به زمان‌بندی دقیق و پیش‌بینی‌پذیر نیاز دارند، استفاده می‌شود، مانند سیستم‌های کنترل صنعتی یا سیستم‌های کنترل پرواز. در این سیستم‌ها، پیش‌بینی‌پذیری اهمیت بالایی دارد و قطع شدن فرآیندها در حین اجرای ناحیه بحرانی ممکن است به نتایج ناخواسته و خطرناک منجر شود.
	\end{enumerate}
\end{qsolve}