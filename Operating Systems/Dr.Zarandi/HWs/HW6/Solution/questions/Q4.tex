\section{سوال چهارم}

دو فرآیند برای حل مسائل ناحیه بحرانی از روش‌های زیر استفاده کرده‌اند (متغیرهای \lr{L1} و \lr{L2} در هر دو مشترک هستند و مقدار \lr{Boolean} دارند و در ابتدا به صورت تصادفی مقداردهی شده‌اند). هر کدام از سه شرط \lr{Mutual Exclusion} و \lr{Progress} و \lr{Bounded Waiting} را بررسی کنید و توضیح دهید.





\begin{multicols}{2}
\centering
\begin{latin}
\begin{lstlisting}[caption=Code of Q4, label=cpp_code_example]
// P2
while (L1 == L2);
//Critical Section
L1 = L2;
\end{lstlisting}
\end{latin}
\begin{latin}
\begin{lstlisting}[caption=Code of Q4, label=cpp_code_example]
// P1
while (L1 != L2);
//Critical Section
L1 = !L2;
\end{lstlisting}
\end{latin}	
\end{multicols}






\begin{qsolve}
	\begin{enumerate}
		\item 
		\textbf{بررسی شرط \lr{Mutual Exclusion}:}
		به دلیل این شرایط متفاوت، تنها یکی از پردازه‌ها می‌تواند همزمان وارد ناحیه بحرانی شود، زیرا شرطی که برای ورود هر فرآیند وجود دارد، دقیقاً خلاف شرط دیگر فرآیند است. بنابراین، شرط \lr{Mutual Exclusion} برقرار است؛ چرا که اگر یکی از فرآیند‌ها وارد ناحیه بحرانی شود، دیگری نمی‌تواند وارد شود تا زمانی که آن فرآیند از ناحیه بحرانی خارج شود و مقادیر $L_1 $ و $L_2 $ تغییر یابند.
		
		
		\item 
		\textbf{بررسی شرط \lr{Progress}:}
			در این الگوریتم، اگر یکی از فرآیند‌ها در حال اجرای ناحیه بحرانی باشد، فرآیند دیگر منتظر می‌ماند. اما مشکل زمانی به وجود می‌آید که مقادیر $L_1 $ و $L_2 $ به صورتی تنظیم شوند که هر دو فرآیند در حالت نامناسب برای ورود قرار گیرند، یعنی:
		\begin{itemize}
			\item 
			اگر $P_1 $ شرط ورود خود را برقرار نبیند (یعنی \texttt{L1==L2} ) و منتظر تغییر در مقدار باشد.
			
			\item 
			و اگر $P_2 $ شرط ورود خود را برقرار نمی‌بیند (یعنی \texttt{L1!=L2} ) و او هم منتظر بماند.
		\end{itemize}
		
		این وضعیت می‌تواند به \lr{Deadlock} منجر شود که هر دو فرآیند به صورت نامحدود در حلقه‌های انتظار خود باقی بمانند، بدون اینکه هیچ‌کدام پیشرفتی داشته باشد. به همین دلیل، شرط پیشرفت به صورت قطعی برقرار نیست و احتمال دارد فرآیند‌ها بدون دلیل منتظر بمانند.
		
		
		
			\item
			\textbf{بررسی شرط \lr{Bounded Waiting}:}
			در این کد، هیچ سازوکار مشخصی برای محدود کردن زمان انتظار وجود ندارد. به عنوان مثال، اگر فرآیند $P_1 $ وارد ناحیه بحرانی شود و پس از خروج مقدار \texttt{L1 != L2} را تنظیم کند، ممکن است فرآیند $P_2 $ در حالت انتظار باقی بماند، چرا که این مقدار به شرط او نمی‌خورد. این مسئله می‌تواند باعث شود که یکی از فرآیند‌ها برای مدت طولانی در حالت انتظار بماند. در نتیجه، شرط انتظار محدود نیز برقرار نیست و ممکن است یک فرآیند به صورت نامحدود منتظر بماند.
		
	\end{enumerate}
\end{qsolve}