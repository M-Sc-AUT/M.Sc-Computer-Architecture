\section{سوال پنجم}

کلاس زیر که پیاده‌سازی سمافور است را کامل کنید و توضیح دهید هر بخش از کد که اضافه می‌کنید چگونه به حفظ سه شرط \lr{Mutual Exclusion} و \lr{Progress} و \lr{Bounded Waiting} کمک می‌کند (فرض کنید که کلاس \lr{Process} دو متد \lr{block} و \lr{wakeup} دارد).


\begin{latin}
\begin{lstlisting}[caption=Code of Q5, label=cpp_code_example]
class Semaphore 
{
	queue :Queue<Process>
	// other class Properties
	
	constructor Semaphore(initialValue: int){
	}
	
	wait(process: Process){
	}
	
	signal(){
	}
}
\end{lstlisting}
\end{latin}




\begin{qsolve}
	کلاس را به‌صورت زیر تکمیل می‌کنیم:
\end{qsolve}

\begin{latin}
\begin{lstlisting}[caption=Complete code of Q5, label=cpp_code_example]
// Semaphore class definition
class Semaphore {
	private:
	int value; // semaphore counter
	Queue<Process> queue; // queue to store waiting processes
	
	public:
	// Constructor to set the initial value of the semaphore
	Semaphore(int initialValue) {
		value = initialValue;
	}
	
	// wait method for entering the critical section
	void wait(Process process) {
		value--; // decrement the semaphore value
		if (value < 0) {
			queue.enqueue(process); // add the process to the queue
			process.block(); // block the process
		}
	}
	
	// signal method for exiting the critical section
	void signal() {
		value++; // increment the semaphore value
		if (value <= 0) {
			Process nextProcess = queue.dequeue(); // dequeue a process
			nextProcess.wakeup(); // wake up the process
		}
	}
};
\end{lstlisting}
\end{latin}


\begin{qsolve}[ادامه پاسخ]
	
	\begin{itemize}
		\item متغیر \texttt{value}: این متغیر شمارنده‌ای است که مقدار سمافور را نگه می‌دارد. اگر مقدار آن منفی شود، فرآیندها باید منتظر بمانند، که باعث برقراری شرط \lr{Mutual Exclusion} می‌شود.
		
		\item سازنده \lr{\texttt{Semaphore(int initialValue)}}: این سازنده مقدار اولیه سمافور را تنظیم می‌کند. این مقدار اولیه به شرط \lr{Mutual Exclusion} کمک می‌کند، زیرا برای سمافور باینری تنها یک فرآیند در هر زمان می‌تواند به ناحیه بحرانی دسترسی داشته باشد.
		
		\item متد \lr{\texttt{wait(Process process)}}: در این متد، اگر مقدار \texttt{value} منفی شود، فرآیند به صف اضافه شده و مسدود می‌شود. این عمل به شرط \lr{Bounded Waiting} کمک می‌کند، زیرا فرآیندها به ترتیب ورود در صف قرار گرفته و به نوبت از صف خارج می‌شوند.
		
		\item متد \texttt{\texttt{signal()}}: این متد مقدار \texttt{value} را افزایش می‌دهد و اگر فرآیندهایی در صف منتظر باشند، فرآیند بعدی را بیدار کرده و اجازه ورود می‌دهد. این امر باعث برقراری شرط \lr{Progress} و \lr{Bounded Waiting} می‌شود.
	\end{itemize}
	
	این پیاده‌سازی سه شرط لازم برای مدیریت نواحی بحرانی را به‌طور کامل فراهم می‌کند و با استفاده از صف و کنترل متغیر \texttt{value}، به حفظ همگام‌سازی بین فرآیندها کمک می‌کند.
	
\end{qsolve}