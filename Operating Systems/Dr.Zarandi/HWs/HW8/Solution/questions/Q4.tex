\section{سوال چهارم}

یک سیستم حافظه قطعه‌بندی شده را با حافظه تخصیص یافته مطابق شکل زیر در نظر بگیرید.

\begin{center}
	\includegraphics*[width=0.2\linewidth]{pics/img1.png}
	\captionof{figure}{حافظه فطعه‌بندی شده}
\end{center}




فرض کنید اقدامات زیر رخ می‌دهد:

\begin{itemize}
	\item فرآیند \lr{E} شروع می‌شود و ۳۰۰ واحد حافظه درخواست می‌کند.
	\item فرآیند \lr{A} مقدار ۴۰۰ واحد حافظه دیگر درخواست می‌کند.
	\item فرآیند \lr{B} خارج می‌شود.
	\item فرآیند \lr{F} شروع می‌شود و ۸۰۰ واحد حافظه درخواست می‌کند.
	\item فرآیند \lr{C} خارج می‌شود.
	\item فرآیند \lr{G} شروع می‌شود و ۹۰۰ واحد حافظه درخواست می‌کند.
\end{itemize}



\begin{enumerate}
	\item وضعیت حافظه را پس از هر عمل با استفاده از الگوریتم اولین برازش توصیف کنید.
	\begin{qsolve}
		\begin{enumerate}
			\item \lr{E}: درخواست ۳۰۰ واحد حافظه دارد. اولین فضای خالی که می‌تواند ۳۰۰ واحد را جای دهد، بخش \lr{B} است (۸۰۰–۱۶۰۰). بنابراین:
			\begin{itemize}
				\item فرآیند \lr{E} ۳۰۰ واحد اول را از این بخش می‌گیرد.
				\item بخش \lr{B} به دو بخش تقسیم می‌شود: ۳۰۰ واحد اشغال‌شده (\lr{E}) و ۵۰۰ واحد خالی.
			\end{itemize}
			
			
			\item \lr{A}: درخواست ۴۰۰ واحد حافظه دارد. اولین فضای خالی که ۴۰۰ واحد حافظه دارد، بخش خالی باقی‌مانده از \lr{B} است (۵۰۰ واحد). بنابراین:
			\begin{itemize}
				\item فرآیند \lr{A} ۴۰۰ واحد از این بخش را می‌گیرد.
				\item باقی‌مانده \lr{B}: ۱۰۰ واحد خالی.
			\end{itemize}
		\end{enumerate}
	\end{qsolve}
\end{enumerate}
\newpage


\begin{enumerate}
	\item [ ]
	\begin{qsolve}
		\begin{enumerate}
			
			\item [(ج)] \lr{B}: خارج می‌شود. این باعث می‌شود کل فضای \lr{B} (۱۰۰ واحد خالی و ۴۰۰ واحد اشغال‌شده توسط \lr{B}) به ۵۰۰ واحد خالی تبدیل شود.
			
			\item [(د)] \lr{F}: درخواست ۸۰۰ واحد حافظه دارد. اولین فضای خالی که می‌تواند این درخواست را برآورده کند، بخش \lr{C} است (۱۹۰۰–۲۴۰۰). بنابراین:
			\begin{itemize}
				\item فرآیند \lr{F} تمام این بخش را اشغال می‌کند.
			\end{itemize}
			
			\item [(ر)] \lr{C}: خارج می‌شود. کل فضای \lr{C} آزاد می‌شود (۵۰۰ واحد).
			
			\item [(ز)] \lr{G}: درخواست ۹۰۰ واحد حافظه دارد. اولین فضای خالی که می‌تواند این درخواست را برآورده کند، بخش \lr{D} است (۳۱۰۰–۳۴۰۰). بنابراین:
			\begin{itemize}
				\item فرآیند \lr{G} تمام بخش \lr{D} را اشغال می‌کند.
			\end{itemize}
			
		\end{enumerate}
		
	\end{qsolve}
	
	
	
	\item [2.] محتویات حافظه را پس از هر اقدام با استفاده از الگوریتم بهترین برازش توصیف کنید.
	\begin{qsolve}
		\begin{enumerate}
			\item \lr{E}: درخواست ۳۰۰ واحد حافظه دارد. بهترین فضای خالی، کوچک‌ترین فضای خالی است که درخواست را برآورده کند. بخش \lr{B} (۸۰۰–۱۶۰۰) ۵۰۰ واحد دارد که بهترین گزینه است. نتیجه مشابه \lr{First Fit} است.
			
			\item \lr{A}: درخواست ۴۰۰ واحد حافظه دارد. بهترین فضای خالی، بخش خالی باقی‌مانده \lr{B} است (۵۰۰ واحد). نتیجه مشابه \lr{First Fit} است.
			
			\item \lr{B}: خارج می‌شود. کل فضای \lr{B} آزاد می‌شود (۵۰۰ واحد).
			
			\item \lr{F}: درخواست ۸۰۰ واحد حافظه دارد. بهترین فضای خالی که می‌تواند این درخواست را برآورده کند، بخش \lr{C} است (۵۰۰ واحد). نتیجه مشابه \lr{First Fit} است.
			
			\item \lr{C}: خارج می‌شود. بخش \lr{C} آزاد می‌شود.
			
			\item \lr{G}: درخواست ۹۰۰ واحد حافظه دارد. بهترین فضای خالی، بخش \lr{D} است. مشابه \lr{First Fit}.
		\end{enumerate}
	\end{qsolve}
	
	\item [3.] الگوریتم، بدترین برازش حافظه را چگونه تخصیص می‌دهد؟
	\begin{qsolve}
		\begin{enumerate}
			\item \lr{E}: درخواست ۳۰۰ واحد دارد. بزرگ‌ترین فضای خالی موجود، بخش \lr{B} (۸۰۰–۱۶۰۰) است. مشابه \lr{First Fit}.
			
			\item \lr{A}: درخواست ۴۰۰ واحد دارد. بزرگ‌ترین فضای خالی باقی‌مانده نیز بخش \lr{B} است. مشابه \lr{First Fit}.
			
			\item \lr{B}: خارج می‌شود. کل فضای \lr{B} آزاد می‌شود.
			
			\item \lr{F}: درخواست ۸۰۰ واحد حافظه دارد. بزرگ‌ترین فضای خالی، بخش \lr{C} است. مشابه \lr{First Fit}.
			
			\item \lr{C}: خارج می‌شود. بخش \lr{C} آزاد می‌شود.
			
			\item \lr{G}: درخواست ۹۰۰ واحد حافظه دارد. بزرگ‌ترین فضای خالی، بخش \lr{D} است. مشابه \lr{First Fit}.
		\end{enumerate}
	\end{qsolve}
	
\end{enumerate}














