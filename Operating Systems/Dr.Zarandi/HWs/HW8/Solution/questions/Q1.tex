\section{سوال اول}


به سوالات زیر پاسخ دهید.

\begin{enumerate}
	\item 
	برای هر یک از حالات زیر توضیح دهید کدام یک از روش‌های \lr{static linking} و \lr{dynamic linking} بهتر است انجام شود:
	
	\begin{itemize}
		\item تعدادی برنامه که از کتابخانه‌های مختلف استفاده می‌کنند.
		
		\begin{qsolve}
	برای برنامه‌هایی که هر کدام از کتابخانه‌های متفاوتی استفاده می‌کنند، روش \lr{static linking} مناسب‌تر است. دلیل آن این است که هر برنامه با کتابخانه‌های مخصوص خود لینک می‌شود و در زمان اجرا نیازی به بارگذاری پویا و مدیریت نسخه‌های متفاوت کتابخانه‌ها نیست. همچنین از آن‌جا که کتابخانه‌ها مشترک نیستند، استفاده از \lr{dynamic linking} صرفه‌جویی چندانی در حافظه ایجاد نمی‌کند و \lr{static linking} پیکربندی و اجرا را ساده‌تر می‌سازد.
		\end{qsolve}
		
		
		\item تعدادی برنامه که همگی از یک کتابخانه استفاده می‌کنند.
		\begin{qsolve}
	برای برنامه‌هایی که همگی از یک کتابخانه‌ی مشترک استفاده می‌کنند، روش \lr{dynamic linking} بهتر است. زیرا می‌توان تنها یک نسخه از کتابخانه را به صورت پویا در حافظه بارگذاری کرده و همه‌ی برنامه‌ها از آن استفاده کنند. این کار منجر به صرفه‌جویی در حافظه و سهولت در به‌روزرسانی کتابخانه می‌شود.
		\end{qsolve}
	\end{itemize}
	
	
	
	\item 
	تفاوت تکه‌تکه سازی خارجی و داخلی را توضیح دهید. در هر بخش زیر مشخص کنید کدام یک از تکه تکه سازی داخلی یا خارجی برای ما می‌تواند مشکل ایجاد کند؟
	\begin{qsolve}
		\begin{itemize}
			\item \lr{External Fragmentation}: زمانی رخ می‌دهد که فضای آزاد حافظه به صورت پراکنده بین بخش‌های اشغال‌شده پخش می‌شود و هرچند مجموع فضای آزاد برای اجرای برنامه‌ای جدید کافی است، اما به صورت یک بخش پیوسته در دسترس نیست.
			\item \lr{Internal Fragmentation}: زمانی رخ می‌دهد که به یک فرایند، بلوکی بزرگ‌تر از نیاز واقعی‌اش اختصاص داده می‌شود و بخشی از آن بلوک استفاده‌نشده و هدر می‌رود.
		\end{itemize}
	\end{qsolve}
	
	
	\begin{itemize}
		\item یک ماشین مدیریت حافظه ساده با استفاده از ثبات‌های \lr{base} و \lr{limit} و بخش‌بندی ایستا.
		
		\begin{qsolve}
	یک ماشین مدیریت حافظه ساده با استفاده از ثبات‌های \lr{base} و \lr{limit} و بخش‌بندی ایستا معمولاً منجر به \lr{Internal Fragmentation} می‌شود. زیرا حافظه به بخش‌هایی با اندازه‌ی ثابت تقسیم شده و ممکن است اندازه‌ی بخش از نیاز واقعی فرایند بیشتر باشد، در نتیجه فضای داخلی هدر می‌رود.
		\end{qsolve}
		
		\item یک ماشین مشابه قسمت قبل با استفاده از بخش‌بندی پویا
	
		\begin{qsolve}
	یک ماشین با بخش‌بندی پویا (\lr{Dynamic Partitioning}) با گذشت زمان و تخصیص و آزادسازی حافظه، دچار \lr{External Fragmentation} می‌شود. در این حالت، شکاف‌های آزاد کوچک و پراکنده بین بخش‌های اشغال‌شده به وجود آمده و ممکن است علی‌رغم وجود فضای آزاد کافی، نتوان یک بخش پیوسته متناسب برای اجرای فرایند جدید یافت.
		\end{qsolve}
		
	\end{itemize}
\end{enumerate}






%\begin{center}
%	\includegraphics*[width=0.3\linewidth]{pics/img1.png}
%	\captionof{figure}{ساختار عمومی مسئله ناحیه بحرانی}
%\end{center}