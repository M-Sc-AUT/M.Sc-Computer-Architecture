\section{سوال سوم}



فرض کنید در یک سیستم حافظه مشخصات زیر داده شده است:

\begin{itemize}
	\item آدرس منطقی: ۲۰ بیت
	\item سایز صفحه: ۸ کیلوبایت (۸۱۹۲ بایت)
\end{itemize}



\begin{enumerate}
	\item تعداد صفحات منطقی موجود در فضای آدرس منطقی چقدر است؟
	\begin{qsolve}
		
		آدرس منطقی ۲۰ بیت است، پس فضای آدرس منطقی کل به صورت زیر محاسبه می‌شود:
		
		\[
		2^{20} = 1,048,576 \; \text{MByte} 
		\]
		
		سایز صفحه $8$ کیلوبایت است. تعداد صفحات منطقی از تقسیم فضای آدرس منطقی بر سایز صفحه به دست می‌آید:
		
		\[
		\text{تعداد صفحات منطقی} = \frac{\text{فضای آدرس منطقی}}{\text{سایز صفحه}} = \frac{1,048,576}{8,192} = 128
		\]
	\end{qsolve}
	
	
	
	\item اگر آدرس منطقی \texttt{0X45F3A} تولید شود شماره صفحه (\lr{Page Number}) و \lr{offset} داخل صفحه (\lr{Page Offset}) را محاسبه کنید.
	\begin{qsolve}
		
		آدرس منطقی \(\texttt{0X45F3A}\) به صورت عددی در مبنای ۱۰ برابر است با:
		
		\[
		\texttt{0X45F3A} = 286522
		\]
		
		برای محاسبه شماره صفحه (\(\text{\lr{Page Number}}\)) و \(\text{\lr{Page Offset}}\):
		\begin{itemize}
			\item اندازه یک صفحه \( 8,192 \; \text{بایت} \) است.
			\item شماره صفحه برابر با تقسیم آدرس منطقی بر سایز صفحه (بدون باقی‌مانده) است:
			\[
			\text{\lr{Page Number}} = \lfloor \frac{\text{آدرس منطقی}}{\text{سایز صفحه}} \rfloor = \lfloor \frac{286522}{8192} \rfloor = 35
			\]
			\item \(\text{\lr{Page Offset}}\) برابر با باقی‌مانده تقسیم آدرس منطقی بر سایز صفحه است:
			\[
			\text{\lr{Page Offset}} = \text{آدرس منطقی} \mod \text{سایز صفحه} = 286522 \mod 8192 = 7994
			\]
		\end{itemize}
		
		بنابراین:
		\[
		\text{\lr{Page Number}} = \mathbf{35}, \quad \text{\lr{Page Offset}} = \mathbf{7994}
		\]
	\end{qsolve}
	
	
	
	
	\item اگر جدول صفحات به صورت زیر باشد، آدرس فیزیکی متناظر با آدرس منطقی \texttt{0X45F3A} را محاسبه کنید:



\begin{latin}
	\begin{itemize}
		\item \lr{Page 0 $\rightarrow$ Frame 7}
		\item \lr{Page 2 $\rightarrow$ Frame 3}
		\item \lr{Page 5 $\rightarrow$ Frame 11}
		\item \lr{Page 8 $\rightarrow$ Frame 6}
	\end{itemize}
\end{latin}

\begin{qsolve}
	
طبق جدول، صفحه منطقی ۳۵ را پیدا می‌کنیم. از آنجا که در جدول صفحات داده شده \lr{Page 35} مشخص نشده است، نتیجه می‌گیریم که این آدرس منطقی به حافظه فیزیکی نگاشت نشده و دسترسی به آن باعث \lr{Page Fault} می‌شود.
	
\end{qsolve}

\end{enumerate}
