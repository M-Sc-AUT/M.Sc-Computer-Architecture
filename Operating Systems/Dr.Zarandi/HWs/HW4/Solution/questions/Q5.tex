\section{سوال پنجم}

توضیح دهید در خروجی قطعه کد زیر چه تعداد \texttt{*} چاپ خواهد شد؟ همچنین درختواره آن را نیز رسم نمایید.


\begin{latin}
\begin{lstlisting}[caption=Code of Q3, label=cpp_code_example]
int main() 
{
	if (fork() || (!fork())) 
	{
		if (fork() && fork()) 
		{
			fork();
		}
	}
	while (wait(NULL) > 0);
	printf("* ");
	
	return 0;
}
\end{lstlisting}
\end{latin}


\begin{qsolve}
با توجه به درختواره زیر و جایگاه تابع \texttt{printf()} در کد، ۹ عدد * چاپ خواهد شد و درختواره آن نیز به‌صورت زیر است:


\begin{center}
	\includegraphics*[width=1\linewidth]{pics/img7.jpg}
	\captionof{figure}{درختواره این سوال}
\end{center}

\end{qsolve}


\begin{qsolve}
	در این کد، ۱ پردازه والد و ۱۵ پردازه فرزند داریم.
	
	در شرط اول، ۲ پردازه با نام‌های $f_1 $ و $f_2 $ ایجاد می‌شود. در شرط دوم، پردازه $f_4 $ ایجاد می‌شود اما پردازه $f_5 $ فقط در پردازه‌های والد تولید می‌شود تا شرط حلقه ارضا شود و وارد آن شویم و درنهایت هم پردازه  $f_5 $ هم برای پردازه‌های والد قبلی خود ایجاد می‌شود.
\end{qsolve}