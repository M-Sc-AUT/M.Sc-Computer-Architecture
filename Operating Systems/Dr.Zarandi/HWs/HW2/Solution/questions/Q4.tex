\section{سوال چهارم}


نحوه عملکرد سیستم‌های چند پردازنده و سیستم‌های خوشه‌ای را توضیح دهید و آن‌ها را با یکدیگر مقایسه کنید. انواع دسته بندی آن‌ها را نیز نام ببرید.


\begin{qsolve}
	
	\begin{enumerate}
		\item 
		سیستم‌های چند پردازنده (\lr{Multi Processor}): سیستم‌های چند پردازنده شامل چندین پردازنده در یک سیستم واحد هستند که به طور همزمان به پردازش داده‌ها می‌پردازند. این پردازنده‌ها از طریق یک حافظه مشترک و یک سیستم‌عامل واحد با هم در ارتباط هستند. هدف اصلی سیستم‌های چند پردازنده افزایش کارایی سیستم است، به طوری که پردازش‌ها بین پردازنده‌ها تقسیم می‌شود و به این ترتیب سرعت اجرای برنامه‌ها افزایش می‌یابد.
		
		در این نوع سیستم‌ها، همزمانی پردازش‌ها و مدیریت منابع به وسیله یک سیستم‌عامل واحد انجام می‌شود. از مزایای این سیستم می‌توان به بهبود کارایی، دسترس‌پذیری بالا و تحمل خطا (\lr{Fault Tolerance}) اشاره کرد، چرا که در صورت خرابی یک پردازنده، پردازنده‌های دیگر می‌توانند وظایف آن را بر عهده بگیرند. 
		
		سیستم‌های چند پردازنده به دو دسته سیستم‌های چند پردازنده متقارن و غیر متقارن تقسیم می‌شوند. در سیستم‌های متقارن هر پردازنده تمام وظایف را انجام می‌دهد اما در سیستم‌های نامتقارن به هر پردازنده یک وظیفه مشخص اختصاص داده شده است. بلوک‌دیاگرام یک سیستم چند پردازنده متقارن در زیر آورده شده است.
		
		\begin{center}
			\includegraphics*[width=0.7\linewidth]{pics/img4.png}
			\captionof{figure}{سیستم‌های چندپردازنده متقارن}
		\end{center}
		
		
		
		\item
		سیستم‌های خوشه‌ای (\lr{Clustered Systems}): سیستم‌های خوشه‌ای شامل مجموعه‌ای از چندین کامپیوتر مستقل هستند که به هم متصل شده‌اند و به عنوان یک سیستم واحد عمل می‌کنند. هر سیستم در این خوشه دارای پردازنده‌ها و حافظه مستقل است و با استفاده از شبکه به سیستم‌های دیگر متصل می‌شود. این نوع سیستم‌ها اغلب برای افزایش مقیاس‌پذیری (\lr{Scalability})، دسترس‌پذیری بالا (\lr{High Availability}) و قدرت محاسباتی استفاده می‌شوند.		
	\end{enumerate}
\end{qsolve}

\begin{qsolve}
	\begin{enumerate}
		\item []
		
		یکی از ویژگی‌های بارز سیستم‌های خوشه‌ای این است که اگر یکی از سیستم‌ها از کار بیفتد، سایر سیستم‌ها می‌توانند همچنان به فعالیت خود ادامه دهند، که این امر باعث افزایش تحمل خطا و دسترس‌پذیری می‌شود. سیستم‌های خوشه‌ای به طور گسترده در مراکز داده و محاسبات علمی استفاده می‌شوند. نمونه‌ای از بلوک دیاگرام سیستم‌های خوشه ای در شکل زیر آورده شده است:
		
		\begin{center}
			\includegraphics*[width=0.7\linewidth]{pics/img5.png}
			\captionof{figure}{سیستم‌های خوشه‌ای}
		\end{center}
		
		این نوع سیستم‌ها نیز به دو دسته متقارن و نامتقارن تقسیم می‌شوند. در سیستم‌های متقارن همه سیستم‌ها همزمان باهم کار می‌کنند و همدیگر را چک می‌کنند اما در سیستم‌های نامتقارن یک سیستم آماده به‌کار دیگر وجود دارد که درصورت خرابی یک سیستم، جاگزین آن بشود.
	\end{enumerate}
	
\end{qsolve}

