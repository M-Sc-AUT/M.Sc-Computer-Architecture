\section{سوال ششم}

	فرض کنید دو برنامه A و B در یک سیستم در حال اجرا هستند. به طور کلی هر نوع فعالیت مرتبط با حافظه ۳۰ میکروثانیه، اجرای ۶۰ دستورالعمل ۴ میکروثانیه و ۲۵ دستورالعمل ۲ میکروثانیه زمان می‌برد. بهره‌وری پردازنده هنگامی که سیستم قابلیت تک برنامه‌ای و چند برنامه‌ای دارد را محاسبه کنید و دیاگرام وضعیت پردازنده در واحد زمان را برای حالت چند برنامه‌ای رسم کنید.

\begin{latin}
	\begin{enumerate}
		\item [A:] 
		Read a record from file \\
		Executing 60 instructions \\
		Write a record to file
		
		\item [B:]
		Read a record from file \\
		Executing 25 instructions \\
		Write a record to file
	\end{enumerate}
\end{latin}

\begin{qsolve}
	\begin{enumerate}
		\item 
		\textbf{تک برنامه:}\\
		
		\begin{enumerate}
			\item 
			 مجموع زمان اجرای برنامه :A\\
			 $$ (2\times 30^{\mu s}) + 4^{\mu s}=64^{\mu s} $$
			 
			 
			 
			 \item
			 مجموع زمان اجرای برنامه :B\\
			 $$ (2\times 30^{\mu s}) + 2^{\mu s}=62^{\mu s} $$
		\end{enumerate}
		
		$$ \rightarrow \text{\lr{CPU Utilization}} = \frac{\text{\lr{Total Execution Time}}}{\sum \text{\lr{Time}}}=\frac{4 + 2}{64 + 62}=\frac{6}{126}=0.0476\approx 4.76 \% $$
		
		
		\item 
		\textbf{چند برنامه:}\\
		در حالت چند برنامه، می‌توان در زمان‌های خالی \lr{CPU} برنامه B را اجرا کرد. دیاگرام وضعیت پردازنده به صورت زیر می‌شود:
		
		
		\begin{center}
			\includegraphics*[width=1\linewidth]{pics/Q6.pdf}
			\captionof{figure}{دیاگرام زمان‌بندی CPU}
		\end{center}
		
		
	\end{enumerate}
\end{qsolve}


\begin{qsolve}
	\begin{enumerate}
		\item [ ]
		بنابراین می‌توان از \lr{CPU} به‌صورت بهینه استفاده نمود. و بهره‌وری \lr{CPU} افزایش می‌یابد:
		
		$$ \text{\lr{CPU Utilization}} = \frac{6}{98}=0.061\approx 6.1 \% $$
		
	\end{enumerate}
\end{qsolve}