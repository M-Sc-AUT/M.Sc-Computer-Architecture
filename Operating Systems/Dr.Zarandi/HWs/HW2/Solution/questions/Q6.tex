\section{سوال ششم}

	فرض کنید دو برنامه A و B در یک سیستم در حال اجرا هستند. به طور کلی هر نوع فعالیت مرتبط با حافظه ۳۰ میکروثانیه، اجرای ۶۰ دستورالعمل ۴ میکروثانیه و ۲۵ دستورالعمل ۲ میکروثانیه زمان می‌برد. بهره‌وری پردازنده هنگامی که سیستم قابلیت تک برنامه‌ای و چند برنامه‌ای دارد را محاسبه کنید و دیاگرام وضعیت پردازنده در واحد زمان را برای حالت چند برنامه‌ای رسم کنید.

\begin{latin}
	\begin{enumerate}
		\item [A:] 
		Read a record from file \\
		Executing 60 instructions \\
		Write a record to file
		
		\item [B:]
		Read a record from file \\
		Executing 25 instructions \\
		Write a record to file
	\end{enumerate}
\end{latin}

\begin{qsolve}
	\begin{enumerate}
		\item 
		\textbf{تک برنامه:}\\
		
		\begin{enumerate}
			\item 
			 مجموع زمان اجرای برنامه :A\\
			 $$ (2\times 30^{\mu s}) + 4^{\mu s}=64^{\mu s} $$
			 
			 
			 
			 \item
			 مجموع زمان اجرای برنامه :B\\
			 $$ (2\times 30^{\mu s}) + 2^{\mu s}=62^{\mu s} $$
		\end{enumerate}
		
		$$ \rightarrow \text{\lr{CPU Utilization}} = \frac{\text{\lr{Total Execution Time}}}{\sum \text{\lr{Time}}}=\frac{4 + 2}{64 + 62}=\frac{6}{126}=0.0476\approx 4.76 \% $$
		
		
		\item 
		\textbf{چند برنامه:}\\
		در حالت چند برنامه، می‌توان در زمان‌های خالی \lr{CPU} برنامه B را اجرا کرد. دیاگرام وضعیت پردازنده به صورت زیر می‌شود:
		
		
		\begin{center}
			\includegraphics*[width=1\linewidth]{pics/Q6.pdf}
			\captionof{figure}{دیاگرام زمان‌بندی CPU}
		\end{center}
		
		
	\end{enumerate}
\end{qsolve}


\begin{qsolve}
	\begin{enumerate}
		\item [ ]
		بنابراین می‌توان از \lr{CPU} به‌صورت بهینه استفاده نمود و در زمان‌های بیکاری سیستم (فاصله بین اتمام برنامه A تا شروع دوباره آن) می‌توان برنامه B را اجرا نمود. اما به دلیل آنکه در صورت سوال دوره تناوب اجرای هر برنامه مشخص نیست، مقدار ماکزیمم زمان اجرا را برای هر دو برنامه درنظر می‌گیریم و بهره‌وری نسبت به اجرای تک برنامه تغییری نمی‌کند! اما اگر تعداد اجرای هر برنامه در یک تناوب مشخص بود، اثبات می‌شد که با اجرای برنامه‌ها به پشت سر هم در زمان‌های بیکاری CPU بهره‌وری افزایش می‌یابد.
		
		$$ \text{\lr{CPU Utilization}} = \frac{6}{126}=0.0476\approx 4.76 \% $$
		
		
		
		همچنین می‌توان اجرای دو برنامه را به‌صورت پایپلاین فرض نمود یعنی ابتدا هر برنامه عملیات I/O خود را انجام می‌دهد (خواندن و نوشتن)، سپس پردازنده دستورالعمل‌ها را اجرا می‌کند. پردازنده به تناوب بین برنامه A و B جابجا می‌شود. دیاگرام این روش به‌صورت زیر می‌شود:
		
		\[
		\begin{array}{|c|c|c|c|c|c|}
			\hline
			0-30 & 30-60 & 30-34 & 60-90 & 90-92 & 92-122 \\
			\text{A: Read} & \text{B: Read} & \text{A: Execute} & \text{A: Write} & \text{B: \text{Write}} & \text{B: Write} \\
			\hline
		\end{array}
		\]
		
		و درنهایت بهره‌وری CPU به‌صورت زیر محاسبه می‌شود:
		
		$$ \text{\lr{CPU Utilization}} = \frac{6}{122}=0.049\approx 4.9 \% $$
		
		
		
	\end{enumerate}
\end{qsolve}