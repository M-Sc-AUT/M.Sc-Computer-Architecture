\section{سوال دوم}
دو حالت اصلی\footnote{Mode} عملیات‌ها در سیستم‌عامل را نام برده و هرکدام را به‌صورت مختصر توضیح دهید.

\begin{qsolve}
	در \lr{OS} دو حالت اصلی عملیات وجود دارد که به آن‌ها \lr{User Mode} و \lr{Kernel Mode} گفته می‌شود. در ادامه به توضیح مختصری از وظایف هریک می‌پردازیم:
	
	\begin{enumerate}
		\item 
		\lr{:User Mode}
		در این حالت، برنامه‌های کاربردی یا نرم‌افزارهایی که توسط کاربر اجرا می‌شوند، عمل می‌کنند. در حالت کاربر، دسترسی مستقیم به منابع حیاتی سیستم مانند سخت‌افزار، حافظه یا تجهیزات ورودی/خروجی وجود ندارد. اگر برنامه‌ای در این حالت نیاز به دسترسی به منابع سیستم داشته باشد، باید از طریق \lr{System Call} ها به حالت کرنل درخواست بدهد. این محدودیت‌ها برای جلوگیری از دسترسی مستقیم برنامه‌ها به سخت‌افزار و حفاظت از امنیت سیستم اعمال می‌شود.
		
		\item 
		\lr{:Kernel Mode}
		در این حالت، سیستم‌عامل به منابع حیاتی و مستقیم سخت‌افزار دسترسی کامل دارد و می‌تواند هرگونه عملیات لازم را اجرا کند. این حالت برای انجام وظایف مهم سیستم‌عامل مثل مدیریت حافظه، مدیریت پردازش‌ها، و کنترل سخت‌افزار استفاده می‌شود. در حالت کرنل، هیچ محدودیتی برای دسترسی به منابع وجود ندارد و دسترسی کامل به حافظه و دستگاه‌ها امکان‌پذیر است.
	\end{enumerate}
	
\end{qsolve}