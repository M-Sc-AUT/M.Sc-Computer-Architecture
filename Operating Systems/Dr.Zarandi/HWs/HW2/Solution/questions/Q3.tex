\section{سوال سوم}

نحوه عملکرد \lr{DMA} را توضیح دهید. نحوه همکاری \lr{DMA} و پردازنده به چه‌صورت است؟ پردازنده به چه‌صورت از به پایان رسیدن کار \lr{DMA} مطلع می‌شود؟


\begin{qsolve}
	وقتی قرار است حجم زیادی از داده‌ها منتقل شود، از تکنیکی به‌نام \lr{Direct Memory Access} یا به‌طور خلاصه \lr{DMA} استفاده می‌شود. این تکنیک به این صورت عمل می‌کند که وقتی پردازنده می‌خواهد یک بلوک داده را بخواند یا بنویسد، دستوری را به ماژول \lr{DMA} ارسال می‌کند که شامل اطلاعات زیر است:
	\begin{enumerate}
		\item 
		اینکه آیا خواندن یا نوشتن درخواست شده است 
		
		\item 
		آدرس دستگاه \lr{I/O} مرتبط 
		
		\item 
		مکان شروع در حافظه برای خواندن داده‌ها یا نوشتن داده‌ها
		
		\item 
		تعداد کلماتی که باید خوانده یا نوشته شوند
		
	\end{enumerate}
	سپس پردازنده به کارهای دیگری ادامه می‌دهد. این عملیات \lr{I/O} به ماژول \lr{DMA} واگذار شده و آن ماژول آن را مدیریت خواهد کرد. ماژول \lr{DMA} کل بلوک داده را به‌طور مستقیم به حافظه منتقل می‌کند بدون اینکه ابتدا آن را به پردازنده بدهد. وقتی انتقال کامل شد، ماژول \lr{DMA} یک سیگنال وقفه به پردازنده ارسال می‌کند که پردازنده را متوجه پایان عملیات کند. بنابراین، پردازنده تنها در آغاز و پایان انتقال درگیر می‌شود.
	
\end{qsolve}
