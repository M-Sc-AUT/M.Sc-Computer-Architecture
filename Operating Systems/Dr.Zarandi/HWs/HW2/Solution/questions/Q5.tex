\section{سوال پنجم}


موارد زیر و نحوه انجام آن‌ها در سیستم عامل را توضیح دهید.
\begin{enumerate}
	\item 
	مدیریت زمان
	\begin{qsolve}
		مدیریت زمان در سیستم عامل به تخصیص منابع پردازشی به پردازه‌ها و وظایف مختلف به‌صورت بهینه مربوط می‌شود. سیستم عامل باید مطمئن شود که همه پردازه‌ها به طور منصفانه به پردازنده دسترسی داشته باشند و منابع به درستی بین پردازه‌های مختلف تقسیم شوند. مدیریت زمان شامل مفاهیم زیر است:
		
		\begin{enumerate}
			\item 
			\lr{:CPU Scheduling}
			سیستم عامل با استفاده از الگوریتم‌های مختلف، پردازنده را به پردازه‌ها تخصیص می‌دهد. الگوریتم‌هایی مانند \lr{FIFO} و الگوریتم \lr{Round Robin} و ... برای برنامه‌ریزی پردازه‌ها استفاده می‌شوند.
			
			\item 
			\lr{:Preemptive Scheduling}
			در این نوع برنامه‌ریزی، سیستم عامل می‌تواند پردازه در حال اجرا را متوقف کند و به پردازه دیگری اجازه اجرا دهد، تا اطمینان حاصل شود که پردازنده بین پردازه‌ها به صورت عادلانه تقسیم می‌شود.
			
			\item 
			مدیریت تایمرها: سیستم عامل از تایمرها برای اجرای کارهایی در بازه‌های زمانی مشخص، مثل مدیریت دوره‌ای وظایف، استفاده می‌کند.
		\end{enumerate}
	\end{qsolve}
	
	
	\item 
	مدیریت پردازه‌ها
	\begin{qsolve}
		مدیریت پردازه‌ها یکی از وظایف اصلی سیستم عامل است. سیستم عامل باید پردازه‌ها را ایجاد، مدیریت و خاتمه دهد و منابع لازم را به آن‌ها تخصیص دهد. مراحل اصلی در مدیریت پردازه‌ها عبارت‌اند از:
		\begin{enumerate}
			\item 
			ایجاد پردازه: زمانی که یک برنامه جدید اجرا می‌شود، سیستم عامل یک پردازه برای آن ایجاد می‌کند و منابع لازم (مانند حافظه و دسترسی به ورودی/خروجی) را به آن تخصیص می‌دهد.
			
			\item 
			حالت‌های پردازه: هر پردازه می‌تواند در یکی از حالت‌های \lr{Running} ، \lr{Ready} و یا \lr{Blocked} باشد. سیستم عامل پردازه‌ها را بین این حالت‌ها مدیریت می‌کند.
			
			\item \lr{:Context Switching}
			زمانی که سیستم عامل باید از یک پردازه به پردازه دیگری تغییر کند، باید وضعیت پردازه فعلی (شامل ثبات‌ها و شمارنده برنامه) را ذخیره کند و سپس وضعیت پردازه جدید را بازیابی کند. این عملیات به نام تعویض زمینه شناخته می‌شود.
			
			\item \lr{:Process Termination}
			پس از پایان کار پردازه، سیستم عامل آن را از حافظه خارج کرده و منابع تخصیص داده شده به آن را آزاد می‌کند.
		\end{enumerate}
	\end{qsolve}
	
	
	\item 
	مدیریت حافظه
	\begin{qsolve}
		مدیریت حافظه در سیستم عامل به نحوه تخصیص و آزادسازی حافظه به پردازه‌ها و برنامه‌های مختلف مربوط 
	\end{qsolve}
	
	
	\begin{qsolve}
		می‌شود. سیستم عامل باید حافظه را به گونه‌ای مدیریت کند که پردازه‌ها بتوانند به داده‌های خود دسترسی داشته باشند و از منابع به صورت بهینه استفاده شود. مراحل اصلی مدیریت حافظه شامل موارد زیر است:
		
		\begin{enumerate}
			\item 
			تخصیص حافظه: سیستم عامل باید حافظه لازم برای اجرای پردازه‌ها را به آن‌ها تخصیص دهد. این تخصیص می‌تواند به صورت تخصیص پیوسته (\lr{Contiguous Allocation)} یا تخصیص غیرپیوسته \lr{(Non-Contiguous Allocation)} انجام شود.
			
			\item 
			حافظه مجازی: سیستم عامل با استفاده از حافظه مجازی به پردازه‌ها اجازه می‌دهد تا حافظه بیشتری از آنچه که به صورت فیزیکی در دسترس است، استفاده کنند. حافظه مجازی با تکنیک‌هایی مثل تبادل \lr{Paging} یا \lr{Segmentation} پیاده‌سازی می‌شود. این تکنیک‌ها به سیستم اجازه می‌دهند تا حافظه پردازه‌ها را به قطعات کوچک‌تر تقسیم کند و آن‌ها را بین حافظه فیزیکی و دیسک جابه‌جا کند.
			
			\item 
			مدیریت فضای خالی: سیستم عامل باید از حافظه بهینه استفاده کند و فضاهای خالی حافظه را به طور مؤثر مدیریت کند. الگوریتم‌هایی مثل \lr{First Fit}، \lr{Best Fit} و \lr{Worst Fit} برای تخصیص و مدیریت فضای خالی حافظه استفاده می‌شوند.
			
			\item 
			حفاظت و دسترسی به حافظه: سیستم عامل با استفاده از تکنیک‌هایی مثل \lr{(Page Table)} و \lr{(Protection Registers)} از دسترسی غیرمجاز به بخش‌های مختلف حافظه جلوگیری می‌کند. این کار برای جلوگیری از تداخل پردازه‌ها با یکدیگر و حفظ امنیت سیستم ضروری است.
		\end{enumerate}
	\end{qsolve}
	
	
\end{enumerate}






