\section{سوال اول}

به سوالات زیر در مورد وقفه‌ها پاسخ دهید.

\begin{enumerate}
	\item 
	وقفه چیست؟ کلاس‌های مختلف وقفه را به صورت مختصر توضیح دهید.
	\begin{qsolve}
		وقفه، به رویداد یا سیگنالی گفته می‌شود که پردازنده را از یک کار جاری متوقف کرده و به پردازش یک کار فوری‌تر یا مهم‌تر هدایت می‌کند. این وقفه می‌تواند توسط سخت‌افزار یا نرم‌افزار ایجاد شود و به منظور مدیریت و پاسخگویی به رویدادهای مختلف در سیستم عامل استفاده می‌شود. به‌طور کلی می‌توان وقفه‌ها را به ۴ کلاس دسته‌بندی نمود:
		
		\begin{enumerate}
			\item 
			برنامه: ایجاد شده توسط شرایطی که در نتیجه اجرای یک دستور رخ می‌دهد، مانند سرریز محاسباتی، تقسیم بر صفر، تلاش برای اجرای یک دستور العمل غیرمجاز، یا ارجاع به خارج از فضای مجاز حافظه کاربر.
			 
			 \item 
			 تایمر: ایجاد شده توسط یک تایمر درون پردازنده. این وقفه به سیستم عامل امکان می‌دهد تا وظایف خاصی را به‌صورت منظم انجام دهد.
			 
			 \item 
			 ورودی/خروجی: توسط کنترل‌کننده ورودی/خروجی ایجاد می‌شود، برای اعلام اتمام عادی یک عملیات یا اعلام انواع مختلفی از شرایط خطا.
			 
		\end{enumerate}
		
	\end{qsolve}
	
	
	\item 
	به هنگام وقوع وقفه، پردازنده چه اطلاعاتی را در پشته ذخیره می‌کند؟ دلیل استفاده از پشته چیست؟ 
	\begin{qsolve}
		هنگام وقوع وقفه، پردازنده باید اجرای برنامه فعلی را متوقف کرده و به پردازش رویداد وقفه بپردازد. برای اینکه بتواند پس از اتمام وقفه به اجرای برنامه اصلی بازگردد، نیاز دارد اطلاعاتی را که مربوط به وضعیت فعلی اجرای برنامه است، ذخیره کند. این اطلاعات شامل:
		\begin{enumerate}
			\item 
			\lr{:Program Counter}
			برای اینکه بدانیم از کجای برنامه اصلی به \lr{ISR}  جامپ زدیمم
			
			\item 
			\lr{:General Purpose Registers}
			برای اینکه مقادیر موجود در ثبات‌های عمومی که ممکن است در حین پردازش وقفه تغییر کنند، در پشته ذخیره می‌شوند تا بعد از اتمام وقفه، مقادیر اصلی بازیابی شوند.
			
			\item
			\lr{:Processor Status Word}
			که شامل اطلاعاتی درمورد پردازنده مانند \lr{Flag} هاست.
			
			\item
			\lr{:Stack Pointer}
			همچنین نیاز است آدرس آخرین مقدار ذخیره‌شده در پشته نیز ذخیره شود.
			
		\end{enumerate}
	
	\end{qsolve}
	
	
	\begin{qsolve}
		پشته یکی از ساده‌ترین و مؤثرترین ساختارهای داده برای مدیریت اطلاعات در زمان وقوع وقفه است. دلیل اصلی استفاده از پشته این است که آخرین دستور یا داده‌ای که ذخیره می‌شود، اولین داده‌ای است که باید بازیابی شود (\lr{Last In, First Out - LIFO}). این ویژگی پشته باعث می‌شود پردازنده بتواند وضعیت برنامه را به درستی ذخیره کرده و پس از پایان وقفه به همان وضعیت بازگردد. استفاده از پشته همچنین امکان مدیریت خودکار و مرتب اطلاعات بدون نیاز به تخصیص دستی حافظه را فراهم می‌کند.
		
		در ادامه تصویری از فرایند رسیدگی به وقفه‌ها از کتاب \lr{Stallings} آورده شده است:
		
		\begin{center}
			\includegraphics*[width=0.8\linewidth]{pics/img1.png}
			\captionof{figure}{فرآیند رسیدگی به وقفه}
		\end{center}
		
	\end{qsolve}
	
	
	\item 
	رویکردهای استفاده شده برای رسیدگی به وقفه‌های متعدد را بیان کنید و آنها را به صورت مختصر توضیح دهید.
	\begin{qsolve}
		مطابق با کتاب آقای \lr{Stallings} دو رویکرد برای مقابله با وقفه‌های متعدد وجود دارد.
		
		\begin{enumerate}
			\item 
			اولین روش این است که در هنگام پردازش یک وقفه، تمامی وقفه‌های دیگر غیرفعال شوند. (وقفه غیرفعال به این معنی است که پردازنده هر سیگنال جدید درخواست وقفه را نادیده می‌گیرد). اگر در این مدت یک وقفه رخ دهد، معمولاً در حالت معلق باقی می‌ماند و بعد از اینکه پردازنده وقفه‌ها را دوباره فعال کرد، بررسی می‌شود. بنابراین، اگر وقفه‌ای هنگام اجرای برنامه‌ی کاربر رخ دهد، بلافاصله وقفه‌ها غیرفعال می‌شوند. پس از تکمیل \lr{ISR} وقفه ایجاد شده، وقفه‌ها دوباره فعال می‌شوند و قبل از ادامه‌ی برنامه‌ی کاربر، پردازنده بررسی می‌کند که آیا وقفه‌های اضافی رخ داده‌اند یا خیر. (شکل زیر)
			
			\begin{center}
				\includegraphics*[width=0.6\linewidth]{pics/img2.png}
				\captionof{figure}{رویکرد اول رسیدگی به وقفه‌ها}
			\end{center}
			
			\item 
			روش دوم این است که برای وقفه‌ها اولویت‌هایی تعریف شود و به یک وقفه با اولویت بالاتر اجازه داده شود تا روتین مدیریت وقفه با اولویت پایین‌تر را قطع کند. (شکل زیر)
			
			\begin{center}
				\includegraphics*[width=0.6\linewidth]{pics/img3.png}
				\captionof{figure}{رویکرد دوم رسیدگی به وقفه‌ها}
			\end{center}
			
		\end{enumerate}
	\end{qsolve}
	
	
	\item 
	پدیده سرریز پشته چه زمانی در وقفه‌های تو در تو رخ می‌دهد و برای حل این مشکل چه تدبیری اندیشیده شده است؟
	\begin{qsolve}
		در وقفه‌های تو در تو، پدیده \lr{Stack Overflow} زمانی رخ می‌دهد که اطلاعات بیشتری نسبت به ظرفیت پشته در آن ذخیره شود. همانطور که در قسمت قبل بحث شد، در وقفه‌های تو در تو، ممکن است یک وقفه جدید در حین پردازش وقفه قبلی رخ دهد، و پردازنده مجبور می‌شود وضعیت جاری خود را (شامل ثبات‌ها و اطلاعات اجرایی) در پشته ذخیره کند. اگر تعداد وقفه‌ها به حدی زیاد شود که ظرفیت پشته پر شود، سرریز پشته اتفاق می‌افتد و این باعث از بین رفتن اطلاعات و خرابی سیستم می‌شود. دو مورد از تدابیر اندیشیده شده برای جلوگیری از این مشکل در قسمت قبل بیان شد. یعنی غیر‌فعال کردن وقفه‌های جدید به‌هنگام انجام \lr{ISR} یک وقفه فعال و اولویت بندی وقفه‌ها. دو راهکار دیگر هم می‌توان پیشنهاد داد که یکی افزایش اندازه پشته است (البته درصورتی که بتوان این کار را انجام داد) و اگر نمی‌توانستیم اندازه پشته را تغییر دهیم می‌توان از یک حافظه جایگذین یا یک پشته مجزا در کنار پشته اصلی استفاده نمود.
	\end{qsolve}
	
\end{enumerate}
