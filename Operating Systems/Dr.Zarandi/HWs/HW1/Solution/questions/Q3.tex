\section{سوال سوم}

میدانیم گاهی اوقات \lr{CPU} در وضعیت \lr{HALT} قرار می‌گیرد. این وضعیت را توضیح دهید و شرح دهید در چه مواردی \lr{CPU} در آن قرار می‌گیرد.


\begin{qsolve}[]
	وقتی \lr{CPU} در حالت \lr{HALT} قرار می‌گیرد، تمام فرآیندهای عادی متوقف می‌شوند و پردازنده به صورت نیمه‌فعال باقی می‌ماند. (پردازنده کلاک میخورد اما الگوریتم فن‌نیومن اجرا نمی‌شود) در این حالت، \lr{CPU} هیچ دستوری از برنامه‌های در حال اجرا را پردازش نمی‌کند و به جای آن، در یک حالت انتظار کم‌مصرف قرار می‌گیرد. این وضعیت به معنی خاموش شدن کامل \lr{CPU} نیست، بلکه پردازنده به نوعی در حالت انتظار قرار دارد تا زمانی که یک رویداد خاص، مانند یک \lr{Interrupt} یا \lr{reset} رخ دهد و آن را از حالت \lr{HALT} خارج کند.
	
	\begin{itemize}
		\item 
		 یکی از رایج‌ترین مواردی که \lr{CPU} در وضعیت \lr{HALT} قرار می‌گیرد، زمانی است که سیستم منتظر یک \lr{Interrupt} است. به عنوان مثال، وقتی پردازنده منتظر دریافت داده از دستگاه‌های \lr{I/O} است (مانند انتظار برای فشردن کلید در کیبورد)، می‌تواند به جای اجرای دستورات بیهوده، وارد حالت \lr{HALT} شود. سپس با وقوع یک وقفه (مثلاً فشار دادن کلید روی کیبورد)، دازنده از حالت \lr{HALT} خارج شده و پردازش را ادامه می‌دهد.
		 
		 \item 
		معمولا در زمان‌هایی که سیستم نیاز به پردازش ندارد یا بیکار است، پردازنده می‌تواند به حالت \lr{HALT} برود تا مصرف انرژی را کاهش دهد. این روش به ویژه در دستگاه‌های قابل حمل مانند لپ‌تاپ‌ها و گوشی‌های هوشمند که مصرف انرژی اهمیت زیادی دارد، کاربرد دارد.
		
		\item 
		 در سیستم‌های چند هسته‌ای، ممکن است یک یا چند هسته‌ی پردازنده در حالتی قرار بگیرند که نیاز به پردازش نداشته باشند. در چنین مواقعی، این هسته‌ها به حالت \lr{HALT} می‌روند تا مصرف انرژی بهینه‌تر شود و تنها در صورتی که نیاز به پردازش مجدد داشته باشند، فعال می‌شوند.
	\end{itemize}
\end{qsolve}