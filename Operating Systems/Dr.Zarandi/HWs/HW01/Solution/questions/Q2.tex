\section{سوال دوم}

تصور کنید یک کامپیوتری دارای چندین دستگاه ورودی/خروجی \lr{(I/O)} مانند کیبورد و اسکنر است. میدانیم که این دستگاه‌ها برای ارسال و دریافت اطلاعات از CPU نیاز به مدیریت دارند. حال توضیح دهید زمانی که کاربر کلیدی را بر روی کیبورد فشار می‌دهد و یا می‌خواهد عکسی با حجم کم را اسکن کند، چه فرآیند و مراحلی میان دستگاه \lr{I/O}، \lr{Device Controller}، \lr{CPU} و \lr{Memory} طی می‌شود.


\begin{qsolve}[]
	وقتی کاربر کلیدی را بر روی کیبورد فشار می‌دهد یا عکسی را اسکن می‌کند، فرآیند تبادل داده بین دستگاه \lr{I/O}، کنترلر دستگاه، \lr{CPU} و حافظه طی مراحلی مشابه به هم انجام می‌شود. که این مراحل را در ادامه برای کیبورد و اسکنر به طور جداگانه توضیح خواهیم داد:
	
	\begin{enumerate}
		\item 
		\textbf{فشردن کلید کیبورد}
		
		\begin{enumerate}
			\item
			ارسال سیگنال از کیبورد به کنترلر دستگاه:\\
			وقتی کاربر کلیدی را روی کیبورد فشار می‌دهد، یک سیگنال الکتریکی به کنترلر کیبورد ارسال می‌شود. این سیگنال نشان‌دهنده‌ی کد آن کلید خاص است که به صورت یک کد اسکی توسط کیبورد تولید می‌شود.
			
			\item 
			ارسال وقفه به \lr{CPU}:\\
			کنترلر کیبورد سیگنال را دریافت کرده و یک وقفه به \lr{CPU} ارسال می‌کند. این وقفه به \lr{CPU} اطلاع می‌دهد که یک داده جدید از کیبورد برای پردازش وجود دارد.
			
			\item 
			سرویس‌دهی به وقفه:\\
			\lr{CPU}
			به وقفه پاسخ داده و اجرای فرآیند جاری را متوقف می‌کند. سپس \lr{CPU} به سراغ دستورالعمل‌های مدیریت وقفه (\lr{Interrupt Handler}) می‌رود. این دستورالعمل‌ها مشخص می‌کنند که وقفه از کیبورد است و باید داده‌ی کیبورد خوانده شود.
		\end{enumerate}
		
		
		
		\item 
		\textbf{اسکن عکس با اسکنر}
	\end{enumerate}
\end{qsolve}