\section{سوال دوم}

تصور کنید یک کامپیوتری دارای چندین دستگاه ورودی/خروجی \lr{(I/O)} مانند کیبورد و اسکنر است. میدانیم که این دستگاه‌ها برای ارسال و دریافت اطلاعات از CPU نیاز به مدیریت دارند. حال توضیح دهید زمانی که کاربر کلیدی را بر روی کیبورد فشار می‌دهد و یا می‌خواهد عکسی با حجم کم را اسکن کند، چه فرآیند و مراحلی میان دستگاه \lr{I/O}، \lr{Device Controller}، \lr{CPU} و \lr{Memory} طی می‌شود.


\begin{qsolve}[]
	وقتی کاربر کلیدی را بر روی کیبورد فشار می‌دهد یا عکسی را اسکن می‌کند، فرآیند تبادل داده بین دستگاه \lr{I/O}، کنترلر دستگاه، \lr{CPU} و حافظه طی مراحلی مشابه به هم انجام می‌شود. که این مراحل را در ادامه برای کیبورد و اسکنر به طور جداگانه توضیح خواهیم داد:
	
	\begin{enumerate}
		\item 
		\textbf{فشردن کلید کیبورد}
		
		\begin{enumerate}
			\item
			ارسال سیگنال از کیبورد به کنترلر دستگاه:\\
			وقتی کاربر کلیدی را روی کیبورد فشار می‌دهد، یک سیگنال الکتریکی به کنترلر کیبورد ارسال می‌شود. این سیگنال نشان‌دهنده‌ی کد آن کلید خاص است که به صورت یک کد اسکی توسط کیبورد تولید می‌شود.
			
			\item 
			ارسال وقفه به \lr{CPU}:\\
			کنترلر کیبورد سیگنال را دریافت کرده و یک وقفه به \lr{CPU} ارسال می‌کند. این وقفه به \lr{CPU} اطلاع می‌دهد که یک داده جدید از کیبورد برای پردازش وجود دارد.
			
			\item 
			سرویس‌دهی به وقفه:\\
			\lr{CPU}
			به وقفه پاسخ داده و اجرای فرآیند جاری را متوقف می‌کند. سپس \lr{CPU} به سراغ دستورالعمل‌های مدیریت وقفه (\lr{Interrupt Handler}) می‌رود. این دستورالعمل‌ها مشخص می‌کنند که وقفه از کیبورد است و باید داده‌ی کیبورد خوانده شود.
			
			\item 
			دریافت داده از کنترلر:\\
			\lr{CPU}
			به کنترلر کیبورد پیام می‌فرستد و داده‌ی مربوط به کلید فشرده شده را درخواست می‌کند. کنترلر کیبورد این داده را که معمولاً کد اسکی مربوط به کلید فشرده‌شده است، به CPU ارسال می‌کند.
			
			\item 
			ذخیره‌سازی داده در حافظه:\\
			\lr{CPU}
			داده دریافت‌شده را به حافظه \lr{RAM} منتقل می‌کند تا در صورت نیاز توسط برنامه‌های در حال اجرا (مثلاً یک نرم‌افزار پردازش متن) استفاده شود.
			
			\item 
			ادامه اجرای برنامه‌ها:\\
			پس از پردازش وقفه، \lr{CPU} به برنامه قبلی بازگشته و اجرای آن را از سر می‌گیرد.
			
			
		\end{enumerate}
		
		
		
		\item 
		\textbf{اسکن عکس با اسکنر}
		\begin{enumerate}
			\item 
			 ارسال سیگنال از اسکنر به کنترلر دستگاه:\\
			 وقتی کاربر درخواست اسکن یک عکس را می‌دهد، اسکنر شروع به جمع‌آوری داده‌های تصویری می‌کند. این داده‌ها به صورت پیکسل به پیکسل به کنترلر اسکنر ارسال می‌شوند. کنترلر دستگاه این داده‌ها را در بخش‌های کوچک بسته‌بندی می‌کند تا برای انتقال آماده باشند.
			 
			 \item 
			 ارسال وقفه به \lr{CPU}:\\
			 همانند کیبورد، کنترلر اسکنر نیز پس از جمع‌آوری بخشی از داده‌ها، یک وقفه به \lr{CPU} ارسال می‌کند تا به آن اطلاع دهد که داده‌های جدید برای پردازش آماده هستند.
			 
			 
			
		\end{enumerate}
	\end{enumerate}
\end{qsolve}


\begin{qsolve}
	\begin{enumerate}
		\item [ ]
		\begin{enumerate}
			\item [(ج)]
			سرویس‌دهی به وقفه:\\
			\lr{CPU}
			به وقفه اسکنر پاسخ داده و دستورالعمل‌های مربوط به مدیریت وقفه را اجرا می‌کند. \lr{CPU} سپس از کنترلر اسکنر درخواست داده می‌کند.
			
			\item [(د)]
			انتقال داده به حافظه:\\
			کنترلر اسکنر داده‌های تصویر را به \lr{CPU} ارسال می‌کند و \lr{CPU} این داده‌ها را به حافظه اصلی منتقل می‌کند. در اینجا ممکن است از \lr{DMA} استفاده شود تا حجم بالای داده‌های تصویر بدون نیاز به پردازنده مستقیماً به حافظه منتقل شود.
			
			\item [(ه)]
			ادامه اسکن و پردازش:\\
			اسکنر همچنان به جمع‌آوری داده‌های جدید ادامه می‌دهد و هر بار که یک بخش از داده‌ها آماده شد، یک وقفه دیگر به \lr{CPU} ارسال می‌شود تا داده‌های جدید به حافظه منتقل شوند. این فرآیند تا زمانی که اسکن کامل شود، تکرار می‌شود.
		\end{enumerate}
	\end{enumerate}
\end{qsolve}