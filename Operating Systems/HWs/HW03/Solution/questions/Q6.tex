\section{سوال ششم}

در این تمرین می‌خواهیم نحوه عملکرد Scheduler های لینوکس را مورد بررسی قرار دهیم. به‌طور کلی در لینوکس سه نوع Scheduler وجود دارد:
\begin{latin}
	\begin{enumerate}
		\item SCHED\_OTHER
		\item SCHED\_FIFO
		\item SCHED\_RR
	\end{enumerate}
\end{latin}

در زبان C می‌توانیم مشخص کنیم که با چه Scheduler ای برنامه‌مان اجرا شود. برای این تمرین نیاز داریم که دو پردازه داشته باشیم و آنها را با Scheduler های مختلف تست کنیم.

برای نوشتن این برنامه به کتابخانه های زیر نیاز داریم:
\begin{latin}
	\texttt{\#include <sched.h>}\\
	\texttt{\#include <stdio.h>}\\
	\texttt{\#include <sys/resource.h>}\\
	\texttt{\#include <unistd.h>}\\
\end{latin}

برای تغییر policy Scheduler برنامه، از دستور زیر استفاده می‌شود:
\begin{latin}
	\texttt{struct sched\_param param}\\
	\texttt{sched\_setscheduler(getpid(), SCHED\_FIFO, \&param)} \\
\end{latin}

ساختار sched\_param پارامتر های scheduler را مشخص می‌کند. برای تغییر اولویت پردازه از این استراکت استفاده می‌کنیم:
\begin{latin}
	\texttt{param.sched\_priority = PRIORITY\_NUM}\\
\end{latin}

نکته: ‫در‬ ‫اینجا‬ ‫هر‬ ‫چه‬ ‫مقدار‬ ‫بیشتر‬ ‫باشد‪،‬‬ ‫اولویت‬ ‫بالاتر‬ ‫است‪.‬‬ ‫و‬ ‫مقداری‬ ‫که‬ ‫اینجا‬ ‫تعیین‬ ‫میشود‬ ‫با‬ ‫مقداری‬ ‫که‬ ‫در‬ ستون Priority پردازه در ستون Top می‌بینید متفاوت است.

‫با‬‫استفاده‬ ‫از‬ ‫این‬ ‫دستورات‪،‬‬ ‫برنامه‬ ‫ای‬ ‫بنویسید‬ ‫و‬ ‫سناریو‬ ‫های‬ ‫زیر‬ ‫را‬ ‫برای‬ ‫دو‬ ‫پردازه‬ ‫همزمان‬ ‫تست‬ ‫کنید‬ ‫(برنامه‬ ‫ای‬ ‫بنویسید‬‫ که‬ ‫زمان‬ ‫زیادی‬ ‫داشته‬ ‫باشد‬ ‫که‬ ‫بتوانید‬ ‫خروجی‬ ‫ها‬ ‫را‬ ‫مقایسه‬ ‫کنید‪.‬‬ ‫برای‬ ‫نمونه‬ ‫می‬ ‫توانید‬ ‫از‬ ‫قطعه‬ ‫کد زیر استفاده کنید: )

\begin{latin}
\begin{lstlisting}
int n = 0;
while(1)
{
  n++;
  if(!(n % 10000000))
  {
  	printf("FIFO running (n=%d) \n", n)
  }
}		
\end{lstlisting}
\end{latin}

\textbf{نکته: }تغییر scheduler یک پردازه نیازمند دسترسی Root است بنابر این برنامه را با sudo اجرا کنید.

\textbf{نکته: }دوپردازه را باید روی یک cpu اجرا کنیم تا نتیجه مد نظر را ببینیم. برای اجرای یک پردازه روی یک core cpu از دستور زیر استفاده می‌کنیم:


\begin{latin}
\texttt{sudo taskset -c 0 ./FIFO.o}
\end{latin}

\textbf{سناریو ۱:‌}\\
پردازه اول:‌ SCHED\_FIFO و priority=1\\
پردازه دوم:‌ SCHED\_FIFO و priority=1\\

\textbf{سناریو ۲:}\\
پردازه اول:‌ SCHED\_RR و priority=1\\
پردازه دوم:‌ SCHED\_RR و priority=1\\

\textbf{سناریو ۳:}\\
پردازه اول:‌ SCHED\_FIFO و priority=1\\
پردازه دوم:‌ SCHED\_FIFO و priority=2\\



\begin{qsolve}
	برنامه نوشته شده به صورت زیر است:
	
\begin{latin}
\begin{lstlisting}[caption=code]
#include <sched.h>
#include <stdio.h>
#include <sys/resource.h>
#include <unistd.h>
#include <sys/wait.h>

int main() 
{
  struct sched_param param;
  param.sched_priority = 1;

  sched_setscheduler(getpid(), SCHED_FIFO, &param);

  pid_t pid = fork();

  if (pid == 0) 
  {
	int n = 0;
	while (1) 
	{
	  n++;
	  if (!(n % 10000000)) 
	  {
	    printf("Running (n=%d)\n", n);
	  }
	  else if (pid > 0) 
	  {
	  	wait(NULL);
	  } 
	  else 
	  {
	  	printf("Fork failed\n");
	  	return 1;
	  }
	  
	  return 0;
	  }
	}
} 
\end{lstlisting}
\end{latin}

سپس برنامه را با دستور زیر کامپایل و اجرا می‌کنیم:
\begin{latin}
\texttt{gcc -o scheduler\_program scheduler\_main.c}\\
\texttt{sudo taskset -c 0 ./scheduler\_program}
\end{latin}




\end{qsolve}








