\section{به سوالات زیر پاسخ دهید}



الف) وقفه چیست؟‌ وقفه‌های سنکرون و آسنکرون را باهم مقایسه کنید.
\begin{qsolve}
وقفه (Interrupt) در محیط برنامه‌نویسی به وقوع پیوستن یک رویداد ناگهانی در حین اجرای برنامه گفته می‌شود که عملکرد طبیعی برنامه را متوقف می‌کند و برنامه‌ای را به اجرای یک کد خاص یا روند دیگر تغییر می‌دهد. وقفه‌ها معمولاً توسط سخت‌افزار و سیستم عامل به منظور پاسخگویی به رویدادهای مهم مانند درخواست‌های I/O، خطاها، تایمرها و سایر رویدادها ایجاد می‌شوند.

از نظر زمان وقوع، وقفه‌ها به دو دسته سنکرون و آسنکرون تقسیم می‌شوند:
\begin{enumerate}
	\item وقفه‌های سنکرون: 
	\begin{enumerate}
		\item وقوع وقفه در زمانی قرار دارد که برنامه در یک نقطه مشخص خود را در حالت انتظار قرار می‌دهد و منتظر وقوع وقفه است.
		\item برنامه‌ای که با وقوع وقفه مواجه می‌شود، به طور مستقیم و بلافاصله وارد روند وقفه می‌شود و ادامه اجرای برنامه بعد از پایان وقفه ادامه می‌یابد.
		\item معمولاً وقفه‌های سنکرون توسط سخت‌افزار ایجاد می‌شوند، مانند درخواست‌های ورودی کاربر، تقاضای دستگاه‌های جانبی و غیره.
	\end{enumerate}
	\item وقفه‌های آسنکرون: 
	\begin{enumerate}
		\item وقوع وقفه در زمانی قرار دارد که برنامه در حال اجرا است و به طور غیرمنتظره با یک رویداد ناگهانی مواجه می‌شود.
		\item وقفه آسنکرون می‌تواند در هر نقطه‌ای از اجرای برنامه رخ دهد و برنامه را به وقفه‌هایی مانند خطاها، سیگنال‌های سیستم عامل، تقاضای دیگر برنامه‌ها و غیره وصل می‌کند.
		\item وقفه‌های آسنکرون برنامه را از جریان اصلی آن جدا کرده و به روند وقفه منتقل می‌کنند. بعد از پایان وقفه، برنامه از جایی که متوقف شده بود، ادامه می‌یابد.
		\item معمولاً وقفه‌های آسنکرون توسط سخت‌افزار (مانند خطاهای سخت‌افزاری) یا سیستم عامل (مانند سیگنال‌های سیستم عامل) ایجاد می‌شوند.
	\end{enumerate}
\end{enumerate}
\end{qsolve}
\newpage




ب) تفاوت‌های بین Interrupt و Trap را توضیح دهید.
\begin{qsolve}
وقفه و Trap هردو پدیده‌هایی در محیط برنامه‌نویسی هستند که به وقوع پیوستن رویدادهای ناگهانی در حین اجرای برنامه را مشخص می‌کنند. اما تفاوت‌هایی بین این دو وجود دارد:
\begin{enumerate}
	\item Interrupt
	\begin{enumerate}
		\item وقفه‌ها معمولاً توسط سخت‌افزار یا سیستم عامل ایجاد می‌شوند و می‌توانند در هر زمانی و در هر نقطه‌ای از اجرای برنامه رخ دهند.
		\item هدف اصلی وقفه‌ها، متوقف کردن عادی برنامه و پاسخگویی به رویدادهای مهم است. مثال‌هایی از وقفه‌ها شامل درخواست‌های ورودی کاربر، تقاضای دستگاه‌های جانبی، خطاها و تایمرها می‌شوند.
		\item وقفه‌ها معمولاً باعث تغییر جریان اجرای برنامه می‌شوند. برنامه به طور مستقیم و بلافاصله به یک روند وقفه منتقل می‌شود و پس از پایان وقفه، به جریان اصلی خود بازگشت می‌کند.
	\end{enumerate}
	\item Trap
	\begin{enumerate}
		\item ترپ‌ها معمولاً توسط خود برنامه نوشته شده و قابلیت اجرای آنها وجود دارد. معمولاً در نقاط خاصی از برنامه قرار داده می‌شوند تا در صورت بروز شرایط خاص، عملیات خاصی انجام دهند.
		\item هدف اصلی ترپ‌ها، نیاز به یک رفتار خاص در برنامه است و معمولاً برای انجام عملیات‌های خاص (مانند خطاها، استثناها و غیره) استفاده می‌شوند.
		\item ترپ‌ها برای تعامل با سیستم عامل یا سخت‌افزار می‌توانند استفاده شوند. به عنوان مثال، یک برنامه می‌تواند ترپی برای درخواست سیستم عامل برای اختصاص حافظه یا فایل‌ها داشته باشد.
		\item ترپ‌ها معمولاً توسط برنامه بررسی می‌شوند و در صورت بروز شرایط، برنامه به طور دستوری به روند ترپ منتقل می‌شود. پس از انجام عملیات ترپ، برنامه به جریان اصلی خود بازگشت می‌کند.
	\end{enumerate}
	به طور خلاصه، وقفه‌ها معمولاً توسط سخت‌افزار یا سیستم عامل ایجاد می‌شوند و هدف اصلی آنها پاسخگویی به رویدادهای مهم است. ترپ‌ها به طور کلی توسط برنامه نوشته شده و هدف اصلی آنها انجام عملیات خاص در برنامه است.
\end{enumerate}
\end{qsolve}
\newpage






پ) فرآیند مدیریت یک وقفه از لحظه ایجاد شدن تا اتمام آن را توضیح دهید. فرض کنید وقفه متعدد نداریم و CPU مشغول انجام برنامه کاربر است.
\begin{qsolve}
فرآیند مدیریت یک وقفه از لحظه ایجاد شدن تا اتمام آن عموماً توسط سیستم عامل و سخت‌افزار انجام می‌شود. در زیر، مراحل اصلی مدیریت یک وقفه را توضیح می‌دهم:
\begin{enumerate}
	\item \textbf{شناسایی وقفه:} سیستم عامل و سخت‌افزار در هر لحظه وقفه‌ها را بررسی می‌کنند. این بررسی ممکن است توسط سخت‌افزار (مانند تایمرها، درخواست‌های دستگاه‌های جانبی و ...) یا سیستم عامل (مانند درخواست‌های ورودی کاربر و ...) صورت گیرد.
	\item \textbf{ذخیره وضعیت فعلی:} سیستم عامل اطلاعات مربوط به وضعیت فعلی برنامه را در یک مکان مناسب ذخیره می‌کند. این کار از طریق استفاده از مکانیزمی به نام Switch Context انجام می‌شود.
	\item \textbf{اجرای روند وقفه:} پس از ذخیره وضعیت فعلی، سیستم عامل به Handler Interrupt منتقل می‌شود. روند وقفه کدی است که توسط سیستم عامل تعریف شده است و وظیفه پاسخگویی به وقفه را دارد. در این مرحله، عملیات مربوط به وقفه انجام می‌شود (مانند پاسخ به درخواست کاربر، خواندن داده از دستگاه جانبی و ...)
	\item \textbf{بازگشت به برنامه اصلی:} بعد از اتمام عملیات وقفه، سیستم عامل وضعیت قبلی را بازیابی می‌کند. این شامل بازگشت به وضعیت قبلی ثبات‌های CPU، شمارنده‌ها و سایر اطلاعات مربوطه است.
	\item \textbf{ادامه اجرای برنامه:} سیستم عامل بعد از بازگشت به وضعیت قبلی، اجرای برنامه را از جایی که قبل از وقفه متوقف شده بود، ادامه می‌دهد. در این مرحله، جریان اجرای برنامه به همان نقطه قبل از وقفه برمی‌گردد و برنامه از همان جایی که متوقف شده بود ادامه می‌یابد.
\end{enumerate}
این فرآیند مدیریت وقفه معمولاً به صورت خودکار توسط سیستم عامل و سخت‌افزار انجام می‌شود و برنامه نویس نیازی به دخالت مستقیم در این فرآیند ندارد.
	
\end{qsolve}