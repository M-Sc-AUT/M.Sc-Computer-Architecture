\section{سوال اول}

در یک سیستم صفحه‌بندی، table Page در حافظه اصلی قرار دارد.

\begin{enumerate}
	\item اگر مراجعه به حافظه ۵۰ نانو ثانیه زمان ببرد، چقدر طول می‌کشد که در قالب سیستم صفحه بندی به داده یا دستور مورد نظر خود دسترسی پیدا کنیم؟
	
	\item فرض کنید که TLB را نیز به سیستم اضافه می‌کنیم و پیدا کردن یک مدخل جدول صفحات در ۲ نانو ثانیه زمان می‌برد. اگر ۷۵ درصد از مراجعات جدول صفحات در TLB نیز یافت شود، زمان موثر دسترسی چقدر خواهد بود؟
\end{enumerate}


\begin{qsolve}
	
	\begin{enumerate}
		\item  زمانی که باید به داده یا دستور مورد نظر دسترسی پیدا کنیم، به توجه به مفهوم صفحه بندی، از فرمول زیر می‌توان استفاده کرد:
		
		زمان دسترسی = زمان مراجعه به حافظه + زمان پیدا کردن داده در حافظه
		
		معمولاً زمان مراجعه به حافظه در مرتبه نانوثانیه است. زمان پیدا کردن داده در صفحه نیز بستگی به سازماندهی و سیاست‌های صفحه‌بندی دارد.
		
		
		\item زمان موثر دسترسی از رابطه زیر محاسبه می‌شود:
		زمان موثر دسترسی = زمان مراجعه به TLB + زمان پیدا کردن داده در صفحه
		
		در اینجا فرض کردیم که زمان مراجعه به TLB ۲ نانوثانیه است. مقدار زمان پیدا کردن داده در صفحه همانند قسمت الف است و به سیاست‌های صفحه‌بندی و سازماندهی حافظه بستگی دارد.
	\end{enumerate}
	
\end{qsolve}