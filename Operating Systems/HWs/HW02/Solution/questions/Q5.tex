\section{سوال پنجم}
با فرض آنکه پردازه‌های \texttt{producer} و \texttt{consumer} به نحو زیر پیاده‌سازی شده اند. اگر سایز \texttt{buffer} برابر با ۵ باشد، و متغیر های in و out برابر با صفر باشد، در هر مورد، خروجی را با ذکر دلیل مشخص کنید.


\begin{latin}
\begin{lstlisting}[label=first,caption=Some Code, language=C]
Producer:

int next_produced = 0;
while(next_produced < 10)
{
  buffer[in] = ++next_produced;
  in = (in + 1) % BUFFER_SIZE
}
\end{lstlisting}
\end{latin}


\begin{latin}
\begin{lstlisting}[label=first,caption=Some Code, language=C]
Consumer:

int next_consumed, sum;
while(next_consumed < 10)
{
  if(in = = out)
    continue;
  next_consumed = buffer[out];
  out = (out + 1) % BUFFER_SIZE
  sum += next_consumed;
}
printf("%d", sum);
\end{lstlisting}
\end{latin}



\begin{qsolve}
	\begin{enumerate}
		\item به ازای هر یک‌بار اجرای بدنه، حلقه \texttt{producer} یک‌بار بدنه حلقه \texttt{consumer} اجرا شود.\\ \textbf{توضیحات: }
اگر تنها یک بار حلقهٔ بدنهٔ مصرف‌کننده و یک بار حلقهٔ بدنه تولیدکننده اجرا شونداگر تنها یک بار حلقهٔ بدنه مصرف‌کننده و یک بار حلقه بدنه تولیدکننده اجرا شوند، مقدار sum برابر با صفر خواهد بود. زیرا تولیدکننده ۱۰ عدد از ۱ تا ۱۰ را در بافر قرار می‌دهد، اما مصرف‌کننده هیچ عددی را مصرف نمی‌کند. این اتفاق به دلیل این است که بعد از قرار دادن اعداد در بافر، مصرف‌کننده هنوز شروع به مصرف نکرده است و در شرایطی که in و out برابر باشند، حلقه مصرف‌کننده به continue می‌رود و از مصرف عدد خودداری می‌کند.

	\item به ازای هر دو‌بار اجرای بدنه، حلقه \texttt{producer} یک‌بار بدنه حلقه \texttt{consumer} اجرا شود.\\ \textbf{توضیحات: }
اگر دو بار حلقه بدنه مصرف‌کننده و یک بار حلقه بدنه تولیدکننده اجرا شوند
در این حالت، مقدار sum برابر با ۵ خواهد بود. تولیدکننده ۱۰ عدد از ۱ تا ۱۰ را در بافر قرار می‌دهد و مصرف‌کننده در دو بار اجرا، پنج عدد از بافر را مصرف می‌کند. در هر بار اجرا، مصرف‌کننده با بررسی شرط \texttt{in == out} به مصرف عدد پرداخته و مقدار \texttt{next\_consumed} را در sum اضافه می‌کند. اما در بین دو بار اجرا، مقدار in و out به هم نزدیک می‌شوند و همیشه شرط \texttt{in == out }برقرار نمی‌شود. به عبارت دیگر، مصرف‌کننده در هر بار اجرا در حداکثر یک عدد را مصرف می‌کند و در بارهای بعدی باید منتظر تولیدکننده بماند. بنابراین، مقدار sum برابر با جمع اعداد ۱ تا ۵ خواهد بود که برابر با ۵ است.
	
	
	\item به ازای هر سه‌بار اجرای بدنه، حلقه \texttt{producer} یک‌بار بدنه حلقه \texttt{consumer} اجرا شود.\\ \textbf{توضیحات: }
اگر سه بار حلقه بدنه مصرف‌کننده و یک بار حلقه بدنه تولیدکننده اجرا شوند
در این حالت، مقدار sum برابر با ۱۵ خواهد بود. تولیدکننده ۱۰ عدد از ۱ تا ۱۰ را در بافر قرار می‌دهد و مصرف‌کننده در سه بار اجرا، همه اعداد موجود در بافر را مصرف می‌کند. در اولین بار اجرا، مصرف‌کننده با بررسی شرط \texttt{in == out} به مصرف اعداد ۱ تا ۵ می‌پردازد و مقدار sum را به ترتیب با ۱، ۲، ۳، ۴ و ۵ افزایش می‌دهد.
	
	\end{enumerate}
\end{qsolve}