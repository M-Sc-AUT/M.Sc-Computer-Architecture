\section{سوال چهارم}
‫با‬ ‫فرض‬ ‫موفقیت‬ ‫آمیز‬ ‫بودن‬ ‫اجرای‬ ‫دستورات‬ \texttt{fork} و ‫‪\texttt{execv}‬‬ ‫خروجی‬‫قطعه‬ ‫کد‬ ‫زیر‬ ‫را‬ ‫به‬ ‫صورت‬ ‫دقیق‬ ‫و‬ ‫با‬ ‫ذکر‬ ‫دلیل‬ بیان‬‫کنید.

\begin{latin}
\begin{lstlisting}[label=first,caption=Some Code, language=C]
int main()
{
  pid_t pid;
  pid = fork();
  if(pid = = 0)
  {
	printf("process 1\n");
	char* args[] = {"ls", "-1", NULL};
	execv("/bin/ls", args);
    printf("‫‪process‬‬ 1 finished\n");
  }
  else if(pid > 0)
  {
  	printf("‫‪process‬‬ 2\n");
  	wait(NULL);
  	printf("‫‪process‬‬ 1 ‫‪terminated‬‬\n");
  }
  return 0;

}
\end{lstlisting}
\end{latin}

\begin{qsolve}
	کد فوق یک برنامه ساده را نشان می‌دهد که از تابع \texttt{fork} برای ایجاد یک فرزند جدید استفاده می‌کند. در ادامه، فرزند ایجاد شده با استفاده از تابع \texttt{execv} اجرای برنامهٔ \texttt{"ls"} را انجام می‌دهد. در ادامه، والد منتظر فرزند خود می‌ماند تا اجرای فرزند به پایان برسد. سپس والد پیامی چاپ کرده و برنامه پایان می‌یابد.
	
	حال به صورت دقیق خروجی برنامه را تحلیل می‌کنیم:
	\begin{enumerate}
		\item در صورت موفقیت آمیز بودن تابع \texttt{fork}، والد یک پردازهٔ فرزند را ایجاد می‌کند و بازگشتی غیر صفر دارد. در صورتی که خطایی رخ دهد، بازگشتی کمتر از صفر خواهد داشت.
		
		\item در صورتی که \texttt{pid} برابر ۰ باشد، به این معنی است که کد در حال اجرا در پردازهٔ فرزند است. در این حالت:
		\begin{itemize}
			\item \texttt{1 process} چاپ می‌شود.
			\item یک آرایه از رشته‌ها به نام \texttt{args} تعریف می‌شود که مسیر برنامهٔ \texttt{"ls"} و پارامترهای آن را مشخص می‌کند.
			\item تابع \texttt{execv} فراخوانی می‌شود تا برنامهٔ \texttt{"ls"} را با استفاده از آرگومان‌های مشخص شده اجرا کند. اگر این تابع با موفقیت اجرا شود، کنترل برنامه به برنامهٔ \texttt{"ls"} منتقل می‌شود و دستورات بعدی در کد اجرا نمی‌شوند.
			\item در صورتی که تابع \texttt{execv} با خطا مواجه شود و اجرای برنامهٔ \texttt{"ls"} انجام نشود، پیام \texttt{finished 1 process} چاپ می‌شود. این پیام هیچگاه نمایش داده نمی‌شود زیرا کنترل برنامه به برنامهٔ \texttt{"ls"} منتقل می‌شود و دستورات بعدی اجرا نمی‌شوند.
		\end{itemize}
		
		\item در صورتی که \texttt{pid} بزرگتر از ۰ باشد، به این معنی است که کد در حال اجرا در پردازهٔ والد است. در این حالت:
		\begin{itemize}
			\item \texttt{process 2} چاپ می‌شود.
			\item با استفاده از تابع \texttt{wait}، والد منتظر اجرای فرزند خود می‌ماند تا به پایان برسد.
			\item پس از اتمام اجرای فرزند، پیام \texttt{terminated 1 process} چاپ می‌شود.
		\end{itemize}
		
		
بنابراین، خروجی نهایی برنامهٔ فوق، به ترتیب چاپ شدن پیام‌در این کد، در صورت موفقیت آمیز بودن تابع \texttt{fork}، یک پردازهٔ فرزند ایجاد می‌شود و بازگشتی غیر صفر دارد. اگر \texttt{pid} برابر 0 باشد، به این معنی است که کد در حال اجرا در پردازهٔ فرزند است. در این حالت، \texttt{Process 1} چاپ می‌شود و سپس تابع \texttt{execv} فراخوانی می‌شود تا برنامهٔ \texttt{"ls"} را با استفاده از آرگومان‌های مشخص شده اجرا کند. اگر این تابع با موفقیت اجرا شود، کنترل برنامه به برنامهٔ \texttt{"ls"} منتقل می‌شود و دستورات بعدی در کد اجرا نمی‌شوند. اگر تابع \texttt{execv} با خطا مواجه شود و اجرای برنامهٔ \texttt{"ls"} انجام نشود، پیام
 \texttt{finished 1 process} چاپ می‌شود.
		
		
	
در صورتی که \texttt{pid} بزرگتر از 0 باشد، به این معنی است که کد در حال اجرا در پردازهٔ والد است. در این حالت، \texttt{process 2} چاپ می‌شود و با استفاده از تابع \texttt{wait}، والد منتظر اجرای فرزند خود می‌ماند تا به پایان برسد. پس از اتمام اجرای فرزند، پیام 
\texttt{terminated 1 process}
چاپ می‌شود.

در نتیجه، خروجی برنامه به صورت زیر خواهد بود:
\begin{latin}
	\texttt{process 2}\\
	\texttt{process 1}\\
	\texttt{<ls results of program>}\\
	\texttt{process 1 terminated}\\
\end{latin}

توجه کنید که خروجی برنامهٔ \texttt{"ls"} به عنوان نتیجهٔ اجرای \texttt{execv} وابسته به محتوای فایل‌ها و دایرکتوری‌های حاضر در مسیر \texttt{bin/ls/} است.
		
		
	\end{enumerate}
\end{qsolve}